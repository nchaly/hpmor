\chapter{Гіпотэза Эфектыўнага Рынку}

\begin{chapterOpeningQuote}
"Сусветная дыктатура" --- такая агідная фраза. Мне болей падабаецца 
называць гэта "глабальнай аптымізацыяй".
\end{chapterOpeningQuote}

\lettrine{С}{топкі} залатых галеонаў. 
Кладкі срэбных сыкляў. Кучы бронзавых кнатаў.

Гары стаяў у сямейным сховішчы, разявіўшы рот. У яго было так шмат пытанняў, што ён не ведаў,
з чаго пачаць.

Звонку расчыненых дзверэй МакГонагал стаяла, раслаблена абапершыся аб сцяну, але позірк яе
уважліва сачыў за Гары. І гэта мела сэнс: паставіць чалавека перад агромнай кіпай манет было
такім чыстым тэстам характару, што існавала толькі ў казках. 

--- Усе гэтыя манеты --- чысты метал? --- нарэшце сказаў Гары.

--- Шшшто? --- прасычэў гоблін Грыпхук, які стаяў у дзвярах. --- Вы ў чымсьці падазраваеце
Грынготс, містэр Потэр-Эванс-Верэс?

--- Не, --- сказаў задумённа Гары, --- зусім не, прабачце, калі так прагучала, сэр. Проста ў мяне
няма ніякага паняцця, як працуе ваша фінансавая сістэма. Мае пытынне было, ці павінны быць усе 
галеоны зроблены з чыстага золата?

--- Само сабой, --- сказаў Грыпхук.

--- І ці можа хто заўгодна чаканіць іх, або гэтае права належыць манаполіі, якая такім чынам
атрымлівае сеньёраж\footnote{{}Прыбытак, які атрымлівае эмітэнт грошаў.}?

--- Што? --- спытала МакГонагал. 

Грыпхук задаволена выскаліўся, паказваючы вострыя зубы.

--- Толькі дурань даверыцца не-гоблінскім манетам!

--- Іншымі словамі, --- працягваў Гары, --- манета павінна каштаваць роўна столькі, колькі
каштуе метал, з якога манета зроблена?

Грыпхук глядзеў на Гары бязглузда. МакГонагал выглядала зацікаўленай.

--- Я маю на ўвазе, уявіце, што я прыйшоў да вас з тонай срэбра. Магу я ў гэтым выпадку загадаць
зрабіць з яго тону сыкляў? 

--- За пэўны працэнт, містэр Потэр-Эванс-Верэс, --- вочы гобліна таямніча бляснулі. --- За пэўны
працэнт... Дарэчы, дзе вы плануеце здабыць тону срэбра?

--- Я казаў гіпатэтычна, --- сказаў Гары. \emph{Прынамсі, пакуль што.} --- І колькі стварае ваш
працэнт?

Грыпхук насцярожыўся.

--- Мне будзе трэба абмеркаваць гэта з начальствам...

--- Скажыце срэдняе значэнне. Я не збіраюся лавіць вас на слове.

--- Дваццатая частка металу будзе прыемным працэнтам за чаканку.

Гары кіўнуў:

--- Вялікі дзякуй, містэр Грыпхук.

\emph{Эканоміка магічнага свету не толькі цалкам ізаляваная ад маглаўскай, дык яшчэ ніхто тут 
нават і 
не чуў пра арбітраж\footnote{{}Паслядоўнасць ўгодаў, якія накіраваны на атрыманне прыбытку праз
розніцу кошту тавара на розных рынках або ў розныя часы.}.} У маглаў курсы абмену пастаянна
вагаліся, але ў магічным свеце ён быў пастаянны: сямнаццаць сыкляў за галеон.
Таму, калі на біржы курс золата-да-срэбра адкланяўся больш чым на пяць адсоткаў
ад гэтай канстанты, можна было выкачваць ці срэбра, ці золата
(у залежнасці ад боку адкланення) з магічнага свету пакуль курс не вернецца назад. 
Напрыклад, срэбра даражэй за золата: прыносіш тону 
срэбра, мяняеш на сыклі (мінус пяць адсоткаў), мяняеш сыклі на галеоны, вязеш золата ў 
маглаўскі банк, мяняеш на срэбра, і атрымліваеш яго больш, чым у цябе было ў пачатку, і
так раз за разам.

Зараз у маглаўскім свеце курс золата да срэбра быў наўродзе каля пяцідзесяці? У любым выпадку, Гары 
быў амаль упэўнены, што не сямнаццаць. І выглядала, што срэбныя манеты былі \emph{меншыя} 
за залатыя.

Але яшчэ раз, Гары знаходзіўся ў банку, які \emph{літаральна} захоўваў твае сродкі ў сховішчы з
залатымі манетамі, якое абаранялі драконы, і куды ты быў павінен прыходзіць і фізічна забіраць 
манеты кожны раз, калі ты хацеў нешта набыць. Гары задавіў у сабе жаданне 
выдаць нешта саркастычнае на конт неэфектыўнасці іх фінансавай  сістэмы...

\emph{Самае сумнае ў тым, што іх падыход мабыць і лепей.}

З другога боку, хапіла бы аднаго талковага менеджера хедж-фонду, каб узяць уладу над фінансамі 
магічнага свету за тыдзень. Гары зрабіў паметку ў галаве вярнуцца да гэтай думкі пазней, у 
выпадку, калі ён раптам страціць свае грошы, або ў яго будзе свабодны тыдзень. 

А пакуль што гіганцкая кіпа залатых манет у сямейным сховішчы павінна была цалкам задаволіць яго
кароткатэрміновыя патрабаванні.

Гары зрабіў шаг наперад, нахіліўся, і пачаў левай рукой браць манеты па адной і класці іх у правую.
Калі ён налічыў дваццаць, МакГонагал кашлянула:

--- Думаю, гэтага болей чым дастаткова для вучэбных матэрыялаў, містэр Потэр.

--- Ммм? --- сказаў Гары праз туман сваіх думак. --- Пачакайце, я падлічваю задачу Фермі.

--- \emph{Каго?} --- спытала МакГонагал устрывожаным тонам.

--- Гэта матэматычны слэнг, назва ў гонар Энрыка Фермі. Задача Фермі --- гэта спосаб хутка
вылічаць прыкладнае значэнне лічбаў ў галаве...

Дваццаць залатых галеонаў важылі... ну, прыкладна сто грам, адну дзясятую кілаграма? І
золата каштавала... ну,
прыкладна дзесяць тысяч брытанскіх фунтаў за кілаграм? Тады кошт аднаго галеона складаў... 
прыкладна пяцьдзясят фунтаў. Далей прыкідваем, што пірамідка манет мае прыкладна шесцьдзясят
манет увышыню і дваццаць --- у аснове. А піраміда займае аб'ём як трэцяя частка куба...
Таму восем тысяч манет на кучку, прыкладна, і ўсяго такіх кучак было пяць, таму ўсяго ---
сорак тысяч галеонаў, або два мільёна брытанскіх фунтаў.

Няблага. Гары ўсміхнуўся з нейкім змрочным задавальненнем. На жаль, ён быў толькі ў самым 
пачатку шляху вывучэння магічнага свету, і ў яго не было часу, каб даследваць свет неймаверна
заможных людзей. Да таго ж, яшчэ адно вылічэнне Фермі паказала, што ён будзе прыкладна ў 
мільярд разоў менш цікавы.

\emph{Затое, больш мне ніколі не прыйдзецца касіць газон за дробязь.}

Гары вярнуўся да дзвярэй. 

--- Прабачце за пытанне, прафесар МакГонагал, але, я так разумею, маім бацькам падчас смерці
было каля дваццаці пяці гадоў. Ці гэта \emph{нармальна} ў магічным свеце мець такую суму грошаў
для маладой пары? 

Калі гэта было нормай, то кубачак кавы павінен быў каштаваць калі пяці 
тысяч. Бо гэта закон эканомікі: грошы нельга есці.

МакГонагал адмоўна пакачала галавой.

--- Ваш бацька быў адзіным нашчадкам даволі старога роду, містэр Потэр. Таксама, магчыма...
--- яна завагалася, --- некаторыя грошы --- гэта ўзнагарода заб... --- яна праглынула гэтае слова, --- 
таму, хто перамог Самі-Ведаеце-Каго. Але дакладна я не ведаю.

--- Цікава... --- сказаў Гары павольна. --- Атрымліваецца, што некаторыя грошы --- мае.
Можна сказаць, я зарабіў іх сам. Магчыма. Кшталту таго. Нават калі я і не памятаю... --- і ён 
рашуча хлопнуў сябе па кішэні: --- Што памяншае маё пачуццё віны за тое, што я патрачу іх
\emph{вельмі маленькую частку. Спакуха, прафесар МакГонагал!}

--- Містэр Потэр! Вы непаўналетні, і таму вам дазваляецца браць толькі \emph{разумныя} сумы з...

--- Я ўсімі рукамі \emph{за} разумны падыход! Цалкам падтрымліваю ідэі фінансавай разважлівасці
і самакантролю. Але па дарозе сюды я \emph{сваімі вачыма} бачыў некалькі рэчаў, якія сапраўды
падыходзяць пад вызначэнне \emph{разумных набыткаў}...

Гары пачаў свідраваць вачыма МакГонагал, якая прыняла вызаў, і моўчкі глядзела ў адказ.

--- Напрыклад? --- сказала яна нарэшце.

--- Куфар з утунтраным аб'ёмам значна больш, чым вонкавым?

МакГонагал нахмурыла бровы. 

--- Яны \emph{вельмі} дарагія, містэр Потэр!

--- Так... але... --- Гары пераключыўся на упрашальны тон, --- я ўпэўнены, што калі вырасту, 
я абавязкова захачу такой. І я магу дазволіць сабе, праўда. І таму гэта мае сэнс  --- набыць
яго тут і зараз, бо ўсё роўна прыйдзецца патраціць тую ж суму, ці не так? Тады якая розніца ---
зараз, або праз гады? А мне захочацца мець якасны, каб унутры было шмат месца, настолькі 
добры, ка не апгрэйдзіць яго потым... --- голас Гары, поўны надзеі, заціх, не скончыўшы сказ.

МакГонагал не адводзіла позірк.

--- І \emph{што менавіта} вы збіраецеся захоўваць у настолькі вялікім куфары, містэр Потэр?

--- Кнігі.

--- Ну, канешне, --- уздыхнула МакГонагал.

--- Вы павінны былі сказаць раней, што існуе магія такога роду! І што я магу дазволіць сабе такое!
Вы толькі сабе ўявіце! Мы з бацькам будзем вымушаны правесці наступныя два дні \emph{ліхаманкава}
бегаючы па букіністах, каб сабраць для мяне ў Хогвартс годную навуковую бібліятэку, і мабыць
нават міні-калекцыю фантастыкі, калі нешта годнае наогул знойдзецца на барахолцы.
І гэта яшчэ не ўсё! --- Гары сам не заўважыў, як перайшоў на мову тэле-крамы, --- Дазвольце мне
палепшыць маю прапанову, проста...

--- \emph{Містэр Потэр!} Ці не думаеце вы прапанаваць мне  \emph{хабар}?

--- Што? \emph{Не!} Вы не так зразумелі. Мая прапанова ў тым, што ўсе ў Хогвартсе могуць
карыстацца маімі кнігамі, калі вы лічаце, што яны змогуць дапоўніць тамтэйшую бібліятэку.
Я ўсё роўна планую траціць на іх як мага менш, і мне не шкада падзяліцца і прынесці нейкую
карысць. Хабар у выглядзе \emph{бібліятэчнага абанементу} --- гэта ж не хабар, ці не так?
Гэта ў нас такая...

--- Сямейная традыцыя.

--- Дакладна.

МакГонагал неяк абмякла. 

--- На жаль, я бачу логіку ў вашых словах, і не змагу яе разбачыць. Я дазваляю вам узяць яшчэ
сто галеонаў, містэр Потэр. Я \emph{ведаю}, што пашкадую аб гэтым, але ўсё роўна дазваляю.

--- Ну вы даёце! А праўда "Махляскінавы кашэль" робіць тое, што кажа рэклама?

--- Да куфара Нават блізка не стаяў, --- адказала МакГонагал, вагаючыся. --- Але 
сапраўды, махляскін
з зачараванямі Аўта-дастаўкі і Незаўважнага Пашырэння можа захоўваць
даволі шмат
рэчаў да таго часу, пакуль іх не выклікае той, хто іх туды паклаў.

--- Так, адзназначна мне такі патрэбны. Гэта як супер пояс бясконцай крустасці! Пояс Бэтмена!
І можна забыць пра швейцарскія нажы, бо можна проста насіць у кішэны поўны набор прылад!
Або \emph{кніг!} Напрыклад, можна мець пад рукой тры лепшых кнігі з тых, што я зараз
чытаю, пастаянна! Я больш не змарную ні хвіліны свайго жыцця! Што скажаце, прафесар
МакГонагал? Ці ёсць тут рацыя?

--- Ладна. Можаце ўзяць яшчэ дзесяць.

Грыпхук глядзеў на Гары, позіркам поўным нават не пашаны, а адкрытага абагаўлення.

--- І крыху на кішэнныя расходы, як вы сказалі раней. Я думаю, што калі вы выйдзем на вуліцу,
я адразу ўспомню некалькі рэчаў, якія адметна падыйдуць да гэтага кашаля.

--- \emph{Не забывайцеся, містэр Потэр!}


--- Ну, прафе-есар МакГонагал, вы ж ламаеце мне ўвесь кайф. Я нагадаю вам, што гэта \emph{той самы}
дзень, калі я ўпершыню даведаўся пра магічныя рэчы! Чаму трэба павоздзіць сабе як бурчлівы
стары дзед? Можна проста ўсміхнуцца і ўспомніць свае бязвіннае дзяцінства, і проста назіраць,
якую асалоду адчувае дзіця, набываючы цацку-другую, патраціўшы пры гэтым малюсенькі
кавалачак свайго багацця, заработанае, паміж іншым, паразай самага жудаснага вядзьмара ў 
гісторыі Брытаніі --- не хачу вас пакрыўдзіць, кажучы, што вы недастаткова ўдзячны, 
але, ну праўда, чаго вартыя пару цацак у параўнаннні?

--- Ургггг, --- МакГонагал прарычала праз зубы нешта незразумелае. Выраз на яе твары 
быў такі жахлівы і
жудасны, што Гары мімаволі ціўкнуў, зрабіў некалькі крокаў назад, і, зачапіўшыся за адну з
залатых пірамід, страціў раўнавагу і ўпаў. Манеты разляцеліся па падлозе з гучным звонам.
Грыпхук уздыхнуў, закрыўшы твар далонню.

--- Думаю, што замкнуўшы вас зараз у гэтым сховішчы, містэр Потэр, я
зраблю вялізную паслугу магічнаму свету, --- сказала яна нарэшце.

І ніякіх праблем з Гары ў МакГонагал больш у банку не было.

