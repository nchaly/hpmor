\chapter{Рацыяналізацыя}

%\begin{chapterOpeningAuthorNote}
%Rowling is whoever does Rowling’s job.
%\end{chapterOpeningAuthorNote}

\lettrine{Г}{ерміёна} Грэнджэр турбавалася, што пераходзіць на Цёмны бок.

Вызначыць розніцу паміж Светлым і Цёмным бакамі звычайна было лёгка,
і яна ніколі не разумела, чаму ў астатніх былі з гэтым праблемы. Светлымі былі 
прафесар Флітвік і прафесар МакГонагал і прафесар Спраут. Цёмнымі былі прафесар 
Снэйп і прафесар Квірэл і Драко Малфой. Гары Потэр... ён быў адным з тых 
рэдкіх выпадкаў, калі нельга было сказаць з першага погляду. Яна ўсё яшчэ 
спрабавала вырашыць, на якім ён баку.  

Але калі справа дайшла да \emph{яе самой...}

Герміёне \emph{вельмі} падабалася раструшчваць Гары Потэра.

Яе адзнакі былі лепей за яго па ўсіх дысцыплінах (за выключэннем лятання на мётлах,
якое было проста кшталту фізры і таму не лічылася). Амаль кожны дзень іх першага
тыдня яна атрымлівала \emph{сапраўдныя} балы для факультэта, і не за нейкія 
геройскія выбрыкі, а за \emph{разумныя} рэчы, напрыклад за тое, што яна хутка 
засвоіла заклён, або дапамагла іншым студэнтам. Яна ведала, што такія балы былі
лепей, і, што самае цудоўнае, Гары Потэр таксама гэта ведаў. Ягоны позірк пасля 
яе чарговага \emph{сапрадаўнага} бала усё казаў без слоў.

Калі ты Светлы, то не павінны адчуваць \emph{такую} асалоду ад перамогі.

Усё пачалося яшчэ ў цягніку. У той мітуслівы дзень яе шмат чаго 
адцягвала, але перад тым, як заснуць, яна ўрэшце асэнсавала, \emph{наколькі}
яна дазволіла вярцець сабой гэтаму фанаберыстаму хлопцу.

Ёй ніколі не хацелася нікога раструшчыць, пакуль яна не сустрэла Гары Потэра. Калі
побач з ёю нехта з вучняў нешта не разумеў, яе абавязкам было дапамагчы, а не кпіць.
Вось што значыла быць Светлым.

А зараз...

...зараз яна \emph{перамагала}. Гары Потэр уздрыгваўся кожны раз, калі яна 
атрымлівала балы факультэту, і гэта было \emph{так прыемна!} Яе бацькі 
калісьці папярэджвалі яе на конт наркотыкаў, і яна шчыра падазравала, што 
перамога была лепей.

Ёй заўсёды падабалася, як усміхаліся настаўнікі, калі ў яе нешта атрымлівалася
выдатна. Ёй падабаліся доўгія радкі адзнак "правільна" у тэстах. Але зараз, 
зрабіўшы нешта на занятку, яна першай справай няўзнак кідала позірк у бок Гары 
Потэра, скрыгаючага зубамі, і ад гэтага відовішча ёй хацелася пець у голас,
нібыта прынцэсе ў дыснееўскім фільме.

Але... гэта была цёмная прымета, ці не?

І таму Герміёна турбавалася, што становіцца Цёмнай.

Але потым яе прасякла думка, якая ў момант сцёрла ўсе страхі.

У іх з Гары пачынаўся \emph{раман!} Ну канешне! Усе ведалі, што значыць, калі 
хлопец і дзеўчына пачыналі адно аднаго раздражняць. Яны проста так \emph{фліртавалі!}
У гэтым не было нічога Цёмнага.

Ну не магло такога быць, што ёй \emph{проста} падабалася выбіваць навуковы пыл з 
амаль самага знакамітага вучня школы, які быў \emph{у кнігах} і нават гаварыў, 
як у кнігах; які падолеў Цёмнага Лорда, і нават расціснуў прафесара Снэйпа, быццам 
нейкага жука; які, як сказаў бы прафесар Квірэл, дамінаваў над усімі першакурснікамі
Рэйвенкло, \emph{за выключэннем} Герміёны Грэнджэр, якая была на галаву вышэй за 
Хлопца-які-выжыў па ўсім прадметам, акрамя мёцел. 

Бо гэта было бы Цёмна.

Не. Гэта была Любоў. Дакладна так. Вось чаму яны пастаянна сварыліся.

Герміёна была рада, што зразумела гэта менавіта сёння, у дзень, калі Гары 
нарэшце прайграе іх чытальнае спаборніцтва, пра якое ведала ўся школа. Ёй хацелася
ўскочыць і пачаць танцаваць ад неймавернай асалоды.

\later

Была субота, 14:45, і ў Гары Потэра заставалася недачытанай добрая палова
\emph{Гісторыі магіі} Бацільды Бэгшот, і Герміёна глядзела на свой гадзіннік,
які жудасна няспешна набліжаўся да 14:47. 

Увесь агульны пакой Рэйвенкло назіраў, прытаіўшы подых.

Тут сабраліся не толькі першакурснікі. Навіны разбегліся па факультэце, нібыта 
малако, разлітае ў лядоўні, і амаль палова Рэйвенкло набілася ў пакой, падпіраючы
сцены і кніжныя шафы, намагчаючы ўціснуцца ўдвая больш у канапы, седзячы на 
бакавінах фатэляў. Прысутнічалі ўсе шасцёра прэфектаў, уключаючы дзеўчыну-старасту
школы. Паветра было настолькі спёртае, што нехта зкаставаў заклён
пасвяжэння паветра. Гоман размоваў паступова перайшоў у шэлест шэпту, які 
канчаткова расчыніўся ў поўнай цішыні.

14:46.

Напружанне было невыноснае. Калі бы на месцы Гары быў нехта іншы, ягоная параза 
была бы ўжо вырашанай справай.

Але гэта быў Гары Потэр, і нельга было выключаць магчымасць, што ён мог у наступныя 
некалькі секунд падняць руку і цокнуць пальцамі.

У жудасным прыступе паніцы Герміёна ўсвядоміла, што можа ў гэтым і ёсць яго план.
Гэта будзе абсалютна ў ягоным стылі, калі ён насамрэч прачытаў апошнюю палову 
кнігі калісьці раней...

У вачах Герміёны пакой пачаў расплывацца. Яна спрабавала прымусіць сябе дыхаць, але 
проста не магла.

Засталося дзесяць секунд, а ён усё яшчэ не падняў руку.

Пяць секунд.

14:47!

Гары Потэр акуратна паклаў закладку ў кнігу, закрыў яе, і паклаў на стол.

--- Хачу заўважыць для гісторыі, --- сказаў спакойным голасам Хлопец-які-выжыў, ---
што ў мяне засталася толькі палова адной кнігі, і пры гэтым я пастаянна сутыкаўся 
з нечаканымі зактрымкамі... 

--- \emph{Ты прайграў!} --- правішчэла Герміёна. --- Ты! Сапраўды! Прайграў! 
Наша! Спаборніцтва!

Пачуўся агульны ўдых, калі ўсе прысутнічаючыя зноў пачалі дыхаць.

Гары Потэр кінуў у Герміёну позірк Палаючага Вогню, але нічога не магло крануцца яе,
атуленую цеплынёй чыстага шчасця.

--- \emph{Ты ўяўляеш, што гэта быў за тыдзень для мяне?} --- сказаў Гары Потэр. 
--- Хтосьці іншы не здолеў бы і Чырвоную Шапачку прачытаць!

--- Ты сам выбраў час.

Палаючы агонь у вачах Гары стаў яшчэ пякельней. 

--- У мяне не было і не магло быць ніякай інфармацыі аб тым, што мне прыйдзецца
ратаваць школу ад Снэйпа, або атрымаць пабоі на ўроку Абароны, і калі я скажу, што 
цалкам згубіў адрэзак часу паміж пяццю і сям'ю ў чацьвер, ты падумаеш, што я
канчаткова з'ехаў з глузду...

--- О, выглядае, быццам \emph{нехта} сам трапіў у пастку \emph{заблуды планавання.}

Гарын выраз стаў выявай чыстага шоку.

--- Дарэчы, ты мне нагадаў. Я скончыла чытаць першую стопку кніг, што ты мне 
пазычыў, --- сказала Герміёна з нявінным позіркам. Некаторыя з іх былі \emph{сапраўды
складанымі.} Ёй было цікава, колькі яму спатрэбіцца, каб прачытаць \emph{іх.}

--- Аднойчы, --- сказаў Хлопец-які-выжыў, --- калі далёкія нашчадкі \emph{Homo Sapiens}
будуць пераглядаць гісторыю галактыкі, азадачаныя пытаннем "як яно магло настолькі
пайсці не так?", яны адзназначна высветляць, што галоўнай памылкай было навучыць 
Герміёну Грэнджэр чытаць.

--- Але ты прайграў, --- сказала Герміёна. Яна задумённа пацёрла падбародак. ---
А зараз... цікава, што канкрэтна ты праспорыў?...

--- \emph{Што?}

--- Ты прайграў спор, --- патлумачыла Герміёна, --- і таму ты мне нешта вінен.

--- Не памятаю, калі я на такое пагаджаўся!

--- Няўжо? --- сказала Герміёна Грэнджэр. Яна зноў прыняла задуменны выгляд. 
Потым, быццам ідэя прыйшла да яе толькі ў гэты момант: --- Ну што. Давай тады 
зладзім галасаванне. Усе, хто лічыць, што Гары Потэр мне вінен, падыміце руку!

--- \emph{Што?} --- зной прасычэў Гары. 

Ён абярнуўся і ўбачыў лес рук.

І калі бы ён прыгледзеўся \emph{палепей,} ён бы заўважыў, што большасць з 
прысутных былі дзеўчыны, і кожная стаяла з паднятай угору рукой.

--- Стойце! --- пралямантаваў Гары Потэр. --- Вы ж нават не ведаеце, што яна хоча!
Ці вы не разумееце, \emph{што} яна робіць? Яна прымушае вас загадзя нешта абяцаць, 
і потым пад пагрозай паказацца непаслядоўнымі вы будзеце вымушаны згадзіцца з 
усім, што яна прыдумае апасля! 

--- Не хвалюйся, --- сказала прэфектэса Перелопа Кліруотэр. --- Нам не будзе 
сорамна перадумаць, калі яна запытае нешта неразважнае. Усе згодныя?

Адказам ёй былі згодня кіўкі дзяўчын, з якімм Пенелопа Кліруотэр раней падзялілася
планам Герміёны.


\later

Бясшумны цень прабіраўся па прахалодных калідорах падзямелля Хогвартс.
Ён павінен быў быць у пэўным пакоі ў 18:00 каб сустрэцца з пэўным некім,
і каб прадэманстраваць павагу, лепей было апынуцца там першым.

Але, калі ягоная рука павярнула ручку і расчыніла дзверы ў гэты цёмны, ціхі,
закінуты пакой, ён убачыў сілуэт у асяроддзі старых пыльных парт. Сілуэт трымаў 
невялічкі зялёны ліхтарых, бледна-зялёнага святла якога ледзве хапала 
на асвятленне гаспадара, не кажучы аб пакоі вакол.

Святло з калідора атрэзалі, зачыніўшыся, дзверы, і вочы Драко пачалі звыкацца з 
цьмяным пакоем.

Сілуэт павольна абярнуўся, так, што толькі частка твара стала асвечана слабым 
зялёным.

Драко ўжо падавалася гэтая сустрэча. Пакінуць зялёнае святво, зрабіць іх абодвух 
вышэй, даць ім плашчы і маскі, замяніць клас на могілкі, і гэта будзе быццам завязка
паловы гісторый, што яму расказвалі сябры яго бацькі пра Пажыральнікаў Смерці.

--- Я хачу, каб ты ведаў, Драко Малфой, --- сказаў сілуэт з магільным спакоем, ---
што я не вінавачу цябе ў маёй нядаўняй паразе.

Драко адчыніў быў рот, каб начаць без разбору пратэставаць, бо не было ніякай 
прычыны вінаваціць \emph{яго...}

--- Прычынай была, болей за астатнее, мая ўласная дурасць, --- размерана працягваў 
сілуэт. --- Бо амаль на кожным кроку я мог шмат чаго знабіць інакш. Ты не запытаў
зрабіць \emph{менавіта} тое, што я зрабіў. Ты проста запытаў дапамогі. І я 
сам абраў спосаб. Але факт застаецца фактам: я прайграў спаборніцтва на палову кнігі.
Дзеянні тваіх ручных ідыётаў, твая просьба, і, так, мая ўласная дурасць яе выканаць
прымусілі мяне сгубіць час. Болей часу, чым ты думаеш. Часу, які быў крытычна патрэбны,
як паказала будучыня. І яшчэ факт застаецца, Драко Малфой, што калі бы не 
твая просьба, я бы выйграў. Замест таго... каб... \emph{прайграць.}

Да Драко ўжо дайшлі чуткі пра паразу Гары, і плату, якую запатрабавала з яго 
Грэнджэр. Навіны разышліся хутчэй, чым іх маглі разнесці совы.


--- Разумею, --- сказаў Драко, --- і мне шкада, --- калі ён хацеў Гарынага сяброўства,
то ён і не мог сказаць нічога іншага.

--- Я не пытаю разумення або шкадавання, --- сказаў цёмны сілуэт з тым жа 
спакоем. --- Але я толькі што правёў дзве гадзіны ў кампаніі Герміёны Грэнджэр,
апрануты ў адзенне, абранае ёю; разглядваючы такія дзіўныя месцы Хогвартс, як 
маленькі вадапад нечага, што выглядала, як соплі; суправаджаемы групай іншых 
дзеўчын, якія ўпарта дадавалі да нашага шпацыру такія элементы, як трансфігураваныя
ружовыя пялёсткі. Я быў на спатканні, нашчадак Малфоеў. Мае \emph{першае}
спатканне. \emph{Таму калі прыйдзе час, ты зробіш мне адказную паслугу без 
аніякіх умоў.}   

Драко змрочна-ўрачыста кіўнуў. Перад тым, як прыйсці, ён даволі мудра вырашыў 
дазнацца дэталяў Гарынага спаткання ва ўсіх падрабязнасцях, каб прыступы 
гістэрычнага смеху канчаткова прайшлі да прызначанага часу і не
не сапсавалі важную сустрэчу.

--- Ты не думаеш, --- сказаў ён, --- што нешта сумнае павінна здарыцца з 
гэтай дзеўкай Грэнжэр?...

--- Давядзі да слізэрынаў, што гэтая дзеўка Грэнджэр --- \emph{мая}, і любому, хто 
суне нос ў \emph{мае} прыватныя справы, прыйдзецца вывучаць карту 
дванаццаці галоўных моў планеты, бо менавіта на такой плошчы іх прыйдзецца збіраць
па кавалачках. І таму што я не грыфіндор, і карыстаю \emph{хітрасць} замест 
лабавой атакі, яны павінны не панікаваць, калі заўважаць, што я ёй ўсміхаюся.

--- Або калі заўважаць цябе на другім спатканні? --- сказаў Драко, дазволіўшы 
сабе толькі маленькую нотку скептыцызму.

--- \emph{Ніякага другога спаткання не будзе!} --- сказаў сілуэт настолькі жудасным 
голасам, што гучаў не проста як Пажыральнік Смерці, а як Амікус Кэраў за секунду 
да таго, як бацька сказаў "супакойся, Амікус, ну які з цябе Цёмны Лорд?"

Але, канешне, гэта ўсё яшчэ быў незламаны хлапчуковы голас, і ў спалучэнні са 
словамі, якія ён казаў, гучала даволі смешна. На выпадак, калі Гары Потэр калісьці 
стане Цёмным Лордам, Драко варта было б знайсці думніцу і зберагчы ўспамін аб гэтым моманце
недзе ў надзейным месцы, і тады Гары Потэр ніколі не пасмее яму здрадзіць.

--- Але час абмяркаваць болей вясёлыя рэчы, --- сказаў цень. --- Аб 
ведах і ўладзе. Драко Малфой, мы будзем размаўляць аб Навуцы.

--- Так, --- сказаў Драко. --- Давай размаўляць.

Драко падумаў, якая частка ягонага ўласнага твара было ў цені, а якая была 
асвечана цьмяным зялёным.

І хаця яго выраз быў сур'ёзны, у сэрцы ён усміхаўся.

\emph{Нарэшце} ў яго будзе сапраўдная дарослая размова.

--- Я прапаную табе ўладу, --- сказаў цень, --- але яна мае кошт.
Калі ведаеш аб тым, як збудавана рэчаіснасць, атрымліваеш моц кіраваць ёю.
Што разумееш --- тое і кантралюеш. Гэтай моцы дастаткова, каб прагуляцца па 
Луне. Кошт гэтай улады ў тым, што ты павінен навучыцца задаваць пытанні аб 
Прыродзе, і, што значна складаней, прыймаць яе адказы. Ты будзеш праводзіць 
эксперыменты і тэсты, і назіраць за вынікам. Ты павінны прыймаць рэзультат,
нават, калі ён кажа, што ты памыляешся. Ты павінен \emph{навучыцца праігрываць} --- 
не мне, але самой Прыродзе. Калі спрачаешся з рэчаіснасцю, яна можа і перамагчы.
Гэта будзе балюча, Драко Малфой, і я не ведаю, ці здолееш ты гэта вытрымаць. 
Ведаючы кошт, ці жадаеш ты працягваць і навучыцца чалавечай моцы?      

Драко глыбока ўдыхнуў. Ён ужо думаў пра гэта. І складана было ўявіць, што ён 
мог адказаць неяк інакш. У яго былі інструкцыі карыстаць любую магчымасць 
пасябраваць з Гары Потэрам. Гэта ўсяго толькі \emph{навучанне,} ён не дае 
абяцанне \emph{зрабіць} нешта. Урокі можна перарваць у любы момант... 

Было, канешне, некалькі рысаў у гэтай сітуацыі, з-за якіх яна выглядала, як 
пастка, але, шчыра кажучы, Драко не ўяўляў, як тут нешта магло пайсці не так.

Плюс, Драко быў не супраць пакіраваць сусветам. 

--- Так, --- сказаў ён.

--- Выдатна, --- сказаў цень. --- У мяне быў даволі заняты тыдзень, і мне трэба 
пэўны час, каб саставіць твой план навучання...

--- У мяне ў самога шмат турбот на конт кансалідацыі ўлады на Слізэрыне, --- 
сказаў Драко, --- не кажучы аб хатніх заданнях. Можа пачнем у кастрычніку?

--- Гучыць рэзонна, --- сказаў цень, --- але я збіраўся сказаць, што каб саставіць 
план твайго навучання, я павінен вырашыць, на чым нам зканцэнтравацца. У мяне ёсць
тры варыянта. Першы --- чалавечы мозг і свядомасць. Другі --- фізічны сусвет,
тыя тэхналогіі, якія ў выніку вядуць да палётаў на Луну. Тут будзе шмат вылічэнняў,
але для чалавека пэўнага складу тыя лікі прыгажэй за ўсё астатнее, што навука можа
прапанаваць. Драко, табе падабаюцца лікі?

Драко адмоўна пакачаў галавой.

--- Я так і думаў. Табе прыйдзецца вывучыць матэматыку калісьці, але не з самога
пачатку. Трэці варыянт --- эвалюцыя, гэнетыка, наследванне, тое, што ты завеш 
"чысціня кры..."

--- Гэты, --- сказаў Драко.

Цень кіўнуў. 

--- Я мяркаваў, што гэты варыянт можа прывабіць цябе болей за астатнія. Але ён 
можа апынуцца і болей траўматычным шляхам для цябе, Драко. Што калі 
твая сям'я і сябры --- ахоўнікі чысціні крыві --- кажуць адно, але ты высветліш,
што эксперыменты кажуць іншае?

--- Тады я прыдумаю, як прымусіць эксперыменты даваць \emph{правільны} адказ!

Прайшло некалькі ціхіх секунд, бо цень некаторы час стаяў з адчыненым ротам.

--- Хм, --- сказаў ён, --- гэта так не працуе. Гэта як раз тое, аб чым я папярэджваў,
Драко. У навуцы ты \emph{не можаш} прымусць адказы быць такімі, як табе падабаецца.

--- Ты \emph{заўсёды} можаш зрабіць адказы такімі, як падабаецца, --- сказаў Драко.
Гэта было практычна першай рэччу, якой яго навучылі настаўнікі ў дзяцінстве. --- 
Проста трэба прымяніць правільныя агрументы.

--- Не, --- сказаў цень, раздражнёна павышаючы голас, --- не, не, не! Тады ты 
атрымліваеш дрэнны адказ, і так да Луны не даляціш! Прырода --- не людзі, нельга 
прымусіць яе верыць у нешта. Ты можаш паспрабаваць пераканаць Луну, што яна зроблена 
з сыра, ты можаш хоць цэлы год спрачацца, але Луна не зменіцца ні на грам!
Гэта не рацыянальнасць, а \emph{рацыяналізацыя}, быццам ты бырэш чысты ліст паперы,
унізе яго пішаш: "і, такім чынам, мы робім вывад, што Луна зроблена з сыра", 
а потым вяртаешся ў пачатак, і дадаеш цэлую кіпу разумных аргументаў. Але калі 
ты напісаў вынік унізе, ён ужо ці праўда, ці мана. Няважна, ці сканчваецца ліст 
праўдай, вынік фіксуецца, як толькі ён напісаны. Напрыклад, калі ты абіраеш 
з двух куфараў больш бліскучы, то няважна, колькі мудрагелістых аргументаў 
ты прыведзеш апасля, бо \emph{сапраўднае} правіла, якое ты ўжо прымяніў, было 
"бяры бліскучы", і якім бы дзіўным яно ні было, яно дазваляе табе зрабіць выбар.
Але \emph{рацыянальнасць} нельга карыстаць, каб апраўдаць нешта, што ўжо адбылося,
толькі тады, калі \emph{ёсць з чаго абіраць}.
Навука не для таго, каб \emph{пераканаць} людзей у тым, што ахоўнікі чысціні 
крыві правыя. Гэта --- \emph{палітыка!} Моц навукі заснавана на тым, што 
\emph{сапраўдныя законы Прыроды нельга змяніць проста праз спрэчку!}
А вось што навука можа зрабіць --- гэта расказаць нам, \emph{як працуе кроў ў рэальнасці}, як 
чараўнікі наследуюць сілу ад сваіх бацькоў, і ці сапраўды магланароджаныя слабей,
або сільней...

--- \emph{Сільней?!} --- Драко, разгублена нахмурыўшыся, шчыра спрабаваў сачыць 
за думкай Гары, і падавалася, \emph{амаль} разумеў яе, але, адназначна, гэтыя ідэі 
не нагадвалі нічога, што ён чуў дагэтуль. Але тут Гары Потэр сказаў такое, што 
Драко не мог прапусціць. --- Ты праўда думеш, бруднакроўкі могуць быць сільней?

--- Я нічога не думаю, --- сказаў цень. --- Я нічога не ведаю. Я ні ва што не веру.
Мае меркаванне пакуль не зафіксавана. Я прыдумаю, як пратэсціраваць 
магічную сілу магланароджаных, і параўнаць яе з сілай чыстакроўных чараўнікоў.
Калі мае тэсты пасведчаць, што магланароджаныя слабей, тады я буду верыць, што яны
слабей. Калі наадварот, я буду верыць, што яны сільней. Высветліўшы гэта, я 
атрымаю чарговы кавалак ведаў і ўлады...

--- І ты чакаеш, што я паверу ўсяму, што ты кажаш? --- спытаў рэзка Драко.

--- Я чакаю, што ты \emph{асабіста} будзеш праводзіць свае тэсты, --- сказаў ціха цень.
--- Ты не баішся, \emph{што менавіта} ты можаш высветліць?

Драко некалькі секунд, прыжмурыўшыся, глядзеў на цьмяную постаць. 

--- Дорбая пастка, Гары, --- сказаў ён. --- Трэба запомніць. Такіх я яшчэ не сустракаў.

Цень пакачаў галавой.

--- Гэта не пастка, Драко. Яшчэ раз --- я \emph{не ведаю}, што мы знойдзем.
Немагчыма зразумець сусвет, спрачаючыся з ім, або патрабуя каб ён выйшаў, і 
зайшоў ізноў, ужо с правільным адказам. Калі апранаеш белы халат навукоўцы,
ты павінен забыць пра палітыку, партыі, і інтрыгі, цалкам адчысціць розум, і 
жадаць толькі пачуць адказ сусвета, --- цень сцішыўся на некалькі секунд.
--- Большасць людзей не здольная на гэта. Настолькі гэта цяжка. Ты ўпэўнены,
што не хочаш проста вывучаць, як працуе мозг?

--- І калі я скажу, што я лепей буду проста вывучаць, як працуе мозг, --- 
сказаў Драко жарстчэй, --- ты будзеш распускаць чуткі, што я сапраўды спужаўся 
таго, што мы можаем знайсці.

--- Не. Я такога не зраблю.

--- Але ты бы мог цішком зрабіць тыя ж самыя тэсты, 
і мяне не будзе побач, каб абмяркаваць рэзультаты да таго, 
як ты імі падзелішся з астатнімі.

--- Тады я бы падзяліўся з табой першым, --- сказаў ціха цень.

Драко задумаўся. Такога ён не чакаў... яму падавалася, што тут таксама была 
прыхаваная нейкая пастка, але не мог спыніцца.

--- Праўда?

--- Ну канешне. Скуль \emph{я} ведаю, каго можна шантажаваць такой інфармацыяй, 
і што канкрэтна ў іх патрабаваць? Калі ласка, павер, што я не задумаў для цябе 
ніякай пасткі. Прынамсі для цябе асабіста. Калі б я размаўляў з некім 
іншым, я мог сказаць "а што калі чыстакроўныя сільней?"

--- Да што ты кажаш.

--- Так! Бо каб стаць вучоным, ты і так павінен заплаціць пэўны кошт!

Драко падняў руку, каб яго сцішыць. Яму трэба было падумаць.

На фоне зялёнага святла цьмяная постаць чакала.

Хаця, аб чым тут было думаць. Калі адкінуць усе незразумелыя тэрміны, то выходзіла... 
Гары  Потэр планаваў нейкае ўмяшальніцтва ў палітычную прастору, якое 
магло выклікаць выбух, і трэба было быць вар'ятвам, каб проста даць яму працягваць 
свае справы без прыгляду.

--- Мы будзем вывучаць кроў, --- сказаў Драко.

--- \emph{Выдатна!} --- сказаў цень і ўсміхнуўся. --- Віншую цябе з тым, што 
жадаеш пачаць задаваць пытанні. 

--- Дзякуй, ---  сказаў Драко, не вельмі хаваючы іронію ў голасе. 

--- Гэй, ты што, думаў, што дасягнуць Луны будзе \emph{проста?} Радуйся, што 
гэта пратрабуе проста змены майндсету, а не чалавечых ахвярапрынашэнняў!

--- Чалавечыя ахвярапрынашэння \emph{лягчэй!}

Пасля невялікай паузы цень кіўнуў.

--- Справядліва.

--- Слухай, Гары, --- сказаў Драко без вялікай надзеі, --- я думаў, ідэя была 
ў тым, каб узяць рэчы, якія ведаюць маглы, спалучыць з тым, што ведаюць чараўнікі,
і стаць валадарамі абодвух сусветаў. Не будзе прасцеў проста вывучыць усё,
што маглы \emph{ўжо} прыдумалі, кшталту лунных палётаў, і проста скарыстаць...

--- \emph{Не}, --- сказала постаць, рэзка пакачаўшы галавой, ад чаго зеленаватыя 
цені прабеглі па твары. Голас Гары стаў змрочны. --- Калі ты не можаш засвоіць 
майстэрства прыняцця рэчаіснасці, тады я \emph{павінен} не раскрываць табе, 
чаго дасягнула чалавецтва гэтым шляхам. Гэта як стары магутны маг кажа табе 
не адчыняць запячатаныя дзверы, пакуль ты не даказаў сваю годнасць, розум,
і дысцыпліну, прайшоўшы праз болей простыя выпрабаванні.

Холад прабяжаў па хрыбту Драко, і ён мімаволі ўздрыгнуўся. Ён ведаў, што гэта было 
заўважна нават пры цьмяным святле. 

--- Добра, --- сказаў ён, --- я зразумеў.

Бацька казаў гэта пастаянна. Калі мацнейшы чараўнік кажа, што ты яшчэ не гатовы
нешта ведаць, то, калі хочаш жыць, ты не соўваеш свой нос, куды не трэба.

Зеленаваты цень нахіліў галаву.

--- Веру. Але ёсць яшчэ нешта, што ты павінен зразумець. Першыя навукоўцы, быўшы 
магламі, не мелі вашых традыцый. У пачатку я проста не маглі ўцяміць паняцце 
"пагрозлівыя веды". Яны шчыра думалі, што могуць свабодна публікаваць усё запар.
Калі яны знайшлі нешта сапраўды пагрозлівае, яны падзяліліся са сваімі палітыкамі...
--- не глядзі так, Драко, то не была проста дурасць, бо яны дастаткова разумныя,
каб зрабіць свае адкрыцці. Але яны былі простыя маглы, і гэта быў першы раз,
калі яны зразумелі нешта \emph{сапраўды} небяспечнае, і традыцыі сакрэтнасці яшчэ не існавала.
Да таго ж, ішла вялікая вайна, і вучоныя на адным баку хваляваліся, што калі 
яны будуць маўчаць, то вучоныя ворага раскажуць сваім палітыкам раней... 
--- ягоны голас важка заціх. --- Яны не разбурылі Зямлю. Але былі блізка.
І \emph{мы} не паўторым гэтую памылку.

--- Дакладна, --- сказаў Драко рашуча. --- \emph{Мы} не паўторым. Мы --- чараўнікі,
але засваенне навукі не робіць нас магламі.

--- Як скажаш, --- адказаў цень. --- Мы прыдумаем нашу ўласную Навуку, 
Магічную Навуку, і яна будзе разумнейшая адразу з пачатку, --- голас стаў 
жарстчэй. --- Я буду дзяліцца ведамі паралельна з тваім прагрэсам у дысцыпліне 
прыняцца праўды, і ўзровень тваіх ведаў будзе адпавядаць тваёй дысцыпліне. 
І ты не будзеш дзяліцца гэтымі ведамі ні з кім, хто не засвоіў навуковы падыход.
Ды прыймаеш гэтыя ўмовы?

--- Так, --- сказаў Драко. Што яму яшчэ заставалася сказаць, "не"?

--- Добра. І тое, што ты вывучыш сам, ты будзеш трымаць пры сабе, калі толькі 
не думаеш, што іншыя вучоныя ўжо пра гэта таксама ведаюць. Тое, чым мы дзелімся
адзін з адным, мы не дзеліся больш ні з кім, пакуль абодва не згадзімся, што 
гэтыя веды не нясуць пагрозу свету. І якімі ні былі б нашыя прыхільнасці,
мы \emph{ўсе} пакараем \emph{любога}, хто выдасць пагрозлівую магію або 
пагрозлівую тэхналогію, няважна, ці ідзе нейкая вялікая вайна.  
З гэтага дня і назаўжды, гэта будзе нашай традыцыяй і законам навукі сярод 
чараўнікоў. Усе згодныя?

--- Так, --- сказаў Драко. Шчыра кажучы, гэта пачынала гучаць даволі прывабна.
Пажыральнікі Смерці спрабавалі стаць жудасней за астатніх кандыдатаў на ўладу,
але ў выніку нічога не атрымалі. Магчыма, настаў час, калі сапраўдны кіроўца 
павінен знаходзіцца ў цені, карыстаючы сакрэтныя тактыкі.
--- І нашая група застаецца ў сакрэтнасці як мага даўжэй, і ўсе ўдзельнікі 
павінны пагадзіцца на нашыя правілы.

--- Безумоўна.

Пасля невякікай паўзы цень сказаў:

--- І нам патрэбныя мантыі палепей. З накідкамі, плашчамі, і ўсё такое...


--- Я \emph{як раз} падумаў тое ж самае, --- адказаў Драко. --- Не трэба мантыі,
проста накідкі з вялікімі капюшонамі, якія можна надзець паверх штодзённай 
вопраткі. Я загадаю адной сваёй сяброўцы зняць твае меркі...

--- Але не расказвай ён, навошта гэта...

--- Ну я ж не ідыёт!

--- І ніякіх масак. Прынамсі пакуль нас двое...

--- Дакладна! --- сказаў Драко. --- Але пазжэй мы можам прыдумаць нейкую 
адмысловую метку, якую будуць мець усе нашыя слугі на правай руцэ... 
Метка Навукі... напрыклад, як змей кусае месяц...

--- Метка навукі завецца "PhD"\footnote{{} Званне "кандыдат навук".}, і не будзе гэта занадта 
простым спосабам вылічыць нашых людзей?

--- У сэнсе?

--- Ну, уяві, нехта на вячэры скажа: "так, я цяпер усім задраць рукавы на правай
руцэ", і нашыя такія: "ой, выбачайце, падаецца, што у мяне тут нейкая дзіўная 
метка..."

--- \emph{Праехалі,} забудзь, --- сказаў Драко, якога раптам прашыб пот па ўсяму
целу. Яму трэба было \emph{хутка} адцягнуць увагу Гары... --- І як мы будзем звацца?
Пажыральнікі Навукі?

--- Н-н-не, --- працягнуў цень, --- н-неяк дзіўна гучыць...

Драко працёр лоб рукавом. І аб чым толькі \emph{думаў} Цёмны Лорд? А бацька казаў,
што той быў дзіка разумны!

--- Прыдумаў! --- раптам сказаў цень. --- Ты пакуль яшчэ не зразумееш, але,
вер мне, яно пасуе.

Драко ў гэты момант згадзіўся бы і на "Пажыральнікі Потэра", калі гэта дазволіла бы
змяніць тэму.

--- Ну?

І стоячы сярод пыльных парт у закінутым класе ў падзямеллі Хогвартс,
цьмяна-зялёны цень Гары Потэра драматычна ўзняў рукі і сказаў:

--- У гэты дзень пачаўся ўсход... \emph{Байесаўскай Змовы!}


\later

Ціхі цень стомлена клэпаў праз калідоры Хогвартс у кірунку Рэйвенкло.

Пасля сустрэчы з Драко, Гары адразу наведаў вячэру, і, хутка праглынуўшы пару 
кускоў, пайшоў у ложак.

Яшчэ нават не было сямі гадзін вечара, але для Гары ўжо даўно быў час спаць.
Учора ён зразумеў, што не зможа скарыстаць часаварот у суботу да заканчэння спаборніцтва
з Герміёнай, але, каб набыць крыху часу, ён усё яшчэ мог зрабіць гэта ў пятніцу.
Таму Гары прымусіў сябе не спаць да 21:00 у пятніцу, калі адчыняўся ахоўны 
кантэйнер, адматаў чатыры гадзіны назад да 17:00, і зваліўся ў сон. Ён прачнуўся 
ў 2 гадзіны ночы ў суботу, як і планаваў, і чытаў без перапынку ўсе дванаццаць 
гадзін... і нават гэтага не хапіла. І некалькі дзён Гары будзе вымушаны класціся спаць 
даволі рана, пакуль цыкл сну не нагоніць.

Партрэт на дзвярах задаў Гары нейкае дурное пытанне, годнае толькі для дзіцячага 
садка, Гары адказаў, нават не ўключыўшы свядомую частку мозга, падняўся ў спальню,
пераапрануўся ў піжаму, і абрушыўся на ложак.

І зразумеў, што пад падушкай ляжала нешта цвёрдае.

Гары, прастагнаўшы, сеў, павярнуўся, і падняў падушку.

Пад ёю ляжалі два залатых галеона, пад імі --- запіска, пад якой была кніга з 
надпісам 
\emph{"Аклюманцыя: патаемныя сведомасці"}. % Арыгінал 
% "Occlumency: The Hidden Arte."  
% Для "arte" мы тут на старабеларускі манер ужываем "сведомасць"

Гары ўзяў запіску і прачытаў:

\begin{writtenNote} 

    Божа, хто б падумаў, што можна так гучна і так хутка ўляпацца ў непрыемнасці? 
    У гэтым ты далёка абскакаў нават свайго бацьку.
 
    У цябе ёсць магутны вораг. Снэйп жэстачайшэ трэніруе кожнага слізэрына
    карыстаць лаяльнасць, страх, або ліслівасць. Не вер нікому з іх, іхняя 
    знешняя жудаснасць або прыязнасць --- толькі маска.
    
    Ніколі не сустракайся са Снэйпам позіркам, інакш ён як леджеламансер 
    зможа чытаць твой розум. Дасылаю табе кнігу, якая дапаможа табе навучыцца 
    абараняцца, хаця без добрага настаўніка ты можаш засвоіць толькі базавые рэчы.
    Прынамсі ты можаш навучыцца вызначаць умяшальніцтва.

    Каб ты змог вызваліць крыху часу для вывучэння аклюманцыі, я таксама дадаў 
    два галеона, што складаюць кошт спісу адказаў усіх хатніх заданняў і катрольных 
    работ па Гісторы Магіі (бо прафесар Бінз, пасля таго як памёр, ніколі іх не мяняў).
    Ты зможаш набыць копію ў сваіх новых сяброў блізнятаў Уізлі. Натуральна,
    не можа быць ніякай гаворкі аб тым, што цябе не павінны злавіць з гэтымі
    шпаргалкамі.

    Аб прафесары Квірэле я ведаю няшмат. Ён слізэрын і прафесар Абароны, і гэта 
    ўжо два пункта супраць яго. Добра абдумвай кожную яго параду, і не кажы яму 
    нічога, што жадаеш трымаць у сакрэце.

    Дамблдор толькі ўдае, што ён вар'ят. Ён неверагодна разумны, і калі ты 
    працягнеш хавацца ў шафах і пасля знікаць, ён адназначна здагадаецца, што ў цябе
    ёсць плачш нябачнасці, калі яшчэ не здагадаўся. Па магчымасці пазбягай 
    кантактаў з ім, і хавай свой плашч недзе ў добрай схованцы (\emph{не}
    ў махляскіне) кожны раз, калі пазбегнуць не атрымаецца, а ў яго 
    прысутнасці --- трымай слых на семярых.

    Калі ласка, будзь болей асцярожным у будучыні, Гары Потэр.

--- Санта Клаус

\end{writtenNote}

Гары на ўсе вочы разглядваў запіску.

Гэта сапраўды гучала як добрая парада. Канешне, Гары не збіраўся спісваць на 
Гісторыі Магіі, нават калі іх настаўнікам будзе мёртвая малпа. Але тое, што 
Северус --- леджеламансер... хто бы ні даслаў ліст Гары, ён ведаў шмат важных сакрэтаў,
і дэманстраваў гатоўнасць дзяліцца імі з Гары. І ён зноў папярэджваў Гары,
што Дамблдор можа хацець скрасці плашч. Было незразумела --- ці пагроза і
праўда была вялікая, ці проста аўтар забыўся, што ўжо рабіў гэта ў першым 
лісце. 

Падобна, што нехта у Хогвартс плёў нейкія вялікія інтрыгі. Магчыма, калі 
Гары мог бы параўнаць гісторыі Дамблдора і аўтара запісак, аб'яднаная выява
будзе дакладнай? Бо калі яны пагаджаліся ў пэўных рэчах, тады... 

...халера іх пабірай...

Гары ўвапхнуў усё ў свой махляскін, выкруціў сцішальнік на максімум, упаў 
на падушку, і памёр.


\later

Была раніца суботы. Гары хуткімі кускамі заглынаў блінцы ў Вялікай Зале, 
кожныя некалькі секунд кідаючы нерваваныя позіркі на гадзіннік.  

Было 8:02, і праз дакладна дзве гадзіны і адну хвіліну будзе \emph{роўна
адзін тыдзень}, як ён сустрэў сям'ю Уізлі і ўвайшоў на платформу  
Дзевяць і Тры Чвэрці.

І Гары падумаў... ён не ведаў, ці правільна было так думаць аб сусвеце,
бо ён ужо не быў у чым упэўнены, але падавалася праўдападобна, што...

\emph{Шэраг цікавых рэчаў, якія здарыліся з ім за гэты тыдзень, яшчэ не фіналізіраваны.}

Ён планаваў, скончыўшы есці, пайсці прама ў сваю спальню, схавацца ў склепе
свайго куфара, і не размаўляць ні з кім да 10:03.

І ў гэты моман Гары ўбачыў, што да яго набліжаюцца блізняты Уізлі. Адзін з іх 
нешта хаваў за спінай.

Яму варта збегчы.

Яму варта збегчы.

Што бы яны ні задумалі, гэта мог быць...

\emph{...Вялікі Фінал...}

Яму сапраўды варта спужацца і збегчы.

З пачуццём пакоры, што сусвет \emph{усё роўна} яго знойдзе, Гары ўжо больш 
павольна пачаў разразаць чарговы блін на кавалачкі. Ён адчуваў сябе цалкам 
спустошаным. Цяпер ён разумеў, ак адчуваюць сябе марафонцы. Як адчуваюць сябе 
людзі, намагаючыся пазбегнуць лёсу, яны проста падаюць на зямлю, і дазваляюць 
дземанам, ашчаціненым зубамі і шчупальцамі, уцягнць сабе на самае глыбокае 
дно самай глыбокай бездані...

Блізняты былі ўжо бліжэй.

І яшчэ бліжэй.

Гары з'еў яшчэ кусок блінца.

Ззяючы ўсмешкамі, блізняты Уізлі нарэшцэ сягнулі Гары.

--- Дароў, Фрэд, --- сказаў безвыразна Гары. Адзін з блізнят кіўнуў. --- 
Дароў, Джордж, --- другі кіўнуў таксам.

--- Выглядаеш стомленым, --- сказаў Джордж.

--- Трэба цябе падбадзёрыць, --- сказаў Фрэд. --- Глядзі, што у нас ёсць для
цябе!

І Джордж дастаў з-за спіны Фрэда...

Торт, на якім гарэлі дванаццаць свечак. 

Павісла пауда. Увесь стол Рэйвенкло глядзеў на іх здзіўлена.

--- Нешта не сыходзіцца, --- сказаў нехта, --- дзень народзінаў Гары Потэра --- 
трыццаць першага ліп...

--- \prophesy{Ён надыходзіць}, --- зычны тужлівы голас перарваў усе размовы як ледзяное 
лязо. --- \prophesy{Той, хто разарве на кавалачкі с...}

Дамблдор спрыгнуў са свайго трона, падбяжаў да настаўніцы, якая сказала гэтыя словы.
З выбліскам над яе галавой з'явіўся Фоукс, і праз імгненне ўсе трое зніклі ў 
полымі.

Вялікая Зала шакавана маўчала...

...і потым галовы павярнуліся ў бок Гары Потэра.


--- Я тут ні пры чым, --- сказаў Гары стомлена.

--- Гэта было \emph{прароцтва!} --- прасычэў нехта побач. --- І б'юся аб заклад, 
што яно было пра \emph{цябе!}

Гары ўздыхнуў.

Ён падняўся, і сказаў велькі гучна, каб перакрыць падымаючыюся гоман:

--- Гэта было не пра мяне! Відавочна, што я не \emph{надыходжу!} Я ўжо тут!

І сеў на свае месца.

Зноў узняўся гоман.

--- Тады пра каго яно было? --- сказаў яшчэ нехта.

І з тупым ледзяным пачуццем Гары зразумеў, \emph{хто} яшчэ быў не ў  Хогвартс.


Здагадка была дзікая, але ў Гары з'явілася прадчуванне, што хутка яны ўбачаць 
Цёмнага Лорда.

Гоман працягваўся.

--- І дарэчы, "разарве на кавалачкі" што?

--- Мне падаецца, Трэлоні пачала казаць нешта на "с", калі Майстра яе схапіў.

--- Кшталту... сонца? 

--- Калі нехта збіраецца разарваць на кавалачкі сонца, у нас \emph{сапраўды}
праблемы!

Гары гэта падавалася даволі малаверагодным, калі, канешне, нейкі жудасны чараўнік
не пазнаёміўся з ідэямі Дэвіда Крызвела аб стар-ліфтынгу\footnote{{} Ідэя аб 
тым, як дастаткова развітая цывілізацыя можа выдаліць частку сваёй зоркі, 
што, імверна, павінна адцягнуць час, калі зорка пераўтварыцца ў гіганта і 
спаліць усе планеты навокал.}. 

--- Дык што, --- сказаў Гары яшчэ стомленей, --- гэта адбываецца кожную нядзелю, ці не?

--- Не, --- сказаў студэнт, па выгляду сямікурснік, моцна нахмурыўшы бровы, --- 
першы раз такое чую.

Гары паціснуў плячыма.

--- Ну, не --- так не. Хто хоча торціка?

--- Але твае народзіны не сёння! --- сказаў той жа студэнт, што і ў першы раз.

Канешне, тут Фрэд і Джордж пачалі гучна смяяцца.

Нават Гары здолеў слаба ўсміхнуцца.

І атрымаўшы першы кусок, ён сказаў:

--- Гэта быў \emph{вельмі доўгі тыдзень.}

\later

Гары сядзеў у склепе свайго куфара, замкнуўшы дзверы знутры --- каб ніхто не змог 
увайсці --- накрыўшыся коўдрай з галавой, чакаючы, калі скончыцца гэты тыдзень.

10:01.

10:02.

10:03, але каб упэўніцца...

10:04, і першы тыдзень скончыўся.

Гары з палёгкай выдыхнуў, і асцярожна сцягнуў коўдру з галавы.

Праз некалькі секунд ён выбраўся з куфара ў яскравае святло спальні.

Яшчэ праз хвіліну ён быў у агульным пакоі Рэйвенкло. Некалькі галоў павярнулася да 
яго, але ніхто не паспрабаваў з ім загаварыць.

Гары знайшоў свабодны стол, і сеў у зручнае крэсла. Потым дастаў з махляскіна 
паперу і аловак.

Тата і мама ясна казалі, што хоць яны і разумелі яго энтузіязм у тым,
каб пакінуць дом і бацькоў, ён усё роўна абавязаны пісаць ім \emph{кожны тыдзень
без выключэння}, каб яны ведалі, што ён жывы, непашкоджаны, і не ў турме.

Гары глядзеў на пусты ліст паперы. \emph{Значыцца так...}

Пакінуўшы бацькоў на платформе нумар дзевяць, ён...

...пазнаёміўся з хлопцам, які гадаваўся ў сям'і дартаў вейдэраў, пасябраваў 
з трымя самымі сумна знакамінымі пранкерамі Хогвартс, сустрэў Герміёну, выклікаў 
Інцыдэнт з Размеркавальным Капелюшом... у панядзелак яму далі машыну часу, каб 
дапамагчы з яго праблемамі са сном, ён атрымаў легендарны плашч нябачнасці ад 
невядомага добразычліўцы, выратаваў сем хафлпафаў, прынізіўны пяцёрых жудасных 
старшакурснікаў, адзін з якіх пагражаў зламаць яму палец, зразумеў, што 
валодае таямнічым цёмным бокам, навучыўся каставаць \emph{Frigideiro} на 
ўроках Чаравання, пачаў спаборніцтва з Герміёнай... аўторак пачаўся з астраноміі,
якую вяла прыемная прафесар Аўрора Сіністра, і Гісторыі Магіі, якую выкладаў прывід, 
якога варта было падвергнуць экзарцызму і замяніць на магнітафон... серада, ён 
атрымаў ганаровае званне самага пагрозлівага студэнта ў класе... чацьвер, давай
нават і не думаць пра чацьвер... пятніца, Інцыдэнт на зеллеварэнні, пасля 
якога ён шантажаваў Майстра, пасля чаго прафесар Абароны арганізаваў над ім гвалт,
пасля чаго высветлілася, што прафесар Абароны --- самы цудоўны з усіх людзей на 
Зямлі... у суботу ён прайграў спор, быў на сваім першым спатканні, і пачаў 
перацягваць Драко на свой бок... і вось, гэтай раніцай недасказанае прароцтва 
прафесара Трэлоні магло сведчыць (або не сведчыць) аб тым, што няўміручы Цёмны Лорд 
збіраецца ўварвацца ў Хогвартс.

Гары мыслена арганізаваў факты, і пачаў пісаць.


\begin{writtenNote} \letterAddress{Дарагія мама і тата:}

    У Хогвартс вельмі весела. На ўроках чаравання я навучыўся парушаць другі
    закон тэрмадынамікі. Яшчэ сустрэў дзяўчыну па імі Герміёна Грэнджэр, якая 
    чытае хутчэй за мяне.

    Лепей скончыць на гэтым.

\letterClosing[З любоўю ваш сын]{Гары Джэймс Потэр-Эванс-Верэс.}
\end{writtenNote}

