\chapter{Вызначэнне гіпотэзы}

\begin{chapterOpeningQuote}
Ты пачынаеш бачыць патэрн, пачынаеш чуць рытм асяроддзя.
\end{chapterOpeningQuote}

\lettrine{Ч}{ацьвер}.\newline
Калі хочаце дакладна, то 7:24 раніцы ў чацьвер.

Гары сядзеў ў сваім ложку, падручнік амаль вываліўся з ягоных разняволеных рук.

Толькі што яму да галавы прыйшла ідэя \emph{сапраўды цудоўнага} эксперыменту.

Гэта патрабавала адкласці сняданак яшчэ на гадзіну, але на такі выпадак 
у яго былі мюслі. Не, гэтая ідэя павінна быць праверана неадкладна, без зацяжак,
зараз.

Гары адклаў кнігу, выпрыгнуў з ложка, падбяжаў да свайго куфара, 
адчыніў люк, збег па лесвіцы, і пачаў перасоўваць каробкі. (Яму варта б 
распакаваць і размеркаваць кнігі па шафах, але ў яго было 
няскончанае спаборніцтва з Герміёнай, у якім ён адставаў, таму на гэта зараз 
не было часу.)

Знайшоўшы кнігу, якую ён шукаў, ён пабег назад.

Астатнія хлопцы збіраліся на сняданак у Галоўную Залу.

--- Прабач, можаш мне дапамагчы? --- сказаў Гары. Ён прагартаў старонкі,
знайшоў месца, дзе былі надрукаваныя першыя дзесяць тысяч простых лікаў,
і сунуў кнігу ў рукі Энтані Голдштэйна. --- Выбяры два трохзначных ліка з гэтага спісу,
і не кажы мне, якія. Перамнож іх, і скажы мне вынік. О, і калі ласка, можаш 
праверыць здабытак два разы? Будзь добры, пераканайся, што здабытак правільны, бо 
я не ўпэўнены, што здарыцца са мной і сусветам, калі ты дзесьці памылішся з падлікам.

Той факт, што Энтані не здівіўся, або не спытаў, што такога можа здарыцца з сусветам,
шмат казала пра тое, якім было жыццё ў гэтай спальне на працягу  некалькіх апошніх дзён.

Энтані без лішніх слоў узяў кнігу, дастаў пергамент і пяро. Гары адвярнуўся і 
заплюшчыў вочы, каб дакладна нічога не ўбачыць, прытанцоўваючы на месцы ад нецярпення.
У адной руцэ у яго быў блакнот, у другой --- механічны аловак, і ён быў гатовы пісаць.

--- Гатова, --- сказаў Энтані, --- сто восемдзесят адна тысяча чатырыста дваццаць дзевяць.

Гары запісаў $181,429$. Ён прачытаў запісанае, і Энтані пацвердзіў.

Потым Гары збег у склеп свайго куфара, паглядзеў на гадзіннік (які паказваў 4:28,
што значыла 7:28) і заплюшчыў вочы. 

Прыкладна праз трыццаць секунд Гары пачуў крокі ў склепе, потым на лесвіцы, 
а потым --- як зачыніўся люк. (Гары не хваляваўся, што задахнецца. Заклён аўтаматычнага
пасвяжэння паветра быў звычайнай часткай каштоўных куфараў. Ну ці не была магія 
цудоўнай? Да таго ж, не трэба думаць пра аплату электрычнасці.)

І калі Гары расплюшчыў вочы, ён убачыў тое, што і чакаў: складзеную паперку на 
падлозе, падарунак ад сябе-будучага.

Давайце дагаварымся называць гэтую паперку "Папера №2".

Гары, не рухаючыся з месца, вырваў старонку са свайго блакнота. 

Гэта --- "Папера №1". Яна была, вядома, тым жа самым кавалкам паперы, што ляжаў 
на падлозе. Калі прыгледзецца, было бачна, што адарваны край быў таго ж узору.

Гары мыслена паўтарыў алгарытм, якому ён будзе следваць.

Калі ён разгарне Паперу №2, і яна будзе пустая, ён напіша “$101\times101$” на Паперы №1,
потым пазаймаецца ўрокамі адну гадзіну, прыгне назад у часе, бросіць на падлогу 
паперку (якая ў гэты момант стане Паперай №2), і потым пакіне свой куфар, каб 
далучыцца да аднакласнікаў на сняданку. 

Калі ён адчыніць Паперу №2, і на ёй будуць напісаны два ліка, Гары перамножыць 
іх. Калі іх здабытак будзе $181,429$, Гары напіша гэтыя два ліка на Паперы №1 і 
адправіць яе ў мінулае.

Інакш, Гары дадасць 2 да правага ліку, калі той не больш за 997, у гэтым выпадку
ён павялічыць левы лік на 2, і скіне правы лік на 101, пасля чаго ён напіша
гэтая два новых ліка на Паперы №1. 

І калі на Паперы №2 будзе напісана $997\times997$, Гары пакіне Паперу №1 пустой.

А гэта значыла, што адзінай \emph{стабільнай} часовай пятлёй будзе тая, дзе 
Папера №2 будзе мець два простых множніка ліка $181,429$.

Калі спрацуе, Гары зможа карыстаць гэты спосаб для адказу на любую задачу, 
які было лёгка праверыць, але цяжка вырашыць.
Ён не проста дакажа, што адказ на адно з асноўных пытанняў 
інфарматыкі $\mbox{P}=\mbox{NP}?$ быў пазітыўным, калі ты меў часаварот.
Гэты прыём адчыняў шмат іншых магчымасцяў. Гары мог карыстаць яго для падбору 
камбінацый для кодавых замкоў, або паролей любой даўжыні. 
Магчыма, нават знайсці ўваход у Таемную Залу Слізэрына, калі ён зможа прыдумаць
сістэматычны спосаб пералічыць ўсе лакацыі замка Хогвартс.
Гэта будзе чыт-код тысячагоддзя, нават па мерках чытынгу самога Гары.

Гары крыху трасучыміся рукамі падняў Паперу №2 і разгарнуў яе.

На Паперы №2 крыху трасучымся почыркам было напісана:

\medskip

{\hpFontFasthand{}Не будзі ліха, прыдурак!}

\medskip

Гары сваёй трасучайся рукой напісаў  {\hpFontFasthand{}Не будзі ліха, прыдурак!} на
Паперы №1, акуратна яе склаў, і вырашыў больш не праводзіць сапраўды цудоўныя
эксперыменты з Часам, прынамсі пакуль яму не споўніцца пятнаццаць.

Па меркаванню Гары, гэта быў самы страшны эксперыментальны вынік за 
ўсю гісторыю навукі.

Наступую гадзіну Гары неяк складана было сканцэнтравацца на чытанні сваіх падручнікаў.

Так пачаўся Гарын чацьвер.


\later

Чацьвер.

Калі хочаце дакладна, то 15:32 ў чацьвер, пасля абеду.

Гары і ўсе астатнія хлопцы-першакурснікі разам з мадам Хуч стаялі на зялёным
палі каля кіпы мёцел з запасаў Хогвартс. Урокі лятання для дзяўчын праводзіліся
паасобку. Па нейкай незразумелай прычыне яны не хацелі вучыцца лятаць у прысутнасці
хлопцаў.

Гары ўвесь дзень крыху каўбасіла. Ён проста не мог перастаць гадаць, якім чынам
\emph{менавіта тая} стабільная часовая пятля была абраная з даволі вялікай --- у рэтраспектыве ---
прасторы магчымасцей. 

Ну і да таго ж: \emph{мётлы?} \emph{Сур'ёзна?} Ён быў павінен лятаць на фактычна 
адрэзку прамой? Ці не было гэта, шчыра кажучы, самая нестабільнай формай, 
якую толькі можна знайсці? Хто мог абраць менавіта такі дызайн для лятаючай прылады
з неверагоднай колькасьці варыянтаў. Спачатку Гары спадзяваўся, што гэта 
проста фігура мовы, але не, тое былі звычайныя мётлы для падмятання лісця. Камусьці 
проста падабалася ідэя з мятлой, і ён проста адкінуў усі іншыя? Відаць, так. 
Ну проста не магло здарыцца, каб дызайн для хатняй прылады і лятальнага апарата 
супадалі бы хоць на некалькі адсоткаў, калі распрацоўваць іх з нуля.

Дзень быў ясны, блакітнае неба было высока над галавой, а яскравае сонца 
так і лезла сляпіцай у вочы, каб перашкаджаць табе ў палёце. Поле выглядала 
сухім, пахла амаль прасмажаным, і адчувалася вельці цвёрдым пад Гарынымі 
ступакамі.

Гары працягваў нагадваць сабе, што  навучальная праграма была разлічана так, што нават самы
імкнуўшыся да нуля адзіннацацтлетка можа яе засвоіць, і што яно не павінна 
было быць насколькі складаным.

--- Працягніце правую руку над мятлой, або левую, калі вы ляўша, --- сказала 
мадам Хуч, --- і скажыце \shout{ўверх!}


--- \shout{Уверх!} --- крыкнулі ўсе.

Мятла ахвотна прыгнула Гары ў руку.

Што ўпершыню паставіла яго на першае месца ў класе. Імаверна, сказаць \shout{Уверх!} 
было значна складаней, чым падавалася, бо большасць мётлаў ці проста ляжалі, ці 
намагаліся адсунуцца ад сваіх будучых ездакоў.

(Канешне, Гары быў гатовы біцца аб заклад, што ў Герміёны атрымалася не горш 
за яго. Калі ён мог вывучыць з першай спробы нешта, што здзівіць Герміёну, 
і гэтае нешта апынецца лятаннем на мятле... ён проста здзейсніць самагубства.)

Патрабіваўся нейкі час, каб усе здолелі узяць мётлы ў рукі. Мадам Хуч паказала,
як садзіцца на мятлу, і потым прайшлася па радах, паправляючы і каментуючы.
Відавочна, нават тыя, каму дазвалялася карыстацца мятлой дома, не навучыліся 
рабіць гэта правільна.

Агледзеўшы ўсіх, яна вярнулася наперад і кіўнула.

--- Так, калі я свісну, усім моцна падпрыгнуць.

Гары з цяжкасцю зглынуў, спрабуючы заглушыць пачуццё млоснаснці.

--- Мётлы трымаць роўна. Падняцца на два-тры футы. Потым крыху нахіліцца наперад, 
і апусціцца на зямлю. Такім чынам, тры, два...

Адна з мёцел стрэльнула ў паветра, суправаджаемая крыкамі хлопца, які на ёй сядзеў.
Крыкі былі ад жаху, а не задавальнення. З дзікай хуткасцю ён круціўся, падымаючыся,
і Гары толькі на імгненне ўбачыў ягоны збялелы твар...

Быццам у замедленай здымке Гары злазіў са сваёй мятлы, адначасова дастаючы 
палачку, хаця ён не зусім разумеў, што плануе рабіць. У іх было ўсяго два урока 
Чаравання, і на мінулым яны праходзілі Заклён Левітацыі, дзе ў Гары атрымлівалася
каставаць яго паспяхова толькі адзін раз з трох спроб, і дакладна, ён не быў
здольны левітавать цэлага чалавека... 

\emph{Калі ва мне ёсць таямнічая сіла, калі ласка, хай яна пакажа сябе \emph{зараз}!}

--- Вярнісь зараз жа! --- крыкнула мадам Хуч. (З боку \emph{настаўніка лятання},
гэта была самая недарэчная парада, якую толькі можна было ўявіць, каб дапамагчы
вучню справіцца са звар'яцелай мятлой, --- і цалкам аўтаномная частка мозгу Гары 
дадала мадам Хуч да ягонага ліста дурняў)

І потым хлоца скінула з мятлы.

Падавалася, што ён падае вельмі павольна, прынамсі спачатку. 

--- \emph{Wingardium Leviosa!} --- крыкнуў Гары.

Зачараванне не спрацавала. Ён адчуў гэта, нават не скончыўшы інкантацыю.

Потым быў громкі "чмяк" і хрыбусценне. Хлопец ляжаў на траве тварам уніз, быццам
кіпа адзення.

Гары паклаў палачку ў карман, і з усіх ног пабяжаў да яго, прыбяжаўшы адначасава 
з мадам Хуч, ён адразу засунуў руку ў махляскін, намагаючыся ўспомніць,
--- о божа як жа яно звалася ды няважна проста скажы "аптэчка", --- і яна прыгнула 
яму ў руку, і...

--- Пералом прыдалоння, --- сказала мадам Хуч. --- Спакойна, хлопча, гэта 
проста пералом!

Яго розум быццам спатыкнуўся, і Гары выйшаў з Ражыма Панікі.

"Emergency Healing Pack Plus" ляжаў перад ім расчынены, у руцэ ў Гары быў 
шпрыц вадкага агню, які павінен быў падтрымліваць аксігенацыю мозга параненага, 
калі той зламаў бы шыю.

--- А-а-а... --- сказаў Гары не вельмі роўным голасам. Яго сэрца бухала так,
што ён амаль не адчуваў, як хапае ротам паветра. --- Пералом... добра...
тады, Гіпсавы Ручнік?

--- Гэта толькі для надзвычайных выпадкаў, --- адрэзала мадам Хуч. --- Прыбяры
свае лекі, ён нармальна, --- яна нахілілася над Нэвілам, падаючы яму руку. --- 
Давай, хлопча, уставай, усё добра, давай-ка...

--- Вы жа не плануеца зноў пасадзіць яго на мятлу? --- з жахам спытаў Гары.

Мадам Хуч кінула ў адказ абураны позірк.

--- Ну вядома ж, не! 

Яна пацягнула за непашкоджаную руку кіпу адзнення, і паставіла яе роўна, --- тут 
Гары з нейкім шокам зразумеў, што гэта \emph{зноў} быў Нэвіл, --- да што ж 
з табой такое? --- і яна павярнулася да ўсіх астатніх вучняў.

--- Нікому нават не рухацца, пакуль я суправаджаю гэтага хлопца ў бальнічнае крыло!
Мётлы нікому не трогаць, інакш выляціце з Хогвартс хутчэй, чым паспееце сказаць 
"квіддзіч". Хадзем, даражэнькі.

І яна ўвяла Нэвіла, які трымаў прыдалонне здаровай рукой, і намагаўся кантраляваць 
свае  ўсхліпы.

Калі яны адышлі дастаткова, адзін са слізерынаў зарагатаў.

За ім --- астатнія.

Гары абярнуўся паглядзець на іх. Добры час, каб запомніць некаторыя твары.

А ўбачыў, як да яго набліжаецца Драко, суправаджаемы містэрам Крэбам і 
містэрам Гойлам. Містэр Крэб не ўсміхаўся. Чаго рашуча нельга было 
сказаць пра містэра Гойла. Сам Драко дэманстраваў вельмі стрыманы твар, які 
часам паторгваўся. Відавочна, падумаў Гары, яму таксама было весела, але ён не бачыў 
палітычнай выгады ў тым, каб пасмяяцца зараз, і яму лепей зрабіць гэта пазней, у 
падзямеллі Слізэрына. 

--- Ну, Потэр... --- сказаў Драко дастаткова ціха, каб астатнія не чулі. Твар яго 
быў усё яшчэ стрыманы, і ўсё яшчэ паторгваўся. --- Проста хацеў сказаць, што 
калі ты плануеш скарыстацца надзвычайнай сітуацыяй для дэманстрацыі свайго 
лідэрства, варта рабіць выгляд, што ў цябе ўсё пад кантролем, замест, скажам,
кідання ў поўную паніку, --- містэр Гойл хіхікнуў, і Драко кінуў у яго гнеўны позірк. 
--- Але, магчыма, некалькі балаў ты заработал. Табе патрэбная дапамога, каб 
запакаваць свае багацце?

Гары павярнуўся да аптэчкі, што яшчэ і дазволіла яму схаваць твар ад Драко. 

--- Усё нармальна, --- сказаў ён, сабраў усе рэчы вакол, паклаў на месца
шпрыц, зашчоўкнуў замкі, і падняўся.

Эрні Макмілан падышоў да яго, калі Гары амаль скончыў скормлівать аптэчку 
махляскіну.

--- Ад імя Хафлпафа, дзякуй, Гары Потэр, --- сказаў ён фармальна. 
--- І думка, і спроба былі добрыя.

--- Сапраўды, добрая думка, --- працягнуў Драко. --- Чаму ніхто з хафлпафаў не 
дастаў палачкі? Мабыць, калі б вы ўсе не аслупянелі, і далучыліся да Потэра, 
дак і паймалі б яго? Я быў пад уражаннем, што хафлпафы павінны трымацца гуртом?

Эрні выглядаў, як раздзіраемы злосцю і сорамам. 

--- Мы не падумалі пра тое...

--- А, --- сказаў Драко, --- не \emph{падумалі}. Вось чаму лепей мець аднаго 
сябра-рэйвенкло, чым цэлы натоўп хафлпафаў.

Вось гад, падумаў Гары, і як зараз з гэтага выбірацца.

--- Ты не дапамагаеш, --- сказаў ён ціха. Спадзяючыся, што Драко здолее перакласці 
гэта як \emph{ты перашкаджаеш маім планам, таму, калі ласка, забі зяпу.}

--- Хэй, а гэта што такое? --- сказаў містэр Гойл. Ён нахіліўся і падняў з 
травы нешта памерам са сліву, шкяны шарык, у якім завіхалася белая дымка.

Эрні міргнуў.

--- Напамінальнік Нэвіла!

--- Што за напамінальнік? --- спытаў Гары.

--- Ён чырванее, калі ты нешта забыў, --- сказаў Эрні. --- 
Але ён не кажа, што менавіта ты забыў. Калі ласка, аддай яго мне, я вярну яго
пазней Нэвілу, --- Эрні працягнуў руку.

Раптоўная усмешка прабяжала па твары містэра Гойла. Ён павярнуўся і пабяжаў прэч.

Пэўную секунду Эрні ад нечаканасці здранцавеў, потым крыкнуў "Гэй!", і 
пабяжаў за ім.

Містэр Гойл адным плаўным рухам схапіў мятлу, ускочыў на яе вярхом, і ўзняўся 
ў паветра.

У Гары сківіца адвалілася. Ці не казала мадам Хуч нешта пра адлічэнне?

--- \emph{Што за ідыёт!} --- прашыпеў Драко. Ён набраў паветра, каб крыкнуць...

--- \emph{Гэй!} --- праравеў Эрні, --- Гэта Нэвіла! \emph{Аддай!}

Слізэрыны пачалі ўлюлюкаць і вітаць містэра Гойла.

Рот Драко рэзка зачыніўся. Гары паспеў заўважыць на яго твары выраз нерашучасці.

--- Драко, --- сказаў Гары ціха, --- калі ты не загадаеш гэтаму ідыёту прызямліцца,
то калі вернецца настаўніца...

--- \emph{А ты адбяры, хафл-пафл!} --- пракрычаў містэр Гойл пад гучныя ўхваляванні 
з боку слізерынаў.

--- \emph{Не магу!} --- прасычэў Драко. --- Усе падумаюць, што я \emph{слабак}!

--- А калі містэра Гойла адлічуць, --- прасычэў Гары ў адказ, --- твой бацька 
падумае, што ты \emph{дэбіл}!

У Драко ажно твар перакасіла.

У гэты момант...

--- Ну ты, \emph{слізня-дрын}, --- крыкнуў Эрні, --- падобна, ніхто не казаў табе,
што хафлпафы трымаюцца гуртом? \emph{Хлопцы, палачкі!}

І раптам шмат палачак былі накіраваны на містэра Гойла.

І праз дзве секунды...

\emph{Слізэрын, дастаць палачкі!} --- сказалі пяць галасоў з групы слізэрынаў.

І шмат палачак былі накіраваны ў бок халфпафаў.

Яшчэ праз секунду...

--- \emph{Грыфіндор!}

--- \emph{Зрабі нешта, Потэр!} --- прашаптаў Драко. --- \emph{Я не магу выйсці і 
спыніць гэта, прыйдзецца табе давай думай я буду табе абавязаны ты ж наўродзе
павінен быць супер-разумны?}

Прыкладна праз пяць з паловай секунд, зразумеў Гары, нехта скастуе Шчучынскі
Штурханец, і калі пыл уляжацца, і настаўнікі скончаць адлічваць злачынцаў, 
адзінымі хлопцамі на першым курсе застануцца рэйвенкло. 

--- \shout{Рэйвенкло! Да бою!} --- завапіў Майкл Корнэр, які, верагодна, адчуў 
заўзятасць далучыцца да катастрофы.

--- \scream{Грэгары Гойл!} --- крыкнуў Гары. --- \shout{Я выклікаю цябе на двубой
дзеля валодання напамінальнікам Нэвіла!}

Раптам усе спыніліся.

--- Што, праўда? --- працягнуў сваім самым нізкім голасам Драко. --- Гучыць цікава.
Што за двубой, Потэр?

Э-э-эммм...

"Двубой" быў максімумам таго, што здолела выдаць Гарына фантазія. А што канатрэтна? 
Ён не мог сказаць "шахматы", бо калі Драко такое прыме, самі слізэрыны будуць 
глядзець на Драко коса, і не мог сказаць "армрэслінг", бо містэр Гойл яму руку адарве...

--- Што на конт такога? --- сказаў Гары гучна. --- Грэгары Гойл і я павінны стаць
адзін насупраць аднаго, і больш нікому нельга да нас набліжацца. Мы не карыстаем свае 
палачкі або любыя іншыя прылады. Мы не сыходзім з нашых месцаў. Калі я змагу дакрануцца
да напамінальніка Нэвіла, тады Грэгары Гойл адмаўляецца ад любых дамаганняў на 
напамінальнік, які ён трымае ў руцэ, і аддае яго мне.

Паследвала яшчэ адна пауза, пакуль людзі абменьваліся разгубленымі позіркамі.

--- Ха, Потэр, --- сказаў таксама гучна Драко. --- Я хачу паглядзець, я ты 
здолееш \emph{такое}! Містэр Гойл прыймае выклік!

--- Тады, пачынаем! --- сказаў Гары.

--- Потэр,  \emph{што ты задумаў?} --- прашаптаў Драко, неймаверным чынам
не рухаючы вуснамі.

Гары не ведаў, як адказаць, не рухаючы вуснамі.

Людзі паволі хавалі свае палачкі, і містэр Гойл зграбна прызямліўся, 
выглядаючы даволі разгублена. Некаторы хафлпафы ўзрушылі да яго, але 
Гары кінуў ім адчайна ўмольны позірк, і яны адступілі.

Гары падышоў да містэра Гойла, і спыніўся ў некалькіх кроках ад яго, каб яны не 
маглі дацягнуцца адзін да аднаго.

Павольна, дэманстратыўна Гары схаваў сваю палачку ў кішэнь.

Астатнія адышлі на некалькі крокаў.

Гары зглынуў. Ён толькі вельмі грубымі мазкамі ўяўляў сабе, што ён \emph{хоча}
зрабіць, але яно павінна было адбыцца такім чынам, каб ніхто не здагадаўся, 
\emph{што} і \emph{хто} гэта зрабіў...

--- Ну, добра, --- сказаў уголас Гары. --- А зараз... --- ён глыбока ўдыхнуў і
падняў правую руку, адзінец упіраўся ў сярэднік. Нехта шумна ўздыхнуў ---
звесткі пра пірагі разышліся хутка. --- Я заклікаю на дапамогу вар'яцтва 
замка Хогвартс! \emph{Happy happy boom boom swamp swamp swamp!} --- і ён цокнуў пальцамі.

Шмат хто ўздрыгануўся...

...і нічога не адбылося.

Гары чакаў. Цішыня зацягнулася, гатовая выбухнуць у любы момант...

--- М-м-м... --- сказаў нехта, --- І гэта ўсе?

Гары зірнуў на хлопца, які гэта сказаў. 

--- Паглядзі перад сабой, бачыш пусты кусок зямлі, дзе не расце трава?

--- М-м, бачу, --- адказаў хлопец-грыфіндор (Дзін нехта?).

--- Капай.

Тут шмат хто паглядзеў на Гары дзіўна.

--- Эм, навошта? --- сказаў Дзін-Нехта.

--- Проста капай, --- сказаў Тэры Бут стомлена. --- Павер мне, ніяма сэнсу пытаць.

Дзін-Нехта апусціўся на калені, і пачаў адкідваць рукамі зямлю.

Праз хвіліну-другую, ён падняўся. 

--- Тут нічога няма.

Хоба. Гары ўжо пачаў быў планаваць, як прыгне ў мінулае, і закапае на гэтым 
месцы мапу скарбаў, якая прывядзе іх да іншай мапы, а ўжо тая --- да 
напамінальніка Нэвіла...

І потым Гары зразумеў, што існаваў прасцейшы шлях, які дарэчы 
не рызыкаваў сакрэтам часаварота.

--- Дзякуй, Дзін! --- сказаў Гары гучна. --- Эрні, будзь ласка, агледзь месца, 
дзе упаў Нэвіл, і скажы, ці не бачыш ты там напамінальніка?

Людзі выглядалі ўсё больш збянтэжанымі.

--- Проста зрабі гэта, --- сказаў Тэры Бут. --- Ён не спыніцца, пакуль нешта не 
спрацуе, і самае жудаснае --- тое, што...

--- \emph{Мерлін!} --- ахнуў Эрні. У руцэ ён трымаў напамінальнік Нэвіла. --- Ён быў 
тут! Роўна дзе упаў Нэвіл!

--- \emph{Што?} --- крыкнуў містэр Гойл. Ён паглядзеў уніз і ўбачыў...

...што ён усё яшчэ трымае напамінальнік Нэвіла ў сваёй руцэ.

Павіcла даволі доўгая пауза. 

--- Э-э, гэта ж не немагчыма, так? -- сказаў Дзін-Нехта.

--- Гэта дзірка ў сцэнары, --- сказаў Гары. --- Я проста на імгненне адцягнуў 
увагу Сусвету, і ён забыў, што Гойл ужо падняў напамінальнік.

--- Што? Не! Я пра тое, што гэта \emph{абсалютна} немагчыма...

--- Прабачце, ці не мы гэта збіраліся толькі што лятаць на мётлах?
Так што лепей маўчы. У любым выпадку, калі я дакрануся да яго, я перамог, і Грэгары Гойл 
павінен адмовіццца ад любых дамаганняў на напамінальнік, які ён трымае ў руцэ, 
і павінен аддаць яго мне. Умовы былі такія, памятаеце? --- Гары працягнуў руку і 
памахаў Эрні. --- Ну і ніхто не павінен набліжацца, так што проста кінь мне яго,
акей?

--- Стаяць! --- пачулася раптоўнае ад слізэрына --- Блэйз Забіні, Гары дакладна не 
забудзе такое імя. --- Скуль мы ведаем, што гэта менавіта напамінальнік 
Лонгботама? Ты мог падкінуць туды нейкі \emph{іншы} напамінальнік...  

--- Слізэрынства моц ў гэтым малым адчуваю я, --- сказаў Гары, усміхаючыся. ---
Але даю слова, што Эрні трымае сапраўдны напамінальнік Нэвіла. На конт таго,
што трымае Грэгары Гойл, --- ніякіх каментароў.

Забіні павярнуўся да Драко.

--- \emph{Малфой!} Ты што, дазволіш яму гэтае махлярства?..

--- Гэй, забі зябу, --- прагрымкаў містэр Крэб, які стаяў ззаду Драко. --- 
Ты хто такі, каб указваць містэру Малфою?

\emph{Добры} міньён.

--- У мяне была змова з Драко, нашчадкам Вялікага і Старажытнейшага Роду Малфоеў, ---
сказаў Гары. --- Не з табою, Забіні. Я зрабіў тое, што ён вельмі хацеў паглядзець,
і судзіць, хто перамог, мы пакінем містэру Малфою, --- Гары нахіліў галаву
ў бок Драко і крыху ўзняў бровы. Драко такім чынам мог зберагчы свае пазіцыі.

Пауза.

--- Ты даеш слова, што тое --- сапраўды напамінальнік Нэвіла? --- спытаў Драко.

--- Так, --- сказаў Гары, --- сапраўды гэта яго арыгінальны напамінальнік, які
павінен вярнуцца да Нэвіла. А той, што трымае Грэгары Гойл адыходзіць да мяне.

Драко рашуча кіўнуў. 

--- Тады я веру слову Вялікага Роду Потэраў, няважна наколькі бязглузда гэта 
выглядае. А Вялікі і Старажытнейшы Род Малфоеў таксама ўмее трымаць слова.
Містэр Гойл, аддайце гэта містэру Потэру...

--- Хэй, --- сказаў Забіні, --- ён яшчэ не выйграў, бо ён павінны дакрануцца...

--- Лаві, Гары! --- крыкнуў Эрні, і шпурнуў напамінальнік.

Гары лёгка злавіў напамінальнік --- у яго заўсёды былі добрыя рэфлексы ў такіх 
гульнях. 

--- Ну, --- сказаў ён, --- я перамог...

Але ягоны голас неяк павольна сцішыўся, як і ўсе астатнія галасы навокал.

Напамінальнік у яго руцэ гарэў яскравым чырвоным святлом, быццам мініятурнае сонейка,
адкідваючы цені хлопцаў на траве, нягледзячы на светлы дзень.


\later

Чацьвер.

Калі хочаце дакладна, то 17:09 ў чацьвер, у кабінеце прафесара МакГонагал,
пасля заняткаў на мётлалятанні. (Плюс дадатковая гадзіна для Гары, усунутая ў гэты 
прамежак.)

Прафесар МакГонагал сядзела на сваім стуле, Гары --- у крэсле перад яе сталом.

--- Прафесар, --- сказаў Гары напружана, --- слізэрыны нацэлілі свае палачкі на 
хафлпафаў, грыфіндоры --- на слізэрынаў, і калі нейкі \emph{ідыёт} крыкнуў 
"Рэйвенкло, да бою!", у мяне заставалася кшталту пяць секунд, каб утрымаць 
сітуацыю ад жудаснага выбуху! Гэта было адзінае, што я паспеў прыдумаць!

Выраз МакГонагал быў нахмураны і ўззлаваны.

--- \emph{Не варта карыстаць часаварот такім чынам, містэр Потэр!} Або канцэпыця 
сакрэтнасці --- нешта па-за межамі вашага разумення?

---  Яны не ведаюць, \emph{як} я гэта зрабіў! Усе ўпэўнены, што я магу рабіць 
дзіўныя рэчы, цокаючы пальцамі! Я ўжо рабіў нават такое, што і з часаваротам не
магчыма, і зраблю \emph{яшчэ} такога, што гэты выпадак ніхто і не успомніць.
У мяне не было ніякага выбару, прафесар! 

--- Выбар быў! --- адрэзала МакГонагал. --- Усё, што вам было трэба --- вярнуць 
гэтага \emph{ананімнага слізэрына} на зямлю, і каб людзі схавалі палачкі! 
Вы маглі прапанаваць партыю ў выбуховы цмок, але не, абавязкова трэба 
выпендрывацца на шырокую нагу, і безразважна рызыкаваць часаваротам!

--- Я паспеў прыдумаць толькі гэта, і дарэчы, я нават не ведаю, што такое гэты 
ваш выбуховы цмок, і я думаў пра шахматы, але яны бы не згадзіліся, і калі бы я
абраў арм-рэслінг, то адразу бы прайграў!

--- \emph{Вам проста трэба было абраць арм-рэслінг!}

Гары міргнуў.

--- Але... але я бы \emph{прайграў!} --- і тут ён зразумеў.

МакГонагал выглядала \emph{вельмі} фрустрыраванай.

--- Прабачце, прафесар, --- сказаў Гары ціха. --- Я праўда пра тое не падумаў,
і вы правы, мне трэба было прайграць... і гэта быў бы цудоўны ход...
але мне гэта проста не прыйшло да галавы, і ведаеце...

Гары заціх, так і не прыдумаўшы, як скончыць фразу. Раптоўна перад ім з'явіўся
цэлы шэраг варыянтаў. Ён мог спытаць меркаванне Драко, ён мог параіцца з 
хлопцамі... яго ідэя скарыстаць часаварот спраўды была безразважным рызыкам. 
Чаму з усёй агромнай прасторы варыянтаў ён выбраў менавіта гэты?

...таму што гэта быў спосаб \emph{перамагчы}. Завалодаць нікчэмнай цацкай, якую
настаўнікі ў любым выпадку вярнулі бы Нэвілу.

Прага перамогі. Вось што яму перашкодзіла.

--- Мне вельмі жаль, --- сказаў Гары, --- за мой гонар і за маю дурасць.

Прафесар МакГонагал пацёрла лоб. Яе гнеў крыху супакоіўся, але голас быў 
вельмі жорсткі:

--- Яшчэ адзін так выбрык, містэр Потэр, і вы будзеце пазбаўлены гэтай прывілеі.
Ці зразумела вам гэта?

--- Так, --- сказаў Гары. --- Вельмі зразумела, і вельмі жаль.

--- Тады, містэр Потэр, дазваляю вам пакінуць часаварот. Пакуль што. 
І ўлічваючы памер бяды, якую вы, праўду кажучы, перадухілілі, 
я не буду здымаць балы з Рэйвенкло.

\emph{Угу, улічваючы, што вы не зможаце растлумачыць, за што знялі.} Але Гары 
быў не настолькі дурны, каб сказаць гэта ўголас.

--- Але важней, чаму напамінальнік адразу пачырванеў у маіх руках? --- спытаў Гары. 
--- Ці значыць гэта, што на мяне прымянялі Забывальны заклён?

--- Мянэ гэта таксама бянтэжыць, --- сказала МакГонагал павольна. --- Калі гэта было б 
так проста, усе суды даўно бы карысталі напамінальнікі. Я падумаю аб гэтым, містэр Потэр, ---
яна ўздыхнула. --- Вы свабодны.

Гары пачаў быў падымацца, але спыніўся.

--- Эмм, прабачце, я хацеў яшчэ нешта вам расказаць...

Уздрыг быў аваль незаўважны.

--- Што, містэр Потэр?

--- Гэта наконт прафесара Квірэла...

--- Я ўпэўнена, што яно не вартае ўвагі, --- сказала прафесар МакГонагал з нечаканым паспехам.
--- Ці вы не чулі, як Майстра загадваў студэнтам не назаляць нам сваімі дакучлівымі скаргамі аб
прафесары Абароны?

Гары разгубіўся.

--- Але \emph{гэта} можа быць важна. Учора ў мяне з'явілася раптоўнае адчуванне бяды, калі...

--- Містэр Потэр! У мяне таксама адчуванне бяды! І мае адчуванне бяды падказвае, што 
\emph{лепей вам не сканчваць гэты сказ!}

Рот Гары застыў, расчынены. Як і прапанавала МакГонагал, Гары не змог дагаварыць.

--- Містэр Потэр, --- сказала прафесар МакГонагал, --- калі наступным разам вы высветліце
нешта цікавае пра прафесара Квірэла, смела трымайце гэта пры сябе, і, калі ласка, не дзяліцеся
са мною або з кімсьці яшчэ. Думаю, вы ўжо патрацілі досыць майго каштоўнага часу...

--- \emph{Гэта на вас не падобна!} --- выбухнуў Гары. --- Выбачайце, але вы паводзіце сябе 
\emph{неверагодна} безадказна! Ходзяць чуткі, што на пасаду прафесара Абароны накладзены
нейкі праклён, і калі вы ўжо ведаеце, што нешта можна пайсці не так, вы павіннны ўжо даўно
навастрыць вушы...

--- Пайсці \emph{не так}, містэр Потэр? \emph{Толькі не гэтым разам.} --- Твар 
МакГонагал быў безвыразны. --- Пасля таго як прафесара Блэйка застукалі ў кладоўцы з 
не менш чым трымя студэнткамі, а за год да таго прафесар Самерс настолькі няздоліўся ў 
навучанні, што яго студэнты былі ўпэўнены, што богарт --- гэта такі від мэблі;
калі ўжо неверагодна кампетэнтны прафесар Квірэл не справіцца, гэта будзе проста 
\emph{катастрофа}. Не кажучы, што большасць студэнтаў праваляць абарону 
ў гэтым годзе на сычах і крумкачах\footnote{{} Экзамены чараўнікоў: СЧ --- Сярэдні Чараўніцкі экзамен пасля пятага года навучання,
КРУМ --- Катавальна РУплівы Магічны экзамен, выпускны. У простай мове завуцца адпаведна
сычы і крумкачы.}.

--- Разумею... --- сказаў Гары павольна, спрабуючы прыняць гэта ўсё. --- Кажучы проста,
што бы ні адбывалася з пафесарам Квірэлам, вы не жадаеце чуць пра тое да канца навучальнаго года.
А ўлічваючы, што зараз яшчэ кастрычнік, то ён можа забіць прэмьер-міністра на ток-шоў у прамым эфіры, 
і яму за гэта нічога не будзе, па-вашаму.

Прафесар МакГонагал глядзела на яго неміргаючым позіркам.

--- Я ўпэўнена, што пацверджанне гэтай заявы ад мяне пачуць было і будзе немагчыма, містэр Потэр. 
У Хогвартс мы імкнемся праактыўна рэагаваць на \emph{любыя} перашкоды адукацыйным 
дасягненням нашых студэнтаў.

\emph{Напрыклад, на рэйвенкло-першаркусніка, які не мае трымаць рот зачыненым.}

--- Думаю, я цалкам вас разумею, прафесар МакГонагал.

--- О, сумняваюся, містэр Потэр. Вельмі сумняваюся, --- МакГонагал падалася наперад,
твар яе зноў стаў жорскім. --- Мы з вамі абмяркоўвалі справы значна больш адчувальныя,
таму скажу вам шчыра. Вы, і толькі вы, паведамілі пра гэтае містычнае прадчуванне бяды. 
Вы, і толькі вы, прыцягваеце да сябе хаос так, як ніхто, каго я ведаю.
Пасля нашага шопінгу ў Дыягон-алеі і \emph{потым} --- інцыдэнта з Капелюшом, і 
пасля \emph{сённяшняга} эпізода, я ясна прадбачу пэўную размову ў кабінеце Майстра, 
дзе ўсе пачуюць выбітную гісторыю, у якой вы, і толькі вы, граеце галоўную ролю, пасля
чаго ў нас не застанецца выбару, як звольніць прафесара Квірэла.
Я ўжо нават не супраціўлюся гэтым думкам. Але, калі гэтая сумная падзея адбудзецца раней за 
пятнаццатага траўня, я прывяжу вас да варот Хогвартс вашымі ўласнымі вантробамі, і 
буду кожны дзень падсыпаць вогняжукоў у ваш нос. Ці разумецце вы мяне \emph{зараз?}

Вочы ў Гары сталі вельмі шырокія. Ён кіўнуў. Потым праз секунду ён спытаў:

--- А што мне будзе, калі я здолею гэта ў апошні школьны дзень?



--- \emph{Прэч з майго кабінета!}

\later

Чацьвер.

Thursday.

There must have been something about Thursdays in Hogwarts.

It was 5:32\pm on Thursday afternoon, and Harry was standing next to Professor Flitwick, in front of the great stone gargoyle that guarded the entrance to the Headmaster’s office.

No sooner had he made it back from Professor McGonagall’s office to the Ravenclaw study rooms than one of the students told him to report to Professor Flitwick’s office, and there Harry had learned that Dumbledore wanted to speak to him.

Harry, feeling rather apprehensive, had asked Professor Flitwick if the Headmaster had said what this was about.

Professor Flitwick had shrugged in a helpless sort of way.

Apparently Dumbledore had said that Harry was far too young to invoke the words of power and madness.

\emph{Happy happy boom boom swamp swamp swamp?} Harry had thought but not said aloud.

“Please don’t worry too much, Mr~Potter,” squeaked Professor Flitwick from somewhere around Harry’s shoulder level. (Harry was grateful for Professor Flitwick’s gigantic puffy beard, it was hard getting used to a Professor who was not only shorter than him but spoke in a higher-pitched voice.) “Headmaster Dumbledore may seem a little odd, or a lot odd, or even extremely odd, but he has never hurt a student in the slightest, and I don’t believe he ever will.” Professor Flitwick gave Harry an encouraging smile. “Just keep that in mind at all times and you’ll be sure not to panic!”

This was not helping.

“Good luck!” squeaked Professor Flitwick, and leaned over to the gargoyle and said something that Harry somehow failed to hear at all. (Of course, the password wouldn’t be much good if you could hear someone saying it.) And the stone gargoyle walked aside with a very natural and ordinary movement that Harry found rather shocking, since the gargoyle still looked like solid, immovable stone the whole time.

Behind the gargoyle was a set of slowly revolving spiral stairs. There was something disturbingly hypnotic about it, and even more disturbing was that \emph{revolving} the spiral ought not to take you anywhere.

“Up you go!” squeaked Flitwick.

Harry rather nervously stepped onto the spiral, and found himself, for some reason that his brain couldn’t seem to visualise at all, moving upwards.

The gargoyle thudded back into place behind him, and the spiral stairs kept turning and Harry kept being higher up, and after a rather dizzying time, Harry found himself in front of an oak door with a brass griffin knocker.

Harry reached out and turned the doorknob.

The door swung open.

And Harry saw the most interesting room he’d ever seen in his life.

There were tiny metal mechanisms that whirred or ticked or slowly changed shape or emitted little puffs of smoke. There were dozens of mysterious fluids in dozens of oddly shaped containers, all bubbling, boiling, oozing, changing colour, or forming into interesting shapes that vanished half a second after you saw them. There were things that looked like clocks with many hands, inscribed with numbers or in unrecognisable languages. There was a bracelet bearing a lenticular crystal that sparkled with a thousand colours, and a bird perched atop a golden platform, and a wooden cup filled with what looked like blood, and a statue of a falcon encrusted in black enamel. The wall was all hung with pictures of people sleeping, and the Sorting Hat was casually poised on a hat rack that was also holding two umbrellas and three red slippers for left feet.

In the midst of all the chaos was a clean black oaken desk. Before the desk was an oaken stool. And behind the desk was a well-cushioned throne containing Albus Percival Wulfric Brian Dumbledore, who was adorned with a long silver beard, a hat like a squashed giant mushroom, and what looked to Muggle eyes like three layers of bright pink pyjamas.

Dumbledore was smiling, and his bright eyes twinkled with a mad intensity.

With some trepidation, Harry seated himself in front of the desk. The door swung shut behind him with a loud \emph{thunk}.

“Hello, Harry,” said Dumbledore.

“Hello, Headmaster,” Harry replied. So they were on a first-name basis? Would Dumbledore now say to call him—

“Please, Harry!” said Dumbledore. “Headmaster sounds so formal. Just call me Heh for short.”

“I’ll be sure to, Heh,” said Harry.

There was a slight pause.

“Do you know,” said Dumbledore, “you’re the first person who’s ever taken me up on that?”

“Ah…” Harry said. He tried to control his voice despite the sudden sinking feeling in his stomach. “I’m sorry, I, ah, Headmaster, you told me to do it so I did—”

“Heh, please!” said Dumbledore cheerfully. “And there’s no call to be so worried, I won’t launch you out a window just because you make one mistake. I’ll give you plenty of warnings first, if you’re doing something wrong! Besides, what matters isn’t how people talk to you, it’s what they think of you.”

\emph{He’s never hurt a student, just keep remembering that and you’ll be sure not to panic.}

Dumbledore drew forth a small metal case and flipped it open, showing some small yellow lumps. “Sherbet lemon?” said the Headmaster.

“Er, no thank you, Heh,” said Harry. \emph{Does slipping a student LSD count as hurting them, or does that fall into the category of harmless fun?} “You, um, said something about my being too young to invoke the words of power and madness?”

“That you most certainly are!” Dumbledore said. “Thankfully the Words of Power and Madness were lost seven centuries ago and no-one has the slightest idea what they are any more. It was just a little remark.”

“Ah…” Harry said. He was aware that his mouth was hanging open. “Why did you call me here, then?”

“\emph{Why?}” Dumbledore repeated. “Ah, Harry, if I went around all day asking \emph{why} I do things, I’d never have time to get a single thing done! I’m quite a busy person, you know.”

Harry nodded, smiling. “Yes, it was a very impressive list. Headmaster of Hogwarts, Chief Warlock of the Wizengamot, and Supreme Mugwump of the International Confederation of Wizards. Sorry to ask but I was wondering, is it possible to get more than six hours if you use more than one Time-Turner? Because it’s pretty impressive if you’re doing all that on just thirty hours a day.”

There was another slight pause, during which Harry went on smiling. He was a little apprehensive, actually a lot apprehensive, but once it had become clear that Dumbledore was deliberately messing with him, something within him \emph{absolutely refused} to sit and take it like a defenceless lump.

“I’m afraid Time doesn’t like being stretched out too much,” said Dumbledore after the slight pause, “and yet we ourselves seem to be a little too large for it, and so it’s a constant struggle to fit our lives into Time.”

“Indeed,” Harry said with grave solemnity. “That’s why it’s best to come to our points quickly.”

For a moment Harry wondered if he’d gone too far.

Then Dumbledore chuckled. “Straight to the point it shall be.” The Headmaster leaned forwards, tilting his squashed mushroom hat and brushing his beard against his desk. “Harry, this Monday you did something that should have been impossible even with a Time-Turner. Or rather, impossible with \emph{only} a Time-Turner. Where did those two pies come from, I wonder?”

A jolt of adrenaline shot through Harry. He’d done that using the Cloak of Invisibility, the one that had been given him in a Christmas box along with a note, and that note had said: \emph{If Dumbledore saw a chance to possess one of the Deathly Hallows he would never let it escape his grasp….}

“A natural thought,” Dumbledore went on, “is that since none of the first-years present were able to cast such a spell, someone else was present, and yet unseen. And if no-one could see them, why, it would be easy enough for them to throw the pies. One might further suspect that since you had a Time-Turner, you were the invisible one; and that since the spell of Disillusionment is far beyond your current abilities, you had an invisibility cloak.” Dumbledore smiled conspiratorially. “Am I on the right track so far, Harry?”

Harry was frozen. He had the feeling that an outright lie would not at all be wise, and possibly not the least bit helpful, and he couldn’t think of anything else to say.

Dumbledore waved a friendly hand. “Don’t worry, Harry, you haven’t done anything wrong. Invisibility cloaks aren’t against the rules—I suppose they’re rare enough that no-one ever got around to putting them on the list. But really I was wondering something else entirely.”

“Oh?” Harry said in the most normal voice he could manage.

Dumbledore’s eyes shone with enthusiasm. “You see, Harry, after you’ve been through a few adventures you tend to catch the hang of these things. You start to see the pattern, hear the rhythm of the world. You begin to harbour suspicions \emph{before} the moment of revelation. You are the Boy-Who-Lived, and somehow an invisibility cloak made its way into your hands only four days after you discovered our magical Britain. Such cloaks are not for sale in Diagon Alley, but there is \emph{one} which might find its own way to a destined wearer. And so I cannot help but wonder if by some strange chance you have found not just \emph{an} invisibility cloak, but \emph{the} Cloak of Invisibility, one of the three Deathly Hallows and reputed to hide the wearer from the gaze of Death himself.” Dumbledore’s gaze was bright and eager. “May I see it, Harry?”

Harry swallowed. There was a full flood of adrenaline in his system now and it was entirely useless, this was the most powerful wizard in the world and there was no way he could make it out the door and there was nowhere in Hogwarts for him to hide if he did, he was about to lose the Cloak that had been passed down through the Potters for who knew how long—

Slowly Dumbledore leaned back into his high chair. The bright light had gone out of his eyes, and he looked puzzled and a little sorrowful. “Harry,” said Dumbledore, “if you don’t want to, you can just say no.”

“I can?” Harry croaked.

“Yes, Harry,” said Dumbledore. His voice sounded sad now, and worried. “It seems that you’re afraid of me, Harry. May I ask what I’ve done to earn your distrust?”

Harry swallowed. “Is there some way you can swear a binding magical oath that you won’t take my cloak?”

Dumbledore shook his head slowly. “Unbreakable Vows are not to be used so lightly. And besides, Harry, if you did not already know the spell, you would have only my word that the spell was binding. Yet surely you realise that I do not \emph{need} your permission to see the Cloak. I am powerful enough to draw it forth myself, mokeskin pouch or no.” Dumbledore’s face was very grave. “But this I will not do. The Cloak is yours, Harry. I will not seize it from you. Not even to look at for just a moment, unless you decide to show it to me. That is a promise and an oath. Should I need to prohibit you from using it on the school grounds, I will require you to go to your vault at Gringotts and store it there.”

“Ah…” Harry said. He swallowed hard, trying to calm the flood of adrenaline and think reasonably. He took the mokeskin pouch off his belt. “If you really \emph{don’t} need my permission…then you have it.” Harry held out the pouch to Dumbledore, and bit down hard on his lip, sending that signal to himself in case he was Obliviated afterwards.

The old wizard reached into the pouch, and without saying any word of retrieval, drew forth the Cloak of Invisibility.

“Ah,” breathed Dumbledore. “I was right…” He poured the shimmering black velvet mesh through his hand. “Centuries old, and still as perfect as the day it was made. We have lost much of our art over the years, and now I cannot make such a thing myself, no-one can. I can feel the power of it like an echo in my mind, like a song forever being sung without anyone to hear it…” The wizard looked up from the Cloak. “Do not sell it,” he said, “do not give it to anyone as a possession. Think twice before you show it to anyone, and ponder three times again before you reveal it is a Deathly Hallow. Treat it with respect, for this is indeed a Thing of Power.”

For a moment Dumbledore’s face grew wistful…

…and then he handed the Cloak back to Harry.

Harry put it back in his pouch.

Dumbledore’s face was grave once more. “May I ask again, Harry, how you came to distrust me so?”

Suddenly Harry felt rather ashamed.

“There was a note with the Cloak,” Harry said in a small voice. “It said that you would try to take the Cloak from me, if you knew. I don’t know who left the note, though, I really don’t.”

“I…see,” Dumbledore said slowly. “Well, Harry, I won’t impugn the motives of whoever left you that note. Who knows but that they themselves may have had the best of good intentions? They did give you the Cloak, after all.”

Harry nodded, impressed by Dumbledore’s charity, and abashed at the sharp contrast with his own attitude.

The old wizard went on. “But you and I are both game pieces of the same colour, I think. The boy who finally defeated Voldemort, and the old man who held him off long enough for you to save the day. I will not hold your caution against you, Harry, we must all do our best to be wise. I will only ask that you think twice and ponder three times again, the next time someone tells you to distrust me.”

“I’m sorry,” Harry said. He felt wretched at this point, he’d just told off Gandalf essentially, and Dumbledore’s kindness was only making him feel worse. “I shouldn’t have distrusted you.”

“Alas, Harry, in this world…” The old wizard shook his head. “I cannot even say you were unwise. You did not know me. And in truth there are some at Hogwarts who you would do well not to trust. Perhaps even some you call friends.”

Harry swallowed. That sounded rather ominous. “Like who?”

Dumbledore stood up from his chair, and began examining one of his instruments, a dial with eight hands of varying length.

After a few moments, the old wizard spoke again. “He probably seems to you quite charming,” said Dumbledore. “Polite—to you at least. Well-spoken, maybe even admiring. Always ready with a helping hand, a favour, a word of advice—”

“Oh, \emph{Draco Malfoy!}” Harry said, feeling rather relieved that it wasn’t Hermione or something. “Oh no, no no no, you’ve got it all wrong, he’s not turning me, I’m turning him.”

Dumbledore froze where he was peering at the dial. “You’re \emph{what}?”

“I’m going to turn Draco Malfoy from the Dark Side,” Harry said. “You know, make him a good guy.”

Dumbledore straightened and turned to Harry. He was wearing one of the most astonished expressions Harry had ever seen on anyone, let alone someone with a long silver beard. “Are you certain,” said the old wizard after a moment, “that he is ready to be redeemed? I fear that whatever goodness you think you see within him is only wishful thinking—or worse, a lure, a bait—”

“Er, not likely,” Harry said. “I mean if he’s trying to disguise himself as a good guy he’s incredibly bad at it. This isn’t a question of Draco coming up to me and being all charming and me deciding that he must have a hidden core of goodness deep down. I selected him for redemption specifically because he’s the heir to House Malfoy and if you had to pick one person to redeem, it would obviously be him.”

Dumbledore’s left eye twitched. “You intend to sow seeds of love and kindness in Draco Malfoy’s heart because you expect Malfoy’s heir to prove valuable to you?”

“Not just to \emph{me}!” Harry said indignantly. “To all of magical Britain, if this works out! \emph{And} he’ll have a happier and mentally healthier life himself! Look, I don’t have enough time to turn \emph{everyone} away from the Dark Side and I’ve got to ask where the Light can gain the most advantage the fastest—”

Dumbledore started laughing. Laughing a lot harder than Harry would expect, almost howling. It seemed positively \emph{undignified}. An ancient and powerful wizard ought to chuckle in deep booming tones, not laugh so hard he was gasping for breath. Harry had once literally fallen out of his chair while watching the Marx Brothers film \emph{Duck Soup,} and that was how hard Dumbledore was laughing now.

“It’s not \emph{that} funny,” Harry said after a while. He was starting to worry about Dumbledore’s sanity again.

Dumbledore got himself under control again with a visible effort. “Ah, Harry, one symptom of the disease called wisdom is that you begin laughing at things that no-one else thinks is funny, because when you’re wise, Harry, you start getting the jokes!” The old wizard wiped tears away from his eyes. “Ah, me. Ah, me. Oft evil will shall evil mar indeed, in very deed.”

Harry’s brain took a moment to place the familiar words…“Hey, that’s a \emph{Tolkien} quote! \emph{Gandalf} says that!”

“Théoden, actually,” said Dumbledore.

“You’re \emph{Muggle-born?}” Harry said in shock.

“I’m afraid not,” said Dumbledore, smiling again. “I was born seventy years before that book was published, dear child. But it seems that my Muggle-born students tend to think alike in certain ways. I have accumulated no fewer than twenty copies of \emph{The Lord of the Rings} and three sets of Tolkien’s entire collected works, and I treasure every one of them.” Dumbledore drew his wand and held it up and struck a pose. “\emph{You cannot pass!} How does that look?”

“Ah,” Harry said in something approaching complete brain shutdown, “I think you’re missing a Balrog.” And the pink pyjamas and squashed mushroom hat were not helping in the slightest.

“I see.” Dumbledore sighed and glumly sheathed the wand in his belt. “I fear there have been precious few Balrogs in my life of late. Nowadays it’s all meetings of the Wizengamot where I must try desperately to prevent any work from getting done, and formal dinners where foreign politicians compete to see who can be the most obstinate fool. And being mysterious at people, knowing things I have no way of knowing, making cryptic statements which can only be understood in hindsight, and all the other small ways in which powerful wizards amuse themselves after they have left the part of the pattern that allows them to be heroes. Speaking of which, Harry, I have a certain something to give you, something which belonged to your father.”

“You do?” said Harry. “Gosh, who would have figured.”

“Yes indeed,” said Dumbledore. “I suppose it is a little predictable, isn’t it?” His face turned solemn. “Nonetheless…”

Dumbledore went back to his desk and sat down, pulling out one of the drawers as he did so. He reached in using both arms, and, straining slightly, pulled a rather large and heavy-looking object out of the drawer, which he then deposited on his oaken desk with a huge thunk.

“This,” Dumbledore said, “was your father’s rock.”

Harry stared at it. It was light grey, discoloured, irregularly shaped, sharp-edged, and very much a plain old ordinary large rock. Dumbledore had deposited it so that it rested on the widest available cross-section, but it still wobbled unstably on his desk.

Harry looked up. “This is a joke, right?”

“It is not,” said Dumbledore, shaking his head and looking very serious. “I took this from the ruins of James and Lily’s home in Godric’s Hollow, where also I found you; and I have kept it from then until now, against the day when I could give it to you.”

In the mixture of hypotheses that served as Harry’s model of the world, Dumbledore’s insanity was rapidly rising in probability. But there \emph{was} still a substantial amount of probability allocated to other alternatives…“Um, is it a \emph{magical} rock?”

“Not so far as I know,” said Dumbledore. “But I advise you with the greatest possible stringency to keep it close about your person at all times.”

All right. Dumbledore was \emph{probably} insane but if he \emph{wasn’t}…well, it would be just too \emph{embarrassing} to get in trouble from ignoring the advice of the inscrutable old wizard. That had to be like \#4 on the list of the Top 100 Obvious Failure Modes.

Harry stepped forward and put his hands on the rock, trying to find some angle from which to lift it without cutting himself. “I’ll put it in my pouch, then.”

Dumbledore frowned. “That may not be close enough to your person. And what if your mokeskin pouch is lost, or stolen?”

“You think I should just carry a big rock everywhere I go?”

Dumbledore gave Harry a serious look. “That might prove wise.”

“Ah…” Harry said. It looked rather heavy. “I’d think the other students would tend to ask me questions about that.”

“Tell them I ordered you to do it,” said Dumbledore. “No-one will question that, since they all think I’m insane.” His face was still perfectly serious.

“Er, to be honest if you go around ordering your students to carry large rocks I can kind of see why people would think that.”

“Ah, Harry,” said Dumbledore. The old wizard gestured, a sweep of one hand that seemed to take in all the mysterious instruments around the room. “When we are young we believe that we know everything, and so we believe that if we see no explanation for something, then no explanation exists. When we are older we realise that the whole universe works by a rhythm and a reason, even if we ourselves do not know it. It is only our own ignorance which appears to us as insanity.”

“Reality is always lawful,” said Harry, “even if we don’t know the law.”

“Precisely, Harry,” said Dumbledore. “To understand this—and I see that you \emph{do} understand it—is the essence of wisdom.”

“So…\emph{why} do I have to carry this rock exactly?”

“I can’t think of a reason, actually,” said Dumbledore.

“…you can’t.”

Dumbledore nodded. “But just because I can’t think of a reason doesn’t mean there \emph{is} no reason.”

The instruments ticked on.

“Okay,” said Harry, “I’m not even sure if I should be saying this, but that is simply not the correct way to deal with our admitted ignorance of how the universe works.”

“It isn’t?” said the old wizard, looking surprised and disappointed.

Harry had the feeling this conversation was not going to work out in his favour, but he carried on regardless. “No. I don’t even know if that fallacy has an official name, but if I had to make one up myself, it would be ‘privileging the hypothesis’ or something like that. How can I put this formally…um…suppose you had a million boxes, and only one of the boxes contained a diamond. And you had a box full of diamond-detectors, and each diamond-detector always went off in the presence of a diamond, and went off half the time on boxes that didn’t have a diamond. If you ran twenty detectors over all the boxes, you’d have, on average, one false candidate and one true candidate left. And then it would just take one or two more detectors before you were left with the one true candidate. The point being that when there are lots of possible answers, \emph{most} of the evidence you need goes into just \emph{locating} the true hypothesis out of millions of possibilities—bringing it to your attention in the first place. The amount of evidence you need to judge between two or three plausible candidates is much smaller by comparison. So if you just jump ahead without evidence and promote one particular possibility to the focus of your attention, you’re skipping over most of the work. Like, you live in a city where there are a million people, and there’s a murder, and a detective says, well, we’ve got no evidence at all, so have we considered the possibility that Mortimer Snodgrass did it?”

“Did he?” said Dumbledore.

“No,” said Harry. “But later it turns out that the murderer had black hair, and Mortimer has black hair, so everyone’s like, ah, looks like Mortimer did it after all. So it’s unfair to Mortimer for the police to \emph{promote him to their attention} without having good reasons already in hand to suspect him. When there are lots of possibilities, most of the work goes into just \emph{locating} the true answer—starting to pay attention to it. You don’t need \emph{proof}, or the sort of official evidence that scientists or courts demand, but you need some sort of \emph{hint}, and that hint has to discriminate that particular possibility from the millions of others. Otherwise you can’t just pluck the right answer out of thin air. You can’t even pluck a possibility worth thinking about out of thin air. And there’s got to be a million other things I could do besides carrying around my father’s rock. Just because I’m ignorant about the universe doesn’t mean that I’m unsure about how I should reason in the presence of my uncertainty. The laws for thinking with probabilities are no less iron than the laws that govern old-fashioned logic, and what you just did is \emph{not allowed.}” Harry paused. “\emph{Unless}, of course, you have some \emph{hint} you’re not mentioning.”

“Ah,” said Dumbledore. He tapped his cheek, looking thoughtful. “An interesting argument, certainly, but doesn’t it break down at the point where you make an analogy between a million potential murderers only one of whom committed the murder, and taking one out of many possible courses of action, when many possible courses of action may all be wise? I do not say that carrying your father’s rock is the one best possible course of action, only that it is wiser to do than not.”

Dumbledore once again reached into the same desk drawer he had accessed earlier, this time seeming to root around inside—at least his arm seemed to be moving. “I will remark,” Dumbledore said while Harry was still trying to sort out how to reply to this completely unexpected rejoinder, “that it is a common misconception of Ravenclaws that all the smart children are Sorted there, leaving none for other Houses. This is not so; being Sorted to Ravenclaw indicates that you are driven by your desire to know things, which is not at all the same quality as being intelligent.” The wizard was smiling as he bent over the drawer. “Nonetheless, you \emph{do} seem rather intelligent. Less like an ordinary young hero and more like a young mysterious ancient wizard. I think I may have been taking the wrong approach with you, Harry, and that you may be able to understand things that few others could grasp. So I shall be daring, and offer you a certain \emph{other} heirloom.”

“You don’t mean…” gasped Harry. “My father…\emph{owned another rock?}”

“Excuse me,” said Dumbledore, “I \emph{am} still older and more mysterious than you and if there are any revelations to be made then \emph{I} will do the revealing, thank you…oh, where \emph{is} that thing!” Dumbledore reached down further into the desk drawer, and still further. His head and shoulders and whole torso disappeared inside until only his hips and legs were sticking out, as though the desk drawer was eating him.

Harry couldn’t help but wonder just how much stuff was in there and what the complete inventory would look like.

Finally Dumbledore rose back up out of the drawer, holding the objective of his search, which he set down on the desk alongside the rock.

It was a used, ragged-edged, textbook with a worn spine: \emph{Intermediate Potion Making} by Libatius Borage. There was a picture of a smoking vial on the cover.

“This,” Dumbledore intoned, “was your mother’s fifth-year Potions textbook.”

“Which I am to carry with me at all times,” said Harry.

“\emph{Which holds a terrible secret}. A secret whose revelation could prove so disastrous that I must ask you to swear—and I do require you to swear it seriously, Harry, whatever you may think of all this—never to tell anyone or anything else.”

Harry considered his mother’s fifth-year Potions textbook, which, apparently, held a terrible secret.

The problem was that Harry \emph{did} take that oaths like that very seriously. Any vow was an Unbreakable Vow if made by the right sort of person.

And…

“I’m feeling thirsty,” Harry said, “and that is not at all a good sign.”

Dumbledore entirely failed to ask any questions about this cryptic statement. “\emph{Do} you swear, Harry?” said Dumbledore. His eyes gazed intently into Harry’s. “Otherwise I cannot tell you.”

“Yes,” said Harry. “I swear.” That was the trouble with being a Ravenclaw. You couldn’t refuse an offer like that or your curiosity would eat you alive, and everyone else knew it.

“And I swear in turn,” said Dumbledore, “that what I am about to tell you is the truth.”

Dumbledore opened the book, seemingly at random, and Harry leaned in to see.

“Do you see these notes,” Dumbledore said in a voice so low it was almost a whisper, “written in the margins of the book?”

Harry squinted slightly. The yellowing pages seemed to be describing something called a \emph{potion of eagle’s splendour}, many of the ingredients being items that Harry didn’t recognise at all and whose names didn’t appear to derive from English. Scrawled in the margin was a handwritten annotation saying, \emph{I wonder what would happen if you used Thestral blood here instead of blueberries?} and immediately beneath was a reply in different handwriting, \emph{You’d get sick for weeks and maybe die.}

“I see them,” said Harry. “What about them?”

Dumbledore pointed to the second scrawl. “The ones in this handwriting,” he said, still in that low voice, “were written by your mother. And the ones in \emph{this} handwriting,” moving his finger to indicate the first scrawl, “were written by me. I would turn myself invisible and sneak into her dorm while she was sleeping. Lily thought one of her friends was writing them and they had the most amazing fights.”

That was the exact point at which Harry realised that the Headmaster of Hogwarts \emph{was}, in fact, crazy.

Dumbledore was looking at him with a serious expression. “Do you understand the implications of what I have just told you, Harry?”

“Ehhh…” Harry said. His voice seemed to be stuck. “Sorry…I…not really…”

“Ah well,” said Dumbledore, and sighed. “I suppose your cleverness has limits after all, then. Shall we all just pretend I didn’t say anything?”

Harry rose from his chair, wearing a fixed smile. “Of course,” Harry said. “You know it’s actually getting rather late in the day and I’m a bit hungry, so I should be going down to dinner, really” and Harry made a beeline for the door.

The doorknob entirely failed to turn.

“You wound me, Harry,” said Dumbledore’s voice in quiet tones that were coming from right behind him. “Do you not at least realise that what I have told you is a sign of trust?”

Harry slowly turned around.

In front of him was a very powerful and very insane wizard with a long silver beard, a hat like a squashed giant mushroom, and wearing what looked to Muggle eyes like three layers of bright pink pyjamas.

Behind him was a door that didn’t seem to be working at the moment.

Dumbledore was looking rather saddened and weary, like he wanted to lean on a wizard’s staff he didn’t have. “Really,” said Dumbledore, “you try anything new instead of following the same pattern every time for a hundred and ten years, and people all start running away.” The old wizard shook his head in sorrow. “I’d hoped for better from you, Harry Potter. I’d heard that your own friends also think you mad. I know they are mistaken. Will you not believe the same of me?”

“Please open the door,” Harry said, his voice trembling. “If you ever want me to trust you again, open the door.”

There was the sound behind him of a door opening.

“There were more things I planned to say to you,” Dumbledore said, “and if you leave now, you will not know what they were.”

Sometimes Harry absolutely \emph{hated} being a Ravenclaw.

\emph{He’s never hurt a student,} said Harry’s Gryffindor side. \emph{Just keep remembering that and you’ll be sure not to panic. You’re not going to run away just because things are getting interesting, are you?}

\emph{You can’t just walk out on the Headmaster!} said the Hufflepuff part. \emph{What if he starts deducting House points? He could make your school life very difficult if he decides he doesn’t like you!}

And a piece of himself which Harry didn’t much like but couldn’t quite manage to silence was pondering the potential advantages of being one of the few friends of this mad old wizard who also happened to be Headmaster, Chief Warlock, and Supreme Mugwump. And unfortunately his inner Slytherin seemed to be much better than Draco at turning people to the Dark Side, because it was saying things like \emph{poor fellow, he looks like he needs someone to talk to, doesn’t he? \emph{and} you wouldn’t want such a powerful man to end up trusting someone less virtuous, would you?} and \emph{I wonder what sort of incredible secrets Dumbledore could tell you if, you know, you became friends with him} and even \emph{I bet he’s got a reaallly interesting book collection.}

\emph{You’re all a bunch of lunatics,} Harry thought at the entire assemblage, but he’d been unanimously outvoted by every component part of himself.

Harry turned, took a step towards the open door, reached out, and deliberately closed it again. It was a costless sacrifice given that he was staying anyway, Dumbledore could control his movements regardless, but maybe it would impress Dumbledore.

When Harry turned back around he saw that the powerful insane wizard was once more smiling and looking friendly. That was good, maybe.

“Please don’t do that again,” Harry said. “I don’t like being trapped.”

“I \emph{am} sorry about that, Harry,” said Dumbledore in what sounded like tones of sincere apology. “But it would have been terribly unwise to let you leave without your father’s rock.”

“Of course,” Harry said. “It wasn’t reasonable of me to expect the door to open before I put the quest items in my inventory.”

Dumbledore smiled and nodded.

Harry went over to the desk, twisted his mokeskin pouch around to the front of his belt, and, with some effort, managed to heave up the rock in his eleven-year-old arms and feed it in.

He could actually feel the weight slowly diminishing as the Widening Lip charm ate the rock, and the burp which followed was rather noisy and had a distinctly complaining sound to it.

His mother’s fifth-year Potions textbook (which held a secret that was in fact pretty terrible) followed shortly after.

And then Harry’s inner Slytherin made a sly suggestion for ingratiating himself with the Headmaster, which, unfortunately, had been perfectly pitched in such a way as to gain the support of the majority Ravenclaw faction.

“So,” Harry said. “Um. As long as I’m hanging around, I don’t suppose you would like to give me a bit of a tour of your office? I’m a bit curious as to what some of these things are,” and that was his understatement for the month of September.

Dumbledore gazed at him, and then nodded with a slight grin. “I’m flattered by your interest,” said Dumbledore, “but I’m afraid there isn’t much to say.” Dumbledore took a step closer to the wall and pointed to a painting of a sleeping man. “These are portraits of past Headmasters of Hogwarts.” He turned and pointed to his desk. “This is my desk.” He pointed to his chair. “This is my chair—”

“Excuse me,” Harry said, “actually I was wondering about those.” Harry pointed to a small cube that was softly whispering “blorple…blorple…blorple”.

“Oh, the little fiddly things?” said Dumbledore. “They came with the Headmaster’s office and I have absolutely no idea what most of them do. Although \emph{this} dial with the eight hands counts the number of, let’s call them sneezes, by left-handed witches within the borders of France, you would not believe how much work it took to nail that down. And \emph{this} one with the golden wibblers is my own invention and Minerva is never, ever going to figure out what it’s doing.”

Dumbledore took a step over to the hat rack while Harry was still processing this. “Here of course we have the Sorting Hat, I believe the two of you have met. It told me that it was never again to be placed on your head under any circumstances. You’re only the fourteenth student in history it’s said that about, Baba Yaga was another one and I’ll tell you about the other twelve when you’re older. This is an umbrella. This is another umbrella.” Dumbledore took another few steps and turned around, now smiling quite broadly. “And of course, most people who come to my office want to see Fawkes.”

Dumbledore was standing next to the bird on the golden platform.

Harry came over, rather puzzled. “This is Fawkes?”

“Fawkes is a phœnix,” said Dumbledore. “Very rare, very powerful magical creatures.”

“Ah…” Harry said. He lowered his head and stared into the tiny, beady black eyes, which showed not the slightest sign of power or intelligence.

“Ahhh…” Harry said again.

He was pretty sure he recognised the shape of the bird. It was pretty hard to miss.

“Umm…”

\emph{Say something intelligent!} Harry’s mind roared at itself. \emph{Don’t just stand there sounding like a gibbering moron!}

\emph{Well what the heck am I \emph{supposed} to say?} Harry’s mind fired back.

\emph{Anything!}

\emph{You mean, anything besides “Fawkes is a chicken”—}

\emph{Yes! Anything but that!}

“So, ah, what sort of magic do phœnixes do, then?”

“Their tears have the power to heal,” Dumbledore said. “They are creatures of fire, and move between all places as easily as fire may extinguish itself in one place and be kindled in another. The tremendous strain of their innate magic ages their bodies quickly, and yet they are as close to undying as any creature that exists in this world, for whenever their bodies fail them they immolate themselves in a burst of fire and leave behind a hatchling, or sometimes an egg.” Dumbledore came closer and inspected the chicken, frowning. “Hm…looking a little peaky there, I’d say.”

By the time this statement registered fully in Harry’s mind, the chicken was already on fire.

The chicken’s beak opened, but it didn’t have time for so much as a single caw before it began to wither and char. The blaze was brief, intense, and entirely self-contained; there was no smell of burning.

And then the fire died down only seconds after it had begun, leaving behind a tiny, pathetic heap of ashes on the golden platform.

“Don’t look so horrified, Harry!” said Dumbledore. “Fawkes hasn’t been hurt.” Dumbledore’s hand dipped into a pocket, and then the same hand sifted through the ashes and turned up a small yellowish egg. “Look, here’s an egg!”

“Oh…wow…amazing…”

“But now we really should get on with things,” Dumbledore said. Leaving the egg behind in the ashes of the chicken, he returned to his throne and seated himself. “It’s almost time for dinner, after all, and we wouldn’t want to have to use our Time-Turners.”

There was a violent power struggle going on in the Government of Harry. Slytherin and Hufflepuff had switched sides after seeing the Headmaster of Hogwarts set fire to a chicken.

“Yes, things,” said Harry’s lips. “And then dinner.”

\emph{You’re sounding like a gibbering moron again} observed Harry’s Internal Critic.

“Well,” Dumbledore said. “I fear I have a confession to make, Harry. A confession and an apology.”

“Apologies are good” \emph{that doesn’t even make sense! What am I talking about?}

The old wizard sighed deeply. “You may not still think so after understanding what I have to say. I’m afraid, Harry, that I’ve been manipulating you your entire life. It was I who consigned you to the care of your wicked step-parents—”

“My step-parents aren’t wicked!” blurted Harry. “My \emph{parents}, I mean!”

“They aren’t?” Dumbledore said, looking surprised and disappointed. “Not even a little wicked? That doesn’t fit the pattern…”

Harry’s inner Slytherin screamed at the top of its mental lungs, \emph{\scream{Shut up you idiot he’ll take you away from them!}}

“No, no,” said Harry, lips frozen in a ghastly grimace, “I was just trying to spare your feelings, they’re actually very wicked…”

“They are?” Dumbledore leaned forward, gazing at him intently. “What do they do?”

\emph{Talk fast} “they, ah, I have to do dishes and wash problems and they don’t let me read a lot of books and—”

“Ah, good, that’s good to hear,” said Dumbledore, leaning back again. He smiled in a sad sort of way. “I apologise for \emph{that}, then. Now where was I? Ah, yes. I’m sorry to say, Harry, that I am responsible for virtually everything bad that has ever happened to you. I know that this will probably make you very angry.”

“Yes, I’m very angry!” said Harry. “Grrr!”

Harry’s Internal Critic promptly awarded him the All-Time Award for the Worst Acting in the History of Ever.

“And I just wanted you to know,” Dumbledore said, “I wanted to tell you as early as possible, in case something happens to one of us later, that I am truly, truly sorry. For everything that has already happened, and everything that will.”

Moisture glistened in the old wizard’s eyes.

“And I’m very angry!” said Harry. “So angry that I want to leave right now unless you’ve got anything else to say!”

\emph{Just \emph{go} before he sets you on fire!} shrieked Slytherin, Hufflepuff, and Gryffindor.

“I understand,” said Dumbledore. “One last thing then, Harry. You are \emph{not} to attempt the forbidden door on the third-floor corridor. There’s no possible way you could get through all the traps, and I wouldn’t want to hear that you’d been hurt trying. Why, I doubt that you could so much as open the first door, since it’s locked and you don’t know the spell \emph{Alohomora—}”

Harry spun around and bolted for the exit at top speed, the doorknob turned agreeably in his hand and then he was racing down the spiral stairs even as they turned, his feet almost stumbling over themselves, in just a moment he was at the bottom and the gargoyle was walking aside and Harry fired out of the stairwell like a cannonball.

\later

Harry Potter.

There must have been something about Harry Potter.

It was Thursday for everyone, after all, and yet this sort of thing didn’t seem to happen to anyone else.

It was 6:21\pm on Thursday afternoon when Harry Potter, firing out of the stairwell like a cannonball and accelerating at top speed, ran directly into Minerva McGonagall as she was turning a corner on her way to the Headmaster’s office.

Thankfully neither of them were much hurt. As had been explained to Harry a little earlier in the day—back when he was refusing to go anywhere near a broomstick again—Quidditch needed solid iron Bludgers just to stand a decent chance of injuring the players, since wizards tended to be a lot more resistant than Muggles to impacts.

Harry and Professor McGonagall did both end up on the floor, and the parchments she had been carrying went all over the corridor.

There was a terrible, terrible pause.

“Harry Potter,” breathed Professor McGonagall from where she was lying on the floor right next to Harry. Her voice rose to nearly a shriek. “\emph{What were you doing in the Headmaster’s office?}”

“Nothing!” squeaked Harry.

“\emph{Were you talking about the Defence Professor?}”

“No! Dumbledore called me up there and he gave me this big rock and said it was my father’s and I should carry it everywhere!”

There was another terrible pause.

“I see,” said Professor McGonagall, her voice a little calmer. She stood up, brushed herself off, and glared at the scattered parchments, which jumped into a neat stack and scurried back against the corridor wall as though to hide from her gaze. “My sympathies, Mr~Potter, and I apologise for doubting you.”

“Professor McGonagall,” Harry said. His voice was wavering. He pushed himself off the floor, stood, and looked up at her trustworthy, \emph{sane} face. “Professor McGonagall…”

“Yes, Mr~Potter?”

“Do you think I should?” Harry said in a small voice. “Carry my father’s rock everywhere?”

Professor McGonagall sighed. “That is between you and the Headmaster, I’m afraid.” She hesitated. “I will say that ignoring the Headmaster completely is almost never wise. I \emph{am} sorry to hear of your dilemma, Mr~Potter, and if there’s any way I \emph{can} help you with whatever you decide to do—”

“Um,” Harry said. “Actually I was thinking that once I know how, I could Transfigure the rock into a ring and wear it on my finger. If you could teach me how to sustain a Transfiguration—”

“It is good that you asked me first,” Professor McGonagall said, her face growing a bit stern. “If you lost control of the Transfiguration the reversal would cut off your finger and probably rip your hand in half. And at your age, even a ring is too large a target for you to sustain indefinitely without it being a serious drain on your magic. But I can have a ring forged for you with a setting for a jewel, a \emph{small} jewel, in contact with your skin, and you can practise sustaining a safe subject, like a marshmallow. When you have kept it up successfully, even in your sleep, for a full month, I will allow you to Transfigure, ah, your father’s rock…” Professor McGonagall’s voice trailed off. “Did the Headmaster \emph{really—}”

“Yes. Ah…um…”

Professor McGonagall sighed. “That’s a bit strange even for him.” She stooped and picked up the stack of parchments. “I’m sorry about this, Mr~Potter. I apologise again for mistrusting you. But now it’s my own turn to see the Headmaster.”

“Ah…good luck, I guess. Er…”

“Thank you, Mr~Potter.”

“Um…”

Professor McGonagall walked over to the gargoyle, inaudibly spoke the password, and stepped through into the revolving spiral stairs. She began to rise out of sight, and the gargoyle started back—

“\emph{Professor McGonagall the Headmaster set fire to a chicken!}”

“He \emph{wha—}”

%  LocalWords:  hursday Remembrall Goyle’s Hufflepuffle Slytherslime Hah
%  LocalWords:  Remembralls Libatius Ehhh reaallly blorple wibblers Ahhh
%  LocalWords:  Umm Grrr wha
