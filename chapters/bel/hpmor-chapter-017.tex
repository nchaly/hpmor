\chapter{Вызначэнне гіпотэзы}

\begin{chapterOpeningQuote}
Ты пачынаеш бачыць патэрн, пачынаеш трапляць у рытм сусвета.
\end{chapterOpeningQuote}

\lettrine{Ч}{ацьвер}.\newline
Калі хочаце дакладна, то 7:24 раніцы ў чацьвер.

Гары сядзеў ў сваім ложку, падручнік амаль вываліўся з ягоных разняволеных рук.

Толькі што яму да галавы прыйшла ідэя \emph{сапраўды цудоўнага} эксперыменту.

Гэта патрабавала адкласці сняданак яшчэ на гадзіну, але на такі выпадак 
у яго былі мюслі. Не, гэтая ідэя павінна быць праверана неадкладна, без зацяжак,
зараз.

Гары адклаў кнігу, выпрыгнуў з ложка, падбяжаў да свайго куфара, 
адчыніў люк, збег па лесвіцы, і пачаў перасоўваць каробкі. (Яму варта б 
распакаваць і размеркаваць кнігі па шафах, але ў яго было 
няскончанае спаборніцтва з Герміёнай, у якім ён адставаў, таму на гэта зараз 
не было часу.)

Знайшоўшы кнігу, якую ён шукаў, ён пабег назад.

Астатнія хлопцы збіраліся на сняданак у Галоўную Залу.

--- Прабач, можаш мне дапамагчы? --- сказаў Гары. Ён прагартаў старонкі,
знайшоў месца, дзе былі надрукаваныя першыя дзесяць тысяч простых лікаў,
і сунуў кнігу ў рукі Энтані Голдштэйна. --- Выбяры два трохзначных ліка з гэтага спісу,
і не кажы мне, якія. Перамнож іх, і скажы мне вынік. О, і калі ласка, можаш 
праверыць здабытак два разы? Будзь добры, пераканайся, што здабытак правільны, бо 
я не ўпэўнены, што здарыцца са мной і сусветам, калі ты дзесьці памылішся з падлікам.

Той факт, што Энтані не здівіўся, або не спытаў, што такога можа здарыцца з сусветам,
шмат казала пра тое, якім было жыццё ў гэтай спальне на працягу  некалькіх апошніх дзён.

Энтані без лішніх слоў узяў кнігу, дастаў пергамент і пяро. Гары адвярнуўся і 
заплюшчыў вочы, каб дакладна нічога не ўбачыць, прытанцоўваючы на месцы ад нецярпення.
У адной руцэ у яго быў блакнот, у другой --- механічны аловак, і ён быў гатовы пісаць.

--- Гатова, --- сказаў Энтані, --- сто восемдзесят адна тысяча чатырыста дваццаць дзевяць.

Гары запісаў $181429$. Ён прачытаў запісанае, і Энтані пацвердзіў.

Потым Гары збег у склеп свайго куфара, паглядзеў на гадзіннік (які паказваў 4:28,
што значыла 7:28) і заплюшчыў вочы. 

Прыкладна праз трыццаць секунд Гары пачуў крокі ў склепе, потым на лесвіцы, 
а потым --- як зачыніўся люк. (Гары не хваляваўся, што задахнецца. Заклён аўтаматычнага
пасвяжэння паветра быў звычайнай часткай каштоўных куфараў. Ну ці не была магія 
цудоўнай? Да таго ж, не трэба думаць пра аплату электрычнасці.)

І калі Гары расплюшчыў вочы, ён убачыў тое, што і чакаў: складзеную паперку на 
падлозе, падарунак ад сябе-будучага.

Давайце дагаварымся называць гэтую паперку "Папера №2".

Гары, не рухаючыся з месца, вырваў старонку са свайго блакнота. 

Гэта --- "Папера №1". Яна была, вядома, тым жа самым кавалкам паперы, што ляжаў 
на падлозе. Калі прыгледзецца, было бачна, што адарваны край быў таго ж узору.

Гары мыслена паўтарыў алгарытм, якому ён будзе следваць.

Калі ён разгарне Паперу №2, і яна будзе пустая, ён напіша “$101\times101$” на Паперы №1,
потым пазаймаецца ўрокамі адну гадзіну, прыгне назад у часе, бросіць на падлогу 
паперку (якая ў гэты момант стане Паперай №2), і потым пакіне свой куфар, каб 
далучыцца да аднакласнікаў на сняданку. 

Калі ён адчыніць Паперу №2, і на ёй будуць напісаны два ліка, Гары перамножыць 
іх. Калі іх здабытак будзе $181429$, Гары напіша гэтыя два ліка на Паперы №1 і 
адправіць яе ў мінулае.

Інакш, Гары дадасць двойку да правага ліку, калі той не больш за 997, у гэтым выпадку
ён павялічыць левы лік на два, і скіне правы лік на 101, пасля чаго ён напіша
гэтая два новых ліка на Паперы №1. 

І калі на Паперы №2 будзе напісана $997\times997$, Гары пакіне Паперу №1 пустой.

А гэта значыла, што адзінай \emph{стабільнай} часовай пятлёй будзе тая, дзе 
Папера №2 будзе мець два простых множніка ліка $181429$.

Калі спрацуе, Гары зможа карыстаць гэты спосаб для адказу на любую задачу, 
які было лёгка праверыць, але цяжка вырашыць.
Ён не проста дакажа, што адказ на адно з асноўных пытанняў 
інфарматыкі $\mbox{P}=\mbox{NP}?$ быў пазітыўным, калі ты меў часаварот.
Гэты прыём адчыняў шмат іншых магчымасцяў. Гары мог карыстаць яго для падбору 
камбінацый для кодавых замкоў, або паролей любой даўжыні. 
Магчыма, нават знайсці ўваход у Таемную Залу Слізэрына, калі ён зможа прыдумаць
сістэматычны спосаб пералічыць ўсе лакацыі замка Хогвартс.
Гэта будзе чыт-код тысячагоддзя, нават па мерках чытынгу самога Гары.

Гары крыху трасучыміся рукамі падняў Паперу №2 і разгарнуў яе.

На Паперы №2 крыху трасучымся почыркам было напісана:

\medskip

{\hpFontFasthand{}Не будзі ліха, прыдурак!}

\medskip

Гары сваёй трасучайся рукой напісаў  {\hpFontFasthand{}Не будзі ліха, прыдурак!} на
Паперы №1, акуратна яе склаў, і вырашыў больш не праводзіць сапраўды цудоўныя
эксперыменты з Часам, прынамсі пакуль яму не споўніцца пятнаццаць.

Па меркаванню Гары, гэта быў самы страшны эксперыментальны вынік за 
ўсю гісторыю навукі.

Наступую гадзіну Гары неяк складана было сканцэнтравацца на чытанні сваіх падручнікаў.

Так пачаўся Гарын чацьвер.


\later

Чацьвер.

Калі хочаце дакладна, то 15:32 ў чацьвер, пасля абеду.

Гары і ўсе астатнія хлопцы-першакурснікі разам з мадам Хуч стаялі на зялёным
палі каля кіпы мёцел з запасаў Хогвартс. Урокі лятання для дзяўчын праводзіліся
паасобку. Па нейкай незразумелай прычыне яны не хацелі вучыцца лятаць у прысутнасці
хлопцаў.

Гары ўвесь дзень крыху каўбасіла. Ён проста не мог перастаць гадаць, якім чынам
\emph{менавіта тая} стабільная часовая пятля была абраная з даволі вялікай --- у рэтраспектыве ---
прасторы магчымасцей. 

Ну і да таго ж: \emph{мётлы?} \emph{Сур'ёзна?} Ён быў павінен лятаць на фактычна 
адрэзку прамой? Ці не было гэта, шчыра кажучы, самая нестабільнай формай, 
якую толькі можна знайсці? Хто мог абраць менавіта такі дызайн для лятаючай прылады
з неверагоднай колькасьці варыянтаў. Спачатку Гары спадзяваўся, што гэта 
проста фігура мовы, але не, тое былі звычайныя мётлы для падмятання лісця. Камусьці 
проста падабалася ідэя з мятлой, і ён проста адкінуў усі іншыя? Відаць, так. 
Ну проста не магло здарыцца, каб дызайн для хатняй прылады і лятальнага апарата 
супадалі бы хоць на некалькі адсоткаў, калі распрацоўваць іх з нуля.

Дзень быў ясны, блакітнае неба было высока над галавой, а яскравае сонца 
так і лезла сляпіцай у вочы, каб перашкаджаць табе ў палёце. Поле выглядала 
сухім, пахла амаль прасмажаным, і адчувалася вельці цвёрдым пад Гарынымі 
ступакамі.

Гары працягваў нагадваць сабе, што  навучальная праграма была разлічана так, што нават самы
імкнуўшыся да нуля адзіннацацтлетка можа яе засвоіць, і што яно не павінна 
было быць насколькі складаным.

--- Працягніце правую руку над мятлой, або левую, калі вы ляўша, --- сказала 
мадам Хуч, --- і скажыце \shout{ўверх!}


--- \shout{Уверх!} --- крыкнулі ўсе.

Мятла ахвотна прыгнула Гары ў руку.

Што ўпершыню паставіла яго на першае месца ў класе. Імаверна, сказаць \shout{Уверх!} 
было значна складаней, чым падавалася, бо большасць мётлаў ці проста ляжалі, ці 
намагаліся адсунуцца ад сваіх будучых ездакоў.

(Канешне, Гары быў гатовы біцца аб заклад, што ў Герміёны атрымалася не горш 
за яго. Калі ён мог вывучыць з першай спробы нешта, што здзівіць Герміёну, 
і гэтае нешта апынецца лятаннем на мятле... ён проста здзейсніць самагубства.)

Патрабіваўся нейкі час, каб усе здолелі узяць мётлы ў рукі. Мадам Хуч паказала,
як садзіцца на мятлу, і потым прайшлася па радах, паправляючы і каментуючы.
Відавочна, нават тыя, каму дазвалялася карыстацца мятлой дома, не навучыліся 
рабіць гэта правільна.

Агледзеўшы ўсіх, яна вярнулася наперад і кіўнула.

--- Так, калі я свісну, усім моцна падпрыгнуць.

Гары з цяжкасцю зглынуў, спрабуючы заглушыць пачуццё млоснаснці.

--- Мётлы трымаць роўна. Падняцца на два-тры футы. Потым крыху нахіліцца наперад, 
і апусціцца на зямлю. Такім чынам, тры, два...

Адна з мёцел стрэльнула ў паветра, суправаджаемая крыкамі хлопца, які на ёй сядзеў.
Крыкі былі ад жаху, а не задавальнення. З дзікай хуткасцю ён круціўся, падымаючыся,
і Гары толькі на імгненне ўбачыў ягоны збялелы твар...

Быццам у замедленай здымке Гары злазіў са сваёй мятлы, адначасова дастаючы 
палачку, хаця ён не зусім разумеў, што плануе рабіць. У іх было ўсяго два урока 
Чаравання, і на мінулым яны праходзілі Заклён Левітацыі, дзе ў Гары атрымлівалася
каставаць яго паспяхова толькі адзін раз з трох спроб, і дакладна, ён не быў
здольны левітавать цэлага чалавека... 

\emph{Калі ва мне ёсць таямнічая сіла, калі ласка, хай яна пакажа сябе \emph{зараз}!}

--- Вярнісь зараз жа! --- крыкнула мадам Хуч. (З боку \emph{настаўніка лятання},
гэта была самая недарэчная парада, якую толькі можна было ўявіць, каб дапамагчы
вучню справіцца са звар'яцелай мятлой, --- і цалкам аўтаномная частка мозгу Гары 
дадала мадам Хуч да ягонага ліста дурняў)

І потым хлоца скінула з мятлы.

Падавалася, што ён падае вельмі павольна, прынамсі спачатку. 

--- \emph{Wingardium Leviosa!} --- крыкнуў Гары.

Зачараванне не спрацавала. Ён адчуў гэта, нават не скончыўшы інкантацыю.

Потым быў громкі "чмяк" і хрыбусценне. Хлопец ляжаў на траве тварам уніз, быццам
кіпа адзення.

Гары паклаў палачку ў карман, і з усіх ног пабяжаў да яго, прыбяжаўшы адначасава 
з мадам Хуч, ён адразу засунуў руку ў махляскін, намагаючыся ўспомніць,
--- о божа як жа яно звалася ды няважна проста скажы "аптэчка", --- і яна прыгнула 
яму ў руку, і...

--- Пералом прыдалоння, --- сказала мадам Хуч. --- Спакойна, хлопча, гэта 
проста пералом!

Яго розум быццам спатыкнуўся, і Гары выйшаў з Ражыма Панікі.

"Emergency Healing Pack Plus" ляжаў перад ім расчынены, у руцэ ў Гары быў 
шпрыц вадкага агню, які павінен быў падтрымліваць аксігенацыю мозга параненага, 
калі той зламаў бы шыю.

--- А-а-а... --- сказаў Гары не вельмі роўным голасам. Яго сэрца бухала так,
што ён амаль не адчуваў, як хапае ротам паветра. --- Пералом... добра...
тады, Гіпсавы Ручнік?

--- Гэта толькі для надзвычайных выпадкаў, --- адрэзала мадам Хуч. --- Прыбяры
свае лекі, ён нармальна, --- яна нахілілася над Нэвілам, падаючы яму руку. --- 
Давай, хлопча, уставай, усё добра, давай-ка...

--- Вы жа не плануеца зноў пасадзіць яго на мятлу? --- з жахам спытаў Гары.

Мадам Хуч кінула ў адказ абураны позірк.

--- Ну вядома ж, не! 

Яна пацягнула за непашкоджаную руку кіпу адзнення, і паставіла яе роўна, --- тут 
Гары з нейкім шокам зразумеў, што гэта \emph{зноў} быў Нэвіл, --- да што ж 
з табой такое? --- і яна павярнулася да ўсіх астатніх вучняў.

--- Нікому нават не рухацца, пакуль я суправаджаю гэтага хлопца ў бальнічнае крыло!
Мётлы нікому не трогаць, інакш выляціце з Хогвартс хутчэй, чым паспееце сказаць 
"квіддзіч". Хадзем, даражэнькі.

І яна ўвяла Нэвіла, які трымаў прыдалонне здаровай рукой, і намагаўся кантраляваць 
свае  ўсхліпы.

Калі яны адышлі дастаткова, адзін са слізерынаў зарагатаў.

За ім --- астатнія.

Гары абярнуўся паглядзець на іх. Добры час, каб запомніць некаторыя твары.

А ўбачыў, як да яго набліжаецца Драко, суправаджаемы містэрам Крэбам і 
містэрам Гойлам. Містэр Крэб не ўсміхаўся. Чаго рашуча нельга было 
сказаць пра містэра Гойла. Сам Драко дэманстраваў вельмі стрыманы твар, які 
часам паторгваўся. Відавочна, падумаў Гары, яму таксама было весела, але ён не бачыў 
палітычнай выгады ў тым, каб пасмяяцца зараз, і яму лепей зрабіць гэта пазней, у 
падзямеллі Слізэрына. 

--- Ну, Потэр... --- сказаў Драко дастаткова ціха, каб астатнія не чулі. Твар яго 
быў усё яшчэ стрыманы, і ўсё яшчэ паторгваўся. --- Проста хацеў сказаць, што 
калі ты плануеш скарыстацца надзвычайнай сітуацыяй для дэманстрацыі свайго 
лідэрства, варта рабіць выгляд, што ў цябе ўсё пад кантролем, замест, скажам,
кідання ў поўную паніку, --- містэр Гойл хіхікнуў, і Драко кінуў у яго гнеўны позірк. 
--- Але, магчыма, некалькі балаў ты заработал. Табе патрэбная дапамога, каб 
запакаваць свае багацце?

Гары павярнуўся да аптэчкі, што яшчэ і дазволіла яму схаваць твар ад Драко. 

--- Усё нармальна, --- сказаў ён, сабраў усе рэчы вакол, паклаў на месца
шпрыц, зашчоўкнуў замкі, і падняўся.

Эрні Макмілан падышоў да яго, калі Гары амаль скончыў скормлівать аптэчку 
махляскіну.

--- Ад імя Хафлпафа, дзякуй, Гары Потэр, --- сказаў ён фармальна. 
--- І думка, і спроба былі добрыя.

--- Сапраўды, добрая думка, --- працягнуў Драко. --- Чаму ніхто з хафлпафаў не 
дастаў палачкі? Мабыць, калі б вы ўсе не аслупянелі, і далучыліся да Потэра, 
дак і паймалі б яго? Я быў пад уражаннем, што хафлпафы павінны трымацца гуртом?

Эрні выглядаў раздзіраемы злосцю і сорамам. 

--- Мы не падумалі...

--- А, --- сказаў Драко, --- не \emph{падумалі}. Вось чаму лепей мець аднаго 
сябра-рэйвенкло, чым цэлы натоўп хафлпафаў.

Вось гад, падумаў Гары, і як зараз з гэтага выбірацца.

--- Ты не дапамагаеш, --- сказаў ён ціха. Спадзяючыся, што Драко здолее перакласці 
гэта як \emph{ты перашкаджаеш маім планам, таму, калі ласка, забі зяпу.}

--- Хэй, а гэта што такое? --- сказаў містэр Гойл. Ён нахіліўся і падняў з 
травы нешта памерам са сліву, шкяны шарык, у якім завіхалася белая дымка.

Эрні міргнуў.

--- Напамінальнік Нэвіла!

--- Што за напамінальнік? --- спытаў Гары.

--- Ён чырванее, калі ты нешта забыў, --- сказаў Эрні. --- 
Але ён не кажа, што менавіта ты забыў. Калі ласка, аддай яго мне, я вярну яго
пазней Нэвілу, --- Эрні працягнуў руку.

Раптоўная усмешка прабяжала па твары містэра Гойла. Ён павярнуўся і пабяжаў прэч.

Пэўную секунду Эрні ад нечаканасці здранцавеў, потым крыкнуў "Гэй!", і 
пабяжаў за ім.

Містэр Гойл адным плаўным рухам схапіў мятлу, ускочыў на яе вярхом, і ўзняўся 
ў паветра.

У Гары сківіца адвалілася. Ці не казала мадам Хуч нешта пра адлічэнне?

--- \emph{Што за ідыёт!} --- прашыпеў Драко. Ён набраў паветра, каб крыкнуць...

--- \emph{Гэй!} --- праравеў Эрні, --- Гэта Нэвіла! \emph{Аддай!}

Слізэрыны пачалі ўлюлюкаць і вітаць містэра Гойла.

Рот Драко рэзка зачыніўся. Гары паспеў заўважыць на яго твары выраз нерашучасці.

--- Драко, --- сказаў Гары ціха, --- калі ты не загадаеш гэтаму ідыёту прызямліцца,
то калі вернецца настаўніца...

--- \emph{А ты адбяры, хафл-пафл!} --- пракрычаў містэр Гойл пад гучныя ўхваляванні 
з боку слізерынаў.

--- \emph{Не магу!} --- прасычэў Драко. --- Усе падумаюць, што я \emph{слабак}!

--- А калі містэра Гойла адлічуць, --- прасычэў Гары ў адказ, --- твой бацька 
падумае, што ты \emph{дэбіл}!

У Драко ажно твар перакасіла.

У гэты момант...

--- Ну ты, \emph{слізня-дрын}, --- крыкнуў Эрні, --- падобна, ніхто не казаў табе,
што хафлпафы трымаюцца гуртом? \emph{Хлопцы, палачкі!}

І раптам шмат палачак былі накіраваны на містэра Гойла.

І праз дзве секунды...

--- \emph{Слізэрын, дастаць палачкі!} --- сказалі пяць галасоў з групы слізэрынаў.

І шмат палачак былі накіраваны ў бок халфпафаў.

Яшчэ праз секунду...

--- \emph{Грыфіндор!}

--- \emph{Зрабі нешта, Потэр!} --- прашаптаў Драко. --- \emph{Я не магу выйсці і 
спыніць гэта, прыйдзецца табе давай думай я буду табе абавязаны ты ж наўродзе
павінен быць супер-разумны?}

Прыкладна праз пяць з паловай секунд, зразумеў Гары, нехта скастуе Шчучынскі
Штурханец, і калі пыл уляжацца, і настаўнікі скончаць адлічваць злачынцаў, 
адзінымі хлопцамі на першым курсе застануцца рэйвенкло. 

--- \shout{Рэйвенкло! Да бою!} --- завапіў Майкл Корнэр, які, верагодна, адчуў 
заўзятасць далучыцца да катастрофы.

--- \scream{Грэгары Гойл!} --- крыкнуў Гары. --- \shout{Я выклікаю цябе на двубой
дзеля валодання напамінальнікам Нэвіла!}

Раптам усе спыніліся.

--- Што, праўда? --- працягнуў сваім самым нізкім голасам Драко. --- Гучыць цікава.
Што за двубой, Потэр?

Э-э-эммм...

"Двубой" быў максімумам таго, што здолела выдаць Гарына фантазія. А што канатрэтна? 
Ён не мог сказаць "шахматы", бо калі Драко такое прыме, самі слізэрыны будуць 
глядзець на Драко коса, і не мог сказаць "армрэслінг", бо містэр Гойл яму руку адарве...

--- Што на конт такога? --- сказаў Гары гучна. --- Грэгары Гойл і я павінны стаць
адзін насупраць аднаго, і больш нікому нельга да нас набліжацца. Мы не карыстаем свае 
палачкі або любыя іншыя прылады. Мы не сыходзім з нашых месцаў. Калі я змагу дакрануцца
да напамінальніка Нэвіла, тады Грэгары Гойл адмаўляецца ад любых дамаганняў на 
напамінальнік, які ён трымае ў руцэ, і аддае яго мне.

Паследвала яшчэ адна пауза, пакуль людзі абменьваліся разгубленымі позіркамі.

--- Ха, Потэр, --- сказаў таксама гучна Драко. --- Я хачу паглядзець, я ты 
здолееш \emph{такое}! Містэр Гойл прыймае выклік!

--- Тады, пачынаем! --- сказаў Гары.

--- Потэр,  \emph{што ты задумаў?} --- прашаптаў Драко, неймаверным чынам
не рухаючы вуснамі.

Гары не ведаў, як адказаць, не рухаючы вуснамі.

Людзі паволі хавалі свае палачкі, і містэр Гойл зграбна прызямліўся, 
выглядаючы даволі разгублена. Некаторы хафлпафы ўзрушылі да яго, але 
Гары кінуў ім адчайна ўмольны позірк, і яны адступілі.

Гары падышоў да містэра Гойла, і спыніўся ў некалькіх кроках ад яго, каб яны не 
маглі дацягнуцца адзін да аднаго.

Павольна, дэманстратыўна Гары схаваў сваю палачку ў кішэнь.

Астатнія адышлі на некалькі крокаў.

Гары зглынуў. Ён толькі вельмі грубымі мазкамі ўяўляў сабе, што ён \emph{хоча}
зрабіць, але яно павінна было адбыцца такім чынам, каб ніхто не здагадаўся, 
\emph{што} і \emph{хто} гэта зрабіў...

--- Ну, добра, --- сказаў уголас Гары. --- А зараз... --- ён глыбока ўдыхнуў і
падняў правую руку, адзінец упіраўся ў сярэднік. Нехта шумна ўздыхнуў ---
звесткі пра пірагі разышліся хутка. --- Я заклікаю на дапамогу вар'яцтва 
замка Хогвартс! \emph{Happy happy boom boom swamp swamp swamp!} --- і ён цокнуў пальцамі.

Шмат хто ўздрыгануўся...

...і нічога не адбылося.

Гары чакаў. Цішыня зацягнулася, гатовая выбухнуць у любы момант...

--- М-м-м... --- сказаў нехта, --- І гэта ўсе?

Гары зірнуў на хлопца, які гэта сказаў. 

--- Паглядзі перад сабой, бачыш пусты кусок зямлі, дзе не расце трава?

--- М-м, бачу, --- адказаў хлопец-грыфіндор (Дзін нехта?).

--- Капай.

Тут шмат хто паглядзеў на Гары дзіўна.

--- Эм, навошта? --- сказаў Дзін-Нехта.

--- Проста капай, --- сказаў Тэры Бут стомлена. --- Павер мне, ніяма сэнсу пытаць.

Дзін-Нехта апусціўся на калені, і пачаў адкідваць рукамі зямлю.

Праз хвіліну-другую, ён падняўся. 

--- Тут нічога няма.

Хоба. Гары ўжо пачаў быў планаваць, як прыгне ў мінулае, і закапае на гэтым 
месцы мапу скарбаў, якая прывядзе іх да іншай мапы, а ўжо тая --- да 
напамінальніка Нэвіла...

І потым Гары зразумеў, што існаваў прасцейшы шлях, які дарэчы 
не рызыкаваў сакрэтам часаварота.

--- Дзякуй, Дзін! --- сказаў Гары гучна. --- Эрні, будзь ласка, агледзь месца, 
дзе упаў Нэвіл, і скажы, ці не бачыш ты там напамінальніка?

Людзі выглядалі ўсё больш збянтэжанымі.

--- Проста зрабі гэта, --- сказаў Тэры Бут. --- Ён не спыніцца, пакуль нешта не 
спрацуе, і самае жудаснае --- тое, што...

--- \emph{Мерлін!} --- ахнуў Эрні. У руцэ ён трымаў напамінальнік Нэвіла. --- Ён быў 
тут! Роўна дзе упаў Нэвіл!

--- \emph{Што?} --- крыкнуў містэр Гойл. Ён паглядзеў уніз і ўбачыў...

...што ён усё яшчэ трымае напамінальнік Нэвіла ў сваёй руцэ.

Павіcла даволі доўгая пауза. 

--- Э-э, гэта ж не немагчыма, так? -- сказаў Дзін-Нехта.

--- Гэта дзірка ў сцэнары, --- сказаў Гары. --- Я проста на імгненне адцягнуў 
увагу Сусвету, і ён забыў, што Гойл ужо падняў напамінальнік.

--- Што? Не! Я пра тое, што гэта \emph{абсалютна} немагчыма...

--- Прабачце, ці не мы гэта збіраліся толькі што лятаць на мётлах?
Так што лепей маўчы. У любым выпадку, калі я дакрануся да яго, я перамог, і Грэгары Гойл 
павінен адмовіццца ад любых дамаганняў на напамінальнік, які ён трымае ў руцэ, 
і павінен аддаць яго мне. Умовы былі такія, памятаеце? --- Гары працягнуў руку і 
памахаў Эрні. --- Ну і ніхто не павінен набліжацца, так што проста кінь мне яго,
акей?

--- Стаяць! --- пачулася раптоўнае ад слізэрына --- Блэйз Забіні, Гары дакладна не 
забудзе такое імя. --- Скуль мы ведаем, што гэта менавіта напамінальнік 
Лонгботама? Ты мог падкінуць туды нейкі \emph{іншы} напамінальнік...  

--- Слізэрынства моц ў гэтым малым адчуваю я, --- сказаў Гары, усміхаючыся. ---
Але даю слова, што Эрні трымае сапраўдны напамінальнік Нэвіла. На конт таго,
што трымае Грэгары Гойл, --- ніякіх каментароў.

Забіні павярнуўся да Драко.

--- \emph{Малфой!} Ты што, дазволіш яму гэтае махлярства?..

--- Гэй, забі зябу, --- прагрымкаў містэр Крэб, які стаяў ззаду Драко. --- 
Ты хто такі, каб указваць містэру Малфою?

\emph{Добры} міньён.

--- У мяне была змова з Драко, нашчадкам Вялікага і Старажытнейшага Роду Малфоеў, ---
сказаў Гары. --- Не з табою, Забіні. Я зрабіў тое, што ён вельмі хацеў паглядзець,
і судзіць, хто перамог, мы пакінем містэру Малфою, --- Гары нахіліў галаву
ў бок Драко і крыху ўзняў бровы. Драко такім чынам мог зберагчы свае пазіцыі.

Пауза.

--- Ты даеш слова, што тое --- сапраўды напамінальнік Нэвіла? --- спытаў Драко.

--- Так, --- сказаў Гары, --- сапраўды гэта яго арыгінальны напамінальнік, які
павінен вярнуцца да Нэвіла. А той, што трымае Грэгары Гойл адыходзіць да мяне.

Драко рашуча кіўнуў. 

--- Тады я веру слову Вялікага Роду Потэраў, няважна наколькі бязглузда гэта 
выглядае. А Вялікі і Старажытнейшы Род Малфоеў таксама ўмее трымаць слова.
Містэр Гойл, аддайце гэта містэру Потэру...

--- Хэй, --- сказаў Забіні, --- ён яшчэ не выйграў, бо ён павінны дакрануцца...

--- Лаві, Гары! --- крыкнуў Эрні, і шпурнуў напамінальнік.

Гары лёгка злавіў напамінальнік --- у яго заўсёды былі добрыя рэфлексы ў такіх 
гульнях. 

--- Ну, --- сказаў ён, --- я перамог...

Але ягоны голас неяк павольна сцішыўся, як і ўсе астатнія галасы навокал.

Напамінальнік у яго руцэ гарэў яскравым чырвоным святлом, быццам мініятурнае сонейка,
адкідваючы цені хлопцаў на траве, нягледзячы на светлы дзень.


\later

Чацьвер.

Калі хочаце дакладна, то 17:09 ў чацьвер, у кабінеце прафесара МакГонагал,
пасля заняткаў на мётлалятанні. (Плюс дадатковая гадзіна для Гары, усунутая ў гэты 
прамежак.)

Прафесар МакГонагал сядзела на сваім стуле, Гары --- у крэсле перад яе сталом.

--- Прафесар, --- сказаў Гары напружана, --- слізэрыны нацэлілі свае палачкі на 
хафлпафаў, грыфіндоры --- на слізэрынаў, і калі нейкі \emph{ідыёт} крыкнуў 
"Рэйвенкло, да бою!", у мяне заставалася кшталту пяць секунд, каб утрымаць 
сітуацыю ад жудаснага выбуху! Гэта было адзінае, што я паспеў прыдумаць!

Выраз МакГонагал быў нахмураны і ўззлаваны.

--- \emph{Не варта карыстаць часаварот такім чынам, містэр Потэр!} Або канцэпыця 
сакрэтнасці --- нешта па-за межамі вашага разумення?

---  Яны не ведаюць, \emph{як} я гэта зрабіў! Усе ўпэўнены, што я магу рабіць 
дзіўныя рэчы, цокаючы пальцамі! Я ўжо рабіў нават такое, што і з часаваротам не
магчыма, і зраблю \emph{яшчэ} такога, што гэты выпадак ніхто і не успомніць.
У мяне не было ніякага выбару, прафесар! 

--- Выбар быў! --- адрэзала МакГонагал. --- Усё, што вам было трэба --- вярнуць 
гэтага \emph{ананімнага слізэрына} на зямлю, і каб людзі схавалі палачкі! 
Вы маглі прапанаваць партыю ў выбуховы цмок, але не, абавязкова трэба 
выпендрывацца на шырокую нагу, і безразважна рызыкаваць часаваротам!

--- Я паспеў прыдумаць толькі гэта, і дарэчы, я нават не ведаю, што такое гэты 
ваш выбуховы цмок, і я думаў пра шахматы, але яны бы не згадзіліся, і калі бы я
абраў арм-рэслінг, то адразу бы прайграў!

--- \emph{Вам проста трэба было абраць арм-рэслінг!}

Гары міргнуў.

--- Але... але я бы \emph{прайграў!} --- і тут ён зразумеў.

МакГонагал выглядала \emph{вельмі} фрустрыраванай.

--- Прабачце, прафесар, --- сказаў Гары ціха. --- Я праўда пра тое не падумаў,
і вы правы, мне трэба было прайграць... і гэта быў бы цудоўны ход...
але мне гэта проста не прыйшло да галавы, і ведаеце...

Гары заціх, так і не прыдумаўшы, як скончыць фразу. Раптоўна перад ім з'явіўся
цэлы шэраг варыянтаў. Ён мог спытаць меркаванне Драко, ён мог параіцца з 
хлопцамі... яго ідэя скарыстаць часаварот спраўды была безразважным рызыкам. 
Чаму з усёй агромнай прасторы варыянтаў ён выбраў менавіта гэты?

...таму што гэта быў спосаб \emph{перамагчы}. Завалодаць нікчэмнай цацкай, якую
настаўнікі ў любым выпадку вярнулі бы Нэвілу.

Прага перамогі. Вось што яму перашкодзіла.

--- Мне вельмі жаль, --- сказаў Гары, --- за мой гонар і за маю дурасць.

Прафесар МакГонагал пацёрла лоб. Яе гнеў крыху супакоіўся, але голас быў 
вельмі жорсткі:

--- Яшчэ адзін такі выбрык, містэр Потэр, і вы будзеце пазбаўлены гэтай прывілеі.
Ці зразумела вам гэта?

--- Так, --- сказаў Гары. --- Вельмі зразумела, і вельмі шкада.

--- Тады, містэр Потэр, дазваляю вам пакінуць часаварот. Пакуль што. 
І ўлічваючы памер бяды, якую вы, праўду кажучы, перадухілілі, 
я не буду здымаць балы з Рэйвенкло.

\emph{Угу, улічваючы, што вы не зможаце растлумачыць, за што знялі.} Але Гары 
быў не настолькі дурны, каб сказаць гэта ўголас.

--- Але важней, чаму напамінальнік адразу пачырванеў у маіх руках? --- спытаў Гары. 
--- Ці значыць гэта, што на мяне прымянялі Забывальны заклён?

--- Мянэ гэта таксама бянтэжыць, --- сказала МакГонагал павольна. --- Калі гэта было б 
так проста, усе суды даўно бы карысталі напамінальнікі. Я падумаю аб гэтым, містэр
Потэр, --- яна ўздыхнула. --- Вы свабодны.

Гары пачаў быў падымацца, але спыніўся.

--- Эмм, прабачце, я хацеў яшчэ нешта вам расказаць...

Уздрыг быў аваль незаўважны.

--- Што, містэр Потэр?

--- Гэта наконт прафесара Квірэла...

--- Я ўпэўнена, што яно не вартае ўвагі, --- сказала прафесар МакГонагал з нечаканым 
паспехам. --- Ці вы не чулі, як Майстра загадваў студэнтам не назаляць нам сваімі 
дакучлівымі скаргамі аб прафесары Абароны?

Гары разгубіўся.

--- Але \emph{гэта} можа быць важна. Учора ў мяне з'явілася раптоўнае адчуванне бяды,
калі...

--- Містэр Потэр! У мяне таксама адчуванне бяды! І мае адчуванне бяды падказвае,
што \emph{лепей вам не сканчваць гэты сказ!}

Рот Гары застыў, расчынены. Як і прапанавала МакГонагал, Гары не змог дагаварыць.

--- Містэр Потэр, --- сказала прафесар МакГонагал, --- калі наступным разам вы высветліце
нешта цікавае пра прафесара Квірэла, смела трымайце гэта пры сябе, і, калі ласка,
не дзяліцеся са мною або з кімсьці яшчэ. Думаю, вы ўжо патрацілі досыць майго 
каштоўнага часу...

--- \emph{Гэта на вас не падобна!} --- выбухнуў Гары. --- Выбачайце, але вы паводзіце сябе 
\emph{неверагодна} безадказна! Ходзяць чуткі, што на пасаду прафесара Абароны накладзены
нейкі праклён, і калі вы ўжо ведаеце, што нешта можна пайсці не так, вы павіннны ўжо даўно
навастрыць вушы...

--- Пайсці \emph{не так}, містэр Потэр? \emph{Толькі не гэтым разам.} --- Твар 
МакГонагал быў безвыразны. --- Пасля таго як прафесара Блэйка застукалі ў кладоўцы 
не менш чым з трымя студэнткамі, а за год да таго прафесар Самерс настолькі няздоліўся ў 
навучанні, што яго студэнты былі ўпэўнены, што богарт --- гэта такі від мэблі;
калі ўжо неверагодна кампетэнтны прафесар Квірэл не справіцца, гэта будзе проста 
\emph{катастрофа}. Не кажучы, што большасць студэнтаў праваляць абарону 
ў гэтым годзе на сычах і крумкачах\footnote{{} Экзамены чараўнікоў: СЧ --- Сярэдні
Чараўніцкі экзамен пасля пятага года навучання;
КРУМ --- Катавальна РУплівы Магічны экзамен, выпускны. У простай мове завуцца адпаведна
сычы і крумкачы.}.

--- Разумею... --- сказаў Гары павольна, спрабуючы прыняць гэта ўсё. --- Кажучы 
проста, што бы ні адбывалася з прафесарам Квірэлам, вы не жадаеце чуць пра тое да
канца навучальнаго года. А ўлічваючы, што зараз яшчэ кастрычнік, то ён можа забіць
прэмьер-міністра на ток-шоў у прамым эфіры, і яму за гэта нічога не будзе, па-вашаму.

Прафесар МакГонагал глядзела на яго неміргаючым позіркам.

--- Я ўпэўнена, што пацверджанне гэтай заявы ад мяне пачуць было і будзе немагчыма,
містэр Потэр. У Хогвартс мы імкнемся праактыўна рэагаваць на \emph{любыя} перашкоды 
адукацыйным дасягненням нашых студэнтаў.

\emph{Напрыклад, на рэйвенкло-першаркусніка, які не мае трымаць рот зачыненым.}

--- Думаю, я цалкам вас разумею, прафесар МакГонагал.

--- О, сумняваюся, містэр Потэр. Вельмі сумняваюся, --- МакГонагал падалася наперад,
твар яе зноў стаў жорсткім. --- Мы з вамі абмяркоўвалі справы значна больш адчувальныя,
таму скажу вам шчыра. Вы, і толькі вы, паведамілі пра гэтае містычнае прадчуванне бяды. 
Вы, і толькі вы, прыцягваеце да сябе хаос так, як ніхто, каго я ведаю.
Пасля нашага шопінгу ў Дыягон-алеі і \emph{потым} --- інцыдэнта з Капелюшом, і 
пасля \emph{сённяшняга} эпізода, я ясна прадбачу пэўную размову ў кабінеце Майстра, 
дзе ўсе пачуюць выбітную гісторыю, у якой вы, і толькі вы, граеце галоўную ролю, пасля
чаго ў нас не застанецца выбару, як звольніць прафесара Квірэла.
Я ўжо нават не супраціўлюся гэтым думкам. Але, калі гэтая сумная падзея адбудзецца раней за 
пятнаццатага траўня, я прывяжу вас да варот Хогвартс вашымі ўласнымі вантробамі, і 
буду кожны дзень падсыпаць вогняжукоў у ваш нос. Ці разумецце вы мяне \emph{зараз?}

Вочы ў Гары сталі вельмі шырокія. Ён кіўнуў. Потым праз секунду ён спытаў:

--- А што мне будзе, калі я здолею гэта ў апошні школьны дзень?

--- \emph{Прэч з майго кабінета!}

\later

Чацьвер.

Адназначна ў Хогвартс было нешта не тое з чацьвяргамі.

Быў чацьвер, 17:52. Гары і прафесар Флітвік стаялі перад вялізнай каменнай гаргульяй,
якая абараняла ўваход у пакоі Майстра.

Як толькі Гары вярнуўся ад МакГонагал ў вежу Рэйвенкло, адзін са студэнтаў перадаў
яму загад наведаць кабінет прафесара Флітвік, дзе Гары даведаўся, што з ім жадае 
пагаварыць Дамблдор.

Гары, раптоўна ўстрывожаны, спытаў прафесара Флітвік, ці сказаў Майстра, пра што
менавіта ён жадае пагаварыць.

Прафесар Флітвік бездапаможна пацепнуў плячыма.

Магчыма, Дамблдор палічыў, што Гары яшчэ надта малы, каб абвяшчаць словы моцы і
вар'яцтва.

\emph{Happy happy boom boom swamp swamp swamp?} падумаў Гары, але не сказаў уголас.

--- Не турбуйцеся, містэр Потэр, --- ціўкнуў прафесар Флітвік дзесцьсі з раёну
Гарынага пляча. (Гары быў прынамсі ўдзячны, што ў яго была агромная калматая барада,
бо было складана звыкнуцца, што настаўнік быў не толькі ніжэй ростам, дык яшчэ і
меў танчэйшы голас.) --- Майстра Дамблдор можа падавацца крыху дзіўным... або не
крыху дзіўным... або вельмі дзіўным, але ён ніколі, ніякім чынам, не нашкодзіў
ніводнаму студэнту, і я думаю, ніколі не нашкодзіць, --- Флітвік усміхнуўся Гары,
каб яго абнадзеіць. --- Проста не забывайце аб гэтым, і паніку як рукой здыме.  

Гэта не дапамагала.

--- Удачы! --- ціўкнуў Флітвік, нахіліўся бліжэй да гаргульі, і сказаў нешта, што
Гары зусім не здолеў пачуць. (Канешне, у чым бы быў сэнс пароляў, калі кожны мог
бы пачуць, як ты іх кажаш?) І гаргулья, зрабіўшы вельмі звычайны і натуральны рух,
збочыла, што было шокам для Гары, бо ўвесь гэты час гаргулья па-ранейшаму 
выглядала цвёрдым нерухомым камянём. 

За гаргульяй была спіральная лесвіца, якая павольна круцілася. У ёй было нешта
трывожна-гіпнатычнае, а самым трывожным было тое, што па такой лесвіцы ты не мог
трапіць нікуды.

--- Уверх! --- ціўкнуў Флітвік.

Гары недаверліва ўзышоў на прыступак, і даведаўся (нават калі яго мозг не і мог 
сабе гэта ўявіць), што ён-такі рушыць уверх.

Унізе гаргулья прагрукатала назад на свае месца, а спіраль працягвала круціцці, 
а Гары працягваў падымацца, і пасля некалькіх галавакружных секунд ён стаяў перад 
дубовымі дзвярыма з грукачом у выглядзе грыфона.

Гары працягнуў руку і павярнуў дзвярную ручку.

Дзверы расчыніліся.

І Гары ўбачыў самы цікавы пакой у сваім жыцці.

Там было мноства маленькіх механізмаў, якія гудзелі, цікалі, нетаропка змянялі форму, або 
выпускалі маленькія струменькі пары. Там былі тузіны загадкавах рознакаляровых вадкасцяў 
у тузінах дзіўнай формы колбах, усе яны булькаталі, кіпелі, сочыліся, змянялі колеры,
стваралі дзіўныя фігуры, якія знікалі праз секунду пасля таго, як ты іх заўважыў.
Такм былі рэчы, якія выглядалі, як гадзіннік з мноствам стрэлак, з незнаёмымі сымбаламі 
на цыферблатах. Быў бранзалет з хрысталём у форме лінзы, які зіхацеў тысячай колераў;
птушка, якая сядзела на залатым насесце; драўляны кубачак, у якім была на выгляд кроў, 
статуя сакола, пакрытая чорнай эмаллю. Усе сцены былі завешаны партрэтамі спячых людзей, 
і Размеркавальны Капялюш быў абыякава насунуты на вешалку, якая таксама трымала два
парасона і тры чырвоных тапкі на левую нагу.

У сярэдзіне гэтага хаоса стаяў пусты пісьмовы стол з чорнага дуба. Перад сталом 
стаяў дубовы стул. А за сталом стаяў агромны, добра укрыты падушкамі трон, у якім
змяшчаўся Альбус Персіваль Вулфрык Браян Дамблдор, які быў упрыгожаны доўгай 
срэбнай барадой, капелюшом, падобным на гіганцкі пляскаты грыб, і чымсьці, што ў 
маглаўскіх вачах выглядала, як тры слоя яскрава-ружовых піжамаў.

Дамблдор усміхаўся, і яго яскравыя вочы зіхацелі вар'яцкай энергіяй.

З трапятаннем Гары сеў на стул перад сталом. Дзверы за яго спінай зачыніліся з гучным 
\emph{грумк.}

--- Вітаю, Гары, --- сказаў Дамблдор.

--- Вітаю, Майстра, --- адказаў Гары. Яны ўжо былі на "ты"? Ці загадае ён зараз клікаць яго...

--- Калі ласка, Гары!, --- сказаў Дамблдор. --- "Майстра" гучыць так фармальна. Проста 
кліч мяне "Эм". 

--- Без праблем, Эм, --- сказаў Гары.

Павісла невялікая пауза.

--- А ты ведаеш, што ты першы, хто гэта зрабіў?

--- А-а... --- сказаў Гары, намагаючыся трымаць свой голас пад кантролем і ігнараваць
адчуванне ў жываце, быццам у хутка спускаючымся ліфце. --- Выбачайце, я... эм... Майстра, 
вы жа самі сказалі мне, таму я так і...

--- "Эм", калі ласка, --- сказаў весела Дамблдор. --- І не трэба так хвалявацца.
Я не выкіну цябе ў вакно проста з-за нейкай памылкі. Я спачатку заўсёды парярэджваю!
І дарэчы, няважна, як людзі з табой размаўляюць, важна тое, што яны пра цябе думаюць.


\emph{Ён ніколі не нашкодзіў ніводнаму студэнту, проста не забывай аб гэтым, і 
паніку як рукой здыме.}

Дамблдор працягнуў наперад маленькую металічную скрыню, і, пстрыкнуўшы, адчыніў 
яе: унутры былі невялікія жоўтыя цукеркі. 

--- Шыпучку? --- спытаў Майстра.

--- Э... не, дзякуй... Эм, --- сказаў Гары. \emph{Ці лічыцца шкодай, калі 
падсаджваеш студэнтаў на ЛСД? Або гэта таксама падзея з катэгорыі вясёлых 
жартаў?} --- Вы, э... казалі нешта пра тое, што я надта малы, 
каб абвяшчаць словы моцы і вар'яцтва?

--- Што ты, канешне, адназначна надта малы! --- сказаў Дамблдор. --- На шчасце,
Словы Моцы і Вар'ятва былі згублены сем стагоддзяў таму, і зараз ніхто не мае
аніякага паняцца, што яны сабою ўяўлялі. Проста, каб ты быў у курсе.

--- А-а... --- сказаў Гары. Ён зразумеў, што ягоны рот расчыены ўжо некаторы
час. --- А чаму вы тады мяне выклікалі?

--- \emph{Чаму?} --- паўтарыў Дабмлдор. --- О, Гары, калі б я ўвесь час пытаў
сябе \emph{чаму} я раблю нешта, у мяне бы не засталося і хвіліны, каб давесці 
нешта да канца! Ведаеш, я і без таго даволі заняты.

Гары кіўнуў, усміхаючыся.

--- Так, у вас уразлівы спіс. Майстра Хогвартс, Галоўны Чарнакніжнік Уізернгамота,
і Вярхоўны Стаўбень Міжнароднай Канфедэрацыі Чараўнікоў. Прабачце за пытанне, але 
ці ёсць магчымасць атрымаць больш за шэсць гадзін, калі скарыстаць больш 
чым адзін часаварот? Бо калі вы ўсё гэта паспяваеце ўсяго за трыццаць гадзін на дзень,
тое вельмі крута. 

Паследвала яшчэ адна амаль незаўважная пауза, падчас якой Гары працягваў 
усміхацца. Ён быў крыху ўстрывожаны, шчыра кажучы, моцна ўстрывожаны, але калі ён зразумеў,
што Дамблдор мэтанакіравана над ім жартуе, нешта ўнутры Гары \emph{абсалютна адмаўлялася}
проста супакоіцца і не супрацівіцца.

--- Баюся, Час не любіць, калі яго занадта моцна расцягваюць, --- сказаў Дабмлдор пасля 
невялікай паузы, --- і ўсё ж, нам заўсёды яго не хапае... гэтая бясконцая барацьба,
каб ўмесціць нашыя жыцця ў Час...

--- Цалкам згодны, --- сказаў Гары з магільнай сур'ёзнасцю. --- Тым больш, лепей 
нам хучэй пераходзіць да сутнасці нашай размовы.

На імгненне Гары задумаўся, ці не перагнуў ён палку.

Дамблдор кекнуў.

--- Што ж, да сутнасці, так да сутнасці, --- Майстра нахіліўся наперад, нахіляючы 
свой пляскатагрыбны капялюш, паклаўшы бараду на стол. --- Гары, гэтым панядзелкам ты 
зрабіў нешта, што лічыцца немагчымым нават з часаваротам. Дакладней, немагчыма з 
\emph{адным толькі} часаваротам. Мне вельмі цікава, адкуль узяліся тыя два пірага? 

Спалох прасяк цела Гары адрэналінам. Ён зрабіў тое з дапамогай Плашча Нябачнасці,
які ён атрымаў як падарунк на Раство, разам з лістом, які казаў \emph{Калі Дамблдор
атрымае магчымасць завалодаць адным са Скарбаў Смерці, ён не выпусціць
яго са сваіх збялелых пальцаў...} 

--- Натуральнае меркаванне, --- працягваў Дамблдор, --- што калі ніхто з прысутнічаючых
першакурснікаў не быў здольны на такі заклён, то там прысутнічаў нехта яшчэ, 
каго ніхто не ўбачыў. І калі яго ніхто не бачыў, то кінуць два пірага "з ніадкуль"
для яго было даволі простай справай. Можна зрабіць далейшую выснову, што, калі 
ты, Гары, меў часаварот, то менавіта ты быў нябачным. А калі заклён Дызілюзорнасці 
яшчэ па-за межамі тваіх здольнасцей, то ты быў пад плашчом нябачнасці, --- 
Дабмлдор па-змоўніцку ўсміхнуўся. --- Ці на правільным я шляху, Гары?

Гары застыў. У яго было прадчуванне, што схлусіць зараз будзе не самым 
разумным варыянтам. Ды яшчэ і не вельмі дапаможа. І наогул пасля складанага дня
ён так і не знайшоў, што сказаць.

Дамблдор махнуў рукой.

--- Не хвалюйся, Гары, ты не зрабіў нічога дрэннага. Плашчы нябачнасці не парушаюць 
правілаў --- я мяркую, яны такія рэдкія, што нікому і да галавы не прыйшло ўключаць
іх у спіс. Але насамрэч мяне цікавіць нешта зусім іншае.

--- О? --- сказаў Гары самым нармальным тонам, якім толькі здолеў.

Вочы Дамблдора зіхацелі энтузіязмам.

--- Разумееш, Гары, пасля пэўнай колькасці прыгодаў, пачынаеш разумець такія рэчы.
Ты пачынаеш бачыць патэрн, пачынаеш трапляць у рытм сусвета. У цябе з'яўляюцца 
прадчуванні яшчэ \emph{да} моманту адкрыцця. Ты --- Хлопчык-які-выжыў, і нейкім 
чынам ўсяго толькі праз чатыры дні, як ты адкрыў для сябе магічную Брытанію, 
у цябе апынуўся нябачнасці. Такія рэчы не купіць у Дыягон-аллеі, але 
існуе толькі \emph{адзін}, які можа сам знайсці свой шлях да свайго прызначанага 
гаспадара. І вось, я не магу не задавацца пытаннем, а што калі па нейкай прымхе лёсу
ты знайшоў не проста нейкі плашч, а Той Самы Плаш Нябачнасці, адзін з трох Скарбаў
Смерці, які, па чутках, можа схаваць таго, хто яго апране, ад вачэй самой Смерці?.. --- 
позірк яго стаў як у фанатыка. --- Магу я паглядзець на яго, Гары?

Гары зглынуў. Адрэналін напаўняў яго па самыя вушы, і ён быў цалкам бескарысны, бо 
гэта --- самы магутны чараўнік у свеце, і ніякім чынам Гары не паспее дасягнуць 
дзвярэй, і нават калі так --- у Хогвартс не было такога месца, дзе ён мог бы схавацца,
ён зараз згубіць свой Плашч, які перадаваўся праз пакаленні Потэраў хто ведае 
колькі гадоў...

Павольна Дамблдор адкінуўся назад у сваім высокім крэсле. Яскравыя агеньчыкі 
сышлі з яго вачэй, ён выглядаў крыху збянтэжана і смучана. 

--- Гары, --- сказаў ён, --- калі ты не хочаш, можаць проста сказаць "не".

--- Я магу? --- прахрыпеў Гары.

--- Так, Гары, --- сказаў Дамблдор. Ягоны голас цяпер гучаў сумна, усхвалявана.
--- Падаецца, што ты мяне баішся, Гары. Магу я спытаць, чым я заслужыў твой недавер?


Гары зноў зглынуў. 

--- Ці ёсць спосаб прымусіць вас прынесці магічную непарушальную клятву, што вы 
не возьмеце мой плашч?

Дамблдор задумённа пакачаў галавой.

--- Непарушны Зарок не карыстаецца па такім дробязям. Акрамя таго, Гары, калі ты 
не ведаеш пра заклён, я магу і падмануць цябе пра яго непрарушальнасць.
Але, ты, канешне, і сам разумееш, што мне не \emph{патрэбны} твой дазвол, каб 
убачыць Плашч. У мяне дастаткова моцы, каб выклікаць яго, з махляскіна або 
скульсьці яшчэ, --- твар Дамблдора быў вельмі змрочны. --- Але я не буду гэтага 
рабіць. Плашч твой, Гары. Я не буду адбіраць яго. Нават, каб проста глянуць, пакуль 
ты не вырашыш паказаць яго сам. Гэта мае абяцанне і зарок. Калі мне спатрэбіцца 
забараніць цябе карыстаць яго на тэрыторыі школы, я загадаю табе адвесці яго ў 
Грынготс і трымаць яго ў тваім сховішчы.

--- А-а... --- сказаў Гары. Ён глыбока ўздыхнуў, спрабуючы супакоіць паток
адрэналіну і пачаць думаць разважліва. Ён зняў свой махляскінавы кашэль з пояса.
--- Калі вам насамрэч не патрэбны мой дазвол... то я дазваляю, --- Гары працягнуў 
кашэль Дамблдору, і адначасова моца ўкусіў сябе за губу, дасылаючы сабе сігнал 
на той выпадак, калі пасля на ім прыменяць забывальны заклён.

Стары чараўнік засунуў руку ў махляскін, і без усялякіх словаў вызаву, выцягнуў 
з яго Плашч Нябачнасці.

--- А... --- выдахнуў Дамблдор. --- Я быў правы... --- Ён працягнуў зіготкую чортную 
тканіну праз далонь. --- Старажытная, як свет, але ў ідэальным стане, як зробленая 
ўчора. Мы страцілі столькі нашага майстэрства за гэтыя гады, і зараз ніхто не можа
стварыць падобную рэч, нават я не магу. Я адчуваю ягонуб сілу, як рэха ў маёй галаве,
быццам бясконцая песня, якая страціла слухачоў... --- Ён падняў позірк ад Плашча.
--- Не прадавай аго, і не давай у пазыку. Падумай двойчы, перад тым, як паказаць яго
камусьці, і тройчы --- калі вырашыш падзяліцца сакрэтам, што гэта адзін з 
Скрабаў Смерці. Стаўся да яго з пашанай, бо гэта сапраўдны Артэфакт Сілы.

На нейкае імгненне аблічча Дамблдора напоўнілася прагай...

...і ён працягнуў Плачш Гары.

Гары схаваў яго ў кашалі.

Дамблдор зноў пазмрачнеў.

--- Магу я спытаць яшчэ раз, Гары, чаму ты мне так не давяраеш?

Раптам Гары адчуў сябе даволі сорамна.

--- З Плашчом была запіска, --- сказаў Гары ціха. --- У ёй казалася, што вы 
паспрабуеце забраць Плашч, калі даведаецеся пра яго. Аднак, я не ведаю, хто 
яе напісаў, праўда.

--- Я... разумею, --- сказаў павольна Дамблдор. --- Ну... я не буду фантазіраваць на 
тэму матывацыі аўтара запіскі. Хто ведае, можа яны і праўда зрабілі гэта з
найлепшых пажаданняў? У рэшце рэшт, яны даслалі табе Плашч.

Гары, уражаны ўзноўнем спачувальнасці Дамблдора, кіўнуў, і адначасова засаромеўся
ад кантрасту са сваім стаўленнем.

--- Але думаю, мы з табою фігуры аднаго колеру ў гэтай гульне, --- працягваў Дамблдор. ---
Хлопчык, які нарэшце адолеў Вальдэморта, і стары, які адцягваў яго ўвагу дастаткова 
доўга, каб табе выпаў шанец выратаваць нас усіх. Я не абураюся за твой недавер, 
бо мы павінны з усёй моцы намагацца, каб быць мудрэйшымі. Я толькі прашу цябе 
падумаць двойчы, тройчы, колькі трэба, калі наступным разам нехта кажа табе,
што я не варты даверу.

--- Прабачце, --- сказаў Гары. Ён адчуваў сябе цалкам спустошаным, быццам ён 
нагрубіў самому Гэндальфу, і дабрыня Дамблдора толькі пагаршала яго становішча.

--- Трэба было даверыцца вам.

--- Ох, Гары, у нашым свеце... --- ён пакачаў галавой. --- Я нават не магу сказаць,
што ты павёў сабе дрэнна. Ты яшчэ не ведаў мяне. І, шчыра кажучы, у Хогвартс 
хапае тых, каму не варта давераць. Магчыма сярод іх ёсць і тыя, каго 
ты лічыш сябрамі.

Гэта прагучала даволі злавесна. 

--- Напрыклад?

Дабмлдор падняўся на ногі і пачаў разглядваць адзін са сваіх прыбораў, 
гадзіннік з васем'ю стэлкамі рознай даўжыні. 

Праз некаторы час ён сказаў:

--- Ён, напэўна, падаецца пабе прывабным. Ветлівым --- прынамсі да цябе.
Добра выхаваны, адукаваны. Заўсёды можа прапанаваць падамогу, зрабіць ласку,
даць добры савет...

--- А, \emph{Драко Малфой!} --- выдахнуў Гары з дзіўнай палёгкай, што гэта было 
не пра Герміёну, або кагосьці яшчэ. --- О, не. Не-не-не, тут вы памыляецеся на ўсе 
сто. Гэта не ён мяне абрачае, а я --- яго. 

Дабмлдор, застыў, усё яшчэ назіраючы за гадзіннікам

--- Ты \emph{што?}

--- Я збіраюся выратаваць Драко ад Цёмнага Бока, --- сказаў Гары. --- Ну, ведаеце,
зрабіць яго адным з  добрых хлопцаў. 

Дамблдор выпрастаўся і павярнуўся да Гары. У яго было неверагодна збянтэжаны выгляд,
які Гары толькі бачыў, не кажучы пра тое, што гэты выгляд быў у чалавека з 
сівой барадой да падлогі. 

--- А ты ўпэўнены, --- спытаю стары чараўнік праз некалькі секунд, --- што ён 
гатовы быць выратаваны? Баюся, што дабрыня, якую, як табе падаецца ты ў ім бачыш,
можа быць усяго ж толькі тваім жаданнем --- або горш, падманам, прыманкай...

--- Э... малаверагодна, --- сказаў Гары. --- Калі ён спрабуе ўдаваць добрага 
хлопца, то ў яго гэта атрымліваецца неверагодна дрэнна. Калі вы думаеце, што 
я падзівіўся, які ён клёвы, і што ў ім ёсць нешта добрае, то гэта не так.
Я абраў яго для выратавання адмыслова таму, што ён нашчадак роду Малфоеў, 
і калі абіраць аднаго чалавека для такой справы, то відавочней кандыдата не 
знайсці.

Левае вока Дамблдора тузанулася. 

--- І ты збіраешся пасеяць зерне любові і дабрыні ў сэрцайка Драко Малфоя,
бо ты плануеш скарыстаць яго на сваю карысць?

--- Не толькі на \emph{маю!} --- сказаў Гары з абурэннем. --- Усёй магічнай
Брытаніі, калі атрымаецца! \emph{Да таго ж}, ён сам атрымае шчаслівейнае
і псіхічна здаравейшае жыццё! Разумееце, у мяне няма часу перацягваць 
\emph{усіх} на Светлы Бок, і таму я пытаю сябе: "Якім чынам Свет можа 
атрымаць максімальную карысць за самы кароткі час?" і...

Дамблдор пачаў смяяцца. Смех яго быў надта мацнейшы, чым мог чакаць Гары --- амаль 
з падвываннямі. Ён быў цалкам пазбаўлены годнасці. Стары і магутны чараўнік павінен 
смяяцца кароткімі глыбокімі смешкамі, а не іржаць так, што яму не хапала паветра.
Аднойчы ў дзяцінстве Гары літаральны выпаў з крэсла, калі глядзеў класічную камедыю
1933 года "Уціны суп", і менавіта так зараз смяяўся Дамблдор.  

--- Гэта не \emph{настолькі} смешна, --- сказаў Гары праз некаторы час. Ён зноў 
пачаў хвалявацца на стан псіхікі Дамблдора.

З заўважным намаганнем Дамблдор узяў сябе ў рукі. 

--- А, Гары, адзін з вядомых сімптомаў хваробы пад назвай "мудрасць" --- гэта смяяцца 
над тым, што ніхто не лічыць вясёлым, бо ты мудрэц, разумееш, Гары, ты пачынаеш 
разумець новыя жарты! --- ён выцер слёзы з вачэй. --- Божачкі... 
Oft evil will shall evil mar, што праўда, то праўда.

Гарынаму розуму спатрэбілася некалькі секунд, каб успомніць, дзе ён чытаў гэтыя словы.

--- Хэй, гэта цытата з \emph{Толкіна!}\footnote{{} Дамблдор сказаў роханскаю прымаўку з
"Уладара Пярсцёнкаў", падобную на нашу "ні капай на другога ямы, бо сам у яе ўвалішся".}
Гэта сказаў Гэндальф!

--- Насамрэч, Тэодэн, --- сказаў Дабмлдор.

--- Вы што, \emph{магланароджаны?} --- Гары быў у шоку.

--- Баюся, што не, --- адказаў Дамблдор, зноў усміхаючыся. --- Я нарадзіўся за 
семдзесят гадоў да публікацыі гэтай кнігі, дзіця. Але падаецца, што ў некаторых
рэчах думкі маіх магланароджаных студэнтаў сходзяцца. У мяне назапасілася не менш 
за дваццаць копій "Уладара Пярсцёнкаў", і тры --- поўнага набору твораў Толкіна,
і кожны з іх мне дарагі, --- ён дастаў палачку, падняў яе, застыўшы ў грознай позе. 
--- \emph{You cannot pass!} Як выглядае?

--- Э-э-э... --- Гарын мозг набліжаўся да стану, калі ён быў гатовы адключыцца назаўсёды.
--- Барлога крыху не хапае... --- таксама ружовыя піжамы і пляскаты грыб аніяк не дапамагалі вобразу.

--- Эх, --- уздыхнуў Дабмлдор і сумна схаваў сваю палачку. --- Баюся, у маім жыцці ў 
апошнія гады назіраецца недарэчны дэфіцыт барлогаў. Зараз усё больш пасяджэнняў ва 
Ўізернгамоце, дзе я павінны адчайна намагацца перадухіліць іх ад прыняцця любых 
рашэнняў, і званыя вячэры, дзе палітыкі спаборнічаюць у дурасці.   
І застаецца толькі быць загадкавым, ведаць рэчы, якія я ніяк не магу ведаць, 
рабіць незразумелыя заявы,
якія можна цалкам зразумець толькі праз пэўны час, і ўсе іншыя дробязі, якімі 
забаўляюцца магутныя чараўнікі пасля таго, як яны сышлі са сцежкі галоўнага героя...
Дарэчы, аб героях, Гары. Я павінен перадаць табе нешта, што 
належыла твайму бацьке.

--- Праўда? --- сказаў Гары. --- Божа, хто б мог падумаць.

--- Дакладна. Гэта настолькі прадказальна? --- Дамблдор зноў стаў змрочным. --- Ну, няважна...

Ён вярнуўся і сеў за свой стол, выцягнуў адну з шуфлядак. Засунуўшы ў яе абодзве рукі 
па самыя плечы, ён з напругай выцягнуў вялікі і на выгляд цяжкі аб'ект, які ён 
паклаў на стол з вялікім грукам.

--- Гэта камень твайго бацькі.

Гары ўтаропіўся. Аб'ект быў шэрым, няправільнай формы, з вострымі краямі: ён быў 
самым звычайным абломкам скалы. Дабмлдор паклаў яго на самую шырокую яго частку, 
але той усё роўна няўстойліва вагаўся з баку на бок.

Гары падняў позірк на Дабмлдора.

--- Гэта такі жарт, праўда?

--- Гэта не жарт, --- сказаў Дабмлдор, качаючы галавой, з вельмі сур'ёзным 
выглядам. --- Я знайшоў яго на руінах дома Джэймса і Лілі ў Годрыкавай Лагчыне, 
там жа, дзе я знайшоў цябе. Я захоўваў яго ўсе гэтыя гады, чакаючы дзень, калі 
я змагу перадаць яго табе, і вось гэты дзень прыйшоў.

У блытанке супярэчлівых гіпотэз, якія былі ў гэты момант Гарынай мадэллю света,
вар'яцтва Дамблдора рэзка падняла сваю імавернасць. Але альтэнатывы таксама 
яшчэ мелі значную яе частку. 

--- Эм... гэта \emph{магічны} камень?

--- Наколькі мне вядома, не, --- сказаў Дамблдор. --- Але я бы саветваў табе 
з усёй належнай дбайнасцю трымаць яго ўвесь час як мага бліжэй да сябе.

Ну ладна. Дабмлдор, хутчэй за ўсё, з'ехаў з глузду, але калі раптам не? 
Будзе даволі недарэчна, калі ты ігнаруеш савет загадкавага старога чараўніка, 
а потым у цябе пачынаюцца праблемы. Гэта будзе дзесці на чацьвёртым месцы з 
рэйтынгу "Сто Самых Відавочных Спосабаў Праваліць Міссію".

Гары падышоў бліжэй,  і паклаў на яго рукі, намагаючыся знайсці спосаб падняць
яго, не парэзаўшыся. 

--- Тады давайце я пакладу яго ў мой кашэль.

Дабмлдор нахмурыўся. 

--- Гэта можа быць недастаткова блізка да цябе. І што, калі твой кашэль 
згубіцца, або хтосьці яго скрадзе?

--- Вы думаеце, мне варта проста таскаць яго ўвесь час у руках?

Позірк Дамблдора быў вельмі сур'ёзны.

--- Гэта будзе даволі мудра. Хто ведае, калі ён прыйдзе на карысць?

--- Э-э... --- сказаў Гары. Камень выглядаў сапраўды цяжкім.  --- Думаю, у маіх 
калегаў гэта будзе вызываць шмат пытанняў.

--- Скажаш, што гэта мой загад, --- сказаў Дабмлдор. --- Да гэтага ні ў каго 
пытанняў не будзе, бо яны ўпэўнены, што я вар'ят, --- ягоны твар быў 
цалкам спакойны і сур'ёзны.

--- Э... гэта зусім не дзіва, калі такія загады для вас --- нармальная справа.

--- Ах, Гары, --- сказаў Дамблдор. Ён зрабіў шырокі жэст, ахінаючы загадкавыя
прыборы, расстаўленыя па кабінеце. --- У дзяцінстве мы думаем, што ведаем усё,
і таму думаем, што калі мы не бачым тлумачэння для чагосьці, то яго і не існуе, 
гэтага тлумачэння. Калі мы сталеем, мы разумеем, што ў сусвета сваі правілы і свой
рытм, нават калі мы самі яго і не бачым. Што мы ўспрымаем, як вар'яцтва --- проста 
нашае неведанне.  

--- Рэальнасць заўсёды следуе законам, --- сказаў Гары, --- нават калі мы не ведаем
саміх законаў.

--- Дакладна, Гары, --- сказаў Дамблдор. --- Гэтае разуменне --- а я бачу, што ты 
яго маеш, --- сутнасць мудрасці. 

--- Так... а чаму \emph{менавіта} я павінен насіць камень?
 
--- Шчыра кажучы, я не магу прыдумаць ніводнай прычыны.

--- Не можаце?

Дамблдор кіўнуў.

--- Але калі я не магу прыдумаць прычыну --- гэта не значыць што яе не існуе.


Механізмы вакол працягвалі цікаць.

--- Акей, --- сказаў Гары, --- не думаю, што варта такое казаць, але гэта адназначна
памылковы спосаб пераадоліць нашае няведанне таго, як працуе сусвет.

--- Праўда? --- стары чараўнік выглядаў здзіўленым і расчараваным.

У Гары было прадчуванне, што ён прайграе ў гэтай спрэчцы, але ўсё роўна працягваў.

--- Праўда. Я не ведаю, ці ёсць афіцыйная назва ў гэтай заблуды, але калі мне трэба 
бы было прыдумаць назву, яна была бы нешта кшталту "ад ліхтарнага слупа". 
Як бы гэта фармалізаваць... вось напрыклад, калі ў вас ёсць мільён каробак, і толькі 
ў адной з іх знаходзіцца дыямент, то, згадзіцеся, абраць адзін з іх выпадкова 
будзе глупствам. Уявіце, што мы знайшлі нейкі датчык. Ён спрацоўвае каля каробкі
з дыяментам, але і выдае памылкі --- ён таксама спрацоўвае ў палове разоў, калі 
яго паднесці да пустой каробкі. Калі мы прыменім такі датчык, у нас застанецца 
500 тысяч каробак, якія ўтрымліваюць дыямент з большай імавернасцю. Потым яшчэ, і яшчэ,
і пасля дваццаці ітэрацый у нас застанецца толькі тры каробкі! Разумееце, сэнс тут 
у тым, што калі ў вас так шмат варыянтнаў, большаць доказаў сыходзіць на тое,
каб \emph{знайсці} найбольш верагодную гіпотэзу з мільёна. Таму калі вы абіраеце
адну з мільёна выпадкава, вы прапускаеце ўсю тую працу, якая дадае імавернасці 
сапраўднай гіпотэзе.

Ён ўдыхнуў і працягваў:

--- Гэта быццам у горадзе жыве мільён жыхароў, і здарылася забойства, і дэтэктыў 
кажа: "Ну, доказаў у нас няшмат. Прапаную разгледзець варыянт, што забойца --- 
Дзед-Барадзед!"

--- А ён забойца?

--- Не, --- сказаў Гары. --- Але пазней высвятляецца, што забойца быў сівы, і
Дзед-Барадзед сівы, таму ўсе кажуць: "Не, ну бачыце, усё ж ясна, як дзень!" З 
боку паліцыі прыцягваць увагу да Барадзеда вось так, не маючы ніякіх 
падстаў падазраваць яго больш за астатніх --- несправядліва.
Калі ёсць шмат варыянтаў, то нават проста каб вызначыць правільный, трэба зрабіць 
шмат працы.
І яшчэ да доказу --- у сэнсе афіцыйнага доказу, які патрабуе суд або вучоныя,
патрэбная нейкая падказка, якая дапамагае нам адрозніть гэты канкрэтны варыянт 
ад мільёну іншых.
Таму вы не можаце выбраць варыянт ад ліхтарнага слупа. Бо можа быць мільён 
іншых рэчаў, якія могуць быць на карысць, якія я магу рабіць замест таго, каб
насіць бацькоўскі камень. Проста тое, што я не ведаю ўсяго аб сусвеце, не значыць,
што я не разумею, як разважаць ва ўмовах нявызначанасці. Законы імавернасці не 
меней строгія, чым законы класічнай логікі, і тое, што вы зараз зрабілі, проста 
\emph{забаронена}, --- Гары зрабіў паузу. --- Толькі калі ў вас няма ніякай 
сакрэтнай \emph{падказкі...}

--- Ох, --- сказаў Дабмлдор задумённа, --- цікавы агрумент, канешне. Аднак я 
не сказаў, што насіць з сабою камень твайго бацькі будзе найлепшым варыянтам,
проста што будзе мудрэй зрабіць гэта, чым адмовіцца.

Ён зноў залез у тую ж шуфляду, што і раней, гэтым разам нешта там абшукваючы, --- 
прынамсі, мяркуючы па рухам ягоных рук. 

--- Паважуся дадаць, --- сказаў Дабмлдор, пакуль Гары прыдумляў, што яму адказаць на 
гэты нечаканы адказ, --- што гэта агульная памылка рэйвенкло --- думаць, што ўсе 
разумныя дзеці размяркоўваюцца на іх факультэт, пакідаючы дурняў астатнім.
Гэта не так. Размеркаванне на Рэйвенкло значыць, што чалавекам кіруе жаданне пазнаваць
рэчы, што зусім не тое ж самае, што быць разумным, --- ён усміхнуўся, усе яшчэ 
займаючыся нечым у шуфлядцы. --- Нягледзячы на гэта,  \emph{ты} выглядаеш  
даволі разумным. Не столькі як звычайны юны герой, больш як юны загадкавы 
сівабароды чараўнік. Я пачынаю думаць, Гары, што абраў не зусім правільный падыход
да цябе, і што ты можаш зразумець рэчы, якія шмат для каго так і застануцца загадкай.
Так што я рызыкну, і дзёрзка прапаную табе яшчэ нейкую, \emph{іншую} 
спадчыну.

--- Вы жа не... --- выдахнуў Гары. --- Яшчэ адзін \emph{бацькоўскі камень?}

--- Я прашу прабачэння, --- сказаў Дамблдор, --- але я ўсё яшчэ старэйшы і больш
загадкавы, чым ты, і калі надыдзе час рабіць выяўленне, я зраблю яго без 
залішняй дапамогі, дзякуй... да \emph{дзе ж} гэтая халера?! --- Дамблдор
засунуў рукі глыбей у шуфляду, потым яшчэ глыбей. Яго галава і плечы схаваліся ў 
стале, потым тулава, пакуль не засталіся толькі тырчаць ногі, быццам стол 
пакрысе еў Майстра.

Гары не мог не дзівіцца, колькі рэчаў магло хавацца ў гэтым стале, і як 
будзе выглядаць іх поўны спіс.

Нарэшце Дабмлдор вылез з шуфляды, трымаючы аб'ект сваіх пошукаў у руцэ.
Ён паклаў яго на стол побач з камянём.

Гэта быў падручнік, быўшы ў карыстанні, з падранымі краямі і пацёртым карэшком:
\emph{Зеллеварэнне, Сярэдні ўзровень}, аўтар Лібатус Борідж. На вокладцы быў
намаляваны флакон, з якога ішоў дымок. 

--- Гэта, --- урачыста абвесціў Дамблдор, --- быў падручнік тваёй маці на пятым 
курсе.

--- Які я павінен насіць як мага бліжэй да сябе ўвесь час, --- сказаў Гары.

--- Які \emph{трымае жудасны сакрэт.} Сакрэт, раскрыццё якога можа апынуцца настолькі 
катастрафічным, што я павінен запытаць у цябе паклясціся --- і я патрабую, каб 
гэтая клятва была сур'ёзная, Гары, няважна, што ты аб гэтым думаеш, --- 
не казаць пра тое нікому і нічому.

Гары яшчэ раз паглядзеў на падручнік яго маці, які, верагодна, утрымліваў 
жудасны сакрэт.

Праблема была ў тым, што Гары і сапраўды ставіўся да падобных клятваў вельмі 
сур'ёзна. Любы зарок быў Непарушным Зарокам, калі яго робіць чалавек 
пэўнага складу характара.

І...

--- У мяне смага, --- сказаў Гары, --- і гэта зусім не добры знак.

Дабмлдор абсалютна не здолеў пракаментаваць гэты незразумелы сказ.

--- Ты клянешся, Гары? --- позірк Дабмлдора свідраваў яго. --- Інакш я не змагу 
з табой падзяліцца.


Гэта было праклёнам Рэйвенкло. 
Ты проста не мог адмовіцца ад такой прапановы, бо твая цікавасць згрызе цябе знутры,
усе гэта ведалі.

--- Так, --- сказаў Гары. --- Я клянуся.

--- І ў адказ я клянуся, --- сказаў Дабмлдор, --- што я раскажу табе чыстую праўду.

Дабмлдор раскрыў кнігу, як падалося, на выпаковай старонке, і Гары нахіліўся паглядзец.

--- Ты бачыш нататкі, --- голас Дамблдора стаў ціхім, амаль шэптам, --- напісаныя 
на палях?

Гары прыжмурыўся. Жоўтыя старонкі апісвалі нешта пад назвай 
\emph{узвар арлінага хараства}, яго інгрыдыентамі былі рэчы, якія Гары не 
распазнаў, або чыі назвы наогул былі не на ангельскай. 
Побач на палях была занатоўка, якая казала \emph{Цікава, што здарыцца, калі скарыстаць тут 
кроў Тэстрала замест сока чарніцаў?} і непасрэдна пад ёю быў адказ, зроблены іншай 
рукой: \emph{Некалькі тыдней хваробы. Магчыма, смерць.}

--- Бачу, --- сказаў Гары. --- Што за яны?


Дабмлдор указаў на другую частку. 

--- Тыя, што зроблены гэтай рукой, --- сказаў ён тым жа ціхім голасам, --- 
былі напісаны тваёй маці. А вось гэтыя, --- ён пасунуў палец, указваючы на першае 
пытанне, --- напісаў я. Я рабіўся нябачным, і прабіраўся ў пакой тваёй маці, пакуль 
яна спала. Лілі думала, што гэта піша адна з яе сябровак, і ў іх на гэты конт былі 
выбітныя сваркі.

У гэты дакладны момант Гары канчаткова ўпэўніўся, што Майстра Хогвартс 
насамрэч \emph{абсалютна з'ехаў з глузду.}1

Дабмлдор сур'ёзна глядзеў на яго. 

--- Ты разумееш наступствы таго, што я расказаў, Гары? 

--- Э-эммм... --- сказаў Гары. Яго голас быццам заліп дзесьці ў горле. --- Выбачайце, 
я не вельмі...

--- А, ну до, --- сказаў Дамблдор, уздыхаючы. --- Вядома, твая разумнасць мае свае межы,
у рэшце рэшт. Можа мы разам удадзім, што я нічога не гаварыў?

Гары падняўся са свайго стула з прыклеенай усмешкай.

--- Натуральна, --- сказаў ён. --- Ведаеце, час ужо позні, і я крыху галодны, 
так што, думаю, мне ўжо пара ісці на вячэру, праўда, --- і ён пайшоў да дзвярэй.

Дзвярная ручка цалкам адмаўлялася павярнуцца.

--- Ты пракрыўдзіў мяне, Гары, --- пачуўся ціхі голас Дабмлдора вельмі блізка ззаду.
--- Ты прынамсі разумееш, што тое, што я табе сказаў --- прызнак даверу?

Гары павольна абярнуўся. 

Перад ім стаяў вельмі магутны і вельі звар'яцелы чараўнік з доўгай 
срэбнай барадой, у капелюшы, падобным на гіганцкі пляскаты грыб, і чымсьці, што ў 
маглаўскіх вачах выглядала, як тры слоя яскрава-ружовых піжамаў.

За спінай Гары былі замкнутыя дзверы, якія не хацелі адчыняцца.

Раптам від у Дамблдора был вельмі сумны і стомлены, быццам ён хацеў абапярэцца аб
посах, якога ў яго не было.

--- Сумна, --- сказаў ён, --- калі спрабуеш нешта новае замест патэрна, якому следваў 
сто дзесяць гадоў, і людзі пачынаюць ад цябе ўбягаць, --- ён патрос галавой у 
смутку. --- Я чакаў ад цябе большага, Гары Потэр. Я чуў, што твае ўласныя сябры лічаць 
цябе звар'яцелым. Я ведаю --- яны памыляюцца. Ці можна прымяніць тую ж логіку да мяне?

--- Калі ласка, адчыніце дзверы, --- сказаў Гары трасучымся голасам. --- Калі вы 
жадаеце, каб я мог калісьці вам давераць --- адчыніце дзверы.

За яго спінай пачуўся гук адчыняючыхся дзвярэй.

--- Я планаваў табе расказаць шмат чаго яшчэ, --- сказаў Дамблдор, --- і калі ты 
сыдзеш зараз, ты ніколі не даведаешся, што гэта было.

Гэта быў адзін з тых момантаў, калі Гары абсалютна \emph{ненавідзеў}, што ён рэйвенкло.

\emph{Ён ніколі не нашкодзіў ніводнаму студэнту,} сказала грыфіндорская частка Гары. 
\emph{Проста не забывай аб гэтым, і паніку як рукой здыме.
Ты жа не плануеш уцячы проста таму, што падзеі сталі сапраўды цікавымі, ці не так?}

\emph{Ты не можаш проста так павярнуцца і сысьці ад Майстра,} сказаў Гарын унутраны
хаплпаф. \emph{Што, калі ён пачне здымаць балы? Калі ён вырашыць, што ты яму не 
падабаешся, ён можа зрабіць твае жыццё вельмі непрыемным.}

Нелюбімая ягоная частка, якую Гары ніяк не мог сцішыць, гучна разважала аб 
патэнцыйных выгадах ад таго факту, што ён будзе адным з нешматлікіх сяброў
звар'яцелага старога чараўніка, які да таго ж так трапна займае пасады Майстра,
Галоўнага Чарнакніжніка, і Вярхоўнага Стаўбеня. І на жаль, ягоны ўнутраны 
слізэрын падаваўся шмат большым майстрам перацягваць людзей на Цёмны бок, бо ён 
казаў \emph{небарака, яму проста трэба с кімсьці пагаварыць, а ты бы не хацеў, 
каб ён даверыўся камусьці не такому крутому, як ты?... цікава, якімі неверагоднымі
сакрэтамі ён мог бы падзяліцца, калі б у цябе было крыху мужнасці пасябраваць 
з ім... і б'юся аб заклад, Дамблдор валодае сапраўды выбітнай калекцыяй кніг...} 


\emph{Вы ўсе --- проста кучка прыдуркаў,} падумаў Гары ўсяму сходу, 
але іх было чатыры супраць аднаго, і яго ўнутраныя часткі пераважылі.

Гары павярнуўся, зрабіў крок да дзвярэй, працягнуў руку, і выразна зачыніў іх.
Фактычна гэта нічога не мяняла, бо Дамблдор мог у любым выпадку кантраляваць яго 
рухі, але можа гэта ўразіць Дабмлдора.

Калі Гары павярнуўся назад, ён убачыў, што магутны вар'ят зноў па-сяброўску 
ўсміхаецца. Гэта было добра. Напэўна.

--- Калі ласка, не рабіце так болей, --- сказаў Гары. --- Я не люблю адчуваць 
сябе ў пастцы.

--- Мне \emph{праўда} шкада за гэта, Гары, --- у голасы Дамблдора, падавалася, было 
нешта падобнае на выбачэнне. --- Але было б вялікай дурасцю адпусціць цябе без 
камяня твайго бацькі.

--- Ну, вядома, --- сказаў Гаы. --- Канешне, дзверы не адчыняцца, пакуль важны 
для прахаджэння квеста артэфакт не апынецца ў маім інвентары. І аб чым я толькі
думаў. 

Дабмлдор з усмешкай кіўнуў.

Гары падышоў да стала, зняў махляскін з пояса, з цяжкасцю падняў сваімі 
адзіннаццацігадовымі рукамі нязграбны камень, і неяк здолеў увапхнуць яго 
ўнутр.

Ён адчуваў рукамі, як лягчае камень, знікаўчы у кашалі, а адбіця, якая 
паследвала, была громкая і выразна абуральная. 

З каменем у кашалі знік і падручнік маці Гары па зеллеварэнню (ягоны сакрэт
апынуўся даволі жудасным у рэшцэ рэшт).

І потым яго ўнутраны слізэрын зрабіў хітрую прапанову падлашчыцца да Майстра, якая,
на жаль, была выказана ў такой пераканаўчай форме, што яе падтрымала бы большасць 
факультэта Рэйвенкло. 

--- Дык гэта, --- сказаў Гары. --- М-м-м... Калі я ўжо тут, ці не хаціце вы зрабіць 
мне экскурсію па вашым кабінеце? Даволі цікаўныя у вас тут рэчы... --- і гэта было 
пераменьшваннем месяца ў яго ўласным рэйтынгу.

Дабмлдор утаропіўся ў яго, потым з лёгкай усмешкай кіўнуў. 

--- Мне ліслівіць твой інтарэс, --- сказаў ён, --- але баюся, я мала што магу 
табе расказаць.

Ён падышоў да сцяны і паказаў на партрэты спячых людзей.

--- Гэта патрэты былых Майстроў Хогвартс.

Ён павярнуўся ў бок свайго стала.

--- Гэта мой стол. 

Ён паказаў на свае крэсла.

--- Гэта мае крэсла...

--- Выбачайце, --- сказаў Гары, --- насамрэч мяне цікавяць яны, --- Гары паказаў 
на невялікі кубік, які ціха шаптаў "блорп... блорп... блорп..."

--- А, гэтыя маленькія дрындушкі? --- сказаў Дамблдор. --- Яны ўжо былі тут, 
калі я стаў Майстрам, і я не маю абсалютна ніякага паняцца, што яны такое.
Хаця... вось гэты гадзіннік, у якога восем стэлак, ён падлічвае... ну давай 
назавем гэта чыхамі, --- падлічвае чыхі ведзьмаў-ляўшэй на тэрыторыі Францыі.
Ты не паверыш, колькі намаганняў патрабавала гэта высветліць. А гэты, з залатымі 
падвескамі --- мая ўласная вынаходка, і Мінерва ніколі, ніколі не здагадаецца,
што ён робіць.

Пакуль Гары абдумваў гэта, Дамблдор зрабіў крок да вешалкі.

--- Гэта, канешне, Размеркавальны Капялюш, я думаю, вы ўжо пазнаёміліся. Ён загадаў 
больш ніколі не надзяваць яго на тваю галаву, ні пры якіх абставінах. Ты --- чатырнаццаты 
студэнт у гісторыі Хогвартс, пра якіх існуе такі загад. Баба Яга таксама ў гэтым спісе,
і я раскажу табе пра астатніх, калі ты крыху падрасцеш... Гэта парасон... Гэта яшчэ 
парасон... І яшчэ адзін парасон, --- Дабмлдор павярнуўся з шырачэзнай ухмылкай. 
--- Ну і канешне, большасць людзей, якія да мяне завітаюць, жадаюць пабачыць 
Фоукса.

Дамблдор стаяў каля птушкі на залатым насесце.

Гары, крыху збянтэжаны, падышоў бліжэй.

--- Гэта Фоукс?

--- Фоукс --- фенікс, --- сказаў Дамблдор. --- Яны вельмі рэдкія, і вельмі 
магутныя магічныя істоты. 

--- А-а... --- сказаў Гары. Ён нахіліўся і ўтаропіўся ў маленькія чорныя вочкі,
якія не выдавалі анікага прызнаку магутнага інтэлекту.

--- А-а-а... --- паўтарыў Гары.

Ён быў упэўнены, што распазнаў від птушкі. Памыліцца было даволі складана.

--- Эммм...

\emph{Скажы нешта разумнае!} --- раўла Гарына свядомасць самой сабе.  
\emph{Хопіць стаяць тут і стагнаць, быццам кончаны ідыёт!}

\emph{Ну дык і што я \emph{наогул} магу сказаць?} --- раздражнёна адказвала Гары свядомасць самой сабе.

\emph{Што заўгодна!}

\emph{У сэнсе, што заўгодна, акрамя, "Фоукс --- курыца"?}

\emph{ТАК! Што заўгодна, акрамя гэтага!}

--- Так... эм... што за магію ўмеюць рабіць феніксы?

--- Іхнія слёзы маюць сілу лячыць раны, --- адказаў Дамблдор. --- Яны вогненныя істоты,
і яны рушаць з аднаго месца ў іншае таксама лёгка, як агонь можна згаснуць у адным месцы,
і перакінуцца ў новае. Неймавернае наруджанне іх адпрыроднай моцы хутка старыць 
іхнія целы, але яны, магчыма, падышлі да бессмяротнасці бліжэй за любых іншых
істот у свеце: калі рэсурс іхняга цела падыходзіць да канца, яны спальваюцца 
ў вогненым выбліску, пакідаючы пасля сябе птушаня або яйка, --- 
Дабмлдор наблізіўся, і, нахмурыўшыся, агледзіў птушку. --- Хм... падаецца мне, 
што выглядае ён нядобра.

Яшчэ да таго, як мозг Гары паспеў успрыняць гэты сказ, курыца ўжо палала агнём.

Яе дзюба адчынілася, але ў яе не было часу, каб нават піскнуць, бо яна ўжо пачала 
абсмальвацца і чарнець. Агонь быў кароткі, моцны, і цалкам аддзелены ад наваколля:
ні тэмпературы, ні паху Гары не адчуў. 

Праз пару секунд агонь знік, пакінуўшы пасля сябе маленькую і нікчэмную кучку 
попелу пад залатым насестам.

--- Не трэба так палохацца, Гары! --- сказаў Дамблдор. --- Фоукс не пацярпеў, ---
ягоная рука, нават не хаваючыся, сунулася ў кішэнь, а потым, пашукаўшы ў попеле,
тая самая рука дастала маленькае жоўтае яйка. --- Глядзі, яйка!

--- О... ну нічога сабе... крута...

--- Але нас чакаюць справы, --- сказаў Дамблдор. Пакінуўшы яйка ў курынам попеле,
ён вярнуўся да свайго трону і сеў. --- У рэшце, ужо амаль час вячэры, і нам не 
хацелася бы карыстаць нашыя часавароты.

Ва ўрадзе Гары адбывалася жорсткая барацьба. Слізэрын і Хафлпаф змянілі бок, пабачыўшы,
як Майстра Хогвартс спаліў курыцу. 

--- Справы, так... --- сказалі Гарыны вусны. --- І вячэра.


\emph{Зноў ты гучыш, быццам абдолбаны ідыёт}, --- заўважыў Гарын Унутраны Крытык.

--- Баюся, --- сказаў Дамблдор, --- я павінен у чымсьці прызнацца, Гары. Прызнацца і 
павініцца.

--- Павініцца --- гэта добра.

\emph{Што я такое кажу?}

Стары цяжка ўздыхнуў. 

--- Ты можаш змяніть меркаванне, калі зразумееш, у чым я павінны прызнацца. Баюся, Гары,
я манпуляваў табою ўсё твае жыццё. Гэта менавіта я абставіў так, што ты трапіў 
да сваіх злосных прыёмных бацькоў...

--- Мае прыёмныя бацькі не злосныя! --- не стрымаўся Гары. --- У сэнсе --- мае \emph{бацькі.}

--- Не злосныя? --- сказаў Дабмлдор, выглядаючы здзіўлена і расчаравана. ---
Нават крыху не злосныя? Гэта неяк не суадносіцца...

Гарын унутраны слізэрын роў ва ўсю моц ягоных ментальных лёгкіх: \emph{\scream{Забейся, 
прыдурак, ён адбярэ цябе ў бацькоў!}}

--- Не, не... --- сказаў Гары, ягоны твар заледзянелы ў жудаснай грымасе, --- я проста 
пашкадаваў вашыя пачуцці, насамрэч яны даволі злосныя...

--- Так? --- Дабмлдор, уперыўшыся ў Гары, падаўся наперад. --- Што яны робяць?

\emph{Кажы хутка, не думай}

--- Яны... э... мяне прымушаюць мыць задачкі, і вырышаць талеркі, і не дазваляюць чытаць... шмат... і...

--- А, добра, добра даведацца, --- сказаў Дабмлдор, адхіляючыся на спінку крэсла. Ён 
сумна ўсміхнуўся. --- Тады я прашу прабачэння за гэта... Дык аб чым гэта я? А, так.
Мне сорамна казаць гэта, Гары, але гэта менавіта я адказны практычна за кожную 
дрэнную рэч, якая адбылася ў тваім жыцці. Я ведаю, што гэта, магчыма, можа 
цябе моцна ўззлаваць.

--- Так, я вельмі ўззлаваны, --- сказаў Гары.

Гарын Унытраны Крытык імгненна выдаў яму ўзнагароду за Найгоршую Акторскую Ігру За Ўсю Гісторыю Часу.

--- Я проста хацеў, каб ты ведаў, --- сказаў Дамблдор. --- Я хацеў расказаць табе
як мага раней (у выпадку, калі з кімсьці з нас нешта здарыцца), што мне вельмі,
вельмі шкада. За ўсё, што адбылося, і што яшчэ адбудзецца.

Старыя вочы блішчэлі ад слёз.

--- А я вельмі ўззлаваны! --- сказаў Гары. --- Я так уззлаваны, што збіраюся 
сысці зараз, калі ў вас больш нечага сказаць!

\emph{Проста \emph{ідзі}, пакуль ён не вырашыў спаліць і цябе!} --- вішчэлі слізэрын,
хафлпаф і грыфіндор.

--- Разумею, --- сказаў Дабмлдор. --- Тады апошняя парада, Гары. Ты нізашто 
\emph{не} паспрабуеш адчыніць забароненыя дзверы ў калідоры трэцяга паверха. 
Для цябе проста немагчыма прайсці праз усе пасткі, і я не хачу чуць, што ты 
пашкодзіўся, спрабуючы. Хм, я сумняваюся, што ты здолееш адчыніць нават першыя 
дзверы, бо яе замкнуў я сам, і ты не ведаеш заклён  \emph{Alohomora...}

Гары павярнуўся і ўзрушыў да выхада з максімальнай хуткасцю, ручка са згодаю 
павярнулася ў ягонай руцэ, і потым ён скаціўся па спіральнай лесвіцы, не чакаючы,
пакуль яна сама яго апусціць, праз секунду ён быў унізе, і гаргулья збочыла, і 
Гары выляцеў з нішы, быццам ядро з гарматы...


\later

Гары Потэр.

Было нешта гэткае ў Гары Потэры.

Бо чацьвер быў для ўсіх, у рэшце рэшт, але такія рэчы больш ні з кім не адбываліся.

Было 18:21, вечарам у чацьвер, калі Гары Потэр, выляцеўшы, быццам ядро з гарматы,
на ўсёй моцы ўрэзаўся ў Мінерву МакГонагал, якая як раз вывярнула з-за рогу,
накіроўваючыся да кабінета Майстра.  

Нашчасце, ніхто з іх моцна не пашкодзіўся. (Як было растлумачана Гары крыху райней,
на занятку па мётлам, бладжэры для для квіддзіча трэба было рабіць з суцэльнага 
кавалка жалеза, каб мець нават малейшыя шанцы параніць гульцоў, бо, як выявілася,
чараўнікі шмат больш трывалыя на сутыкнення, чым маглы.) Яны абодва ўпалі на падлогу,
і світкі, якія несла МакГонагал, разляцеліся па ўсяму калідору.

І павісла жудасная, жудасная пауза.

--- Гары Потэр, --- выдахнула прафесар МакГонагал, і тут яе голас перайшоў у віск:
--- \emph{Што вы рабілі ў кабінеце Майстра?!}

--- Нічога! --- ціўкнуў Гары.

--- \emph{Вы размаўлялі пра прафесара Абароны?}

--- Не! Дабмлдор выклікаў мяне сам, і потым даў мне вялікі камень, і сказаў, што 
той належыў майму бацьку, і што я павінны насіць яго паўсюль!

Была яшчэ адна жудасная пауза.

--- Зразумела, --- сказала МакГонагал болей спакойным голасам. Яна паднялася, 
змахнула пыл, і кінула позірк на раскіданыя світкі, якія адразу падпрыгнулі, і 
склаліся ў акуратны стос ўздоўж сцяны, быццам намагаючыся схавацца ад яе вачэй.

--- Прыміце мае спачуванні, містэр Потэр, і прабачце, што сумнявалася ў вас.

--- Прафесар, --- сказаў Гары няўпэўнена. Ён абаперся аб сцяну, падняўся, і паглядзеў 
на яе спакойны, \emph{цвярозы} твар. --- Прафесар МакГонагал...

--- Я слухаю, містэр Потэр.

--- Вы думаеце, мне варта? --- сказаў Гары ціха. --- Насіць бацькоўскі камень пры сябе?

МакГонагал уздыхнула. 

--- Баюся, тут я дапамагчы не магу, --- яна завагалася. --- Скажу толькі, што 
цалкам ігнараваць Майстра яшчэ нікога да дабра не даводзіла. Мне шкада чуць пра 
вашую дылему, містэр Потэр, і калі я магу нечым дапамагчы, калі вы адзначыцеся з 
выбарам...

--- Хм, --- сказаў Гары, --- дарэчы, я як раз думаў: калі я даведаюся, якім 
чынам, то было б няблага трансфігураваць камень у пярсцёнак, які я змагу насіць на пальцы. Калі 
вы можаце навучыць мяне падтрымліваць трансфігурацыю...

--- Добра, што спачатку вырашылі спытаць мяне, --- сказала МакГонагал, і яе 
твар стаў крыху стражэй. --- Калі бы вы страцілі кантроль, то адваротная трансфармацыя
адарвала бы вам палец, і, магчыма, палову рукі. У вашым узросце нават пярцёнак --- занадта
вялікі аб'ект, і любая працяглая трансфармацыя будзе траціць вашу магію вельмі хутка.
Але... я магу даць вам пярсцёнак, і вы зможаце практыкавацца трансфігураваць
у дыямент нейкі бясшкодны прадмет, скажам, зефірку... Калі вы здолееце ўтрымліваць 
форму прадмета ўвесь час --- і нават у сне --- на працягу, скажам, месяца, я 
дазволю вам трансфігураваць, о божа, ваш бацькоўскі камень... --- яна сціхла. 
--- Майстра праўда?...

--- Праўда. Эм...

Прафесар МакГонагал уздыхнула. 

--- Гэта крыху дзіўна нават па ягоным меркам, --- яна нахілілася і падняла стос 
світкаў. --- Прабачце мяне яшчэ раз за недавер. Але зараз мая ўласная чарга 
пабачыцца з Майстрам.

--- А... удачы, мабыць... эм...

--- Дзякуй, містэр Потэр.

МакГонагал падышла да гаргульі, нячутна сказала пароль, і ўзышла на спіральную лесвіцу.
Яна пачыла падымацца, і гаргулья пачала задвігацца назад...

--- \emph{Прафесар МакГонагал, Майстра спаліў курыцу!!!}

--- Ён \emph{ШТ!?...}
