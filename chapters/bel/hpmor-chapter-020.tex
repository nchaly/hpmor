\chapter{Тэарэма Баеса}

\lettrine{Г}{ары} глядзеў у шэрую столь маленькага пакоя, дзе ён ляжаў на 
даволі мягкай раскладушцы. Ён з'еў шмат з пачастункаў прафесара Квірэла ---
цукеркі з сумесі розных відаў шакаладу і чагосьці яшчэ, зверху абсыпаныя 
зіхоткай пудрай і аздобленыя цукровымі дыяментамі, якія выглядалі надзвычайна 
каштоўнымі, і апынуліся такімі ж на смак. Гары не адчуваў ні кроплі 
віны, бо ён гэта заслужыў.

Ён не спрабаваў спаць, бо падазраваў, што калі ён заплюшчыць вочы, у яго 
розуме адбудзецца нешта не вельмі прыемнае.

Ён не спрабаваў чытаць, бо ўсё роўна не змог бы сфакусавацца.

Цікава, што Гарын мозг думаў і думаў без перапынку,
не сцішаючыся ні на імгненне, як бы Гары ні намагаўся яго спыніць.
Чым далей, тым яго розум станавіўся дурней, але ён проста адмаўляўся \emph{выключацца.}

Але ў ім было, шчырае і сапраўднае, адчуванне трыумфу. 

Фраза "плюс адзін бал у праграмме Гары-анты-цёмны-лорд" не магла апісаць і 
сотай долі. Гары падумаў, што б сказаў Капялюш \emph{зараз}, калі ён мог бы 
яго надзець. 

Не дзіва, што прафесар Квірэл вінаваціў Гары ў тым, ён ішоў шляхам да Цёмнага Лорда.
Гары не зразумеў, хаця павінен быў убачыць паралель адразу: \emph{"...Цёмны Лорд 
\emph{не} перамог у той дзень. Яго мэтай было паступіць у навучанне
баявым майстэрствам, але ён пакінуў тое месца, не атрымаўшы ні воднага ўрока."}

Гары таксама прыйшоў на занятак па зеллеварэнню з мэтай вучыцца. І таксама пакінуў
кабінет, не атрымаўшы ўрок.

І прафесар Квірэл зразумеў, прабраўся ў розум Гары з гэтай жахлівай дасканаласцю,
і працягнуў руку, і выдраў Гары з гэтага шляха, шляха, які вёў Гары да 
пераўтварэння ў Самі-Ведаеце-Каго.

У дзверы пагрукалі. 

--- Заняткі скончыліся, --- сказаў ціхі голас прафесара Квірэла.

Гары ўстаў, накіраваўся да дзвярэй, і раптоўна адчуў нерваванае ўсхваляванне.
Потым напружанне крыху сціхла, калі ён пачуў крокі прафесара, які адышоў ад дзвярэй.

\emph{Ды што ж гэта такое адбываецца? Гэта неяк звязаны з тым, што ён будзе 
вымушаны сысці напрыканцы года?}

Гары адчыніў дзверы, і ўбачыў, што прафесар Квірэл стаіць метрах у пяці ад яго.

\emph{Ён таксама гэта адчувае?}

Яны дайшлі па зараз пустой сцэне да стала прафесара Квірэла, 
і Гары, як і раней, спыніўся, не заходзячы на платформу.

--- Так, --- сказаў прафесар Квірэл. Нейкім чынам, ягоны голас гучаў прыязна, 
хаця аблічча трымала яго звычайны сур'ёзны выраз. --- Што вы хацелі са мной 
абмяркаваць, містэр Потэр?

\emph{У мяне ёсць таямнічы цёмны бок.} Але Гары не мог вываліць гэта проста так.

--- Прафесар, --- сказаў Гары, --- я сышоў са шляху да Цёмнага Лорда? Пасля...

Прафесар Квірэл паглядзеў яму ў вочы. 

--- Містэр Потэр, --- сказаў ён урачыста, але з ледзьве заўважнай усмешкай, ---
вось вам крыху парады. Часам сваю акторскую гульню можна сапсаваць, зрабіўшы яе 
занадта ідэальнай. Ніхто ў сваім розуме пасля пятнаццаці хвілін ганьбы 
не падымецца, каб міласэрна прабачыць сваіх катаў. Рэч такога роду
зробіць нехта, хто адчайна \emph{намагаецца пераканаць} усіх у тым, 
што ён не Цёмны... 

--- \emph{Я не магу ў гэта паверыць! Вы не можаце перакручваць кожнае назіранне 
так, каб яно падтрымлівала вашу гіпотэзу!}

--- А гэта было \emph{крышаку} занадта пафасна.

--- \emph{Божа мой, што мне зрабіць, каб пераканаць вас?}

--- Пераканаць мяне, што вы не выношваеце планаў стаць Цёмным Лордам? --- 
сказаў прафесар Квірэл, па выгляду якога было зразумела, што яго гэта 
вельмі забавіць. --- Думаю, вам проста дастаткова падняць правую руку.

--- Што? --- сказаў бязглуда Гары. --- Але я магу падняць правую руку ў незалежнасці 
ад таго ці я... --- ён спыніўся, адчуваючы сябе няёмка.

--- Сапраўды, --- сказаў прафесар Квірэл. --- Вы лёгка можаце зрабіць гэта ў 
любым выпадку. Вы нічым не можаце мяне пераканаць, бо я буду ведаць, што вы 
зрабілі гэта, адмыслова намагаючыся мяне пераканаць. І калі вы хочаце 
дакладней: хаця я і думаю, што верагоднасць існавання ідэальна добрых людзей 
ледзьзе заўважная, але \emph{неймаверна,} каб нехта, каго білі пятнаццаць хвілін,
адчуў бы неверагодны прыліў дабрыні і даравання да сваіх катаў. З другога боку,
давайце ўявім, што гэты вучань успрымае гэта як \emph{ролю} дзеля 
пераканання настаўніка і сваіх аднагрупнікаў, што ён не ідзе шляхам Цёмнага Лорда.
Верагоднасць гэтага сцэнару ўжо не \emph{такая} малая, пагадзіцеся.
Важнасць любой ролі, містэр Потэр, не ў тым, як ваша ігра выглядае звонку, 
а ў стане вашага розуму, які робіць вашую ігру больш або менш верагоднай. 

Гары міргнуў. Ён толькі што атрымаў тлумачэнне розніцы паміж рэпрэзентатыўнасцю 
і Баесаўскім вызначэннем доказу \emph{ад чараўніка.}

--- Але, з трэцяга боку, --- працягваў прафесар Квірэл, --- людзі могуць жадаць
уразіць сваіх сяброў. Для гэтага не трэба быць Цёмным. Таму містрэр Потэр, 
скажыце мне шчыра. Аб чым вы думалі ў той момант, калі забаранілі чыніць помсту? 
Ці быў гэта сапраўдны імпульс дабрыні? Або вы думалі аб тым, як успрымуць 
ваш учынак аднагрупнікі?

\emph{Часам мы прыдумляем свае уласные спевы фенікса.}

Але Гары не сказаў гэта ўголас. Было ясна, што прафесар Квірэл яму не паверыць,
і магчыма, за спробу такой нахабнай хлусні, Гары страціць павагу настаўніка.

Праз некалькі секунд цішыні прафесар Квірэл задаволена ўсміхнуўся.

--- Хаціце верце, хаціце --- не, містэр Потэр, --- сказаў ён, --- але вам не 
варта баяцца мяне за тое, што раскрыў ваш сакрэт. Я \emph{не} збіраюся 
адгаворваць вас ад таго, каб стаць Цёмным Лордам. Калі б я мог адкруціць час назад,
і сцерці падобныя амбіцыі з розума таго вучння, якім я быў у школе, то я-сённяшні,
безумоўна, значна бы страціў ад такой перамены. Доўгі час я верыў, што гэта было 
маёй мэтай, яна заклікала мяне вучыцца, шукаць новыя тэхнікі, станавіцца 
лепей і мацней. Мы становімся тымі, хто мы ёсць, следуючы сваім самым 
патаенным жаданням, куды б яны ні вялі. У гэтым --- празарлівасць Салазара.
Нагадайце калісьці паказаць вам аддзел нашай бібліятэкі
з кнігамі, якімі я зачытваўся ў трынаццаць гадоў --- я з радасцю пакажу вам шлях.

--- Да чорт вас пабірай, --- сказаў Гары, сеў на цвёрдую мармуровую падлогу, 
потым лёг на спіну, гледзячы да аркі столі. Яшчэ крыху, і ён зваліцца ў 
незваротны адчай.

--- Усё яшчэ шмат пафасу, --- заўважыў прафесар Квірэл. Гары яго не бачыў,
але чуў у голасе стрыманы смех.

І тут Гары зразумеў.

--- Насамрэч, я думаю, я разумею, чаму вы так упэўнены, --- сказаў ён. --- 
Пра гэта я і хацеў пагаварыць. Прафесар, я думаю, што вы здолелі разглядзець 
мой таямнічы цёмны бок.

Была пауза.

--- Ваш... цёмны бок?..

Гары сеў. Прафесар Квірэл разглядваў яго з самым дзіўным выразам, які Гары толькі 
бачыў на тварах людзей у сваім жыцці.

--- Ён прасыпаецца, калі я злуюся, --- патлумачыў Гары. --- Мая кроў халаднее --- 
літаральна, і ўсё становіцца такім ясным... Рэтраспектыўна я разумею, што ён 
заўсёды быў са мной --- у першым класе маглаўскай школы нехта паспрабаваў 
забраць у мяне мячык, а я ўдарыў яго нагой у сонечнае спляценне, бо я чытаў, 
што гэта слабае месца, і астатнія дзеці больш мяне не турбавалі.
І яшчэ я ўкусіў настаўніцу матэматыкі, калі яна не прыняла маё дамінаванне.
І толькі ў Хогвартс падчас монцага стрэсу я стаў заўважаць, што гэта --- 
звычайны, ведаеце, таямнічы цёмны бок, гэта не праблема кіравання гневам, 
як казаў маглаўскі школьны псіхолаг. Але з цёмным бокам,
нажаль, да мяне не прыходзіць супер-магічная сіла, я праверыў гэта першай 
справай.

Прафесар Квірэл пацёр пераносіцу.

--- Дайце мне падумаць аб гэтым, --- сказаў ён.

Гары чакаў моўчкі цэлую хвіліну. Ён скарыстаў гэты час, каб падняцца на ногі, 
што апынулася складаней, чым яму падавалася.

--- Ладна, --- сказаў прафесар Квірэл. --- Я думаю, \emph{ёсць} нешта, 
што вы можаце сказаць, каб пераканаць мяне.

--- Я ўжо зразумеў, што мой цёмны бок --- проста іншая мая частка, і 
адказ тут не тое, што я ніколі не павінен злавацца, а прыняць яго, і 
навучыца жыць з ім, і трымаць яго пад кантролем. Я сустракаў такія гісторыі
шмат разоў, і магу ўцяміць, што мне рабіць, але гэта цяжка, і вы, як мне 
падаецца, зможаце мне дапамагчы.

--- Хм... так... вельмі пранікліва, містэр Потэр. Я павінен адзначыць...
гэты ваш бок, як вы, напэўна, ужо зразумелі, і ёсць ваш смяротны намер, і ён,
як вы сказалі, таксама ваша частка...

--- ...і таксама патрабуе трэніроўкі, --- скончыў сказ Гары.

--- Так, патрабуе, --- на твары настаўніка ўсё яшчэ быў дзіўны выраз. --- 
Містэр Потэр, калі вы шчыра не жадаеце стаць Цёмным Лордам, тады якую, 
на вашае меркаванне, уласцівасць убачыў у вас Размеркавальны Капялюш, уласцівасць,
якая прымусіла яго накіраваць вас на Слізэрын?

--- Мяне размеркавалі на \emph{Рэйвенкло!}

--- Містэр Потэр, --- сказаў прафесар Квірэл, ужо са сваёй звычайнай сухаватай 
усмешкай, --- я ведаю, вы звыкліся, што па большасці вас абкружаюць ідыёты, але, калі 
ласка не рабіце памылку, далучаючы мяне да гэтай катэгорыі. 
Імавернасць таго, што Капялюш разыграў свой першы за васемсот гадоў пранк 
у момант, калі ён апынуўся на вашай галаве --- насколькі малая, што не вартая 
ніякай увагі. Таксама я не надта веру, што вы маглі цокнуць пальцамі, і
за секунду прыдумаць, як абысці абарончыя заклёны, накладзеныя на Капялюш.
Нават я не ведаю такіх метадаў. Паверце, самае верагоднае тлумачэнне --- што 
Дамблдор вырашыў, што ён не вельмі задаволены выбарам Капелюша на конт 
Хлопца-які-выжыў. Гэта відавочна любому, хто мае хаця бы грам жыццёвага разумення,
і таму ў Хогвартс ваш сакрэт у поўнай бяспецы.

Гары раскрыў рот, і з адчуваннем поўнай бездапаможнасці закрыў. Прафесар Квірэл
памыляўся, але памыляўся так пераканаўча, што Гары пачынаў думаць, што 
ён проста "рацыянальна ацэньвае сітуацыю, ўлічваючы ўсе наяўныя фактары".
Часам --- немагчыма было прадказаць, калі гэта менавіта здарыцца, --- але часам 
ты мог атрымаць пацьвержанне нават для памылковай тэорыі, або самая верагодная здагадка 
можа быть неправільнай. Калі ў цябе ёсць медыцынскі тэст, які памыляецца толькі 
раз на тысячу, усё роўна --- часам ён будзе памыляцца для цябе.

--- Магу я спытаць вас трымаць у сакрэце тое, што я зараз скажу? --- спытаў Гары.

--- Безумоўна, --- сказаў прафесар. --- Лічыце, што ўжо спыталі.

Гары не быў прастачком таксама.

--- Мачу я лічыць, што адказалі пазітыўна?

--- Вельмі добра, містэр Потэр. Канешне, можаце лічыць так.

--- \emph{Прафесар!...}

--- Я буду трымаць у сакрэце тое, што вы зараз скажаце, --- сказаў прафесар Квірэл з 
усмешкай.

Яны абодва засмяяліся, і потым Гары зноў стаў сур'ёзны.

--- Размеркавальны Капялюш, падавалася, сапраўды думаў, што я ў выніку стану 
Цёмным Лордам, толькі калі не пайду на Хафлпаф, --- сказаў Гары. --- Але 
я не \emph{хачу} ім станавіцца.

--- Містэр Потэр, --- сказаў прафесар Квірэл. --- Не ўспрымайце гэта няправільна ---
і я абяцаю не ставіць адзнаку за гэтае пытанне, бо я проста хачу пачуць ваш 
шчыры адказ. Чаму не?

Гары зноў адчуў бездапаможнасць. \emph{Не забівай, не крадзі, не станавіся Цёмным Лордам}
было падмуркам маральнай сістэмы любога нармальнага чалавека, і было складана 
абгрунтоўваць настолькі відавочныя рэчы.

--- Эм... могуць пацярпець людзі?

--- Відавочна, вам знаёма жадане, каб некаторыя людзі сапраўды пацярпелі, --- сказаў прафесар Квірэл.
--- Вы хацелі, каб тыя сённяшнія слізэрыны пакутвалі. Быць Цёмным Лордам значыць,
што пацярпелымі стануць тыя, каго ты \emph{запісаў} у гэты спіс.

Гары павагаўся, падбірачы словы, і ўрэшце вырашыў сказаць відавочнае.

--- Па-першае, калі \emph{я} хачу камусці нашкодзіць, гэта не значыць, што гэта 
правільна...

--- Што робіць учынкі правільнымі, як не вашае жаданне?

--- А, --- сказаў Гары, --- прэферэнцыяльны ўтылітарызм.

--- Прабачце? --- сказаў прафесар Квірэл.

--- Гэта этычная тэорыя, паводле якой добрыя рэчы --- гэта тое, што задавальняе жаданням большасці людзей.

--- Не, --- сказаў прафесар Квірэл. Ён пацёр пальцем пераносіцу. --- Не думаю, 
што я меў на ўвазе такое. Містэр Потэр, у рэшце рэшт, людзі 
робяць тое, што хочуць. Часам яны называюць свае ўчынкі "добра" ці "правільна", 
але ў выніку на што нам яшчэ абапірацца ў нашых учынах, як не на нашыя жаданні?


--- Відавочна, --- сказаў Гары. --- Я бы не мог дзейнічаць згодна з нейкай 
маральнай сістэмай, якой я не веру. Але гэта не значыць, 
што мае жаданне нашкодзіць слізэрынам мацней за мае маральныя прынцыпы!

Прафесар Квірэл міргнуў.

--- Не кажучы, --- працягваў Гары, --- што быць Цёмным Лордам таксама значыць,
што шмат нявінных простых людзей таксама могуць пацярпець!

--- Чаму гэта вас хвалюе? --- сказаў прафесар Квірэл. --- Што яны вам зрабілі?

Гары засмяяўся.

--- О, а гэта было так сама тонка, як і ў \emph{"Атлант расправіў плечы".}

--- Прабачце? --- зноў сказаў прафесар Квірэл.

--- Гэта кніга, якую бацькі не давалі мне чытаць, яны думалі, што яна мяне 
"сапсуе", і канешне, я яе прачытаў употай, і мяне абурыла, што яны думалі я 
пападуся ў такія відавочныя пасткі. Бла-бла-бла, "я лепей за шэрую масу",
"яны спрабуюць мяне кантраляваць", і гэтак далей.

--- Я так разумею, што мнэ трэба зрабіць мае пасткі меней відавочнымі? --- 
сказаў прафесар Квірэл. Ён задумённа пацёр падбародак. --- Над гэтым я магу 
папрацаваць.

Яны абодва засмяяліся.

--- Але, калі вярнуцца да майго пытання, --- сказаў прафесар Квірэл, --- 
\emph{што} ўсе гэтыя незнаёмыя вам людзі зрабілі для вас?

--- Незнаёмыя людзі зрабілі для мяне \emph{процьму} рэчаў! --- сказаў Гары. 
--- Калі мае бацькі памерлі, мяне ўзялі на выхаванне незнаёмыя людзі, таму што 
яны \emph{добрыя}, і стаць Цёмным Лордам будзе здрадзіць ім!


Прафесар Квірэл некаторы час маўчаў.

--- Прызнаюся, --- сказаў ён ціха, --- калі я быў у вашым узросце, падобная думка
ніколі не магла прыйсці мне да галавы.

--- Мне шкада, --- сказаў Гары.

--- Не шкадуйце, --- сказаў прафесар Квірэл. --- Гэта было даўно, і я здавальняючы 
вырашыў праблемы з бацькамі. Дык значыць вас утрымліваюць думкі аб 
бацькоўскай рэакцыі? Ці значыць гэта, што 
калі б яны раптоўна памерлі ў выніку няшчаснага выпадку, не застанецца нічога паміж 
вамі і...

--- Не, --- сказаў Гары. --- Зусім не так. Іх \emph{імкненне да дабрыні} --- вось 
што мяне трымае. І гэтае імкненне не толькі ўласціва маім бацькам. І гэтае 
імкненне будзе здраджана.

--- У любым выпадку, містэр Потэр, вы не адказалі на мае пачатковае пытанне, --- 
сказаў прафесар Квірэл. --- Якія ў вас амбіцыі?

--- О, --- сказаў Гары, --- эм-м-м, --- ён прывёў думкі ў адносны парадак. --- 
Зразумець пра сусвет усё, што варта пра яго ведаць, прымяніць гэтыя веды, каб
стаць усемагутным, і скарыстаць гэтую сілу на змяненне рэальнасці, бо 
ў мяне ёсць некаторы пярэчанні на конт таго, як яно працуе зараз. 


--- Прабачце мяне, калі гэта дурное прытанне, містэр Потэр, --- пасля невялікай 
паузы сказаў прафесар Квірэл, --- але вы \emph{ўпэўнены}, што толькі што не 
прызналіся ў жаданні стаць Цёмным Лордам?

--- Гэта толькі калі карыстаць сілу на злыя рэчы, --- патлумачыў Гары. --- 
Калі карыстаць яе на дабро, то будзеш Светлым Лордам.

--- Зразумела... --- сказаў прафесар Квірэл. Ён задумённа пацёр пальцамі падбародак.
--- Думаю, з гэтым можна працаваць. Але, містэр Потэр, хоць вашыя амбіцыі 
і вартыя самога Салазара, як менавіта вы збіраецеся ажыццяўляць іх? Які першы крок?
Стаць баявым магам? Кіраўніком Аддзела Таямніц? Міністрам Магіі, або... 

--- Першы крок --- стаць вучоным.

Прафесар Квірэл паглядзеў на Гары так, нібыта той пераўтварыўся ў котку.

--- Вучоным, --- сказаў Квірэл праз некаторы час.

Гары кіўнуў.

--- \emph{Вучоным?} --- паўтарыў прафесар Квірэл.

--- Так --- сказаў Гары. --- Я збіраюся ажыццявіць свае мэты праз моц \emph{Навукі!}

--- \emph{Навукі!} --- паўтарыў прафесар Квірэл. На яго твары быў выраз шчырага 
абурэння, ягоны голас стаў мацней і рэзчэйшым. --- Вы маглі бы стаць найлепшым
з усіх маіх студэнтаў! Найвялікшым баявым магам-выпускніком Хогвартс за апошнія 
пяцьдзясят гадоў! Я не магу ўявіць сабе, як вы марнуеце дзень за днём у белым халаце,
рэжучы пацукоў!

--- Гэй! --- сказаў Гары. --- Навука --- гэта значна больш, чым рэзаць пацукоў! 
Канешне, ў эксперыментах над пацукамі няма нічога дрэннага. Але навука --- 
гэта сапраўдны спосаб зразумець і кантраляваць сусвет... 

--- Дурасць, --- сказаў Квірэл голасам, поўным ціхай, але горкай моцы. --- І вы 
дурань, містэр Потэр, --- ён правёў рукой па сваім твары, і паспакайнеў. --- Або 
дакладней, вы яшчэ не знайшлі сваю сапраўдную амбыцыю. Магу я параіць вам замест 
усёж-такі паспрабаваць стаць Цёмным Лордам? Са свайго боку абяцаю дапамагаць вам 
ва ўсім --- у якасці паслугі для грамадства.

--- Вы не любіце навуку, --- сказаў павольна Гары. --- Чаму?

--- Гэтыя дурні-маглы калісці заб'юць нас усіх! --- сказаў раздражнёна прафесар. 
--- Канец Зямлі! Канец усяму!

Гары крыху згубіўся.

--- Вы маеце на ўвазе ядзерную зброю?

--- \emph{Натуральна,} ядзерную зброю! --- прафесар Квірэл ужо амаль крычаў. ---
Нават Той-каго-нельга-называць не карыстаў яе, магчыма таму, што ён не хацеў 
правіць кучкай попелу! Не варта было яе прыдумляць! І з часам усё будзе толькі 
горш! --- прафесар Квірэл перастаў абапірацца аб стол і стаў роўна. --- Ёсць дзверы,
якія не варта адчыняць, печаці, якія не варта ламаць. Як толькі першыя з тых, хто сунулі
свае насы куды не трэба, памерлі, астатнія вучоныя павінны былі адразу зразумець,
што ў іх руках апынуліся сакрэты, якімі \emph{нельга} дзяліцца з тымі,  
каму не хапае розума і дысцыпліны, каб самім зрабіць такое вынаходства! 
Любы нармальны чараўнік гэта ведае! Нават самы жудасны Цёмны Лорд гэта ведае.
А ідыётам-маглам такая думка ў галаву не прыйшла! Дурні, якія адкрылі сакрэт ядзернай
зброі, не здолелі трымаць роты зачыненымі. Як толькі пра яе празнаў першы 
\emph{дурань-палітык}, мы ўсе павінны жыць у пастаяннай пагрозе анігіляцыі!

Такі погляд даволі моцна адрознівіўся ад ідэй, на якіх выхоўвалі Гары. Яму ніколі 
не прыхадзіла да галавы, што ядзерныя фізікі павінны былі стварыць загавар,
і трымаць ядзерную зброю ў сакрэце ад усіх, хто недастаткова разумны, каб 
стаць ядзерным фізікам. Думка інтрыгавала, што і казаць. У іх былі бы 
паролі і сакрэтныя сходы? Яны насілі бы маскі?

(Насамрэч, наколькі яму было вядома, навукоўцы \emph{сапраўды} хавалі ад людзей
шмат якіх пагрозлівых разбуральных тэхналогій, і ядзерная зброя было адзінай, якая ўцякла ў 
публічную прастору. Свет выглядаў бы для Гары такім самым, нават, калі яна б і 
не ўцякла.)

--- Мне трэба пра гэта падумаць, --- сказаў Гары. --- Гэта для мяне новая ідэя. 
Бо адзін з \emph{сапраўдных} галоўных сакрэтаў навукі, які стагоддзі перадаваўся ад рэдкіх 
настаўнікаў вучням --- пачуўшы ідэю, якая табе не падабацца, не змываць яе ва ўнітаз 
у той жа момант.

Прафесар Квірэл зноў міргнуў.

--- Ці ёсць некая часта навукі, з якой вы згодны? --- спытаў Гары. --- Медыцына,
напрыклад?

--- Касмічная тэхніка, --- сказаў прафесар Квірэл. --- Але, падаецца, што яны 
наўмысна зацягваюць з гэтым адзіным праектам, які можа дазволіць роду чараўнікоў 
уцячы, пакуль яны не ўзарвалі ўсю планету.

--- Я таксама фанат касмічнай праграмы, --- кіўнуў Гары. --- Прынасмі гэта ў нас 
агульнае.

Прафесар Квірэл паглядзеў на Гары. Нешта бліснула ў ягоных вачах.

--- Вы дадзіце мне слова, абяцанне, і клятву трымаць у сакрэце тое, што будзе
зараз.

--- Я даю, --- паспешна сказаў Гары.

--- Трымайце слова, інакш вынік вам не спадабаецца, --- сказаў прафесар Квірэл.
--- Зараз я зкастую рэдкі і моцны заклён, не на вас, але на аўдыторыю, дзе мы
знаходзімся. Калі яно пачнецца, стойце на месцы, каб не парушыць межы заклёну.
Вы павінны ніяк не ўзаемадзейнічаць з маёй магіяй. Толькі глядзець. 
Інакш я адразу спыню заклён, --- ён зрабіў паузу. --- І пастарайцеся не ўпасці.

Гары кіўнуў, не разумеючы, але ў нецярплівам чаканні.

Прафесар Квірэл падняў палачку і сказаў нешта, што ні вушы, ні розум Гары не здолелі
ўспрыняць, словы, якія прайшлі праз свядомасць, і згінулі ў забыцці.

Усё вакол, акрамя невялікага круга мармура пад нагамі Гары, знікла ў цемры ---
падлога, сцены, столь.

Гары знаходзіўся пасярэдзіне бяскоцай прасторы, поўнай зор, якія свяцілі жахліва 
моцна і не міргаючы. Не было ні Зямлі, ні Луны, ні Сонца. Прафесар Квірэл 
знаходзіўся на той жа адлегласці ад Гары, што і ў аудыторыі, таксама ў сярэдзіне 
зорнай бездані. Млечны Шлях выбіваўся патокам святла, і праз некалькі секунд 
стаў яшчэ яскравей, калі вочы Гары звыкліся з цемрай. 

Відовішча кранула нешта ў Гарыным сэрцы так, як нішто дагэтуль.

--- Мы... ў космасе?

--- Не, -- сказаў прафесар Квірэл. Ягоны голас быў сумным і ўрачыстым. --- Але 
выява сапраўдная.

На вачах Гары з'явіліся слёзы. Ён хутка іх выцер, бо не збіраўся губляць ні 
секунды відовішча на гэтую дурную ваду, якая размывала яму зрок.

Зоры персталі быць зіхоткімі кропкамі, раскіданымі па аксамітаваму купалу, як гэта 
бачна ноччу з Зямлі. Тут не было неба над галавой, не было купала. Проста ідэальнае
святло супраць ідэальнай цемры, бяскоцае і пустое нічога з незлічонай колькасьцю
дзірачак, праз якія праходзіў бляск нейкага іншага неймавернага сусвета. 

У космасе зокркі падаваліся жудасна далёка.

Гары працягваў выціраць вочы, зноў і зноў.

--- Часам, --- пачуўся голас прафесара Квірэла, такі ціхі, амаль неіснуючы, ---
калі наш нікчэмны свет падаецца мне надзвычайна агідным, я задумваюся, можа быць 
існуе нейкае далёкае месца, дзе я ў гэты момант павінен быць. І я нават не магу 
ўявіць сабе, што гэта можа быць за месца, і калі я не магу яго ўявіць, тады як 
я магу верыць у яго існаванне? Але сусвет такі неабсяжны... і магчыма дзесьці 
яно ўсё-такі існуе? Але зоркі вельмі, вельмі далёка. Нават калі б я і ведаў 
дарогу, гэта зойме так шмат часу... Што будзе мне сніцца, калі я засну на 
доўгі, вельмі доўгі час? 


Парушаць цішыню падавалася святатацтвам, але Гары здолеў прашаптаць.

--- Калі ласка, дазвольце мне яшчэ...

Прафесар Квірэл кіўнуў са свайго месца, дзе ён вісеў паміж зор.

Было лёгка забыцца пра невялікае кола мармуру, на якім ты стаіш, пра свае цела,
і проста стаць кропкай свядомасці, якая магла рухацца, або вісець нерухома. 
З гэтымі невыразна вялікімі адлегласцямі было цяжка сказаць.

Быццам нават час перастаў існаваць.

І потым зоркі зніклі. Яны зноў былі ў аудыторыі.

--- Выбачайце, --- сказаў прафесар Квірэл, --- але ў нас госці.

--- Нічога, --- прашаптаў Гары. --- Гэтага было дастаткова.

Ён ніколі не забудзе гэты дзень, і не за тое, што здарылася раней. Ён навучыцца 
гэтаму заклёну, нават калі гэта будзе апошнім, чаму ён навучыцца. 

У гэты момант цяжкія дубовыя дзверы аудыторыі зляцелі з пяцель і з жудасным 
віскам праехаліся па мармуровай падлозе. 


--- \shout{Квірынус! Як вы смееце!}

Як цёмны шторм, стары і магутны чараўнік уляцеў у аудыторыю з такой распаленай 
лютасцю, што ранішні строгі выгляд у параўнанні падаваўся Гары проста лёгкай хмурынкай.

Асноўная частка Гарынага розуму жадала ўбяжаць як мага далей ад гэтага шторма,
але ў Гары было і нешта, што магло перажыць гэты шок, і ўтрымала яго на месцы.
Агульна, усе часкті ягонага розуму былі вельмі незадаволены перарваным зорным 
відовішчам.

--- Майстра Альбус Персіваль... --- пачаў Гары сваім ледзяным тонам.

\emph{Бах!} --- моцна жмахнула далонь прафесара Квірэла па стале. 

--- \emph{Містэр Потэр!} --- гыркнуў ён. --- Перад вамі сам Майстра Хогвартс,
а вы ўсяго толькі студэнт. Вы будзеце звяртацца да яго з усёй належнай павагай!

Гары паглядзей на Квірэла. Той глядзеў на яго строгім позіркам.

Ніхто з іх не ўсміхаўся.

Дамблдор шырокімі крокамі падышоў да платформы, і спыніўся, з пэўным здзіўленнем 
разгледжваючы абодвух.

--- Прашу прабачэння, --- сказаў Гары значна больш ветлівым тонам. --- Майстра,
дзякуй за вашае жаданне абараніць мяне, але прафесар Квірэл усё зрабіў правільна.

Вельмі павольна позірк Дамблдора змяніўся з нечага, што магло выпар\'аць сталь,
да проста гнеўнага стану.

--- Да мяне дайшлі чуткі, што гэты чалавек абразіў вас, выкарыстоўваючы старэйшых
слізэрынаў! Што ён забараніў вам адбівацца!

Гары кіўнуў.

--- Ён дасканала зразумеў маю праблему, і паказаў, як яе вырашыць.

--- Гары, \emph{што такое ты гаворыш?}

--- Я навучыў яго, як праігрываць, --- суха сказаў прафесар Квірэл. --- Гэта 
важны жыццёвы ўрок.

Было бачна, што Дамблдор усё яшчэ не разумее, але ягоны голас крыху сцішыўся. 

--- Гары... --- сказаў ён павольна, --- калі прафесар Абароны пагражаў табе за тое,
што ты паскардзішся...

\emph{Дурань, няўжо пасля ўсіх сённяшніх падзей ён сапраўды...}

--- Майстра, --- сказаў Гары, спрабуючы прыняць здзіўлены выгляд, --- 
вы, напэўна, ужо зразумелі, як я рэагую на аб'юзіўных настаўнікаў.

Прафесар Квірэл усміхнуўся. 

--- Да ідэала, канешне, далёла, але зусім не дрэнна для першага дня. Майстра, 
ці засталіся вы да моманта, калі я выдаў пяцьдзесят адзін бал Рэйвенкло, або 
вас хапіла толькі на першую частку?

На імгненне выгляд Дамблдора стаў збятэжаны, потым яго змяніла здзіўленне. 

--- Пяцьдзесят адзін бал Рэйвенкло?

Прафесар Квірэл кіўнуў.

--- Для яго гэта было нечакана, але мне падалося гэта будзе дарэчна. Перадайце 
прафесару МакГонагал не толькі гэты факт, але і тое, праз што прайшоў містэр Потэр,
каб іх вярнуць. Не, Майстра, містэр Потэр не расказваў мне нічога. Нескладана 
здагадацца, на якую частку сённяшніх падзей яна паўплывала, такім жа чынам, як 
і тое, што фінальны кампраміс быў вашай ідэяй. Але ўсё роўна я не перастаю здзіўляцца,
як містэр Потэр здолеў перамагчы абодвух вас і Снэйпа, і ў выніку прайграў МакГонагал.

Неверагодным чынам Гары здолеў утрымаць покер-фэйс. Няўжо ўсё было \emph{настолькі}
відавочна для сапраўдных слізэрынаў?

Дамблдор падышоў да Гары бліжэй, і ўгледзеўся.

--- Ты крыху збялелы, Гары, --- сказаў стары, прыплюшчваючы вочы. --- Што ты еў 
сёння на абед?

--- Што? --- розум Гары забіўся ў раптоўнай збянтэжанасці. З чаго бы гэта Дамблдору
цікавіцца запечаным ягнём з брокалі, бо гэта магло быць \emph{апошнім} у спісе 
рэчаў, якія мелі дачыненне да... 

Старажытны чараўнік выпраміўся.

--- Ну, не --- так не, --- сказаў ён. --- Думаю, ты і праўда ў поўным парадку.

Прафесар Квірэл кашлянуў, чотка і гучна. Зірнуўшы на яго, Гары ўбачыў, што прафесар
глядзіць на Дамблдора з рэзкім дакорам.

--- \emph{Ах-хым!} --- зноў ненатуральна кашлянуў Квірэл.

Ён і Дамблдор свідравалі адзін аднаго позіркамі, быццам паміж імі адбываўся 
незаўважны двубой.

--- Калі вы яму не раскажаце, --- сказаў урэшце Квірэл, --- тады я раскажу, 
нават калі вы мяне за гэта звольніце.

Дамблдор уздыхнуў і звярнуўся да Гары.

--- Я прашу прабачэння за тое, што ўварваўся ў вашу ментальную прыватнасць,
містэр Потэр, --- сказаў фармальнай мовай Майстра. --- У мяне не было ніякага іншага 
замыслу, як высветліць, ці не зрабіў прафесар Квірэл таго ж самага. 

\emph{Што?}

Збянтэжанасць Гары доўжылася роўна сколькі, колькі яму спатрэбілася зразумець,
што зараз адбылося.

--- \emph{Вы!...}

--- Паветлівей, містэр Потэр, --- сказаў Квірэл, аднак ягоны выраз усё яшчэ быў 
цверды, і ён усё яшчэ глядзеў на Дамблдора.

--- Леджэламанцыю часта блытаюць са звычайным разуменнем, --- сказаў Дамблдор. ---
Але яна пакідае сляды, якія можна высветліць іншы здольны леджэламансер. 
Я шукаў толькі тыя сляды, містэр Потэр, і я задаў вам недарэчнае пытанне,
каб упэўніцца, што вашая галава не будзе занятая важнымі рэчамі ў гэты час.

--- \emph{Вам трэба было спачатку спытаць!}

Прафесар Квірэл пакачаў галавой. 

--- Не, містэр Потэр, тут Майстра мае рацыю. Калі б ён першай справай запытаў 
ваш дазвол, то вы бы толькі і думалі аб рэчах, якія хацелі бы трымаць у сакрэце,
--- ягоны голас стаў разчэй. --- Але мяне больш турбуе, Майстра, што вы не бачылі 
неабходнасці расказаць яму пра гэта пасля!

--- Вы зараз непамерна ўскладнілі мне будучыя спробы пацвердзіць ментальную некранутасць
Гары, --- адказаў Дамблдор, вяртаючы прафесару Квірэлу такі ж халодны позірк. --- 
Цікава, з якім намерам? 

Выраз Квірэла быў непарушны.

--- У гэтай школе занадта шмат леджэламансераў. Я настойваю на тым, што містэр 
Потэр павінен прайсці курс аклюманцыі. Ці дазволіце вы мне быць ягоным трэнерам?

--- Ні ў якім разе, --- адразу сказаў Дамблдор.

--- Я так і думаў. Калі \emph{вы} пазбавілі яго маіх бескаштоўных паслугаў, 
тады \emph{вы} і павінны заплаціць за курс містэра Потэра ў 
ліцэнзаванага трэнера. 

--- Такія заняткі --- рэч не танная, --- сказаў Дамблдор, с пэўным 
здзіўленнем гледзячы на прафесара Квірэла. --- Хаця, у мяне ёсць некаторыя сувязі...

Прафесар Квірэл выразна пакачаў галавой.

--- Не. Містэр Потэр запытае свайго акаунт-менеджэра ў Грынготс парэкамендаваць 
яму незалежнага інструктара. З усёй павагай, Майстра, але пасля падзей сённяшняй
раніцы я супраць таго, каб вы, або нехта з вашых сяброў мелі доступ да 
розуму містэра Потэра. Я таксама вымушаны настойваць, каб трэнер-аклюмансер 
даў непарушны зарок трымаць заняткі ў сакрэце, і каб ён пагадзіўся прыняць 
забывальны заклён пасля кожнага ўрока.

Дамблдор нахмурыўся.

--- Такія паслугі неверагодна каштоўныя, які вы выдатна ведаеце, і я не магу не 
задумвацца аб прычынах, чаму вы на іх так настойваеце.

--- Калі пытанне ў грошах, --- уваткнуў Гары, --- у меня ёсць некалькі ідэй, як 
хутка вырабляць вялікія...

--- Дзякуй, Квірынус, твая мудрасць мне цяпер стала цалкам зразумелай, і я выбачаюся
за свае ранейшыя сумневы. Таксама твой клопат аб Гары Потэры дадае табе павагі.

--- Калі ласка, --- сказаў прафесар Квірэл. --- Я спадзяюся, вы не будзеце супраць,
калі я працягну трымаць яго пад маёй пільнай увагай, --- яго выраз быў вельмі 
сур'ёзны.

Дамблдор глянуў на Гары.

--- Я далучаюся, --- сказаў Гары.

--- Значыць, вось як яно павінна стацца... --- сказаў стары чараўнік задумённа. 
Нешта дзіўнае праляцела на ягоным твары. --- Гары... ты павінен зразумець, што калі 
ты абярэш яго, як свайго ментара і сябра, свайго першага настаўніка, то калі ў 
будучыні з нім нешта здарыцца, то магчыма, ты ніколі не здолееш вярнуць яго назад.

Гары спачатку нічога не зразумеў, але праз секунду яго праняло. Ён зусім забыў 
пра праклён пасады прафесара Абароны... якое належна 
працавала вось ужо колькі год...

--- Гэта магчыма, --- сказаў ціха прафесар Квірэл, --- але пакуль я тут, 
мы будзем карыстаць мяне па максімуму.

Дамблдор уздыхнуў.

--- Думаю, што гэта рэзонна, прынамсі, бо як прафесар Абароны, вы ўжо асуджаны на 
нешта невядомае.

Гары патрабаваліся ўсе сілы, каб не выдаць здзіўленне, калі ён зразумеў, што менавіта
меў на ўвазе Дамблдор.

--- Я праінфармую мадам Пінц, што містэру Потэру дазваляецца браць кнігі па 
аклюманцыі, --- сказаў Дамблдор.

--- У іх ёсць кароткі ўводны курс, які вы павінны прайсці самастойна, --- сказаў 
Квірэл Гары. --- І я прапаную з ім не зацягваць.

Гары кіўнуў.

--- Тады я вас пакідаю, --- сказаў Дамблдор, кіўнуў абодвум Гары і Квірэлу,
і накіраваўся да дзвярэй, крыху павольней, чым звычайна.

--- Вы можаеце яшчэ раз зкаставаць той заклён? --- сказаў Гары, як толькі Дамблдор знік.

--- Не сёння, --- сказаў ціха прафесар, --- і, баюся, не заўтра таксама. Яно патрабуе
шмат сіл, каб пачаць, але падрымліваць яго лягчэй, таму я звычайна намагаюся 
трымаць яго як мага даўжэй. Гэтым разам я зкаставаў яго нечакана нават для сябе.
Калі б я толькі падумаў, што нас могуць перарваць...

У гэты момант Дамблдор быў самым нелюбімым чалавекам для Гары ва ўсім свеце.

Абодва ўздыхнулі.

--- Нават калі тое быў і адзіны раз, --- сказаў Гары, --- я ніколі не перастану 
быць вам удзячным.

Прафесар Квірэл кіўнуў.

--- Вы чулі аб праграме "Піянер"? --- спытаў Гары. --- Гэта зонды, якія 
павінны былі праляцець побач з рознымі планетамі, і зрабіць іх фотаграфіі.
Два з іх апынулся на траекторыях, якія дазволілі ім пакінуць сонечную сістэму,
і накіравацца ў адкрыты космас. І вучоныя паклалі ў іх залатыя пласцінкі з выявамі 
мужчыны і жанчыны, і карту, як знайсці Сонца ў нашай галактыцы.

Некалькі секунд прафесар Квірэл маўчаў, потым усміхнуўся. 

--- Скажыце мне, містэр Потэр, ці можаце вы здагадацца, якая думка прыйшла мне 
да галавы, калі я скончыў свой ліст з трыццаці сямі рэчаў, якія я ніколі не 
зраблю, будзь я Цёмным Лордам? Пастаўце сябе на мае месца, і скажыце, што 
прыйдзе вам да галавы.

Гары ўявіў, як ён агледжвае ліст з трыццаці сямі рэчаў.

--- Вы падумалі, што калі заўсёды выконваць \emph{усе} забароны \emph{ўвесь} час,
то сэнсу станавіцца Цёмным Лордам не вельмі шмат.

--- \emph{Дакладна}, --- сказаў прафесар Квірэл, усміхаючыся. --- Таму я парушу 
правіла нумар два --- "не выхваляйся", і раскажу вам пре нешта, што я зрабіў. 
Тым больш, што мне гэта ніяк не пашкодзіць. І думаю, вы і так калісьці здагадаліся бы. 
У любым выпадку, вы паклянецеся не казаць нікому 
пра тое, што пачуеце.

--- Натуральна! --- у Гары было прадчуванне, што гэта будзе нешта цудоўнае.

--- Я падпісаны на маглаўскі часопіс, які інфармуе мяне аб прагрэсе ў галіне 
даследвання космасу. Я прапусціў падрыхтоўку запуска Піянера 10, бо 
пра гэта напісалі надта позна. Але пачуўшы, што і Піянер 11 пакіне межы сонечнай сістэмы назаўсёды,
--- Гары ніколі не бачыў яго з насколькі шырокай усмешкай, --- я прабраўся 
ў НАСА, праўда, і зкаставаў невялічкі прыгожы заклён на тую прыгожую залатую 
пласцінку, якое павялічыць тэрмін яе існавання ў шмат разоў.

…

…

…

--- Менавіта, --- сказаў прафесар Квірэл, які меў выгляд, быццам стаў на 
дваццаць метраў вышэй, --- менавіта так я і ўяўляў вашу рэакцыю.

…

…

…

--- Містэр Потэр?

--- ...я не ведаю, што сказаць.

--- "Вы перамаглі" было б даволі слушна, --- сказаў прафесар.

--- Вы перамаглі, --- сказаў Гары.

--- Бачыце, як проста, --- сказаў прафесар Квірэл. --- Мы толькі можам 
гадаць, колькі ў вас з'явілася бы праблем, калі б вы не здолелі гэта вымавіць.

Абодва засмяяліся.

--- А вы часам не пакінулі на пластінцы нейкай дадатковай інфармацыі? --- прыйшла 
Гары да галавы лагічная думка. 

--- Дадатковай інфармацыі? -- спытаў прафесар Квірэл, быццам такая ідэя была для 
яго неваргодна новай і інтрыгуючай.

Што вызывала ў Гары падазрэнне, бо самому Гары патрабавалася б меньш за хвіліну,
каб падумаць пра гэта.

--- Можа быць, вы пакінулі там галаграфічнае паведамленне, як ў \emph{Зорных войнах?} ---
спытаў Гары. --- Або... хм. Выяву, якая магла ўмесціць інфармацыю чалавечага 
мозга цалкам?.. Вы не маглі дадаць лішняй масы да зонда, магчыма вы маглі 
пераўтварыць ужо існуючую частку карабля ў партрэт самога сябе? Або знайшлі 
валанцёра ў тэрмінальнай стадыі, тэлепартавалі яго ў НАСА, і зрабілі заклён,
каб удрукаваць ягонага \emph{прывіда} у пласцінку...  

--- Містэр Потэр, --- сказаў прафесар Квірэл, нечакана рэзка, --- заклён, які 
патрабуе чалавечай смерці, безумоўна класіфікуецца Міністэрствам як Цёмнае 
Майстэрства, нягледзячы на абставіны. Студэнту Хогвартс не варта казаць такія 
рэчы ўголас.

Самае неверагоднае было ў тым, \emph{як} ён гэта прамовіў. Гэта было сказана тонам чалавека, які не меў ніякага жадання 
абмяркоўваць гэтую тэму, і які шчыра думаў, што студэнтам і думаць пра такое не 
варта. Гары раптам падумаў, ці не адкладае прафесар Квірэл гэтыя размовы на 
той час, калі Гары навучыцца абараняць свой розум ад леджэламанцыі.

--- Зразумела, --- сказаў Гары. --- Я не буду размаўляць пра гэта ні з кім болей.

--- Калі ласка, містэр Потэр, --- сказаў прафесар Квірэл. --- Мне падабаецца 
ісці па жыцці, не прыцягваючы залішнюю публічную ўвагу. Вы не знойдзеце 
ніводнага слова аб прафесары Квірынусе Квірэле ў газетах да таго дня, калі я 
вырашыў навучаць Абароне ў Хогвартс.

Гэта было крыху сумна, але Гары зразумеў. Потым ён зразумеў, што гэта вызначала.

--- Так значыць усе тыя \emph{крутыя} штукі, што вы расказавалі --- ніхто пра іх 
не ведае?...

--- Не, некаторыя ведаюць, --- адказаў прафесар. --- Але я думаю, на сёння дастаткова.
Прызнаюся, я крыху стаміўся...

--- Я разумею. І \emph{дзякуй}. За усё.

Прафесар Квірэл кіўнуў, яшчэ мацней абапіраючыся на свой стол.

Гары паспешна паінуў аудыторыю.
