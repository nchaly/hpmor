\chapter{Грунтоўная памылка атрыбуцыі}

\begin{chapterOpeningQuote}
Патрабавалася бы боскае умяшальніцтва, каб надзяліць яго тваёй мараллю ў дадзеным асяроддзі. 
\end{chapterOpeningQuote}

\lettrine{М}{ахляскінавая} Крама была маленькая, але мудрагелістая (некаторыя маглі нават сказаць 
"сімпатычненькая"). Яна прочна ўсталявалася за лаўкай гародніны, якая хавалася за магазінам 
магічных пальчатак, які стаяў на самым канцы глухога завулка, які збочаў ад
Дыягон-аллеі. За прылаўкам, да расчаравання Гары, стаяў не сухарлявы і загадкавы мудрэц, 
а ўсяго толькі дзеўчына крыху знерваванага выгляду ў паношанай жоўтай мантыі. У дадзены
момант яна трымала ў руках Махляскінавы Супер-Кашэль \abbrev{qx31}, выбітнымі якасцямі якога былі
наяўнасць зачараваній Гумавага Хайла і Незаўважнага Пашырэння: у ім можна было насіць
вялікія рэчы, хаця агульны аб'ём быў абмежаваны.

Гары настаяў на тым, каб наведаць гэтую краму першай справай --- настаяў цвёрда, але так, каб 
не вызваць у МакГонагал падазрэнняў. Бо ў Гары было нешта, што трэба было пакласці ў кашэль
неадкладна. Гэты быў не мяшочак галеонаў, якія МакГонагал дазволіла яму ўзяць у банку. Гэта была
кучка іншых галеонаў, якія Гары незаўважна схапіў, калі ён выпадкова ўпаў
на кучу манет. Упаў ён і сапраўды выпадкова, але Гары быў не з тых, хто прапускае добрыя 
магчымасці. Таму ўвесь гэты час ён быў вымушаны нязграбна несці мяшок з дазволенымі 
галеонамі побач з кішэнню, каб звон стуль яго не выдаў.

Але заставалася пытанне аб тым, як незаўважна рэлацыраваць \emph{іншыя} галеоны ў кашэль. Манеты,
вядома, належылі яму, але яны былі атрыманы ў выніку... самакражы?.. аўтакрадзяжу?..

Гары ўважліва агледзеў Махляскінавы Супер-Кашэль \abbrev{qx31}, які ляжаў перад ім на прылаўке.

--- Можна мне паспрабаваць? Каб пераканацца, што ён працуе, аммм, надзейна? --- яго вочы 
былі па-дзіцячаму вялікія і бліскучыя.

Натуральна, пасля дзясяці разоў змяшчэння мяшочка з манетамі ў кашэль, засоўвання 
туды рукі, шэпту "мяшок з манетамі", і
яго даставання, МакГонагал адышла, і пачала аглядываць астатнія рэчы ў краме, а прадаўшчыца, у
сваю чаргу, звярнула ўвагу на МакГонагал.

Гары палажыў мяшок ў кашэль левай рукой, потым дастаў з кішэні правую руку з моцна сціснутымі ў ёй 
галеонамі, засунуў яе ў кашэль, адпусціў манеты, і з шэптам "мяшок золата" дастаў мяшочак зноў.
Потым той перакачаваў у левую руку, адтуль --- у кашэль, пакуль правая рука Гары зноў пацягнулася 
ў кішэнь...

Аднойчы МакГонагал глянула ў яго бок, але Гары здолеў не ўздрыгнуць або замярэць, і яна нічога
не заўважыла. Але ніколі не ведаеш, з гэтымі дарослымі, якія мелі пачуццё гумару. Праз тры ітэрацыі 
аперацыя была выканана, і Гары прыкінуў, што ў яго атрымалася скрасці ў сябе каля 
трыццаці галеонаў.

Гары выцер крыху мокры лоб і выдахнуў.

--- Калі ласка, я яго бяру.

Палягчэўшы на пятнаццаць галеонаў (дарэчы, двойчы больш за кошт чароўнай палачкі) і пацяжалеўшы на
адзін Махляскінавы Супер-Кашэль \abbrev{qx31}, Гары і МакГонагал выйшлі на вуліцу. Дзверы сфармавалі 
паўнаразмерную далонь, якая махала ім, пакуль яны выходзілі. Прарастанне рукі праз дрэва крыху 
знервавала Гары.

І потым зноў, трасца...

--- Ты \emph{сапраўды} Гары Потэр? --- прашаптаў стары мінак, вялікая сляза цякла па ягонай шчацэ. 
--- Ты ж не будзеш хлусіць пра такое? Бо хадзілі чуткі, што насамрэч ты не перажыў Забіваючы
Заклён, і таму ніхто нічога больш пра цябе не чуў.

...відавочна, маскіровачны заклён МакГонагал не працаваў на больш-менш здольных вядзьмарах.

МакГонагал схапіла Гары за плячо і пацягнула ў бліжэйшы завулак, толькі паспеўшы пачуць
"Гары Потэр?". Стары пайшоў за імі, але прынасмі падавалася, што больш ніхто іншы не пачуў.

Гары абдумваў пытанне. Ці \emph{быў} ён сапраўды Гары Потэрам? 

--- Я ведаю толькі тое, што мне казалі іншы людзі, --- сказаў ён уголас. --- Не магу сказаць, што
памятаю, як я нарадзіўся, --- яго рука кранулася ілба. --- Гэты шнар быў у мяне, колькі
я сябе памятаю, і мне казалі, што мае імя Гары Потэр, колькі я сябе памятаю. Але, ---
сказаў Гары задумённа, --- калі нават мы і не можам абгрунтаваць наяўнасць змовы, нічога 
не перашкаджала ім узяць любога магічнага сірату, і ўзгадаваць яго так, каб ён верыў, што 
ён і ёсць Гары П..

МакГонагал адчайна стукнула сябе па ілбе.

--- На выгляд вы выліты Джэймс, якім ён быў у год залічэння ў Хогвартс, за выключэннем
вачэй, яны ў вас ад маці. І па адным толькі \emph{характары} магу аўтарытэтна заявіць, 
что вы адназначна прамы нашчадак Грыфіндорскага Бізуна.

--- \emph{Яна} таксама можа быць у змове, --- ціха адзначыў Гары.

--- Не, --- праскрыпеў стары, --- яна права. У цябе матуліны вочы.

Гары нахмурыўся.

--- Бадай, \emph{вы} таксама можаце быць ў змове...

--- Дастаткова, містэр Потэр, --- сказала МакГонагал.

Стары падняў руку, быццам хацеў дакрануцца да Гары, але потым апусціў яе.

--- Я проста рады, што ты жывы, --- прамармытаў ён. --- Дзякуй, Гары Потэр. Дзякуй 
за ўсё, што ты зрабіў... Пакідаю вас адных.

І яго кій павольна пастукаў уздоўж завулка да галоўнай вуліцы Дыягон-аллеі.

МакГонагал з напруджаным і змрочным выразам на твары агледзелася. Гары аутаматычна таксама
паглядзеў навокал. Але акрамя кучак леташняга лісця, вакол было пуста, толькі часам 
хтосьці хутка праходзіў міма ўвахода ў завулак. 

Нарэшце МакГонагал разняволілася. 

--- Гэта было даволі груба, --- сказала яна ціха. --- Я ведаю, што вы да такой павагі не звыклыя,
містэр Потэр, але людзі праўда клапоцяцца аб вас. Будзьце да іх дабрэйшыя.

Гары разгледжваў свае туфлі.

--- Ім не варта, --- сказаў ён з адценнем горычы. --- У сэнсе, клапоціцца пра мяне.

--- Вы пазбавілі іх ад Самі-Ведаеце-Каго, --- сказала МакГонагал. --- Як ім не клапоціцца?

Гары ўздыхнуў і паглядзеў на МакГонагал. 

--- Мяркую, што словы \emph{фундаментальная памылка атрыбуцыі} вам нічога не кажуць?

Тая адмоўна пакачала галавой. 

--- Не, але калі ласка, патлумачце.

--- Ну... --- сказаў Гары, прыкідваючы, як апісаць гэтую ідэю маглаўскай навукі чалавеку з
магічнага свету. --- Уявіце, вы прыходзіце раніцай на працу ў офіс, і бачыце, што ваш 
калега з усёй дуры б'е кулаком па стале. Вы думаеце: "Які агрэсіўны чалавек". Але насамрэч,
ваш калега проста ўспомніў, як у метро нехта зачапіў яго плячом, а потым яшчэ і налаяўся на яго.
Сам калега не лічыць сябе асабліва злым або агрэсіўным. \emph{Хто заўгодна абазліцца 
пасля такога,} --- думае ён. Вось так вельмі часта --- мы тлумачым паводзіны людзей праз
рысы іх характару, але свае ўласныя паводзіны амаль заўсёды тлумачым 
пэўнай сітуацыяй. Мы ведаем свой кантэкст, і можам распавесці гісторыю, як мы прыйшлі да дадзенага
моманту, але гісторыі іншых людзей не цягнуцца за імі ў паветры, як дым за цягніком. Мы бачым 
іх толькі ў адной сітуацыі, і мы не ведаем, як яны паводзяць сябе ў іншых. У гэтым і ёсць 
фундаментальная памылка атрыбуцыі, калі мы тлумачым нешта праз усталяваныя рысы характару тое,
што лепей тлумачыцца кантэкстам і асяродзем.

Гары ведаў пра некалькі дасціпных эксперыментаў, якія праводзілі псіхолагі для праверкі гэтай
тэорыі, але вырашыў абысціся без паглыблення ў дэталі.

Бровы МакГонагал падняліся ўверх. 

--- Думаю, я разумею... --- сказала яна павольна. --- Але як гэта тычыцца вас?

Гары пхнуў нагой цагляную сцяну завулка, дастаткова моцна, каб забалелі пальцы.

--- Людзі вераць, што я перамог Самі-Ведаеце-Каго, бо я нейкі вялікі светлы ваяр. 

--- ...надзелены сілай падолець Цёмнага Лорда... --- сказала МакГонагал, скрыпучым ад 
нейкай незразумелай для Гары іроніі голасам.

--- Так, --- сказаў Гары, у яго голасе змагаліся раздражненне і расчараванне, ---
калі я забіў Цёмнага Лорда, значыць у мяне ад нараджэння павінна быць нейкая 
парамагайская-цёмных-лордаў ўласцівасць. Да мне было пятнаццаць месяцаў усяго!
Я не \emph{ведаю}, што менавіта здарылася, але я магу \emph{здагадацца}, што ўсяму
віной былі, як кажуць, "абставіны непераадольнай сілы". І ніякаія мае ўласцівасці 
былі не пры чым. Людзі клапоцяцца не пра \emph{мяне}, яны нават не звяртаюць увагі на
сапраўднага \emph{мяне}, яны проста жадаюць паціснуць руку \emph{дрэннай інтэрпрэтацыі}, ---
Гары замаўчаў, паглядзеў на МакГонагал, і дадаў: --- \emph{Вы} ведаеце, што сапраўды адбылося?

--- У мяне склалася гіпотэза, --- сказала МакГонагал. --- Я маю на ўвазе, пасля сустрэчы з вамі.

--- Так?

--- Цёмны Лорд паў, калі вы зрабілі нешта жудаснае нават па яго меркам. А Забіваючы Заклён
сам спужаўся, бо сустрэча з вамі страшней за Смерць.

-- Ха. Ха. Ха, --- Гары зноў пхнуў сцяну.

МакГонагал усміхнулася.

--- Зараз хадзем да мадам Малкін. Хутчэй за ўсё, ваша маглаўскае адзенне прыцягвае ўвагу.

Па дарозе яны сустрэлі яшчэ двух фанатаў Гары Потэра.

МакГонагал прыпынілася перад уваходам у "Мантыі Мадам Малкін". Фасад крамы быў звычайным, у асноўным
з цэглы, якая выглядала, як звычайная чырвоная цэгла, а ў вітрыне віселі звычайныя чорныя 
мантыі. Не было, напрыклад, мантый, якія свецяцца, або мяняюць колеры, або, такіх, каб
выпускалі нябачныя промні, якія бы праходзілі праз майку і казыталі бы свайго ўладальніка. Проста 
звычайныя чорныя мантыі --- прынамсі гэта было ўсё, што было бачна звонку. Дзверы былі 
шырока расчыненыя, быццам абвяшчаючы на ўвесь свет, што ніякіх сакрэтаў тут няма, і хаваць
тут нечага.

--- Мне трэба адысці на некалькі хвілін, пакуль вам будуць падбіраць мантыю, --- сказала 
МакГонагал. --- Управіцеся без мяне?

Гары кіўнуў. Ён ненавідзеў паходы за адзеннем з палаючай страснасцю, і не мог вінаваціць
МакГонагал за жаданне прамінуць гэты працэс.

МакГонагал кранулася яго галавы чароўнай палачкай.

--- Трэба, каб мадам Малкін успрымала вас як ёсць, таму я здымаю абфускацыю.

--- Эммм... --- сказаў Гары. Гэта яго крыху непакоіла.

--- Мы вучыліся ў Хогвартс разам з мадам Малкін, --- сказала МакГонагал. --- Нават тады 
яна была адным з самых ураўнаважаных людзей, каго я ведала. Яна не павядзе бровам, нават калі
Самі-Ведаеце-Хто зойдзе ў краму, --- яе голас крыху змяніўся, быццам яна ўспомніла нешта
з мінулага. --- Мадам Малкін не будзе вас турбаваць, і не дасць нікому вакол.

--- А вы куды? --- спытаў Гары. --- На выпадак, ну ведаеце, калі нешта здарыцца.

МакГонагал кінула на Гары жорсткі скептычны позірк. 

--- Я буду вось там, --- сказала яна, паказваючы на будынак на іншым баку вуліцы, на якім вісела выява
драўлянай бочкі, --- набываючы сабе келіх, які мне зараз вельмі патрэбны. У гэты час, пакуль вам
будуць падганяць адзенне, ад вас патрабуецца стаяць смірна, \emph{і больш нічога}.
Вельмі хутка я загляну праверыць, як вы там, і я спадзяюся знайсці краму мадам Малкін у парадку,
цэлую і непатрэсканую, як і яе саму.

Мадам Малкін была дзелавітай пажылой жанчынай, якая не сказала ні слова, калі ўбачыла
шнар на яго ілбе; яна нават кінула рэзкі позірк на асістэнтку, якая быццам збіралася 
нешта сказаць. Мадам Малкін дастала з кішэні пучок дзіўных стужак, якія пачалі мітусіцца вакол
Гары, верагодна, іграючы ролю рулетак, і вымяраючы яго ўздоўж і поперак.

Побач з Гары, бледны хлопец з вострым тварам і \emph{найкруцейшым} плацінава-бялявым хаерам 
праходзіў праз апошнія стадыі працэсу падгонкі адзння. Другая з дзвюх асістэнтак мадам
Малкін уважліва агледжвала бялявага хлопца і яго мантыю ў шахматную клетку. Час ад часу яна 
краналася мантыі сваёй палачкай, і тая падцягвалася або адпускалася.

--- Прывіт, --- сказаў хлопец. --- Таксама Хогвартс?

Гары мог прадказаць, куды вяла гэтая размова, і ў секундным прыступе раздражнення ён вырашыў,
што з яго хопіць.

--- Святыя нябёсы, --- прашаптаў Гары, --- гэта жа... --- ён дазволіў сваім вачам 
пашырыцца, --- Ваша імя, сэр?

--- Драко Малфой, --- сказаў Драко Малфой, крыху збянтэжаны.

--- Гэта вы! Драко Малфой. Я... я ніколі не думаў, што атрымаю такі гонар, сэр, --- 
Гары спрадзяваўся, што здолее выдавіць слязу з вачэй. Яго ўласныя фанаты да гэтага часу 
ўжо звычайна пачыналі плакаць.

--- О, --- прагучала разгублена з боку Драко. Потым яго вусны расцягнуліся ў самазадаволенай
ўсмешцы. --- Як добра ўрэшце сустрэць кагосці, хто ведае свае месца.

Першая асістэнтка, тая, што пазнала Гары, выдала гук, падобны на сціснуты кашэль.

Гары працягваў балбатаць:

--- Я так рады пазнаёміцца з вамі, містэр Малфой. Проста невыразна рады. І паступіць у Хогвартс 
у адзін год з вамі! У мяне ажно сэрца замірае.

Ой. Апошняе прагучала крыху занадта. Такое хутчэй магла сказаць закаханая дзеўчына.

--- А ў мяне сэрца ажно пяе, калі я бачу, калі да мяне ставяцца з павагай, адпаведнай 
сям'і Малфоеў, --- адказаў хлопец, і ўсміхнуўся такой усмешкай, якой кароль мог узнагароджваць
самых нікчэмных сваіх падданых, хоць і бедных, але сумленных.

Эммм... Чорт, Гары не мог сабе ўявіць, што сказаць далей. Звычайна ўсе фанаты жадалі паціснуць
руку Гары Потэра, таму...

--- Калі мы скончым з падгонкай, сэр, не маглі б вы паблажыць і паціснуць маю руку? Гэта 
будзе самая вялікая падзея за сёння... за месяц... да чаго там, за ўсё мае жыццё.

Драко кінуў на яго гнеўны позірк.

--- Ну гэта ўжо занадта нахабна! Чаго такога ты зрабіў для сям'і Мафлоеў,
 каб заслужыць гэтую літасць?

\emph{Ого, я стоадсоткава паспрабую такую тактыку наступным разам!} 

Гары схіліў галаву.

--- Не, не, сэр, я разумею. Выбачайце, што спытаў. Магчыма я магу пачысціць вашыя боты, або
нешта такое...

--- Згодны, --- кінуў у адказ Драко. Яго твар крыху прасвятлеў. --- Але твае імкненне можна 
зразумець. Скажы, як думаеш, на які факультэт ты трапіш? Я --- адназначна на Слізэрын, як і мой
бацька Люцыус да мяне. А ты, думаю, на Хафлпаф. А можа нават адразу ў эльфы.

Гары сарамліва ўсміхнуўся. 

--- Прафесар МакГонагал кажа, што я самы рэйвенколошны чалавек з усіх, каго яна ведае,
і з усіх, пра каго кажуць у легендах, настолькі рэйверклошны, што сама Равена параіла 
бы мне праветрывацца больш --- што гэта б ні значыла, --- і што я без усялякага сумневу 
апынуся ў Рэйвенкло, канешне, толькі ў тым выпадку, калі жахлівыя крікі Размеркавальнага 
Капелюша не перашкодзяць нам разабраць, што ён там вярзе, канец цытаты.

--- Хрэнасе, --- сказаў Драко, крыху ўражаны, і ўздыхнуў неяк мройліва. --- А ліслівіў ты крута,
мне спадабалася. Ты бы мог няблага зладзіцца і ў Слізэрыне. Звычайна ўся такая ўгода дастаецца 
майму бацьку, і я спрадзяюся, што зараз, у Хогвартс, астатнія слізэрынцы будуць падлашчвацца
да мяне... Так што гэта добры знак.

Гары кашлянуў.

--- Выбачай, але я не маю ні малейшага разумення, хто ты такі насамрэч.

\emph{--- ДА ЛАДНА!} --- прароў Драко ў дзікім расчараванні. --- Навошта камусьці можа спатрэбіцца 
рабіць такое? --- ягоныя вочы прыжмурыліся з раптоўным падазрэннем. --- І якім чынам ты можаш
наогул \emph{не} ведаць пра Малфоеў? І што гэта на табе за вопратка? Твае бацькі што, \emph{маглы}?

--- Двое з маіх бацькоў памёрлі, --- сказаў Гары. Сэрца ў яго ёкнула. Калі ён казаў гэта такім
чынам... --- Двое маіх іншых бацькоў --- якія мяне вырасцілі --- яны маглы. 

--- \emph{Што?} --- сказаў Дркако. --- Да \emph{хто} ты такі?

--- Гары Потэр, рады з табой пазнаёміцца.

--- \emph{Гары Потэр?} Той самы... --- Драко раптам замаўчаў.

Некаторы час стаяла поўная цішыня.

І потым, з наіўным энтузіязмам:

--- Гаты Потэр? Гары ПОТЭР?? \emph{Той самы} Гары Потэр? Божа, я заўседы хацеў з табой пазнаёміцца!

Асістэнтка, што займалася Драко, моцна стрымлівался, каб не засмяяцца, але 
працягнула працу, падняўшы рукі Драко дагары, і акуратна сцягваючы з яго клятчатую мантыю.

--- Забі зяпу, --- прапанаваў Гары.

--- Можна твой аўтограф? Не, чакай, спачатку мы зробім фотку разам!

--- Забізяпузабізяпузяпу!

--- Я проста невыразна ў захапленні ад нашай сустрэчы!

--- Гар\'ы ў полымі і сдохні ўжо.

--- Але ты жа Гары Потэр, слаўны вытаравальнік магічнага свету, пераможца Цёмнага Лорда!
Ты жа мой герой, Гары Потэр! Я заўсёды хацеў стаць як ты, калі вырасту, каб я таксама 
мог адольваць Цёмных Лордаў налева і...

Голас Драко абарваўся. На яго твары застыў выраз абсалютнага жаху.

Высокі, белавалосы, халодна-эленгантны, у чорнай мантыі найвышэйшай якасці.
Кій са срэбнай ручкай, які атрымліваў аўру смяротнай зборі проста праз тое, чыя рука яго
трымала. Вочы агледжвалі асяроддзе безуважным позіркам к\'ата --- чалавека,
для якога забойства было не балючым, не заклікальна забароненым, а проста
руціннай справай, кшталту дыхання. \emph{Дасканаласць} было слова, якое аўтаматычна
прыйшло да галавы.

Гэта быў мужчына, які ў гэты самы момант зайшоў праз расчыненыя дзверы.

--- Драко, --- сказаў ён ціха і вельмі злосна, --- \emph{што} ты такое кажаш?

За паўсекунды спачувальнай панікі Гары сфармуляваў план выратавання.

--- Люцыус Малфой! --- вохнуў Гары. --- \emph{Сам} Люцыус Малфой?

Адна з асістэнтак мадам Малкін адвярнулася, быццам убачыла штосьці вельмі цікавае на суседняй
сцяне.

Халодныя вочы забойцы сустрэліся з вачыма Гары.

--- Гары Потэр.

--- Гэта такі... такі гонар для мяне --- сустрэць вас!

Вочы Люцыуса пашырыліся, смяротную пагрозу змяніла шакаванае здзіўленне.

--- Ваш сын \emph{усё} мне аб вас расказаў, --- тарабаніў Гары, амаль не думаючы аб тым, што
кажа, проста намагаючыся гаварыць як мага хутчэй. --- Але, безумоўна, я і так ведаў, усе ведаюць
пра вас, пра вялікага Люцыуса Малфоя! Самы заслужаны выпускнік факультэта Слізэрын, я і сам,
ведаеце, падумваў трапіць у Слізэрын, проста калі дазнаўся, што вы вучыліся там 
у маім узросце... 

--- \emph{Што вы такое кажаце, містэр Потэр?} --- пачаўся з вуліцы голас на грані крыку, за 
якім праз імгненне ў краме ўзнікла прафесар МакГонагал.

На яе твары быў выраз такой чыстай жудасці, што Гары, адкрыўшы рот, закрыў яго, так і не здолеўшы
знайсці, што сказаць. І тут...

--- Прафесар МакГонагал! --- закрычаў Драко. --- Гэта сапраўды вы? Я так шмат чуў пра вас ад бацькі,
і ведаеце што, я думаў, якім чынам я магу размеркавацца ў Грыфіндор, каб я мог...

--- \emph{ШТО?} --- Люцыус Малфой і прафесар МакГонагал пракрычалі ў ідэальны ўнісон. Іх галовы
адначасова і аднолькава павярнуліся адно да аднаго, і таксама рэзка адвярнуліся, быццам у
сінхронным танцы.

Сарваўшыся з месца, Люцыус схапіў Драко і вывалак яго вонкі.

І потым зноў наступіла цішыня.

МакГонагал паглядзеша на шклянку з віном у сваёй руцэ. Яна і забылася пра яго ў паніцы, і зараз на дне 
 заставалася толькі пара кропель.

МакГонагал прайшла прама, пакуль не апынулася супраць мадам Малкін.

--- Мадам Малкін, --- сказала МакГонагал спакойна, --- што тут у вас адбылося?

Мадам Малкін чатыры секунды глядзела ў адказ моўчкі, а потым яе прарвало. Яна засмяялася так, што 
ён прыйшлося апярэцца на сцяну за ёю. Адна з асістэнтак проста упала, істэрчына
стукаючы далонню па падлозе. 

МакГонагал павольна павярнулася да Гары. Яе твар быў каменным.

--- Варта мне толькі пакінуць вас на пяць хвілін. Пяць хвілін, містэр Потэр, секунда ў секунду.

--- Я проста пажартаваў, --- запратэстарваў Гары, амаль не чутны праз выбухі смеху.

--- \emph{Драко Малфой перад сваім бацькам сказаў, што хацеў бы трапіць у Грыфіндор!}
Для \emph{такога} недастаткова проста пажартаваць! --- МакГонагал цяжка ўздыхнула. --- 
Якім, скажыце, чынам загад "ад вас патрабуецца стаяць смірна" вы ўспрынялі як 
\emph{"калі ласка, скастуйце заклён Confundus на ўвесь сусвет"}?

--- Драко знаходзіўся ў сітуацыйным кантэксце, для якога тыя словы мелі вельмі
лагічны сэнс і...

--- Не. Не тлумачце. Я не хачу ведаць, што тут адбылося. Ніколі. Ёсць рэчы, якія мне 
не дадзена зразумець, і гэтая --- адна з іх. Якая б дэманічная сіла хаосу ні валодала вамі,
яна \emph{заразлівая,} і я не хачу апынуцца на месцы беднага Драко Малфоя, беднай мадам 
Малкін і яе дзвюх бедных асістэнтак.

Гары ўздыхнуў. Было ясна, што прафесар МакГонагал не была ў гуморы слухаць разумныя тлумачэнні.
Ён паглядзеў на мадам Малкін, якая ўсё яшчэ адбымала сцяну, яе асістэнтак, якія ўжо
проста ляжалі на падлозе, і на сябе, усё яшчэ апаітава вымяральнымі стужкамі.

--- Мне яшчэ не скончылі падганяць адзенне, --- сказаў ён вельмі ветліва. --- Чаму б вам
не вярнуцца ў бар, і не ўзяць яшчэ чарку?

