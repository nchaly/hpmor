\chapter{Рэальнасць і яе альтэрнатывы}

\begin{chapterOpeningQuote}
Але пытанне ўсё ж застаецца --- хто?
\end{chapterOpeningQuote}

\lettrine{-Б}{ожа}-божа, --- сказаў бармэн, вылупіўшы зенкі на Гары. --- Ці гэта... ці можа
гэта быць...

Гары нахіліўся да барнай стойкі Дзіравага Катла, нягледзячы на тое, што яна была дзесьці на 
ўзроўні яго броваў. \emph{Такое} пытанне патрабавала поўнай самаатдачы.

--- Ці я?... ці магу я?... складана сказаць... ці \emph{я} гэта наогул... але пытанне ўсё ж
застаецца --- \emph{хто?}

--- Якое шчасце... --- прашаптаў стары бармен, --- Гары Потэр... дзякуй вам за гонар.

Гары міргнуў, але хутка сабраўся і працягнуў:

--- Ну, так. Вы вельмі ўважлівы. Астатнія не так спрытна разумелі.

--- Даволі, --- сказала прафесар МакГонагал. Яе рука сціснула плячо Гары. --- Не дакучай хлопцу,
Том, ён да гэтага не звыклы.

--- Але гэта праўда ён? --- дрыжачым голасам спытала старая жанчына побач. --- Гэта сапраўды
Гары Потэр? --- Яна паднялася са стула, які пры гэтым выдаў рэзкі скрып.

--- Дорыс, --- сказала МакГонагал з папярэджаннем у голасе. Позірку, якім яна акінула пакой, было
дастаткова, каб сцішыць усіх астатніх.

--- Я проста хачу паціснуць яго руку, --- сказала ціха жанчына. Яна нізка нахілілася і працягнула
сваю маршчыністую руку. Гары, адчуваючы сябе збянтэжана і няёмка як ніколі ў жыцці, асцярожна яе
паціснуў. Слёзы пакаціліся з яе вачэй на іх рукі. 

--- Мой унучак быў Аўрорам, --- прашаптала яна. --- Памёр у семдзесят дзявятым. Дзякуй, Гары Потэр.
Хвала нябёсам за цябе.

--- Да няма за што, --- сказаў Гары, абсалютна аўтаматычна, і кінуў на МакГонагал спалоханы,
умольны позірк.

МакГонагал стукнула нагой па падлозе так, што паціху ўздымаючыся гоман зноў сціх. Гэты грукат 
пабіў рэкорд Гары ў намінацыі "Грукат Лёсу", а ўсе прысутныя замерлі на месцы.

--- Мы спяшаемся, --- сказала МакГонагал голасам, які гучаў цалкам, неверагодна спакойна.

Яны пакінулі бар без далейшых перашкодаў.

--- МакГонагал? --- сказаў Гары, калі яны выйшлі. Ён меў намер спытаць аб тым, што толькі што
адбылося, але вельмі дзіўным чынам чамусці задаў цалкам іншае пытанне: --- Хто быў бледны 
мужчына? Той, у якога вока торгалася?

--- Хмм? --- сказала крыху здзіўлена МакГонагал; магчыма яна і сама не чакала такога пытання. --- То 
быў прафесар Квірэл. Ён будзе сёлета весці Абарону ад Цёмных Майстэрстваў ў Хогвартс. 

--- У мяне было дзівоснае адчуванне, што я ўжо недзе яго бачыў... --- Гары пацер лоб. --- І што
лепей не паціскаць яго руку, --- нейкае сумнае пачуццё, адчуванне страты. --- І што там увогуле
адбылося?

МакГонагал дзіўна паглядзела на яго ў адказ. 

--- Містэр Потэр... што вы ведаеце... што менавіта вам расказвалі пра тое, як загінулі вашы бацькі?

Гары вярнуў ён сур'ёзны позірк.

--- Мае бацькі жывыя і здаровыя, і яны заўсёды адмаўляліся расказваць пра тое, як загінулі мае 
\emph{генетычныя} бацькі. Скуль я раблю выснову, што здарылася нешта нядобрае.

--- Выдатная вернасць, --- сказала МакГонагал. Яна сцішыла голас. --- Але было даволі прыкра чуць, як
гэта прагучала. Лілі і Джэймс былі маімі сябрамі.

Гары адвёў позірк, раптам адчуўшы сябе сорамна. 

--- Прабачце, --- сказаў ён ціха, --- але ў мяне \emph{ёсць} мама і тата. І я ведаю, што
параўноўваць рэальнасць з... нечым ідэальным, што я пабудаваў у сваёй галаве --- прамая дарога да
дэпрэсіі...

--- Гэта неверагодна мудра з вашага боку, --- сказала МакГонагал таксама ціха. --- Але вашыя
\emph{генетычныя} бацькі загінулі адважна, абараняючы вас.


\emph{Абараняючы мяне?}

Быццам нешта схапіла сэрца Гары.

--- Што... \emph{што} адбылося?

МакГонагал уздыхнула. Яна лёкка кранулася палачкай ілба Гары, і яго позірк на імгненне расплыўся. 

--- Крыху маскіроўкі, --- сказала яна, --- каб больш такога не паўтаралася, прынамсі 
пакуль вы не гатовыя.

Яна зноў махнула чароўнаяй палачкай, і тры разы грукнула па цаглянаму муру...

...цэглы якога пачалі разбягацца ў бакі, зрабіўшы спачатку невялічку дзірку, якая 
расчынілася ў вялікую арку. За ёю была вуліца з радамі крамаў паабапал, вывескамі і 
рэкламамі дзівосных рэчаў кшталту магічных катлоў або драконавай печані.

Гары нават і не міргнуў. Гэта і блізка не ляжала да пераўтварэння чалавека ў котку.

Яны абодва шагнулі наперад, і ўвайшлі ў магічны свет.

Адразу каля іх быў тарговец, які заклікаў набыць Брыкаючыя Боты ("Зробленыя з сапраўднай
лягумы\footnote{{} англ. Flubber або Fly Rubber --- лятаючая гума.}"), побач з ім чулася 
"Лязы з бонусам плюс тры! Відэльцы плюс два! Лыжыкі плюс чатыры!". Там былі акуляры, якія
рабілі зялёным усё, на што ты глядзіш, і цэлая чарада фатэляў з катапальтуючыміся
сядушкамі, на выпадак надзвычайных сітуацый.

Галава ў Гары варочалася, нібыта намагалася адкруціцца ад шыі. Падавалася, што ён трапіў
у раздзел падручніка D\&D, прысвечаны магічным рэчам (ён сам ніколі не гуляў, але яму вельмі 
падабалася чытаць іхнія матэрыялы). Гары намагаўся не прамінуць не адзіны прадмет, на той выпадак,
калі яму раптам сустрэнуцца інгрэдыенты заклёна для стварэння бясконцых заклёнаў.

Потым Гары заўважыў тое, што прымусіля яго абсалютна бяздумна пайсці па прамой да дзвярэй 
з бронзавым дэкорам у краму з блакітнай цэглы. У рэальнасць яго вярнула толькі тое, што перад
ім узнікла прафесар МакГонагал.

--- Містэр Потэр?

Гары міргнуў, і зразумеў, што менавіта ён зрабіў толькі што.

--- Выбачайце! На секунду я забыўся, што я не з бацькамі, --- ён паказаў на вітрыну, дзе
завіхаліся вогненыя яскравыя літары, складваючы словы \emph{Брыльянтавы Букініст Бігмаума}.
--- Калі бачыш кнігарню, дзе ты яшчэ не бываў, абавяскова трэба зайсці і агледзіцца. 
Гэта сямейнае правіла. 

--- Самае рэйвенклоўскае правіла, якое я толькі чула.

--- Што?

--- Нічога. Містэр Потэр, першай справай нам трэба наведаць Грынготс, банк чарадзейскага свету.
У вашай \emph{генетычнай} сям'і там ёсць сховішча, у якім захоўваецца спадчына, якую
покінулі вам вашы \emph{генетычныя} бацькі, і там вы зможаце знайсці сродкі на набыццё
школьных рэчаў, --- яна ўздыхнула. --- І, магчыма, нейкая сума на кнігі таксама знойдзецца.
Хаця, я бы параіла зберагчы іх на будучыню. У Хогвартс ёсць вялікая бібліятэка з кнігамі на 
любыя магічныя тэмы. І башня, дзе, я падазрую, вы будзеце жыць, можа пахваліцца не менш
разнастайнай бібліятэкай. Усё, што вы сёння возьмеце, хутчэй за ўсё будзе дубліравацца.

Гары кіўнуў, і яны пайшлі далей.

--- Не прыміце за бестактоўнасць, \emph{гэта ўсё} вельмі адцягвае ўвагу, --- сказаў Гары,
па-ранейшаму варочваючы галавой ва ўсе бакі, --- магчыма, лепей за любыя сродкі, якія да 
мяне прымянялі ў мінулым, але не думайце, што я раптам забыў пра нашу дыскуссію.

МакГонагал уздыхнула.

--- З боку тваіх бацькоў... або прынамсі тваёй маці, магчыма, было даволі мудра не расказваць.

--- То бок, вы жадаеце і далей трымаць мяне ў блажэнным няведанні? Ведаеце, прафесар МакГонагал,
у гэтым плане ёсць адзін вялікі недахоп.

--- Думаю, план і сапраўды бязглузды, --- сказала ведзьма напруджана. --- Улічваючы, што
ўся вуліца гатовая выстраіцца ў чаргу, каб расказаць табе ўсю гісторыю. Ну што ж...

І яна расказала яму пра таго, каго нельга называць, Цёмнага Лорда, Вальдэморта.

--- Вальдэморт? --- прашаптаў Гары. Павінна было гучаць смешна, але не гучала. Імя апаляла
халодным пачуццём, бязлітаснасцю, дыяментавай яснасцю. Яна было як тытанавая кувалда, падаючая
на слабую плоць. Нават проста калі Гары сказаў імя, яго працяла дрыготка, і ён адразу вырашыў
карыстаць больш бяспечны тэрмін кшталту "Самі-Ведаеце-Кто".

Цёмны Лорд лютаваў па ўсёй чарадзейнай Брытаніі, як дзікі воўк, ён ірваў і раздзіраў тканіну
звычнага жыцця людзей. Суседнія краіны заламвалі рукі, але ўтрымліваліся ад інтэрвенцыі, праз
апатычны эгаізм, або проста страх, бо першы, кто ўзнімаў голас супраць Цёмнага Лорда, 
апынаўся наступнай мэтай яго тэрору.

(\emph{Эфект сведкаў,} падумаў Гары. Яму адразу ўспомніўся эксперымент Дарлі і Латанэ, які 
паказаў, што чалавек у стане эпілептычнага прыпадку хутчэй атрымае дапамогу, калі гэта бачыць
адзін чалавек, чым, калі мінакоў трое. \emph{Размыццё адказнасці, кожны спадзяецца, што 
ініцыятыву праявяць іншыя.})

Пажыральнікі Смерці следвалі за Цёмным Лордам, як сцярвятнікі, якія дзяўблі параненых; або ў яго
авангардзе, быццам змеі, каб жаліць і паслабляць ахвяру. Яны былі не такіх жахлівыя, як 
Цёмны Лорд, але яны былі жахлівыя, і іх было шмат. І Пажыральнікі давалі свайму правадыру не 
толькі свае палачкі: за маскамі хаваўся палітычны ўплыў, капітал, сувязі, або сакрэты дзеля
шантажу --- выдатныя інструменты, каб паралізаваць грамадства, якое паспрабуе супраціўляцца.

Адзін стары і паважаны журналіст, Йермі Уібл, заклікаў да падняцця налогаў і мабілізацыі. Ён 
адкрыта паказваў абсурднасць страху столькіх людзей перад невялікай кучкай. Яго скура, адна толькі 
яго скура, была знойдзена наступнай раніцай прыбітай да сцяны ў рэдакцыі яго газеты. Побач былі
скуры яго жаны і дачок. Кожны тады жадаў, каб нешта было зроблена ў адказ, але ніхто не смеў
устаць першым, каб прапанаваць гэта. Той, хто выдзяляўся больш за астатніх, станавіўся
наступным прыкладам.

Пакуль Джэймс і Лілі Потэр не падняліся на вяршыню гэтага спісу.

І яны памёрлі з палачкамі ў руках, і шкадавалі не аб сваім выбары, бо яны былі сапраўднымі героямі, 
а толькі аб тым, што яны пакінулі сіратой свае дзіця, Гары Потэра.

Слёзы пацяклі з вачэй Гары. Ён выцер іх, з нейкім злосным адчаем. \emph{Я не ведаў гэтых 
людзей, яны не мае сапраўдныя бацькі... няма ніякага сэнсу адчуваць такі жаль...}

Калі Гары перастаў плакаць у мантыю МакГонагал, ён паглядзеў наверх, і яму крыху палепшыла,
калі ён убачыў, што яе вочы таксама блішчаць.

--- Дык... дык што здарылася? --- спытаў ён усё яшчэ няроўным голасам.

--- Цёмны Лорд прыйшоў у Лагчыну Годрыка, --- сказала МакГонагал шэптам. --- Ваша сям'я была
схаваная, але вам здрадзілі. Цёмны Лорд забіў Джейсма, потым Лілі, і дабраўся да вашых
ясляў. Ён скаставаў на вас Забіваючы Заклён, і ў гэты момант усё скончылася. Забіваючы Заклён
зроблены з чыстай нянавісці, ён б'е прама ў душу, адрываючы яе ад цела. Яго немагчыма 
заблакаваць. Адзіная абарона ад яго --- не быць у месцы ўдара. Але вы выжылі. Вы --- адзіны
чалавек у гісторыі, які выжыў. Заклён адбіўся і замест трапіў у Цёмнага Лорда, пакінуўшы толькі 
яго абпаленае цела і шнар у вас на ілбе. Гэты быў канец тэрору, краіна была свабодна. Вось чаму,
Гары Потэр, людзі жадаюць разгледзіць ваш шнар, і вось чаму яны жадаюць паціснуць вам руку.

Падавалася, што мінулы прыступ плачу страціў усе ягоныя слёзы, бо Гары Потэр не мог больш плакаць.
Ён быў высмактаны, як губка.

\emph{(Але дзесці ў глыбіні душы ў яго ўзнікла пытанне да гэтай гісторыі. Штосці з ёю было не так.
Заўважаць такія дробязі было натуральнай часткай характару Гары, але ў той момант яго ўвага была
адцягнута. Так здараецца, што ў тыя часы, калі майстэрства рацыянальнасці нам асабліва патрэбна,
мы часцей за ўсё пра яго забываемся.)}

Гары адчапіўся ад бока МакГонагал. 

--- Мне... трэба пра гэта падумаць, --- сказаў ён, намагаючыся трымаць голас пад кантролем, але
глядзеў сабе пад ногі. --- Эмм...  вы можаце называць іх маімі бацькамі, калі хочаце, не трэба
казаць "генетычныя бацькі", або нешта такое. Думаю, нічога не перашкаджае мне мець дзвюх маці і 
двух татаў.

МакГонагал нічога не сказала ў адказ.

Яны ішлі далей у цішыні, пакуль не дайшлі да вялікага белага будынку з магутнымі бронзамымі
дзвярыма.

--- Грынготс, --- сказала МакГонагал.
