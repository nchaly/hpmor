\partchapter{Самаўсведамленне}{I}


\begin{chapterOpeningQuote}
Ніколі не ведаеш, якое
малейшае нечаканае здарэнне можа разбурыць твой стратэгічны план.
\end{chapterOpeningQuote}

\lettrinepara{-Э}{бат,} Ганна!

\hplettrineextrapara
Пауза.

--- ХАФЛПАФ!

--- Боунс, Сюзан!

Пауза.

--- ХАФЛПАФ!

--- Бут, Тэры!

Пауза.

--- РЭЙВЕНКЛО!

Гары хутка глянуў на свайго новага калегу па факультэту, проста каб запомніць выгляд.
Ён усё яшчэ намагаўся атрымаць над сабой кантроль пасля сустрэчы з прывідамі. 
Самым сумным было тое, што ў яго атрымлівалася. Гэта было падазрона,
быццам яму павінна было патрабавацца цэлы дзень на гэта, а можа і ўсё жыццё, а можа --- вечнасць.


--- Корнер, Майкл!

Доўгая пауза.

--- РЭЙВЕНКЛО!

Прафесар МакГонагал стаяла за кафедрай перад галоўным сталом, выглядаючы
строга, і строга агледжваючы залу, выклікаючы імёны адно за адным (%
але Герміёне і некалькім яшчэ яна ўсміхнулася).
За яе спінай у самым высокім крэсле --- дакладней, троне --- 
сядзеў маршчыністы старык у акулярах, з сівой барадой, якая, падавалася,
была даўжынёй да падлогі, калі б яе было бачна; ён дабразычліва назіраў за 
размеркаваннем. Ён быў настолькі стэрэатыповым Старым Мудрацом
з Сівой Барадой, наколькі можна было сабе ўявіць --- але Гары навучыўся не 
давяраць стэрэатыпам, бо падобнае няправільнае ўражанне на яго зрабіла сама 
МакГонагал пры іх першай сустрэчы. 
Старажытны чараўнік са шчырай усмешкай пляскаў у далоні кожнаму, хто выходзіў 
з-пад Капелюша, кожны раз радуючыся па новай.

Па левы бок ад трону сядзеў суровы чалавек з пранізлівым позіркам, які не 
ляскаў нікому, і які неверагодным чынам умадраўся глядзець 
Гары ў вочы кожны каз, калі Гары глядзеў на яго. За ім сядзеў
бледны малады чалавек, якога Гары ўжо бачыў у Дзіравым Катле, ягоныя вочы
тузалізя па баках, быццам у паніцы, і час ад часу ён ўздрыгваўся у сваім 
крэсле. Гары заўважыў, што яго позірк вяртаўся часцей на гэтага настаўніка.
Яшчэ лявей сядзелі ў рад тры старэйшыя вядзьмаркі, якіх не вельмі цікавіла 
тое, што адбывалася навокал. Направа ад трону сядзела ведзьма сярэдняга ўзросту,
з круглым тварам, у жоўтым капялюшы, якая пляскала ўсім студэнтам, акрамя 
Слізэрына. Далей быў вельмі кароткі чалавек з калматай
белай барадой, які стаяў на сваім крэсле, ён пляскам усім, але ўсміхаўся толькі
тым, хто ішоў у Рэйвенкло. І на самым краі сядзеў, займаючы тры месца, чалавек-гара, які вітаў іх, калі яны сышлі з цягніка, назваўшыся Хагрыдам, Загадчыкам па
Гаспадарцы.

--- Той, што стаіць на крэсле --- дэкан Рэйвенкло? --- Гары прашаптаў
Герміёне. Тая стаяла  побач з Гары, і тузалася з боку ў бок так энергічна, што
падавалася, яна зараз узнімецца ў паветра. 

--- Так, --- сказала адна з прэфектэс, якія прывялі атрад першакурснікаў у залу,
дзеўчына ў блакітнай мантыі, колеру Рэйвенкло --- міс Кліруотэр,
калі Гары не памыляўся. У яе голасе было чуваць ноткі гонару:
--- Гэта Прафесар Чаравання Філіус Флітвік, самы вядомы эксперт па 
зачараванні з жывучых зараз, і ў мінулым --- дуэлянт-чэмпіён... 

--- А чаму ён такі \emph{кароткі?} --- спытаў нехта з дзяцей. --- Ён што,
\emph{паўкроўка?}

Адказам быў халодны позірк маладой прэфектэсы. 

--- У прафесара сапраўды ў раду былі гобліны...

--- Што? --- сказаў Гары ўголас нечакана нават для сябе, так што Герміёна і бліжэйшыя 
да яго шыкнулі на яго.

Позірк, які міс Кліруотэр кінула ў бок Гары, быў ўжо проста палохаючым. 

--- У сэнсе... --- перайшоў на шэпт Гары, --- не тое, каб для мяне гэта было
\emph{праблемай}, проста... гэта... проста як гэта наогул \emph{магчыма?} 
Немагчыма проста змяшаць два розных біялагічных віда, і атрымаць жыццяздольнае патомства! З пункту гледжання генетыкі... гэта быццам як калі пытаешся сабраць... ---
у іх тут не было машын, таму Гары не мог скарыстаць аналогію з пераблытанымі
чарцяжамі для рухавіка, --- паўлодку-паўцялегу, кшталту таго...

Прэфектэса Рэйвенкло паглядзела на яго сурова.

--- І чаму гэта немагчыма атрымаць паўлодку-паўцялегу?

--- \emph{Хшшш!} --- хшукнуў іншы прэфект, хаця яны размаўлялі вельмі ціха.

--- Я маю на ўвазе... --- сказаў Гары яшчэ цішэй, прыдумляючы, як лепей спытаць,
ці гобліны эвалюцыюнавалі ад чалавека, або ў гоблінаў і людзей быў агульны
продак як  \emph{Homo erectus}, або гобліны былі нейкім чынам \emph{зроблены}
з людзей... ---  скажам, калі генетычна яны ўсё яшчэ людзі, проста знаходзяцца 
пад уздзеяннем нейкага наследаванага маічнага эфекту, гэта бы магло 
патлумачыць магчымасць скрыжавання відаў... і таму, на жаль, гобліны не з'яўляюцца 
цэнным відам, які эвалюцыянаваў у разумную форму паасобку... ---
і сапраўды, гобліны ў Грынготс не выглядалі прышэльцамі з нечалавечым розумам, кшталту
ДырДыр або Куклаводаў\footnote{
    Іншапланетныя расы з раманаў пісьменнікаў-фантастаў Джэка 
    Вэнса і Лары Нівена.
}, --- карацей, скуль гобліны наогул \emph{узяліся?}

--- З Літвы, --- адстранённа прашаптала Герміёна, пакуль яе вочы неадрыўна сачылі за
Размеркавальным Капелюшом. 

Герміёна атрымала ўсмешку ад прэфектэсы.

--- Няважна, --- адказаў Гары.

С кафедры прафесар МакГонагал выклікала:

--- Голдштэйн, Энтані!

--- РЭЙВЕНКЛО!

Герміёна ўжо так моцна тузалася, што яе падэшвы адрываліся ад падлогі.

--- Гойл, Грэгары!

Прайшло некалькі напруджаных секунд, якія расцягнуліся амаль на хвіліну. 

--- СЛІЗЭРЫН!

--- Грэнджэр, Герміёна!

Герміёна сарвалася з месца і пабяжала да Размеркавальнага Капелюша, узяла яго ў рукі,
і моцна насунула сабе на галаву. Гары скурчыўся. Герміёна, якая і расказала Гары пра 
Капялюш, падавалася, сама не ўспрымала яго, як незаменны,
жыццёва важны, 800-летні артэфакт забытай старажытнай магіі, які збіраўся
здзейсніць сэанс складанай тэлепатыі над яе могзам, нават не гледзячы, што капялюш 
выглядаў старым і спарахнелым.  

--- РЭЙВЕНКЛО!

Гары не разумеў, чаму Герміёна так хвалявалася. У якім перакручаным альтэрнатыўным
сусвеце гэтая дзеўчына магла трапіць \emph{не} на Рэйвенкло? Калі Герміёна пайшла бы 
на іншы факультэт, то ў Рэйвенкло не заставалася ніякай прычыны для існавання.

Герміёна села за стол Рэйвенкло пад гром гучных вітянняў. Ведалі б яны, які 
ўзровень спаборніцтва яны толькі што павіталі. Гары ведаў "пі" да шостага знаку
--- 3.141592 --- таму што адна мільённая было дастатковай дасканаласцью ў большасці
практычных задач. Герміёна ведала сто знакаў "пі", таму што менавіта столькі іх было 
надрукавана на задняй вокладцы яе падручніка па матэматыцы.

Нэвіл Лонгботам пайшоў на Хафлпаф, што Гары было прыемна бачыць. Калі ён сапраўды 
знойдзе там падтрымку і таварыства, якім славіўся гэты факультэт, гэта
пайдзе Нэвілу толькі на карысць.
Вучоныя --- у Рэйвенкло, палітыкі і кар'ерысты --- у Слізэрын, героі-выскачкі --- у
Грыфіндор, а ў Хафлпаф траплялі адзіныя, хто сапраўды рабіў нешта карыснае.

(Але Гары быў правы, калі першай спытай прэфектэсу Рэйвенкло. Міс Кліруотэр, нават
не падняўшы воч ад свайго чытання, проста махнула
палачкай у бок Нэвіла і нешта прамармытала, ад чаго Нэвіл увайшоў у нейкі транс,
пайшоў у чатвёртае купэ  пятага вагона, где пад лаўкай яны знайші ягоную жабу.)

"Малфой, Драко!" быў размеркаваны на Слізэрын, і Гары з палёгкай выдахнуў. 
Гэта падавалася, што яно само сабой разумеецца, але ніколі не ведаеш, якое
малейшае нечаканае здарэнне можа разбурыць твой стратэгічны план.

Прафесар МакГонагал сказала:

--- Перкс, Салі-Эн! --- і ад іх групы аддзялілася бледная калматая дзяўчына, якая
выглядала дзіўна-эфемерна, быццам яна магла знікнуць, калі на ігненне адвесці ад яе вочы,
і болей ніхто яе ніколі не ўбачыць і не ўспомніць.

--- ХАФЛПАФ!

І потым (з ледзьве чутным трапятаннем у голасе і твары, якое маглі заўважыць толькі
тыя, хто добра ведаў яе) Мінерва МакГонагал глыбока ўдыхнула і сказала:

--- Потэр, Гары!

Раптам ва ўсёй зале натупіла цішыня.

Усе размовы сціхлі.

Усе вочы глядзелі на яго.

Гэта была выдатная магчымасць для таго, каб адчуць страх перад сцэнай, як падалося Гары.

Ён задавіў у сабе гэтыя пачуцці. Калі ён планаваў жыць у магічнай Брытаніі ---
або зрабіць нешта цікавае ў жыцці, --- яму
прыйдзецца звыкнуцца быць у цэнтры ўвагі вялікіх натоўпаў.
Прыляпіўшы фальшывую ўсмешку на твар, ён пачаў рабіць крок наперад...

--- Гары Потэр! --- крыкнуў ці то Фрэд, ці то Джордж Уізлі.

--- Гары Потэр! --- далучыўся ягоны блізнюк.

Праз секунду да іх далучыўся ўвесь стол Грыфіндора, а потым і добрая частка сталоў
Рэйвенкло і Хафлпафа.

--- \emph{Гары Потэр! Гары Потэр! Гары Потэр!}

Гары Потэр пайшоў наперад. Занадта павольна, як ён зразумеў праз некалькі крокаў, але
было позна мяняць хуткасць, бо гэта будзе выглядаць нязграбна.

\later

--- \emph{Гары Потэр! Гары Потэр! \shout{Гары Потэр!}}

Мінерва з патаемным страхам паглядзела на галоўны стол.

Трэлоні ліхаманкава абмахвалася веерам, Флітвік глядзеў на Потэра зацікаўлена, 
Хагрыд пляскаў у далоні ў такт, Спраут глядзела сурова, 
Вектар і Сіністра --- са здзіўленым задавальненнем,
Квірэл асцякленела глядзеў у нікуды. Альбус дабразычліва ўсміхаўся, а Снэйп сціскваў у 
руцэ пусты кубак для віна так моцна, што тоўстае срэбра пачало згібацца.

З вырокай усмешкай, ківаючы то ў адзін бок, то у другі, Гары Потэр ішоў 
паміж чатырох доўгіх сталоў, ішоў велічна спакойнам крокам, прынц, успадкаваўшы
свой замак. 

--- \emph{Ратуй нас яшчэ ад Цёмных Лордаў!} --- крыкнуў адзін з блізнят Уізлі, 
і яго брат дадаў: --- \emph{Асабліва калі яны настаўнікі!} --- пад агульны смех
з усіх сталоў, акрамя Слізэрына.

Вусны Мінервы сціснулся ў тонкую паласу. Ёй трэба абмяркамаць з Кашмарнымі Уізлі
гэтую апошнюю частку. Калі яны думалі, што ў першы школьны дзень яшчэ не было 
факультэтных балаў, каб іх сняць, і калі іх не хвалявалі звычайшыя школьныя пакаранні,
яны прыдумае нешта яшчэ.

З лёгкім прыступам панікі яна глянула на Северуса, які дакладна павінен зразумець, аб кім гэта Уізлі...

Твар Северуса перайшоў ад стану раз'юшанасці да спакойнай абыякавасці. Лёгкая 
усмешка хадзіла па яго вуснах. Ён глядзеў на Гары Потэра, не ў бок стала Грыфіндора,
у яго руцэ былі скамечаныя рэшкі кубка.

\later

Гары Потэр ішоў наперад з наклеенай усмешкай, адчуваючы ўнутры адначасова 
цеплыню і жах.

Агульныя апладысменты за тое, што ён зрабіў ва ўзросце аднаго году. І зрабіў не да канца.
Цёмны Лорд усё яшчэ быў жывы. Ці віталі б яго, калі б гэта было вядома?

Але аднойчы сіла Цёмны Лорд пала.

І Гары Потэр абароніць іх зноў. Прынамсі, калі пра гэта ёсць прароцтва. А, чорт з 
ім, з прароцтвам. Усё роўна абароніць.

Усе гэтыя людзі, якія віталі ў яго і верылі --- Гары не мог гэтаму здрадзіць.
Успыхнуць і згаснуць, як тысячы вундэркіндаў. Стаць расчараваннем. Не дайсці 
ў жыцці да высокага стандарту, які ён сам жа і ўсталяваў, як сымбал Святла ---
нягледзячы, як ён гэта зрабіў. Ён абавяскова, няважна, колькі часу гэта зойме, 
і нават калі гэта яго заб'е, апраўдае іх чаканні. А потым \emph{перавысіць}
іх чаканні, каб яны потым здзіўляліся, як мала ад яго калісьці чакалі...

--- \emph{ГАРЫ ПОТЭР! ГАРЫ ПОТЭР! ГАРЫ ПОТЭР!}

Гары зрабіў апошнія крокі да Размеркавальнага Капелюша. Абярнуўшыся, ён пакланіўся ў бок
Ордэна Хаоса за сталом Грыфіндора, потым --- у другі бок залы, і 
пачакаў, пакуль апладысменты і гоман сціхнуць.

У глыбіні сваёй свядомасці ён думаў, ці сапраўды Капялюш \emph{свядомы},
у сэнсе ўсведамлення сваёй свядомасці, і калі так, ці задавальняла яго,
што яму дазволена размаўляць толькі з адзіннаццацілеткамі, і толькі раз на год. 
Згодна з песняй, якую прапеў Капялюш, падавалася, што так:

\begin{verse}%[\versewidth]
    \itshape
    Я --- капялюш размеркавальны,\\
    Я пачуваюся нармальна,\\
    Увесь год ляжу, як той камень,\\
    Працую ў годзе толькі дзень...
\end{verse}

Калі настала цішыня, Гары сеў на стул, і \emph{акуратна} надзеў на галаву 
800-летні артэфакт забытай старажытнай магіі, думаючы з усёй моцы: 
\emph{Пачакай, пакуль не размяркоўвай мяне! У мяне ёсць да цябе шмат  
пытанняў! На мне калісьці рабілі забывальны заклён? Калі ты размяркоўваў Цёмнага Лорда,
можаш расказаць мне пра яго слабасці? Ты ведаеш, чаму я атрымаў чароўную палачку,
якая сястра палачкі Сам-Ведаеш-Каго? Ці ўтрымлівае мой шнар дух Цёмнага Лорда,
і таму я часам моцна злуюся?.. Гэта самыя важныя пытанні, але, калі ў цябе ёсць час,
можаш расказаць мне нешта пра тое, як пераадкрыць забытую магію, якая цябе стварыла?}

У прасторы Гарынай свядомасці, дзе дагэтуль гучаў толькі адзін голас, 
пачулася вельмі ўсхваляванае:

\emph{Ох, божачкі вы мае. Ніколі такога не было, і вось...}
