\chapter{Іерархія дамінавання} 

\begin{chapterOpeningQuote}
Гучыць як нешта з майго рэпертуара, ці не так?
\end{chapterOpeningQuote} 

\lettrine{Б}{ыла} пятніца, час сняданка. Гары адгрыз чарговы агромны кавалак ад
свайго бутэрброда, і зноў паспрабаваў нагадаць сабе, што хуткае знішчэнне
сняданка ніяк не набліжае момант, калі ён апынецца ў падзямеллі. У любым
выпадку, у іх была яшчэ цэлая гадзіна самападрыхтоўкі перад першым заняткам па
зеллеварэнню. 

Але падзямеллі! У Хогвартс! Гарына ўяўленне ўжо малявала вузкія мосты над
безданню, паходні на каменных сценах, плямы моха, якія свецяцца ў цемры. Ці
будуць там пацукі? Ці будуць там \emph{драконы?} 

--- Гары Потэр, --- сказаў ціхі голас ззаду. 

Гары абярнуўся і ўбачыў Эрні Макмілана, апранутага ў мантыю з жоўтай аблямоўкай,
і выглядаючага крыху ўстрывожана. 

--- Нэвіл думае, што лепей цябе папярэдзіць, --- сказаў ён ціхім голасам. ---
Напэўна, ён правы. Будзь вельмі асцярожны з настаўнікам зеллеварэння. Хлопцы са
старэйшых курсаў расказвалі, што прафесар Снэйп можа быць
вельмі злосным да тых, хто яму не падабаецца, і яму не падабаюцца большасць тых,
хто не слізэрыны. Калі ты будзеш... ну ведаеш... выпендрывацца, гэта можа
скончыцца для цябе вельмі дрэнна, вось што я чуў. Проста не высоўвайся і давай
яму зачэпак. 

Некаторы час Гары абдумваў гэта, потым ён узняў бровы. (Ён жадаў умець узнімаць
толькі адно брыво, як Спок, але ў яго ніколі не атрымлівалася.)

--- Дзякуй, --- сказаў Гары. --- Думаю, ты абараніў мяне ад шмат якіх
непрыемнасцяў. 

Эрні кіўнуў, павярнуўся і пайшоў да стала Хафлпафа. 

Гары вярнуўся да свайго бутэрброда. 

Прыкладна каля чацьвёртага ўкуса нехта сказаў "Прабач", і, абярнуўшыся, Гары
ўбачыў старэйшага рэйвенкло, які выглядаў крыху ўстрывожана... 

Праз некаторы час, калі Гары сканчваў сваю трэцюю порцыю бекона (Гары ўжо звыкся
есці на сняданак як мага мацней. Ён заўсёды мог прапусціць абед, калі ў той
дзень ён не карыстаў часаварот.), нехта яшчэ побач сказаў "Гары?" 

--- Так, --- сказаў ён стомлена, --- я паспрабую не прыцягваць увагу прафесара
Снэйпа...

--- Дохлы нумар, --- сказаў Фрэд.

--- Абсалютна дохлы, --- сказаў Джордж.

--- Таму мы загадалі эльфам спячы табе торцік, --- сказаў Фрэд.

--- І мы паставім на яго па свечцы за кожны бал, які ты страціш для Рэйвенкло,
--- сказаў Джордж.

--- І мы адсвяткуем усё гэта на абедзе за сталом Грыфіндора, --- сказаў Фрэд.

--- Мы думаем, гэта цябе крыху суцешыць пасля ўсяго, --- скончыў Джордж.

Гары праглынуй свой апошні кавалак бекона, і павярнуўся да іх.

--- Ну ладна, --- сказаў ён. --- Я не хацеў задаваць гэтае пытанне, улічваючы,
што ў нас ёсць прафесар Бін, але ўсё ж задам.  Калі прафесар Снэйп
\emph{настолькі} злосны, чаму яго не вытурылі?

--- Вытурылі? --- спытаў Фрэд.

--- У сэнсе, звольнілі? --- сказаў Джордж.

--- Так, --- сказаў Гары. --- Менавіта так ты абыходзішся з дрэннымі
настаўнікамі.  Вытурываеш, а потым замест наймаеш лепшага настаўніка. У вас
тут няма паняцца аб прафсаюзах, або перманентных кантрактаў\footnote{У ЗША і
Канадзе існуе пажыццёвы кантракт прафесара без права адміністрацыі ўстановы
звольніць такога працоўніка, на ангельскай "tenure".}, так?

Фрэд і Джордж задумённа хмурыліся, быццам яны былі чальцы племені
збіральнікаў-паляўнічых, якім толькі што расказалі на дыферэнцыяльнае вылічэнне. 

--- Не ведаю, --- сказаў нарэшце Фрэд. --- Я пра такое ніколі не думаў.

--- І я, --- сказаў Джордж.

--- Вядома, --- сказаў Гары. --- Мне так часта адказваюць. Ладна, хлопцы,
пабачымся на абедзе, і не абурайцеса, калі на торціке наогул не будзе свечак.

Фрэд і Джордж засмяяліся, быццам Гары сказаў нешта смешнае, і пайшлі назад да
сваіх месцаў.

Гары павярнуўся назад да стала, і схапіў бліжэйшы кекс. Ягоны страўнік адчуваўся
як вельмі поўны, але Гары падазраваў, што гэтай раніцай яму спатрэбіцца шмат
калорый.

Разбіраючыся з кексам, Гары думаў аб найгоршым настаўніке, якога ён дагэтуль
сустракаў, аб прафесары Гісторыі Магіі. Прафесар Бінз быў прывід. З таго, што пра прывідаў
расказвала Герміёна, не падавалася, што яны цалкам мелі самаўсведамленне. Не
было важных адкрыццяў, зробленых прывідамі, або любых іншых дасягненняў, кім бы
яны ні былі пры жыцці. Было падобна, што яны не вельімі разумелі, якое зараз
стагоддзе.  Герміёна казала, што яны быццам выпадковыя партрэты, адбітак
чалавека на тканіне сусвету, які пакідае выбух псіхічнай энергіі падчас
раптоўнай смерці чараўніка.

Гары сустракаў пару дурных настаўнікаў падчас яго няўдалых вылазак ў маглаўскую
адукацыйную сістэму --- канешне, выбар аспірантаў для Гары ягоны бацька
праводзіў значна скрупулёзней, --- але занякі па Гісторыі былі першым выпадкам,
калі ён сутыкнуўся з настаўнікам, які літаральна не меў уласнаго розуму.

І гэта было заўважна. Праз пяць хвілін першага ўроку Гары здаўся, і пайшоў
чытаць падручнік. Калі стала зразумела, што "прафесар" Бінз не будзе супраць,
Гары таксама дастаў з махляскіна затычкі для вушэй.

Цікава, а прывідам трэба было плаціць зарплату? Як яны яе атрымлівалі?  І яшчэ:
ці праўда, што немагчыма звольніць нікога з Хогвартс, нават \emph{калі яны
памерлі?}

І вось: прафесар Снэйп адкрыта збіраецца ставіцца абсалютна злосна 
да ўсіх, хто не слізэрын, і нікому да галавы нават \emph{не завітала} думка аб 
тым, каб парваць з ім кантракт.

І Майстра спаліў курыцу...

--- Выбачай, -- сказаў устрывожаны голас за яго спінай.

--- Клянуся, --- сказаў Гары, паварочваючыся, --- гэтае месца амаль на восем 
з паловай адсоткаў настолькі ж дрэннае, як тата расказваў пра Оксфард.


\later

Гары тупаў уздоўж калідора, выглядаючы абражаным, уззлаваным, і расчараваным --- 
і ўсё гэта адначасова. 

--- Падзямеллі! --- сычэў ён. --- \emph{Падзямеллі!} Гэта --- не падзямеллі!
Гэта --- сутарэнне! Звычайны \emph{склеп!}

Хтосьці з дзяўчат паглядываў на яго дзіўна. Хлопцы-рэйвенкло былі ўжо звыклыя.

Верагодна, паверх, дзе знаходзіўся кабінет па зеллеварэнню, называлі 
"падзямеллі" проста таму, што ён быў ніжэй за узровень зямлі, і быў крыху халадзейшы
за астатні замак. 

У \emph{Хогвартс!} Божа ты мой! Гары чакаў такой магчымасці ўсё жыццё, і думаў,
што калі нешта на гэтай зямлі і мела годныя падзямеллі, то гэта проста павінен 
быць Хогвартс! Няўжо чалавеку, каб выканаць свае жаданне паглядзець на 
невялічкую пякельную бездань, трэба будаваць асабісты замак?  

Але, калі праз некаторы час яны сягнулі кабінета зёлкаў, настрой у Гары 
значна палепшыўся.

На паліцах, якія пакрывалі кожны сантыметр сцен паміж шафамі, стаялі агромныя
шкляныя жбаны з заспіртаванымі дзіўнымі пачварамі. Гары прадвінуўся ў чытанні 
падручніка настолькі, што мог ідэнтыфікаваць некаторых з із, кшталту Забруйскай 
Фантэмы. Пяцідзесяцісантыметровы павук быў занадта малы для таго, каб 
быць Акрамантулай, ці не? Ён спрабаваў пытаць Герміёну, але яе погляд быў 
кожны раз прыкаваны да зусім іншага месца на паліцах.

Гары разглядваў нешта, што выглядала, як вялкі кавалак пылу з ножкамі і вочкамі,
калі ў пакой уварваўся ас\'асін.

Гэта было першае, што прыйшло яму да галавы, калі ён убачыў прафесара Снэйпа.
Было нешта ціхае і смяротнае ў тым, як настаўнік слізгаў паміж парт. Ягоная 
мантыя была памятая, валасы нямытыя і калматыя. Было ў ім нешта падобнае на 
Люцыуса, і негледзячы на тое, што выглядалі яны амаль як супрацьлегласці,
Гары адчуваў, што калі Люцыус мог бы зрабіць гэта з бездакорнай
элегантнасцю, то Снэйп цябе проста прыб'е і пойдзе далей.

--- Сядзець, --- сказаў прафесар Северус Снэйп. --- Зараз жа.

Гары і некалькі іншых вучняў, якія стаялі групкамі, нешта абмяркоўваючы,
кінуліся да сваіх парт. Гары планаваў сядзець побач з Герміёнай, але нейкім 
чынам апынуўся на бліжэйшым да сябе незанятым стуле, побач з Джасцінам Фінч-Флэчлі
(бо гэты быў аб'яднаны ўрок для Рэйвенкло і Хафлпаф), на две парты лявей ад 
Герміёны.

Северус сеў за свой стол і без аніякага пераходу або ўводзінаў сказаў:

--- Ганна Эбат.

--- Тут, --- сказала Ганна крыху трасучымся голасам.

--- С'юзан Боўнз.

--- Я.

І так далей, усё ішло нармальна, пакуль...

--- А, Гары Потэр. Наша новая... \emph{зорка.}

--- Зорка на месцы, \emph{сэр.}

Палова класа ўздрыгнула, некаторыя выглядалі так, быццам ацэньвалі адлегласць 
да дзвярэй, і за які час яны здоляць яе перасекчы.

Северус усміхнуўся, быццам чакаў чагосьці яшчэ, і потым выклікнуў наступнае 
імя ў лісце.

Гары мыслена выдахнуў. Ну, ладна. Відавочна, гэты чалавек яго знелюбіў, па нейкай 
прычыне. Але наогул, лепей ён абярэ Гары, чым, скажам, Нэвіла або Герміёну.
Гары значна болей здольны абараніцца ад такога. Так што мабыць гэта і да лепшага.

Калі пярэклічка была скончана, Северус агледзеў увесь клас. Ягоныя вочы былі 
настолькі ж выразныя, як начное неба, пазбаўленае зор.

--- Вы прыйшлі, --- сказаў Северус ціхім голасам, які амаль было не чуць 
з задніх радоў, --- каб навучыцца дэлікатнаму і дасканаламу майстэрству
зеллеварэння. Тут настолькі мала бязглуздага махання палачкамі, што не ўсе 
адразу павераць, што гэта --- магія. Не ўсім дадзена зразумець прыгажосць 
павольна кіпячага катла, зіготкіх пераліваў пары, тонкую сілу ўзвару, які 
прабіраецца па чалавечым венам, --- яго самазадавалены тон стаў амаль зларадным,
--- чаруючы розум, туманячы пачуцці...

Уся гэтая сцэна пужала больш і больш. 

--- Я навучу вас, як выварыць славу, наліць у прабірку 
гонар, і нават закаркаваць смерць --- калі вы толькі не кучка дурняў,
якіх мне звычайна прыходзіцца вучыць.

Северус нейкім чынам заўважыў скептычны Гарын выраз, прынасмі яго позірк раптам 
скочыў у Гарын бок.

--- Потэр! Што атрымаецца, калі дадаць мелены корань асфадэля да настойкі палыні?

Гары міргнуў.

--- А гэта было ў "Магічныя настоях і ўзварах"? --- сказаў ён. --- Я толькі што 
скончыў іх чытаць, і не памятую нічога, дзе карысталася палынь... 

Рука Герміёны ўзнялася ў паветра. Гары кінуў на яе позірк, ад чаго яе рука 
ўзнялася нават вышэй.

"Ц-ц-ц", --- пацокаў языком Снэйп, і алейным голасам сказаў: 

--- Відавочна, слава не дадае розуму.

--- Праўда? --- сказаў Гары. --- А вы самі толькі што абяцалі нас навучыць 
нас выварыць славу. Як яно дакладна працуе, скажыце? Выпіваеш, і адразу --- 
знакамітасць?

Тры чвэрці класа ўздрыгануліся.

Рука Герміёны павольна апусцілася. Добра. Яны былі супраціўнікі, але яна была не 
з тых, што будзе падыгрываць такому настаўніку, асабліва, калі было відавочна, 
што ён намагаецца зняважыць Гары.

--- Паспрабуем яшчэ раз, --- сказаў Северус. --- Потэр, калі я загадаю вам 
знайсці безоар, дзе вы будзеце яго шукаць?

Гары намагаўся трымаць свой гнеў пад кантролем. Першы адказ, які прыйшоў яму да 
галавы, быў "абракадабра".

--- Гэтага таксама няма ў падручніку, --- сказаў Гары, --- але я чытаў ў адной 
маглаўскай кніжке, што "трыхабезоар" --- гэта маса зацвярдзелых валасоў, якія 
знаходзяць у чалавечых страўніках, і маглы калісьці верылі, што яны маглі 
лячыць любыя атруты... 

--- Няверна, --- сказаў Северус. --- Безоар можна знайсці ў страўніку коз,
ён не складаецца з валасоў, і ён вылячыць ад большасці ядаў, але не ад усіх. 

--- Я не \emph{сказаў}, што ён вылячыць, я сказаў, што чытаў пра нешта такое ў 
адной маглаўскай кнізе...

--- Нікому тут не цікавыя вашыя \emph{нікчэмныя} маглаўскія кнігі. Апошняя спроба.
Потэр, у чым розніца паміж прастрэл-травой і воўчым коранем?

Гэта было апошняй кроплей.

--- Ведаеце, --- сказаў Гары ледзяным тонам, --- у адной з маіх \emph{выбітных}
маглаўскіх кніг расказваецца пра феномен, калі людзі падаюцца сабе
вельмі разумнымі праз тое, што задаюць пытанні пра дробныя факты, вядомыя толькі ім.
Простыя сведкі адзначаюць толькі, што той, хто пытае, ведае адказ, а тыя, каго ён пытае --- не,
і без кантэксту не здольныя зрабіць папраўку на несправядліваць такой гульні.
Такім чынам, прафесар, вы можаце мне сказаць, сколькі электронаў знаходзіцца на 
знешняй арбіталі атама вуглярода?

Усмешка Снэйпа пашырылася.

--- Чатыры, --- сказаў ён. --- Аднак гэта бязглузды дробны факт, таму не 
клапоцьцеся гэта запісваць, клас. Да вашага ведама, Потэр: асфадэль і палынь 
даюць сонны ўзвар такой моцы, што яго называюць "жывой смерцю". Прастрэл-трава і 
воўчы корань --- гэта адна і тая ж расліна, таксама вядомая як "аканіт", і вы 
ведалі бы гэта, калі прачыталі бы \emph{Тысячу Магічных Зёлкаў і Грыбоў}. Думалі,
што няма сэнсу адчыняць кнігу да навучальнага года, так, Потэр? Усе астатнія ---
запісвайце, каб не стаяць потым, як гэты вось, --- Северус зрабіў паузу, 
выглядаючы вельмі самазадаволеным. --- І гэта будзе... пяць балаў? Не, хай будзе 
дзесяць балаў з Рэйвенкло за спрэчкі з настаўнікам.

Герміёна ахнула, як і яшчэ некалькі студэнтаў.

--- Прафесар Северус Снэйп, --- адчаканіў Гары, --- я не ведаю ні малейшай прычыны,
чым я заслужыў такую варожасць з вашага боку. Калі ў вас існуе нейкая праблема ў 
маім дачыненні, пра якую мне не вядома, я прапаную...

--- Маўчаць, Потэр. Яшчэ дзесяць балаў з Рэйвенкло. Астатнія, адкрыць 
падручнікі на старонцы нумар тры.

Гарына горла толькі крыху, зусім крыху сціснулася, і вочы яго былі зусім сухія.
Плакаць было дрэннай стратэгіяй для перамогі над настаўнікам зёлак, таму плакаць 
не было сэнсу.

Павольна Гары сеў вельмі роўна. Уся ягоная кроў змянілася на вадкі азот. Ён ведаў,
што намагаецца стрымліваць свой гнеў, але ніяк не мог прыпомніць, чаму.

--- Гары, --- шалёна прашаптала Герміёна праз дзве парты, --- спыніся, калі ласка,
ты не вінаваты, мы гэта не залі...

--- Размовы ў класе, Грэнджэр? Тры...

--- Як тут у вас, --- сказаў голас, халаднейшы за нуль кельвінаў, --- падаецца 
фармальная скарга на аб'юзіўнага настаўніка? Уголас Намесніцы Майстра, або 
патрэбна заява ў трох экземплярах у Папячыцельскую Раду?.. Нехта можа 
патлумачыць, як гэта працуе?

Клас замер.

--- Спагнанне на адзін месяц, Потэр, --- сказаў Северус, усміхаючыся яшчэ шырэй.

--- Я адмаўляюся прымаць ваш аўтарытэт у якасці настаўніка, і я не буду адрабляць 
ніякае вашае спагнанне.

Людзі перасталі дышаць.

Усмешка Северуса знікла. 

--- Тады цябе... --- ён раптам змоўк.

--- Адлічаць, вы гэта хацелі сказаць? --- Гары цяпер таксама ўсміхаўся, але
тонка сціснутымі вуснамі. --- Але падобна, што вы сумняваецеся ў сваёй здольнасці 
выканаць такую пагрозу, або вас пужаюць яе наступству. Аднак я, з другога боку,
не баюся і не сумняваюся ў сваім намеры знайсці школу з меней аб'юзіўнымі 
настаўнікамі. Або хатніх настаўнікаў, звыклая для мяне практыка. У мяне 
дастаткова грошаў у сховішчы. Нейкія ўзнагароды за перамогу над Цёмным Лордам,
ведаеце? Аднак у Хогвартс ёсць настаўнікі, якія мне падабаюцца, і я думаю, 
што будзе прасцей знайсці спосаб ад вас пазбавіцца.

--- Пазбавіцца ад мяне? --- Северус таксама тонка ўсміхаўся. --- Якая забаўная 
самаўпэўненасць. І як ты збіраешся гэта зрабіць, Потэр?

--- Я так разумею, што ўжо ёсць колькі скаргаў на вас з боку студэнтаў і іх 
бацькоў, --- гэта была толькі здагадка Гары, але даволі верагодная, --- што 
прымушае мяне задаць сабе адзінае пытанне --- чаму вы ўсё яшчэ тут? 
Няўжо ў Хогвартс так дрэнна з фінансамі, што яны не могуць дазволіць сабе 
сапраўднага настаўніка зеллеварэння? Я магу дапамагчы, калі так. Думаю, калі 
падвоіць вашу зарплату, можна лёгка знайсці тузін лепшых настаўнікаў.

Два ледзяных полюса ў класе выпраменьвалі марознае паветра праз увесь клас.

--- Баюся, вы не знойдзеце, --- сказаў ціха Северус, --- ні грама спачування
ад Папячыцельскай Рады.

--- Люцыус... --- сказаў Гары. --- \emph{Вось} чаму вы ўсё яшчэ тут! Магчыма 
мне варта абмяркаваць гэта з ім непасрэдна. Думаю, ён і так не супраць са мной 
пазнаёміцца. Цікава, ці ёсць у мяне нешта, што быць яму патрэбна?..

Герміёна адчайна качала галавой. Гары бачыў гэта краем вока, але ўся ягоная ўвага 
была на Снэйпе.

--- Ты вельмі бязглуздае дзіця, --- сказаў Северус. Болей ён не усміхаўся. --- У 
цябе няма нічога, што Люцыус ацэніць болей за нашае сяброўста. І нават калі так, 
у мяне ёсць іншыя сябры, --- яго голас стаў жарстчэй. --- І дарэчы, мяне ўсё больш 
цікавіць неймавернасць твайго размеркавання не на Слізэрын. Якім чынам ты здолеў 
пазбегнуць майго факультэта? Ах, так, бо Капялюш сказаў, што ён \emph{пажартаваў}.
Першы раз у гісторыі. Аб чым вы сапраўды з ім размаўлялі, Потэр? Ці было ў цябе нешта,
што было яму патрэбна?

Гары паглядзеў прама ў халодныя вочы Северуса, і ўспомніў, як Размеркавальны 
Капялюш папярэдзіў не сустракацца ні з кім позіркам, пакуль ён пра гэта думаў...
Гары хутка перавёў вочы на стол Снэйпа.

--- Ты дзіўным чынам баішся паглядзець мне ў вочы, Потэр!

Раптоўны шок разумення.

--- Дык гэта пра \emph{вас} мяне папярджваў Капялюш!

--- Што? --- Северус гучаў сапраўды здзіўлена, хаця, канешне, Гары не 
мог глядзець яму ў твар.

Гары падняўся з-за стала.

--- Сядзець, Потэр, --- сказаў злосны голас з таго месца, куды ён не глядзеў.

Гары праігнараваў яго, і агледзеў клас. 

--- У мяне няма намеру дазваляць чарговаму некампетэнтнаму настаўніку 
сапсаваць мой час у Хогвартс, --- сказаў Гары са смяротным спакоем. --- Я 
збіраюся пакінуць гэты клас, і знайсці сабе іншага настаўніка. Калі вы вырашыце,
што з вас хопіць гэтага булінга, запрашаю вас далучыцца да мяне.

--- \emph{Потэр, сядзь на месца!}

Гары падышоў да дзвярэй і схапіўся за ручку.

Яна не павярнулася.

Гары павольна абярнуўся, і заўважыўшы краем вока злосную ўсмешку Снэйпа,
паспешна адвеў позірк.

--- Адчыніце дзверы.

--- Не, --- сказаў Северус.

--- Вы прымушаеце мяне адчуваць сябе ў пастцы, --- ягоны голас быў настолькі 
ледзяны, што гучаў, быццам чужы, --- і гэта памылка.

Северус засмяяўся.

--- І што нашае дзіцяня збіраецца з гэтым рабіць, хм?

Гары зрабіў шэсць доўгіх крокаў ад дзвярэй, пакуль не параўняўся з апошнім 
шэрагам парт.

Потым ён павярнуўся да Северуса, і драматычным рухам падняў правую 
руку, гатовы цокнуць пальцамі.

Нэвіл ускрыкнуў, і схаваўся пад сталом. Астатнія вучні інстынктыўна адхілілся, 
або паднялі рукі, каб закрыцца.

--- \emph{Гары, спыніся!} --- прасычэла Герміёна. --- Што б ты ні задумаў зрабіць
з ім, не рабі гэта!

--- Вы тут што, усе \emph{звар'яцелі?} --- пачуўся рэзкі голас Снэйпа.

Марудна Гары апусціў руку. 

--- Я не думаў яму шкодзіць, Герміёна, --- сказаў Гары ціха. --- Я проста хацеў 
паўзрываць дзверы.

Але ў гэты момант ён успомніў, што нельга трансфігураваць нешта, што можа гарэць,
што значыла, што ідэя вярнуцца назад у часе, і спытаць Фрэда і Джорджа трансфігураваць
пэўную (акуратна падлічаную) колькасць выбухоўкі была не такой ужо і добрай...


--- \emph{Silencio,} --- пачуўся голас Снэйпа.

Гары хацеў сказаць "Што?", але не змог выдаць ніякага гука.

--- Гэты балаган ужо ні ў якія вароты не лезе. Думаю, твой сённяшні час на
пошук для сябе новых праблем скончыўся, Потэр.  Ты самы злачынны і некіруемы 
студэнт, якога я калісьці сустракаў, і я не памятаю дакладна, колькі зараз 
балаў у Рэйвенкло, але я ўпэўнены, што злодею зняць іх усе. Дзесяць балаў 
з Рэйвенкло. Дзесяць балаў з Рэйвенкло. Дзесяць балаў з Рэйвенкло!..
Колькі ўжо ўсяго? Пяцьдзесят балаў з Рэйвенкло! А зараз сядзь і назірай за тым, як вучацца 
астатнія! 

Гары сунуў руку ў махляскін, і паспрабаваў сказаць "маркер", але, натуральна, 
нічога не атрымалася. На пэўную долю секунды гэта яго прыпыніла, але потым 
ён здагадаўся паказаць літары азбукай глухіх "М-А-Р-К-Е-Р", і гэта спрацавала.
"С-Ш-Ы-Т-А-К" --- і ў яго ў руцэ з'явіўся сшытак. Ён сеў за бліжэйшую пустую 
парту, зрабіў надпіс, схаваў маркер у карман мантыі для больш хуткага доступа, 
і падняў сшытак так, каб паведамленне было бачна астатнім вучням.



\begin{writtenNote} 
    Я ПАКІДАЮ ГЭТЫ КЛАС\\ ХТО ХОЧА ДАЛУЧЫЦЦА?
\end{writtenNote}


--- Ты канчаткова звар'яцеў, Потэр, --- сказаў Снэйп з халоднай пагардай.

Больш ніхто нічога не сказаў.

Гары іранічна пакланіўся ў бок настаўніцкага стала, падышоў да шафы, адчыніў 
дзверы, зайшоў у шафу, і бразнуў дзвярыма за сабой.

Потым адтуль пачуўся прыглушаны гук цокаючых пальцаў, а потым наступіла цішыня.

Вучні паглядзелі адно на аднаго збянтэжана, спужана. 

З лютым абліччам прафесар зеллеварэння ў некалькі жудасных крокаў перасек пакой, 
і расхінуў дзверы шафы.

Шафа была пустая.


\later

Гадзінай раней Гары прыслухваўся ў зачыненай шафе. Звонку было ціха, але 
рызыку ўсё роўна выключаць было нельга.

"П-Л-А-Ш-Ч", --- паказаў ён знакамі махляскіну.

Стаўшы нябачным, ён вельмі асцярожна і павольна адчыніў дзверы і выглянуў з шафы.
У класе нікога не было. 

Дзверы ў клас не былі замкнутымі.

І толькі калі Гары апынуўся далёка ад месца пагрозы, у калідоры, нябачны, ягоны 
гнеў адступіў, і ён зразумеў, \emph{што} ён толькі што зрабіў.

Што ён толькі што зрабіў.

На нябачным абліччы Гары застыла маска абсалютнага жаху.

Ён паўстаў супраць настаўніка, што было на тры парадка бязглудзей усяго, што 
ў Гары атрымлівалася дагэтуль. Ён пагразіўся пакінуць Хогвартс, і, магчыма,
будзе вінны давесці пагрозу да канца. Ён страціў усе балы Рэйвенкло. І потым ён 
скарыстаў часаварот.

Ягонае ўяўленне паказвала яму розныя выявы: як паглядзяць на яго бацькі дома;
як на твары МакГонагал прарастае расчараванне... Гэта было балюча і невыносна,
і ён проста не мог прыдумаць ніякага спосаба выбрацца з гэтай пасткі...

Думка, якую Гары дазволіў сабе думаць, казала яму, што калі гнеў завёў яго сюды,
магчыма, ён паможа Гары і выбрацца, бо ў раз'юшаным стане 
нейкім чынам яго галава думала ясней і хутчэй.

А думка, якую Гары не дазваляў сабе думаць, казала, што ён проста не перажыве 
гэты дзень, калі ён не ўззлуецца.

Ён адкінуў усе думкі, і прымусіў сябе ўспомніць пякучаю абразу...



\emph{Ц-ц-ц, відавочна, слава не дадае розуму.}

\emph{...дзесяць балаў з Рэйвенкло за спрэчкі...}

Спакой і прахалода прабеглі на яго венах, быццам хваля, адбітая хвалярэзам, і 
Гары павольна выдахнуў.

Добра. Сардэчна вітаем вас зноў у разумным стане.

Ён адчуваў расчараванне за сябе не-гнеўнага --- за тое, што так хутка 
зламаўся і пастараўся як мага хутчэй пакінуць бойку. Прафесар Северус Снэйп быў 
праблемай \emph{для ўсіх}. Звычайны Гары забыў пра тое, і намагаўся выратаваць 
толькі сябе. І пакінуць усіх астатніх ахвяр у бядзе? Пытанне змянілася з 
"як выратаваць сябе?" на "як знішчыць прафесара зеллеварэння?"

\emph{Відаць, вось ён які, мой цёмны бок. Крыху зняважальны тэрмін, бо 
мой светлы бок болей эгаістычны і труслівы, не кажучы пра бязглуздасць і паніку.}

І зараз, калі яго галава прачысцілася, яму было відавочна, што рабіць далей.
У яго ўжо была гадзіна на падрыхтоўку, і калі спатрэбіцца --- ў запасе было яшчэ 
пяць...


\later

Мінерва МакГонагал чакала ў кабінеце Майстра.

Дамблдор сядзеў на сваім троне за сваім сталом, апрануты ў чатыры слоя 
фармальных мантый колеру лаванды. Мінерва сядзела ў крэсле па другі бок стала,
насупраць ад Северуса Снэйпа. Перад імі стаяў пусты стул.

Яны чакалі Гары Потэра.

\emph{Гары}, --- падумала Мінерва ў роспачы, --- \emph{ты абяцаў не кусаць 
настаўнікаў!}

І яна ясна бачыла гнеўнае аблічча Гары і чула ўззлаваны адказ: \emph{Я абяцаў
не кусаць нікога, акрамя тых, хто не ўкусіць мяне першым.}

У дзверы пагрукалію

--- Увайдзіце! --- адказаў Дамблдор.

Дзверы адчыніліся, і ўвайшоў Гары Потэр. Мінерва амаль не ахнула ўголас.
Хлопец выглядаў спакойным, сабраным, цалкам упэўненым у сабе.

--- Добрай ра... --- голас Гары раптам перарваўся, і яго сківіца адвалілася.

Мінерва прасачыла за яго позіркам, і убачыла, што Гары глядзіць на Фоукса, які 
сядзеў на залатым насесце. Фоукс узмахнуў сваімі чырвона-залатымі крыламі,
быццам языкамі ангю, і крыху нахіліў галаву ў паклоне да хлопца. 

Гары ўтаропіўся ў Дамблдора.

Дамблдор яму падміргнуў.

Мінерва адчувала, што чагосьці не ведае.

Нечаканая нерашучасць прабегла па твары Гары. Ягоная ўпэўненасць захісталася.
У яго вачах паказаўся страх, потым злосць, і потым ён зноў супакоіўся.

Па спіне Мінервы прайшоў халадок. Нешта з ім было зусім не так.

--- Калі ласка, прысядзь, --- сказаў Дамблдор. Ён зноў быў сур'ёзны.

Гары сеў на стул.

--- Гары, --- сказаў Дамблдор, --- прафесар Снэйп далажыў мне аб падзеях 
сённяшняга ўрока. Ці хочаш ты расказаць, што здарылася, сваімі словамі?

Гары кінуў кароткі позірк у бок Северуса.

--- Гэта не складана. Ён спрабаваў буліць мяне такім жа чынам, як ён буліў 
усіх не-слізэрынаў у Хогвартс з таго дня, як Люцыус навязаў вам яго. Што 
датычыцца іншых дэталяў, я прашу асабістай размовы з вамі. У рэшце рэшт, нельга 
чакаць, што 
студэнт, які падае скаргу на аб'юзіўныя паводзіны настаўніка, будзе казаць 
шчыра ў прысутнасці гэтага настаўніка. 

Тут Мінерва МакГонагал не ўтрымалася, і ахнула ўголас.

Северус проста засмяяўся.

Аблічча Майстра стала змрочным.

--- Містэр Потэр, --- сказаў ён, --- нікому не дазволена гаварыць пра 
прафесара Хогвартс такім тонам. Баюся, што вы зыходзіце з памылковых 
дапушчэнняў. Я цалкам давяраю прафесару Северусу Снэйпу, і ён ыслужыць Хогвартс 
па майму запрашэнню, і не быў навязаны мне Люцыусам Малфоем.

На некалькі хвілін кабінет апанавала ціша.

Калі хлопец загаварыў, ягоны голас быў ледзяны.

--- Я чагосьці не ведаю?

--- Даволі шмат чаго, містэр Потэр, --- сказаў Майстра. --- Для пачатку
вы павінны зразумець, што мэта дадзенага мітынгу --- абмяркаваць вашае спагнанне 
за падзеі гэтай раніцай. 

--- Гэты чалавек гадамі тэрарызуе вашую школу. Я размаўляў са студэнтамі, 
і сабраў дастаткова гісторый, каб пачаць супраць яго кампанію ў СМІ. Некаторыя 
з малодшых вучняў плакалі, пакуль расказалі гэта. Я амаль сам плакаў, калі 
чуў іх гісторыі! \emph{І вы самі паклікалі гэтага аб'юзера ў сваю школу?
За што вы караеце сваіх вучнях? За што?} 

Мінерва адчула камок у горле. Яна... яна часам думала пра гэта, але кожны
раз яна так і не...

--- Містэр Потэр, --- сказаў Майстра, голас яго быў строгі, --- мы сабраліся 
не дзеля таго, каб абмяркоўваць прафесара Снэйпа. А дзеля вас і вашай пагарды да 
школьнай дысцыпліны. Прафесар Снэйп прапанаваў, і я пагадзіўся, што тры месяца 
спагнання будзе адпаведнай...

--- Адхілена, --- сказаў холадна Гары.

Мінерва згубіла дар мовы.

--- Гэта не просьба, містэр Потэр, --- сказаў Дамблдор. Поўная моц яго позірку 
сканцэнтравалася на хлопцы. --- Гэта вашае спагна...

--- Вы растлумачыце мне, чаму вы дазваляеце гэтаму чалавеку здзеквацца над
падапечнымі вам дзецьмі, і калі тлумачэнне мяне не задаволіць, я пачну кампанію 
ў СМІ супраць \emph{вас}.

Гэта была такая несхаваная і моцная хваля \emph{lèse
majesté}\footnote{\emph{фр.} "Абражанне маястату", шукайце ў вікіпедыі.}, што 
Мінерва пахіснулася, як ад удару.

Нават Снэйп выглядаў шакаваным.

--- Гэта, Гары, было бы эктрэмальна бязглузда, --- сказаў павольна Дамблдор.
--- Я --- галоўная фігура на гэтай шахматнай дошцы, якая супярэчыць Люцыусу.
Такі ўчынак дасць яму вялікую перавагу, і я не думаю, што той бок --- гэта твой 
свядомы выбар.

Хлопец сядзеў нерухома некалькі доўгіх секунд.

--- Гэтая размова становіцца асабістай, --- сказаў ён нарэшце. Ягоны позірк 
прыгнуў у бок Северуса. --- Адышліце яго.

Дамблдор пакачаў галавой. 

--- Гары, ці не сказаў я толькі што, што цалкам давяраю Северусу Снэйпу?

--- Тое, чым займаецца гэты чалавек, пазбаўляе вас перавагі! Я не адзіны, хто 
можа пачаць публічную кампанію супраць вас! Гэта нейкае вар'яцтва! Навошта 
вы гэта робіце?

Дамблдор уздыхнуў. 

--- Выбачай, Гары. Гэта звязана з рэчамі, пра якія ты яшчэ не гатовы дазнацца.

Хлопец некалькі секунд узіраўся ў Дамблдора. Потым павярнуўся паглядзець на Снэйпа.
Потым зноў на Дамблдора.

--- Гэра дакладна вар'яцтва, --- сказаў Гары павольна. --- Вы не звальняеце яго,
таму што думаеце, што ён --- частка патэрна. Што Хогвартс, калі хоча звацца 
годнай магічнай школай, павінен мець злоснага настаўніка зеллеварэння, таксама,
як і павінен мець прывіда, які выкладае гісторыю.

--- Гучыць як нешта з майго рэпертуара, ці не так? --- сказаў Дамблдор, усміхаючыся.

--- Непрымальна, --- сказаў Гары безвыразна. Ягоны позірк быў цёмным і халодным.
--- Я не буду цярпець булінг і аб'юз. Я абдумаў шмат спосабаў вырашэння гэтай 
праблемы, але давайце ўсё спрасцім. Ці ён сыходзіць, ці я.

Мінерва зноў ахнула. Нейкі дзіўны выраз прабяжаў у вачах Северуса.

Цяпер і позірк Дамблдора стаў халодным.

--- Адлічэнне, містэр Потэр, --- гэта найвышэйшае пакаранне, якое можа накласці
на студэнта. І звычайна яго не карыстаюць студэнты, каб пагражаць Майстру. 
Гэта --- найлепшая магічная школа ў свеце, і магчымасць атрымаць тут адукацыю
даецца не кожнаму. Вы, магчыма, упэўнены, што Хогвартс не зможа працягваць 
без вас?

Гары сядзеў моўчкі, усміхаўчыся сціснутымі вуснамі.

Раптоўны жах абуяў Мінерву. Гары што, вырашыў...

--- Вы забываеце, --- сказаў Гары, --- што не вы адзін здольны бачыць патэрны.
\emph{Гэта прыватная размова. Адышліце яго...} --- Гары падняў руку ўказаці на 
Спэйпа, і раптам спыніўся пасярод фразы, і пасярод жэста. 

Мінерва бачыла, як змяніўся выраз Гары, калі ён успомніў.

У рэшце рэшт, гэта яна яму расказала.

--- Містэр Потэр, --- сказаў Майстра, --- яшчэ раз: Северус Снэйп мае 
мой поўны давер.

--- Вы сказалі яму, --- прасычэў хлопец. --- Ну што за дурань...

Дамблдор прапусціў абразу міма вушэй.

--- Сказаў што?

--- Што Цёмны Лорд жывы.

--- \emph{Дзеля Мерліна, што ты такое вярзеш, Потэр?} --- крыкнуў Снэйп, 
выразна здзіўленым і гнеўным тонам.

Гары глянуў на яго, змрочна ўсміхаючыся.

--- О, да мы ўсё ж такі слізэрын, у рэшце рэшт, --- сказаў ён. --- А я ўжо быў 
пачаў сумнявацца.

Настала цішыня.

Першым яе перарваў Дамблдор. Ягоны голас быў мяккі і спакойны. 

--- Гары, аб чым \emph{канкрэтна} ты хочаш сказаць?

--- Прабач, Альбус, --- прашаптала Мінерва.

Северус і Дамблдор павярнуліся да яе.

--- Прафесар МакГонагал мне не сказала, --- пачуўся голас Гары, паспешны, меней
упэўнены. --- Я здагадаўся. Як я і казаў, я таксама віжу патэрны. Я пажартаваў пра 
Цёмнага Лорда, а яна стрымала сваю рэакцыю такім жа чынам, як і Снэйп зараз. 
Але рэакцыя была крышачку ненатуральнай, і я зразумеў, што гэта самакантроль, падман.

--- І потым я ўсё расказала, --- працягнула Мінерва дрыжачым голасам, --- і што 
вы, я, і Северус --- адзіныя, хто ведае праўду.

--- Расказала толькі дзеля таго, каб перадухіліць мае пытанні да ўсіх 
прысутнічаўшых на вуліцы, --- сказаў Гары. Потым ён выдаў кароткі смешок. --- 
Сапрады, было б варта застаць аднаго з вас асабіста, і сказаць, што я пра ўсё 
даведаўся, паглядзець, ці высветлілася б чагосьці новага. Верагоднасць малая, 
але паспрабаваць было б варта. Дарэчы, я не жартаваў на конт сваёй пагрозы, 
і ўсё яшчэ чакаю поўнага тлумачэння.

Северус глядзеў на Мінерва позіркам, поўным чыстай пагарды. Яна глядзела ў адказ,
не хаваючыся. Яна ведала, што заслужыла гэта.

Дамблдор адкінуўся на падушкі свайго трона. Такім змрочным яна не бачыла яго 
са дня, калі памёр ягоны брат. 

--- І твая пагроза --- перайці на бок Вальдэморта, калі мы не выканаем 
твае пажаданні.

Голас Гары быў вострым, як брытва. 

--- З вялікім жалем хачу данесці да вас, што вы не цэнтр сусвету. Я не збіраюся
пакідаць магічную Брытанію. Я пакіну \emph{вас}. Я не маленькі нікчэмны Фрода. 
Гэта \emph{мой} квест, і калі вы хочаце быць яго часткай, гуляце па маіх 
правілах.


--- Я пачынаю сумнявацца ў вашай годнасці на ролю героя, містэр Потэр, --- 
вочы Дамблдора былі па-ранейшаму халодныя.

Адказны позірк Гары быў той жа тэмпературы.

--- Я пачынаю сумнявацца ў вашай годнасці на ролю майго Гэндальфа, \emph{містэр
Дамблдор.} 
Ну максімум --- на Бораміра. % Originaфl says "Boromir was at least a plausible mistake."
% this does not make much sense. 
І дарэчы, што гэты \emph{назгул} робіць у маім Братэрстве?

Мінерва канчаткова разгубілася. Яна паглядзела на Северуса, але той адвярнуўся,
каб Гары не было бачна яго ўсімхаючыся твар.

--- Я думаю, --- сказаў задумённа Дамблдор, --- што з твайго пункта гледжання 
гэтае пытанне мае рацыю... Так... Містэр Потэр, калі прафесар Снэйп з 
гэтага дня пакіне вас у спакоі, ці будзе гэта апошні ваш такой візіт да мяне,
або мне чакаць вас кожны тыдзень з новымі патрабаваннямі?

--- Пакінуць \emph{мяне} у спакоі? --- голас Гары быў поўны абурэння. --- Я не 
адзіны пацярпелы, і дакладна --- не самы безабаронны. \emph{Ці вы забылі,
якімі ўразлівымі могуць быць дзеці? Што яны могуць пакутваць?} З гэтага моманту
прафесар Снэйп будзе ставіцца да \emph{кожнага} студэнта Хогвартс з усёй належнай 
прафесійнай ветлівасцю, інакш шукайце або новага настаўніка, або новага героя.

Дамблдор зарагатаў. Ва ўсё горла, цёплым, вясёлым смехам, быццам Гары толькі 
што выканаў перад ім нейкі цудоўны жартоўны танец.

Мінерва не смела паварушыцца. Павярнуўшы вочы ў бок Северуса, яна ўбачыла, 
што ён у такім жа стане.

Гары ад гнева пабляднеў.

--- Калі вы думаеце, што я жартую, Майстра, то вы моцна памыляецеса. Гэта не 
просьба. Гэта вашае спагнанне.

--- Містэр Потэр!.. --- пачала казаць Мінерва, але яна нават не ведала, што сказаць
далей, і вырашыла не працягваць.

Гары толькі махнуў у яе бок, каб яна маўчала, і сказаў Дамблдору:

--- Калі гэта падаецца вам не вельмі ветліва з майго боку, --- ягоны голас крыху 
змягчыўся, --- мне таксама вашыя словы падаліся не вельмі ветлівымі. Вы ніколі 
не падумалі бы сказаць такое тым, каго лічыце нармальнымі чалавечымі істотамі,
а не бязпраўным вучнем, таму я адказваю вам тым жа...

--- О, ну, канешне, ка-неш-не, гэта маё спагнанне, калі гэта можна так назваць.
Ну \emph{канешне}, вы шантажуеце мяне дзеля выратавання сваіх бедных калег, не 
дзеля сябе. Не магу ўявіць сабе, чаму такая думка не прыйшла мне да галавы раней!
--- Дамблдор смяяўся нават грамчэй, і нават пачаў грукаць па стале кулаком. 

Гары няўпэўнена паглядзеў на Мінерву, і першы раз за час размовы звярнуўся да яе:

--- Прабачце, эм... яму трэба прыняць нейкія лекі, або нешта такое?

--- Э-э... --- у Мінервы не было ні малейшай ідэі, што на такое можна было 
адказаць.

--- Так, --- сказаў Дамблдор, і суцішыўся. Ён выцер слёзы з вачэй. --- 
Шчыра прашу прабачэння за перашкоду. Прашу вас, працягвайце свой шантаж.

Гары раскрыў рот, потым закрыў. Ён завагаўся.

--- І... ён перастане чытаць розумы студэнтаў.

--- Мінерва, --- сказаў Снэйп, --- ты...

--- Капялюш мяне парярэдзіў, --- сказаў паспешна Гары.

--- \emph{Што?}

--- Больш нічога сказаць не магу. Але думаю, што гэта ўсё. Я скончыў.

Цішыня.

--- І што цяпер? --- сказала Мінерва, калі стала зразумела, што ніхто больш 
не збіраецца нічога казаць.

--- Што цяпер? --- адклікнуўся Дамблдор. --- Ну што... цяпер герой перамагае,
натуральна.

--- \emph{Што?} --- сказалі Северус, Мінерва, і Гары.

--- Так, ён адназначна здолеў загнаць нас у кут, --- сказаў Дамблдор са 
шчаслівай усмешкай. --- Але Хогвартс \emph{сапраўды} патрэбны злосны майстра 
зелляў, інакш яна не будзе годнай магічнай школай, ці не так? А што калі, 
скажам... прафесар Снэйп будзе ставіцца злосна толькі да пяцікурснікаў і 
старэй?

--- \emph{Што?} --- сказалі зшоў усе трая.

--- Гэта калі ваш клопат накіраваны на самых уразлівых ахвяр. Можа і 
праўда, Гары, можа я і \emph{сапраўды} за гэтыя годы пазабыў, што гэта такое --- 
быць дзіцём. Я прапаную кампраміс. Северус працягне несправядліва прысуджаць 
балы Слізэрыну, і таксама дазваляць паслабленне дысцыпліны сваім студэнтам. 
І ён будзе злосны да не-слізерынаў пачынаючы з пятага курса і вышэй. Для астатніх
ён будзе страшны, але не аб'юзіўны. Ён дасць абяцанне чытаць розумы вучняў толькі 
калі іх бяспецы будзе нешта пагражаць. Хогвартс не згубіць свайго гадкага 
настаўніка зёлак, і найшбольш уразлівыя ахвяры, як вы гэта называеце, будуць 
у бяспецы.

Мінерва МакГонагал не была ў такім шоку ніколі ў жыцці. Яна няўпэўнена 
паглядзела на Северуса, выраз ягога быў цалкам нейтральным, быццам ён так і 
не вырашыў, якую маску апрануць. 

--- Думаю, гэта падыходзіць, --- сказаў Гары. Ягоны голас гучаў крыху дзіўна.

--- Вы сур'ёзна? --- сказаў Северус. Ягоны голас быў настолькі ж пазбаўлены 
выразу, як і аблічча.

--- Мне вельмі падабаецца такое рашэнне, --- павольна сказала Мінерва. 
Рашэнне ёй так падабалася, што сэрца яе дзіка калацілася. --- Але як мы можам 
растлумачыць гэта студэнтам? Калі Северус ставіўся дрэнна да ўсіх --- гэта 
было нешта звычайнае, але зараз...

--- Гары скажа вучням, што ён высветліў жудасны сакрэт Северуса, і прымяніў 
крышачку шантажу, --- сказаў Дамблдор. --- У рэшцэ рэшт, гэта праўда. Ён дазнаўся,
што Северус чытаў розумы, і ён дакладна нас шантажаваў.

--- Гэта вар'яцтва! --- выбухнуў Северус.

--- Гва-ха-ха! --- д'ябальскім смехам адказаў Дамблдор.

--- Эм... --- сказаў Гары, няўпэўнены. --- А калі нехта спытае, чаму я не дамовіўся
пра старэйшыя курсы? Я не магу іх вінаваціць за абурэнне, бо гэтая частка --- 
не зусім мая ідэя...

--- Скажы ім, --- сказаў Дамблдор. --- што не ты прапанаваў такі кампраміс,
і ты зрабіў усё, што мог. І адмаўляйся казаць штосьці яшчэ. Бо гэта таксама праўда.
Гэта тонкае майстэрства, з цягам часу будзе атрымлівацца ўсё лепей.

Гары задумёна кіўнуў.

--- А балы, знятыя з Рэйвенкло?

--- Яны павінны застацца знятымі.

Гэта сказала Мінерва.

Гары паглядзеў на яе.

--- Я прашу прабачэння, містэр Потэр, --- сказала яна. Ёй было вельмі шкада, 
але гэта быў яе доўг. --- \emph{Абавязкова} павінны быць наступствы за ваш
учынак, інакш гэтая школа перавернецца ўверх дном.

Гары паціснуў плячыма.

--- Прыймаецца, --- сказаў ён без выразу. --- Але ў будучыні Северус не будзе 
караць маіх сяброў здымаючы балы з  мяне, і таксама не будзе марнаваць мой
час сваімі спагнаннямі. Калі ў яго будзе ўражанне, што мае паводзіны патрабуюць 
карэктыроўкі, ён можа давесці тое да ведама прафесара МакГонагал. 

--- Гары, --- сказала Мінерва, --- ці будзеце вы прытрымлівацца школьнай 
дысцыпліны, або вы адчуваеце сябе вышэй законаў, як дагэтуль Северус?

Гары глянуў на яе. Нешта цёплае кранула яго позірк, і хутка знікла.

--- Я буду працягваць быць звычайным вучнем для ўсіх настаўнікаў, якія не 
звар'яцелі, або не злыя. Або не пад іх уплывам, --- Гары паглядзеў на Снэйпа,
потым на Дамблдора. --- Пакіньце Мінерву ў спакоі, і я буду звычайным студэнтам 
у яе прысутнасці. Ніякіх асаблівых прывілей і статусаў.

--- Выбітна, --- сказаў шчыра Дамблдор. --- Сказана, як сапрадным героем.

--- І, --- дадала Мінерва, --- містэр Потэр публічна папросіць прабачэння за 
ягоны сённяшні ўчынак. 

Гары перавеў позірк на яе. Гэты раз позірк быў крыху скептычны.

--- Вашыя дзеянні сур'ёзна падарвалі стан школьнай дысцыпліны, містэр Потэр. 
Гэта мусіць быць выпраўлена.

--- Я думаю, прафесар МакГонагал, вы значна пераацэньваеце важнасць таго,
што вы клічаце дысцыплінай, калі яе падрывае жаданне мець жывога настаўніка, або 
не пакутваць на занятках. Патрыманне школьнай іерархіі і бяспрэчнае следванне 
правілам падаецца значна больш мудрым і важным, калі ты на версе гэтай піраміды.
Я магу пачаць цытаваць вам мае размовы са студэнтамі на гэты конт, магу рабіць 
гэта некалькі гадзін безперарынна, але...

Мінерва патрасла галавай.

--- Містэр Потэр, вы недаацэньваеце важнасць дысцыпліны, таму што самі не маеце 
ў ёй патрэбы... --- яна замаўчала. Прагучала не вельмі, і ўсе трая глядзелі на 
яе дзіўна, --- ... у сэнсе, каб вучыцца. Не кожны чалавек можа выконваць 
школьную праграму ў адсутнасці строгага настаўніка. Бо калі астатнія вучні 
вырашаць, што могуць паследваць вашаму прыкладу, гэта можа скончыцца для іх вельмі
дрэнна. 

Вусны Гары скрывілся ва ўсмешку. 

--- Першая і апошняя інстанцыя --- праўда. Праўда ў тым, што мне не варта было 
злавацца, не варта было срываць занятак. Не трэба было гэта рабіць, і паказваць 
дрэнны прыклад. Але праўда ў тым, што Северус паводзіў сабе не так, як варта 
прафесару Хогвартс, і з гэтага дня ён будзе больш уважлівы да параненых пачуццяў 
вучняў чацьвёртага курса і ніжэй. Мы абодва можам падняцца і расказаць праўду. 
На такое я згодны. 

--- Вільготныя мары, Потэр! --- сыкнуў Северус.

--- У рэшце рэшт, --- сказаў Потэр, па-ранейшаму змрочна ўсміхаючыся, ---
калі студэнты ўбачаць, што правілы існуюць таксама і для настаўнікаў, а не 
толькі для бедных бездапаможных студэнтаў, якія не атрымліваюць ад сістэмы 
нічога, акрамя пакутаў... чаму не, пазітыўны эфект для школьнай дысцыпліны 
можа быць \emph{неймаварным.}

Пасля невялікай паузы Дамблдор хіхікнуў.

--- Мінерва думае, што ты правей, чым нават маеш права быць.

Вочы Гары мігам прыгнулі прэч ад твара Дамблдора, кудысьці на падлогу.

--- Вы што, чытаеце яе розум?

--- Звычайнае разуменне часта блытаюць з леджэламанцыяй, --- сказаў Дамблдор. ---
Я абмяркую гэта з Северусам, і табе не прыйдзецца выбачацца, калі ён адмовіцца.
І зараз я афіцыйна абвяшчаю гэтую справу закрытай, прынамсі да абеда, --- ён 
памаўчаў. --- Хаця... Гары, баюся, што Мінерва жадае праясніць нешта яшчэ. І гэта 
--- не вынік майго ўплыву. Мінерва, калі ласка?

Мінерва паднялася з свайго крэсла. У яе крыві было занадта шмат адрэналіну, і 
яе сэрца выскоквала з грудзей.

--- Фоукс, --- клікнуў Дамблдор, --- суправадзі Мінерву, калі ласка.

--- Не трэ... --- пачала яна казаць.

Дамблдор кінуў на яе позірк, і яна замаўчала.

Фенікс пераляцеў праз пакой, быццам узняўшыся язык полымя, і сеў на яе плячо.
Праз мантыю яна адчула цяпло, а потым і ўсім сваім целам.

--- Ідзіце за мной, містэр Потэр, --- сказала яна сваім звычайным цвёрдым голасам,
і яны пакінулі кабінет Майстра.


\later

Яны стаялі на спіральнай лесвіцы, апускаючыся ў цішыні.

Мінерва не ведала, што сказаць. Чалавек, які стаяў побач, быў ён незнаёмы.

Фоукс пачаў свой спеў.

Ён быў пяшчотны, цёплы --- так мог бы гучаць хатні камін, калі ён мог бы пець, 
і гэты спеў прайшоў праз розум Мінервы, распавольваючы, супакойваючы ўсё, чаго 
кранаўся...

--- \emph{Што} гэта такое? --- пачуўся Гарын шэпт недзе побач з ёю.

--- Спеў фенікса, --- сказала Мінерва, не вельмі ўсвядамляючы гэта, бо ўся яе 
ўвага захаплялася ціхай дзіўнай музыкай. --- Ён таксама лечыць.

Гары адвярнуўся ад яе, але яна паспела заўважыць краем вока, як 
чамусьці скрывіўся ягоны твар.

Спуск заняў доўгі час, або можа проста падавалася, што спеў цягнецца вельмі доўга,
і калі яны выйшлі праз нішу, дзе стаяла гаргулья, МакГонагал моцна трымала яго руку ў 
сваёй.

Калі гаргулья вярнулася на месца, Фоукс скокнуў з яе пляча, і моцна замахаў крыламі,
завіснуўшы ў паветры перад Гары.

Гары глядзеў на фенікса, як людзі зачаравана глядзяць у глыбіню няспынна зменлівага
агню.

--- Што мне рабіць, Фоукс? --- прашаптаў Гары. --- Я не мог бы абараніць іх, калі б 
не ўззлаваўся.

Крылы працягвалі бязгучна біць паветра, фенікс вісеў на адным месцы. Потым --- 
выбліск, падобны на тое, калі як полымя ратпам ўздымаецца, і Фоукс знік.

Абодва міргнулі, быццам абуджаючыся ад сну.

Мінерва глянула ўніз.

Вочы ў Гары былі крыху пачырванелыя.

--- Феніксы --- людзі? --- спытаў ён. --- У сэнсе, яны дастаткова разумныя, 
каб лічыцца людзьмі? Я мог бы пагаварыць з Фоксам, калі бы ведаў, як гэта робіцца?

Мінерва  зноў міргнула. Потым моцна зажмурылася, і зноў расплюшчыла вочы.

--- Н-не, --- сказала яна, --- феніксы --- істоты магутнай магічнай сілы. Магія 
дадае іх існаванню вагу, якую простым жывёлам ніколі не сягнуць. Феніксы --- гэта 
агонь, свет, перараджэнне, лекі. Адказ на вашае пытанне --- не.

--- Дзе я магу набыць фенікса?

Мінерва нахілілася, і абняла яго. Яна не планавала гэта секунду назад, а зараз 
падавалася, што іншага выбару ў яе і не было.

Калі яна паднялася, ёй было цяжка гаварыць. Але яна была павінна спытаць. 

--- Што сёння здарылася, Гары?

--- Я таксама не ведаю адказы ні на адзін з важных пытанняў. Не ўлічваючы, што 
мне праўда хацелася бы нейкі час пра гэта наогул не думаць.

Мінерва ўзяла яго за руку, і яны прайшлі рэшту шляху моўчкі.

Шлях быў кароткі, бо, натуральна, кабінет намесніцы быў побач з кабінетам 
Майстра.

Мінерва села за свой стол.

Гары сеў насупраць.

--- Так... --- прашаптала Мінерва. Яна аддала бы амаль што ўгодна, каб не 
рабіць гэта, або не быць тым, хто павінны зрабіць гэта, або каб гэта зрабілася  
ў любы іншы час. 

--- Засталося невырашаным толькі пытанне школьнай дысцыпліны. З якой вы не
выключэнне.

--- А канкрэтна? --- сказаў Гары.

Ён не ведаў. Ён яшчэ не здагадаўся. Яна адчула, як сціскаецца горла. Але справа 
павінна быць зробленай, і яна не адступіць.

--- Містэр Потэр, --- сказала прафесар МакГонагал. --- Я бы хацела паглядзець 
на ваш часаварот, калі ласка.

Увесь спакой фенікса раптоўна пакінуў яго аблічча, і яна адчула, быццам 
яна зараз ўдарыла яго нажом. 

--- \emph{Не!} --- сказаў ён з панікай у голасе. --- Ён мне пратрэбны, я не змагу 
хадзіць на заняткі, я не змагу спаць!

--- Вы зможаце спаць, --- сказала яна. --- З Міністэрства даслалі ахоўны кантэйнер 
для вашага часаварота. Я зачарую яго адчыняцца толькі паміж дзевяццю і дванаццацю
вечара. 

Яго твар скурчыўся.

--- Але... але... я...

--- Містэр Потэр, колькі разоў вы карысталі часаварот з панядзелка? Колькі гадзін?

--- Я... --- сказаў Гары. --- Пачакайце... мне трэба падлічыць... --- ён паглядзеў 
на свой гадзіннік.

Мінерва адчула хвалю смутку. Яна так і падазравала. 

--- Значць, не проста дзве гадзіны на дзень. Думаю, калі я спытаю вашых аднакурснікаў,
я даведаюся, што кожны дзень вы клаліся ў звыклы час, але прачыналіся ўсё раней 
і раней, так?

На абліччы Гары ўсё было напісана і так.

--- Містэр Потэр, --- сказала яна мякка, --- некаторым студнтам нельга даверыць 
часаварот, бо ў іх пачынаецца залежнасць. Яны злоўжываюць часаваротам, карыстаючы 
яго не толькі, каб вырашыць свае праблемы са сном. У такіх выпадках мы вымушаны 
забраць часаварот. Містэр Потэр, вы звыкліся карыстаць часаварот як універсальнае 
вырашэнне ўсіх праблем, часта даволі бязглузда. Вы скарысталі яго, каб дабыць 
напамінальнік. Вы зніклі з шафы пры поўнам класе, калі трэба было, вярнуўшыся 
назад у часу спытаць кагосьці проста адчыніць вам дзверы звонку.

Па Гары бало ясна, што пра такое ён не падумаў.

--- І, што самае важнае, --- працягвала яна, --- вам трэба было проста 
сядзець далей на ўроку прафесара Снэйпа. І глядзець, і слухаць. І сысьці з 
усімі ў канцы. Як бы вы і зрабілі, калі б у вас не было часаварота. Ёсць 
студэнты, якім нельга даверыць часаварот. І вы адзін з іх. Мне шкада.

--- Але ён мне \emph{патрэбны!} --- амаль крыкныў Гары. --- Што калі нейкі 
слізэрын будзе мне пагражаць, і трэба будзе збегчы? Ён --- гарантыя маёй
\emph{бяспекі...}

--- Усе астатнія вучні ў замке рызыкуць такім жа чынам, і дазвольце вас заўпэўніць,
ніхто яшчэ не памёр. Цэлых пяцьдзясят гадоў ніхто не паміраў. Містэр Потэр, 
вы аддадзіце мне часаварот, і зробіце гэта зараз жа.

Агонія скурчыла твар Гары, але ён зняў з шыі ланцужок, і працягнуў ёй.

Мінерва дастала з шуфляды кантэйнер, які даслалі з Міністэрства, паклала 
часаварот унутр, замкнула, і дакранулася да кантэйнера палачкай, каб 
завершыць працэдуру зачаравання. 

--- \emph{Гэта неспавядліва!} --- прасычэў Гары. --- Сёння я выратаваў палову 
Хогвартс ад Снэйпа! І за гэта --- пакаранне? Я бачыў, як вы глядзелі, вы жа 
самі робіце гэта з агідай!

Мінерва маўчала, яна была занятая зачараваннем.

Калі яна нарэшце скончыла і падняла вочы, яна ведала, што яе выраз быў строгі. 
Магчыма, яны былі неправы. Але, магчыма, толькі так і магчыма было працягваць. 
Перад ёю знаходзілася непаслухмянае дзіця, а не канец свету.

--- \emph{Несправядліва,} містэр Потэр? --- гаркнула яна. --- Я была вымушана 
даслаць \emph{два} тлумачальных рапарта ў Міністэрства Магіі на конт публічнага карыстання 
часаваротам \emph{за два дня запар!} Вам варта быць \emph{бясконца} ўдзячным, што 
вам дазволілі працягваць карыстацца часаваротам, нават з абмежаваннямі!  
Сам Майстра наведаўся туды праз лятучае сеціва, каб асабіста прасіць за вас,
і каб не вашая ганаровае званне Хлопца-які-выжыў, то нават і гэтага было б 
недастаткова!   

Гары глядзеў на яе шырока расплюшчанымі вачыма.

Яна ведала, што ён бачыць гнеўнае аблічча прафесара МакГонагал.

Потым яго вочы напоўніліся слязьмі.

--- Прабачце... --- прашаптаў ён зламаным голасам. --- Мне вельмі шкада... што 
расчараваў вас...

--- Мне таксама шкада, містэр Потэр, --- сказала яна, таксама строга, і працягнула 
яму запакаваны часаварот. --- Вы свабодны.

Гары выйшаў з кабінета, хлюпаючы носам. Мінерва чула ягоныя крокі ўздоўж 
калідора, а потым дзверы зачыніліся. 

--- Мне таксама шкада, Гары, --- прашаптала яна ў цішыню. --- Мне таксама...


\later

Прайшло пятнаццаць хвілін з пачатку абеда.

Ніхто не замаўляў з Гары. Некаторыя рэйвенкло кідалі на яго раз-пораз позіркі, 
хто --- абураны, хто --- спачуллівы, некаторыя --- самыя малодшыя --- з захапленнем,
але ніхто да яго не звяртаўся. Нават Герміёна не спрабавала наблізіцца.

Потым асцярожна падышлі Фрэд і Джордж. Таксама моўчкі. Было і без слоў ясна, 
чаго яны хацелі, і што ён мог адмовіцца. Ён сказаў ім толькі, каб чакалі яго, 
калі пададуць дэсерт, не раней. Яны кіўнулі і хуценька зніклі.

Магчыма, абсалютна безвыразнае аблічча Гары рабіла такі эфект.

Людзі вакол яго думалі, напэўна, што ён спрабуе трымаць пад кантролем свой 
гнеў, а можа і трывогу. Яны ведалі, што ў яго былі прычыны, бо  
Флітвік мог асабіста прыйсці за ім толькі каб адвесці ў кабінет 
Майстра.

Гары вельмі моцна намагаўся не ўсміхацца, бо калі ён усміхнецца, ён пачне 
смяяцца, а калі ён пачне смяяцца, ён не спыніцца, пакуль за ім не прыйдуць 
клапатлівыя людзі ў белых халатах.

Гэта было занадта. Гэта ўсё было проста овердафіга занадта. Гары амаль перайшоў 
на Цёмны Бок. Цёмны Гары сёння рабіў рэчы, якія рэтраспектыўна бачыліся яму 
вар'яцкімі; цёмны Гары атрымаў неверагодную перамогу, якая магла быць як 
сапраўднай, так і проста выбрыкам шалёнага Майстра; цёмны Гары прыдумаў, як 
абараніць сяброў, і давёў справу да канца. Гары больш не мог кантраляваць
свой цёмны бок. Ён прагнуў зноў пачуць спеў фенікса. Ён прагнуў скарыстацца 
часаваротам, каб атрымаць ціхую гадзіну, каб супакоіцца і адпачыць, але ён 
згубіў такую магчымасць, і гэтая страта была як дзірка ў ягоным сусвеце, але 
ён не мог пра тое думаць, бо ён дакладна пачне смяяцца.    

Дваццаць хвілін. Усе студэнты, хто планаваў трапіць на абед, ужо прыбылі.

Звон відэльцам па шкянцы пранізаў Вялікую Залу.

--- Калі я магу запытаць крыху вашай ўвагі, калі ласка... --- сказаў Дамблдор, 
--- у Гары Потэра ёсць чым з намі падзяліцца.

Гары глыбока ўдыхнуў, і падняўся. Пакуль ён ішоў да галоўнага стала, кожная 
пара вачэй неадрыўна за ім сачыла.

Дасягнуўшы стала, ён абярнуўся, і агледзеў чатыры стала факультэтаў.

Было ўсё складаней не усмізацца, але Гары здолеў пратрымаць безвыраз падчас 
сваёй кароткай прамовы, якую ён вывучыў на памяць.

--- Праўда --- гэта святасць, --- сказаў ён голасам, такім жа безвыразным,
як і яго твар. --- Адна з самых каштоўных маіх рэчаў --- гэта значок з дэвізам:
"Кажы праўду, нават калі твой голас трасецца". Вось я кажу вам праўду. Памятайце
пра гэта. Я кажу гэта не таму што мяне прымусілі, а таму, што гэта праўда. 
Тое, што я зрабіў на ўроку прафесара Снэйпа, было дурным, бязглуздым, здзіцянелым,
і невыбачальным парушэннем правілаў Хогвартс. Я сарваў заняткі, і пазбавіў 
маіх аднакласнікаў каштоўнага часу навучання. Усё гэта таму, што я не 
стрымаў свой гнеў. Спадзяюся, што ніводны з вас не ніколі не паследуе майму 
прыкладу. Я дакладна маю намер ніколі такое не паўтараць.

Шмат хто глядзеў на Гары сумна, быццам яны былі сведкамі таго, як сістэма
ўрэшце паламала Гары. У маладзейшых раёнах грыфіндорскага стала такі выгляд быў 
амаль на ўсіх тварах.

Пакуль Гары не падняў руку.

Ён не падняў яе высока. Ён дакладна не падняў яе ў бок Северуса. Ён проста 
падняў яе на ўзровень пляча, і ціха цокнуў пальцамі --- гэта было больш бачна,
чым чутна. Было магчыма, што большасць з настаўнікаў наогул гэта не бачылі.

Гэты жэст супраціву раптам вызваў хвалю ўсмешак з боку Грыфіндора, насмешлівае 
чмыханне з боку Слізэрына, і ўсхваляваныя позіркі ад астатніх. 

Гары неймаверным чынам працягваў дэманстраваць адсутнасць якога-небудзь
выразу.

--- Дзякуй, --- сказаў ён. --- У мяне ўсё.

--- Дзякуй, містэр Потэр, --- сказаў Майстра. --- А зараз прафесар Снэйп 
таксама хацеў бы сказаць пару слоў.

Северус падняўся з-за стала. 

--- Да маёй увагі было даведзена, --- сказаў ён, --- што мае дзеянні маглі 
быць часткай таго, што справакавала безумоўна непрыяльныя паводзіны містэра
Потэра, і падчас наступнай дыскусіі я ўсведаміў, што я мог забыць, якімі 
лёгка ўразлівымі могуць быць пачуцці маладых і няспелых...

Зал выдаў гук, быцам вялікая колькасць народу адначасова прыдушылася.

Северус працягваў, быццам не чуў.

--- Кабінет зеллеварэння --- месца небяспечнае, і я па-ранейшаму прытрымліваюся
меркавання, што на занятках патрэбна строгая дысцыпліна, але я пастараюся быць 
болей уважлівым да... эмацыянальнай крохкасці студэнтаў чацьвёртых курсаў і 
ніжэй. Знятыя балы з Рэйвенкло застаюцца ў сіле, але спагнанне з містэра 
Потэра здымаецца. Дзякуй за ўвагу.

З боку Грыфіндора пачуўся адзіны хлапок, і палачка Северуса хутчэй за 
маланку прыгнула яму ў руку і \emph{Quietus!} сцішыў парушыцеля.

--- Я па-ранейшаму буду патрабаваць дысцыпліну і павагу на \emph{усіх} маіх
занятках, --- холадна сказаў Северус, --- і любы, хто будзе валяць дурня, 
пашкадуе аб гэтым.

Ён сеў.

--- Дзякуй і вам! --- сказаў весела Дамблдор. --- Калі ласка, працягвайце абед!

І Гары безвыразна пайшоў да свайго месца.

Зала выбухнула размовай. Два слова былі даволі добра чуваць у пачатку. Першае 
было "што", за якім ішло нешта кшталту "гэта было?", або "за чорт?". 
Іншае слова было "\emph{Scourgify!}", якім людзі прыбіралі разлітыя напоі і ежу 
з падлогі і суседзяў.

Некаторыя вучнкі адкрыта плакалі. Таксама і прафесар Спраут.

За сталом Грыфіндора, дзе стаяў торт з пяцьюдзесяццю незапаленымі 
свечкамі, Фрэд шапнуў: 

--- Ну, думаю, сёння ён паклаў нас на лапаткі, Джордж.

І з гэтага самага дня, што бы ні спрабавала пярэчыць Герміёна, у Хогвартс 
агульна было прызнана як факт, што Гары Потэр можа зрабіць абсалютна што 
заўгодна, проста цокнуўшы пальцамі.
