\chapter{Творчы падыход}

\begin{chapterOpeningQuote}
Я не псіхапат, я проста вельмі крэатыўны.
\end{chapterOpeningQuote}

\lettrine{Я}{к} толькі яны ўвайшлі ў пакой для заняткаў па Абароне, Гары адчуў, што 
гэты прадмет будзе моцна \emph{адрознівацца}.

Па-першае, гэта быў найвялікшы пакой, які ён да гэтага часу бачуў у Хогвартс. Пакой
быў падобны на ўніверсітэтскія лекцыйныя залы, з уздымаючыміся, як ступені, сталамі 
для вучняў, якія абкружалі гіганцкі бела-мармуровы памост. Зала знаходзілася на 
пятым паверсе Хогвартс, і як падазраваў Гары, іншга тлумачэння, як нешта такіх 
памераў магло ўмясціцца ў замку, ён не атрымае. Ён пачынаў верыць у тое, што 
Хогвартс проста \emph{не меў} геаметрыі, эўклідавай або іншай; у ім былі месцы,
і пераходы паміж імі, але паняцце кірунку проста адсутнічала.

Але, у адрозненне ад лекцыйнай залы, там не было вузкіх сталоў і складваючыхся
сядзенняў. Замест іх былі парты і крэслы, 
выстраеныя ў форме амфітэатра. Звычайные драўляныя крэслы, і звычайныя драўляныя
парты Хогвартс, за тым выключэннем, што пасярэдзіне кожнай быў усталяваны плоскі
прамавугольны загадкавы прадмет.

У сярэдзіне гіганцкага памосту стаяў самотны настаўніцкі стол. За якім 
паўсядзеў-паўляжаў Квірэл, адкінуўшы галаву, пускаючы тонкую нітку сліны на 
сваю мантыю.

\emph{І што ж гэта мне нагадвае?..}

Гары прыбыў так рана, што больш нікога з вучняў яшчэ не было. (У нашай мове 
не хапае слоў, каб адэкватна апісаць вандроўкі ў часе; у асабістасці,
няма слоў, каб апісаць, наколькі зручныя такія вандроўкі.)
Квірэл быў не вельмі... функцыянальным?.. у дадзены момант, і ў любым 
выпадку, Гары не падабалася думка да яго набліжацца. 

Гары выбраў сабе парту, забраўся да яе, сеў, і дастаў падручнік па Абароне. 
Ён прачытаў каля сямі восьмых, і як раз планаваў скончыць кнігу перад гэтым урокам,
але ён адставаў ад графіка, і сёння ён ўжо двойчы карыстаў часаварот.

Праз некаторы час зала пачала напаўняцца. Гары ігнараваў асяродзе.

--- Потэр? Што ты тут робіш?

\emph{Гэтага} голаса Гары не чакаў тут пачуць. Ён падняў галаву.

--- Драко? Што \emph{ты} тут робіш абожамой у цябе ёсць \emph{міньёны!}

Адзін з хлопцаў, якія стаялі ззаду Драко, падавалася, меў зашмат мускулатуры для 
адзінаццацілеткі, а другі стаяў у падазрона збалансаванай позе,
быццам быў гатовы кінуцца ў бойку.

Бялявы хлопец самазадаволена ўсміхнуўся і махнуў рукой у іх напрамку. 

--- Потэр, дазволь прадставіць табе містэра Крэба, --- яго рука рушылася ад Мускулаў 
да Балансу, --- і містэра Гойла. Вінсэнт, Грэгары, гэта --- Гары Потэр.

Містэр Гойл нахіліў галаву на бок і выдаў Гары позірк, які, верагодна, быў 
павінны нешта вызначаць, але выглядала, быццам ён сонна прыжмурыўся.

--- Прыемна пазнаёміцца, --- сказаў містэр Крэб, намагаючыся
прымусць свой голас гучаць як мага ніжэй.

Разугублены выраз прабяжаў па твары Драко, але быў хутка заменены 
ягонай выдатнай усмешкай.

--- У цябе ёсць \emph{міньёны!} --- паўтарыў Гары. --- Дзе \emph{я} магу набыць міньёнаў?

Самазадаволеная ўсмешка Драко стала шырэй.

--- Баюся, Потэр, першы шаг для гэтага --- размяркавацца на Слізэрын...

--- Што? Гэта несправядліва!

--- ...і вашыя рады павінны заключыць пагадненне яшчэ да вашага нараджэння.

Гары паглядзеў на містэра Крэба і містэра Гойла. Яны вельмі моцна стараліся 
дэманстраваць пагрозу. У сэнсе, яны навісалі над ім, згорбіўшыся на плячах, і 
глядзелі зпадылба.

--- Эмм... пачакай, --- сказаў Гары, --- гэта было запланавана шмат гадоў таму?

--- Дакладна, Потэр. Непашанціла табе.

Містэр Гойл дастаў зубачыстку і пачаў калупацца ў зубах, усё яшчэ навісаючы.

--- І, па мяркаванню Люцыуса, ты павінен быў расці, не ведаючы сваіх целаахоўнікаў,
і вы павінны былі ўпершыню сустрэцца ў першы школьны дзень?

Гэта сцёрла ўсмешку з твара Драко.

--- Так, Потэр, мы ўсе ведаем, што ты дужа разумны, уся школа ўжо ведае, 
таму можаш перастаць выпендрывацца...

--- Дык і ім таксама \emph{ўсё} жыццё казалі, кім яны калісьці будуць,
і яны \emph{гадамі} ўяўлялі сабе, якімі павінны быць міньёны...

Драко скурчыўся.

--- ...і што самае дрэннае, яны ведаюць адзін аднаго і яны \emph{трэніраваліся...}

--- Бос казаў та'э заткнуцца, --- прагрымеў містэр Крэб. Містэр Гойл, працягваючы 
грызці зубачыстку зубамі, пархусцеў касцяшкамі пальцаў.

--- \emph{Каму я казаў не займацца гэтай лухтой перад Гары Потэрам?}

Абодва крыху засаромеліся, і містэр Гойл хутка схаваў зубачыстку назад у кішэнь.

Але калі Драко павярнуўся назад да Гары, яны зноў пачалі навісаць.

--- Я прашу прабачэння, --- сказаў сціснута Драко, --- за абразу,
якую гэтыя \emph{дэбілы} маглі нанесці.

Гары з разуменнем паглядзеў на містэра Крэба і містэра Гойла. 

--- Я бы сказаў, што ты занадта жорстка да іх ставішся, Драко. Я бы сказаў, яны 
паводзяць сябе адпаведна з маім бачаннем маіх уласных міньёнаў. У сэнсе, 
калі б яны ў мяне былі.

У Драко сківіца адвалілася.

--- Хэй, Грэг, ты не думаеш, што ён спрабуе пера'аніць нас ад боса, ці што?

--- Містэр Потэр не тастолькі дурны.

--- О, не, нічога такога, --- сказаў Гары. --- Гэта проста каб вы мелі на ўвазе, 
калі раптам ваш бягучы работадаўца перастане вам падабацца. І дарэчы, мець 
дадаковы козыр для перамоваў аб умовах працы --- 
ніколі не залішне, праўда?

--- І што \emph{ён} ро'іць у Рэйвенкло?

--- Ні 'алейшага паняцця, містэр Крэб.

--- Абодва \emph{заткнуліся}, --- сказаў Драко праз сціснутыя зубы. --- Гэта загад, ---
з напругай ён зноў вярнуў увагу да Гары. --- У любым выпадку,
што ты робіш на ўроке Слізэрына па Абароне?

Гары нахмурыўся.

--- Чакай, --- ён залез у махляскін. --- Расклад, --- ён паглядзеў на пергамент.
--- Абарона, 14:30, а зараз... --- ён паглядзеў на гадзіннік, які паказваў 11:23. --- 14:23,
калі я не памыляюся. Я памыляюся? --- Калі так, Гары ведаў спосаб трапіць на любы 
занятак, дзе ён павінен быў быць. Божа, як ён быў улюбёны ў свой часаварот, і калісцьці,
калі ён стане дастаткова дарослы, яны абавяскова ажэняцца.

--- Не, падобна на праўду, ---- сказаў Драко, выглядаючы разгублена. Ён прайшоўся
вачыма па аўдыторыі, якая была поўная мантый з зялёнай аблямоўкай, і...

--- \emph{Грыфіндурні!} --- цыркнуў Драко. --- А  \emph{яны} што тут згубілі?

--- Гхм, --- сказаў Гары, --- прафесар Квірэл казаў... я не памятаю дакладныя словы,
але было нешта, што ён будзе парушаць некаторыя вучэбныя практыкі Хогвартс. 
Можа ён проста сабраў разам цэлы курс.

--- Мм... --- сказаў Драко. --- Ты першы рэйвенкло тут.

--- Угу, я паспеў раней за ўсіх.

--- Чаму ты тады заняў месца ў апошнім радзе?

Гары міргнуў.

--- Не ведаю, мне падалося нармалёвым месцам?

Драко фыркнуў.

--- Далей ад настаўніка забрацца немагчыма, --- ён нахіліўся крыху бліжэй. ---
У любым выпадку, гэта праўда пра тое, што ты сказаў Дэрыку і яго бандзе?

--- Што за Дэрык?

--- Ты кінуў у яго пірог?

--- Насамрэч, два. І што я, па чуткам, яму сказаў?

--- Што ён займаўся не мудрагелістымі камбінацыямі, а лухтой, і ганьбіў імя Салазара Слізэрына, --- 
Драко глядзеў на Гары вельмі ўважліва.

--- Ну... кшталту таго, --- сказаў Гары. --- Я думаю, яно гучала як "гэта ў вас такі
хітры спосаб атрымаць новыя факультэтныя балы, або  
ганьба імі Салазара Слізэрына, бо выглядае як другое", або нешта такое. Не памятаю 
дакладна.

--- Ведаеш, ты бянтэжыш шмат каго, --- сказаў Драко.

--- Гэта як? --- Гары шчыра сам збянтэжыўся.

--- Уорынгтан кажа, што доўгі час пад Размеркавальным Капелюшом --- верны знак 
цёмнага мага. Нашыя шушукаюцца, думаюць, ці не варта ўжо пачынаць падлашчвацца да цябе.
І тут ты выходзіш і абараняеш \emph{хаплпафаў,} дзеля Мерліна! \emph{І потым} 
ты кажаш Дэрыку пра ганьбу імі Салазара Слізэрына! \emph{Што} яны ўсе тады падумаюць? 

--- Што Капялюш размеркаваў мяне на "Слізэрын! Проста пажартаваў! Рэйвенкло!" 
і што вяду сябе адпаведна з гэтым.

Містэр Гойл хіхікнуў. Містэр Крэб хутка паспеў закрыць свой рот далонню.

--- Лепей нам заняць месцы, --- сказаў Драко. Ён павагаўся, потым стаў раўней, і 
сказаў больш фармальна:

--- Але я хацеў бы працягнуць нашую апошнюю размову, і я прымаю твае ўмовы.

Гары кіўнуў.

--- Ці магу я запытаць цябе пачакаць за суботы? У мяне тут як раз ідзе невялікае
спаборніцтва.

--- Спаборніцтва?

--- Праверка, ці здолею я прачытаць усі мае падручнікі з той жа хуткасцю, што і 
Герміёна Грэнджэр.

--- Грэнджэр, --- эхам паўтарыў Драко. Ягоныя вочы звузіліся. --- Бруднакроўка,
якая зафанабэрылася, што яна --- сам Мерлін? Тады ўсе слізэрыны  \emph{вельмі} жадаюць 
табе ўдачы, Потэр, і я пачакаю да суботы, --- Драко паважліва нахіліў галаву,
і сышоў, суправаджаемы сваімі міньёнамі.

\emph{Ох, як жа \emph{весела} будзе балансаваць гэта ўсё, магу сказаць загадзя.}

Аўдыторыя ўжо хутка запаўнялася мантыямі ўсіх магчымых аблямоўкаў: зялёны,
чырвоны, жоўты, блакітны. Драко і яго сябры спрабавалі заняць тры месца
побач у першым радзе, якія, каненшне, былі ўжо занятыя. Містэр Крэб і містэр 
Гойл навісалі вельмі пагрозліва, але гэта не вельмі шмат уражвала.

Гары працягнуў чытаць падручнік па Абароне.


\later

У 14:35, калі зайшлі апошнія вучні і большасць месцаў была занятая, прафесар Квірэл 
раптам таргануўся, сеў роўна, і ягоны твар з'явіўся на плоскіх прамавугольных 
панэлях, якія стаялі на партах.

Гары ад нечаканасці сам здрыгануўся, не столькі ад раптоўнага з'яўлення твара прафесара,
колькі з таго, як гэта нагадвала маглаўскае тэлебачанне. Было нешта такое
сумнае і настальгічнае ў гэтым, яно так нагадвала дамашні спакой, але не было ім...

--- Добры дзень, мае юныя вучні, --- сказаў прафесар Квірэл. Ягоны голас сыходзіў з
экрану і быў накіраваны прама на Гары. --- Вітаю вас на вашым першым уроку 
Баявой Магіі, як бы гэта назвалі заснавальнікі Хогвартс; або, як гэты прадмет 
чамусьці клічуць у канцы дваццатага стагоддзя, "Абарона ад Цёмных Майстэрстваў".

Некаторыя студэнты пачалі адчыняць свае пергаменты, рыхтуючыся канспектаваць.

--- Адставіць, --- сказаў прафесар Квірэл. --- Не хвалюйцеся запісваць, як нешта
некалі называлася. Ніводнае такое бессэнсэўнае пытанне не заработае вам адзнакі 
на маіх уроках. Абяцаю вам.

Шмат хто са студэнтаў сеў роўна, выглядаючы шакавана.

На твары прафесара Квірэла з'явілася тонкая ўсмешка. 

--- Тыя з вас, якія змарнавалі час, чытаючы вашыя падручнікі...

Нехта зрабіў гук, быццам захлынуўся. Гары падумаў, ці не Герміёна то была.

--- ...напэўна склалі мяркаванне, што прадмет навучыць вас абароне ад Кашмарных 
Матылькоў, якія часам выклікаюць дрэнныя сны, або ад Кіслотных Смаўжоў, якія могуць
растварыць дзвюхцалевую дошку, калі ім выдзеліць на гэта цэлы дзень.

Прафесар Квірэл устаў. Экран на парце Гары сачыў за кожным ягоным рухам. 
Прафесар прайшоў на край памоста і прагрымеў:

--- Вянгерскі Рагахвост вышэй, чым тузін людзей! Ён плюе агнём так хутка і так дасканала,
што можа расплавіць Зніч у палёце! Адзін Забівальны Заклён справіцца з ім за секунду!

Нехта громка вохнуў.

--- Горны Троль болей пагрозлівы за Вянгерскага Рагахвоста! Ён можа прагрысці 
сталь! Ягоная скура настолькі моцная, што яе не прабіваюць аглушальныя або праразныя
закляцці! Ягоны нюх такі востры, што ён па паху можа сказаць, ці ахвяра побач са 
зграяй, або яна ў адзіноце і безабаронная! І самае дрэннае --- троль унікальны 
сярод усіх магічных істот ў сваёй здольнасці падтрымліваць трансфігурацыю самога сябе ў 
свае уласнае цела. Калі вам пашанціць адарваць яго руку, яму спатрэбіцца 
пару секунд на тое, каб адрасціць новую! Агонь і кіслата зрабяць шнары, якія 
крыху прыпыняць яго рэгенератыўныя здольнасці --- на гадзіну-дзве!
Тролі дастаткова разумныя, каб карыстаць свае дубінкі як прылады! 
Горны Троль займае трэцяе месца ў рэйтынгу самых дасканалых машын для забойства 
ў прыродзе! Адзін Забівальны Заклён справіцца з ім за секунду!

Шакаваная цішыня апанавала аўдыторыю.

Усмешка прафесара Квірэла была даволі змрочнай.

--- Нікчэмная пародыя на падручнікі па Абароне для трэцяга курса прапануе вам 
выманіць троля на сонца, ад чаго ён адразу павінен акамянець. Гэта, мае юныя вучні,
прыклад бессэнсоўных ведаў, якія вы ніколі не знойдзеце на маіх занятках.
Вы не сустрэнеце троля там, дзе яго можна выманіць на сонца! Гэтая ідэя --- 
вынік паказухі аўтарамі таго, як добра яны ведаюць дробязі, губляючы ўсякае
практычнае прымяненне. Проста таму, што існуе нейкі таямнічы спосаб перамагчы троля,
не значыць, што вы павінны спрабаваць яго. Забівальны Заклён нельга спыніць,
блакаваць, і ён працуе кожны раз, калі яго прымяніць на нешта, што мае мозг. 
Калі вы ў сваім дарослым жыцці па якім-небудзь прычынам няздольныя карыстаць 
забівальны заклён, вы проста апарыруеце як мага далей. Таксама, калі вы сустракаеце
ўладальніка другога месца ў рэйтынгу --- Дэментара, --- самы проста спосаб --- 
апарацыя!

--- Канешне, --- працянуў ён пасля невялікай паузы, голас яго быў цішэй і жарстчэй,
--- калі вы не пад уздзеяннем анты-апрарыцыйнага праклёну. Існуе толькі адзін монстр,
які можа параўнацца з вамі, калі вы дасягнеце піка сваёй сілы. 
Гэты монстр такі небяспечны, што нішто ў  свеце не можа параўнацца з ім.
Гэта Цёмны маг. Гэта адзіная істота, якая заўсёды будзе вам пагражаць.

Вусны яго сціснуліся ў тонкую лінію.

--- Я стаўлюся скептычна на конт праграмы падрыхтоўцы, якая дазволіць вам 
здаць экзамен, афіцыйна зацведжаны Міністэрствам. Вашыя адзнакі ніяк не паўплываюць
на вашае жыццё пасля Хогвартс, таму любы, хто жадае, калі ласка, марнуйце ваш час 
самастойна над сваімі пародыямі на падручнікі. Назва маёй дысцыпліны ---
не "Абарона ад Дробных Паразітаў". Вы тут, каб навучыцца абараняцца ад 
Цёмных Майстэрстваў. Што вызначае --- дазвольце нам быць шчырымі прынасмі 
з сабою, --- абарону ад цёмных магаў. Ад людзей з чароўнымі палачкамі, якія 
жадаюць нанесці вам шкоду, і нанясуць, калі толькі вы не пашкодзіце іх першымі!
Не бывае абароны без нападу! Няма абароны без бойкі! Тлустыя, разбэшчаныя, абкружаныя 
аўрорамі чыноўнікі палічылі гэтую рэальнасць занадта жорсткай для 
адзінаццацілетніх дзяцей. У пекла гэтых дурняў! Вы будзеце вывучаць прадмет,
які вядзецца ў Хогвартс на працягу васьмі стагоддзяў! Вітаю вас на першым курсе 
Баявой Магіі!

Гары пачаў пляскаць у далоні, натолькі гэта было натхняльна.

З баку Грыфіндора пачуліся некалькі адказнах апладысментаў, некалькі --- ад 
Слізэрына, але большасць студэнтаў падавалася паралізаванай.

Прафесар Квірэл рубануў далонню паветра, і апладысменты імгненна сціхлі.

--- Дзякуй, --- сказаў ён. --- Бліжэй да справы. Я аб'яднаў усе групы першага 
курса ў адзіны паток, што дае вам падвоены аб'ём заняткаў у парўнанні са стандартным 
раскладам... 

Па аўдыторыі пранёсся жахлівы ўздых.

--- ...але павялічаная нагрузка кампенсуецца тым, што я не буду задаваць 
хатнія заданні.

Жахлівы ўздых раптам абарваўся.

--- Так, вы не памыліліся. Я буду навучаць вас, як біцца, а не як напісаць 
эссэ пра бойкі даўжынёй у дванаццаць цалей да панядзелка.

Гары адчайна жадаў сядзець побач з Герміёнай, каб убачыць яе выраз у гэты момант,
але, у цэлым, ён быў упэўнены, што ўяўляў яго даволі дакладна.

І Гары, падавалася, закахаўся. У ніх будзе тройчы звяз: ён, часаварот, і прафесар
Квірэл.

--- Для ўсіх жадаючых, я арганізаваў некаторыя пазакласныя актыўнасці, якія вы, 
спадзяюся, знойдзеце цікавымі і карыснымі. Вы жадаеце паказаць усяму свету 
\emph{свае} здольнасці, замест таго, каб назіраць, як чатырнаццаць чалавек 
гуляюць у Квіддзіч? Для арміі патрабуецца больш за сем чалавек.

\emph{Ну ні храна сабе.}

--- На занятках і па-за класам вы будзеце атрымліваць балы Квірэла. Што за балы такія,
спытаеце вы. Сістэма факультэтных балаў не адпавядае маім мэтам, бо яны надта рэдкія.
Мне трэба трымаць маіх студэнтаў у курсе іх паспяховасці пастаянна. У тыя рэдкія 
выпадкі, калі я буду даваць вам пісьмовы тэст, ён сам будзе правяраць вашыя адказы,
і калі шмат іх будзе неправільнымі, ваш тэст пакажа вам імёны студэнтаў,
якія адказалі правільна, і тыя студэнты змогуць заработаць балы 
Квірэла, дапамагаючы астатнім.

Ух ты... Ну чаму астатнія настаўнікі не былі такімі?

--- У чым сэнс яшчэ адной сістэмы балаў, вы пытаеце? Для пачатку, дзесяць 
балаў Квірэла раўняюцца аднаму балу факультэта. Але за іх можна атрымаць 
і іншыя прывілеі. Напрыклад, захочаце здаць экзамен у час, які падабаецца вам?
Або прапусціць пэўны ўрок? Вы дазнаецеся, што магу быць вельмі гібкім у 
дачыненні да студэнтаў, якія накопяць дастаткова балаў. Таксама, балы Квірэла 
ўплываюць на тое, хто камандуе арміямі. А на Раство --- перад пачаткам вакацый ---
я выканаю жаданне кагосьці з вас. Любыя школьныя рэчы, якія ў маёй уладзе,
пад маім уплывам, і больш за ўсё, у межах маёй вынаходлівасці.
Так, я выпускнік Слізэрына, і я прапаную выканаць закручаную інтрыгу ад 
вашага імя, калі гэта можна здзейсніць вашае жаданне. Гэтая прывілея дастанецца 
таму, хто набярэ найбольшую колькасць балаў Квірэла з усіх сямі курсаў.

Гэта будзе Гары.

--- А зараз, калі ласка, пакіньце вашыя кнігі і сумкі на партах, --- экраны будуць 
назіраць за імі, не хвалюйцеся, --- і спускайцеся сюды на памост. Надышоў нам 
час правесці гульню пад назвай "Самы Пагрозлівы Студэнт у Класе".


\later

Гары махнуў палачкай і сказаў "\emph{Ma-ha-su!}"

Паследваў высокі чысты "блямк" --- гук, які выдаў лятаючы блакітны шар, які 
прафесар Квірэл прызначыў Гары ў якасці мішэні. Менавіта такі гук вызначаў 
ідэальна трапны стрэл, якія Гары здолеў зрабіць дзевяць з апошніх 
дзесяці спробаў.

Дзесьці прафесар Квірэл знайшоў заклён, які было неймаверна проста вымавіць,
які патрабаваў неймаверна простага руху палачкай, і ляцеў туды, куды ты глядзеў 
у мамент вымаўлення. Прафесар Квірэл пагардліва абвесціў, што сапраўдная баявая магія
была значна складаней. Гэты заклён быў абсалютна бескарысны ў рэальнай бойке. 
Ён быў слаба ўладкаваным выбліскам магіі, і асноўным яго кампанентам было
прыцэльванне, і калі ён трапіў ў чалавека, то эфектам быў кароткі боль,
прыкладна падобны на ўдар кулаком у нос. Што адзінай мэтай гэтага 
тэсту было вызначэнне вучняў, якія лавілі новыя рэчы на ляту, бо прафесар
Квірэл быў упэўнены, што ніхто з іх дагэтуль не сустракаў нічога падобнага.

Гары ўсё гэта было абыякава.

--- \emph{Ma-ha-su!}

\emph{Чырвоны згустак энергіі} вырваўся з ягонай палачкі і трапіў у шар, які зноў
выдаў чыстую ноту. \emph{Як жа ён быў выдатны!}

Гары адчуваў сябе сапраўдным магам упершыню за весь свой час у Хогвартс.
Яму хацелася, каб мішэнь ухілялася, як тыя, якімі Бен Кенобі трэніраваў 
Люка, але прафесар Квірэл выстраіў студэнтаў у адну доўгую шарэнгу, проста каб 
яны не трапілі адно ў аднаго. 

Тады Гары апусціў руку, рэзка кінуўся направа, узняў руку, махнуў, і крыкнуў:

--- \emph{Ma-ha-su!}

Мішэнь выдала ніжэйшы гук: Гары не трапіў дасканала ў цэнтр.

Гары схаваў палачку ў кішэнь, прыгнуў у левы бок, у прыжку выхапіў палачку і 
стрэліў яшчэ раз.

Высокі чысты блямк быў адным з самых задавальняючых гукаў, якія ён чуў у жыцці.
Гары хацелася пераможна крычаць ва ўсю моц сваіх лёгкіх. 
\emph{\scream{Я магу рабіць магію! Беражыцеся, законы фізікі,
я іду вас парушаць!}}

--- \emph{Ma-ha-su!} --- Гары казаў гучна, але не вельмі заўважна, бо шум ад 
падобных крыкаў вакол стаяў неймаверны.

--- Дастаткова, --- сказаў узмоцнены голас прафесара Квірэла. (Ён не быў моцна гучным.
Гэта быў нармальны голас, проста ён ішоў з-за твайго левага пляча, няважна, дзе 
ты знаходзіўся адносна прафесара.) 

--- Я бачу, што ўсе вы справіліся прынамсі аднойчы.

Шары-мішэні сталі чырвонымі, і пачалі павольна падымацца да столі.

Прафесар Квірэл стаяў у сярэдзіне памосту, крыху абапіраючыся на свой стол адной рукой.

--- Я казаў вам, што мы згуляем у гульню пад назвай "Самы Пагрозлівы Студэнт у Класе".
Сярод вас ёсць чалавек, які здолеў засвоіць заклён Шчучынскага Штурханца хучтэй за астатніх...

О, бла-бла-бла.

--- ...і потым дапамагчы сямёрым іншым студэнтам. За гэта яна заработала першыя сем балаў 
Квірэла, прызначаныя сёлета. Падыдзіце, Герміёна Грэнджэр. Надышоў час наступнага этапу 
гульні. 

Герміёна выйшла з-за парты, выгляд у яе быў адначасова трыумфальны і насцярожаны.
Студэнты Рэйвенкло праваджалі яе ганарлівымі поглядамі, слізэрыны --- з пагардай,
Гары --- са шчырым раздражненнем. Ён зрабіў усё правільна. Магчыма ён нават быў 
у ліку першых, хто засвоіў незнаёмы заклён. Да таго ж, ён прачытаў усю "Тэорыю Магіі"
Адальберта Уафлінга. І ўсё роўна  \emph{Герміёна яго абставіла}.

Дзесьці глыбока ў розуме Гары прачнуўся страх, што Герміёна проста была 
разумней за яго.

Але зараз Гары ўскладаў усе сваі надзеі на вядомыя факты: а) Герміёна прачытала 
болей падручнікаў, чым Гары, і б) Адальберт Уафлінг лічыў першакурснікаў нікчэмнымі
ідыётамі.

Унізе Герміёна падышла да настаўніцкага стала.

--- Герміёна Грэнджэр засвоіла незнаёмы заклён за дзве хвіліны, амаль на цэлую 
хвіліну раней, чым наступны ў спісе, --- прафесар Квірэл павольна агледзеў 
усю аўдыторыю. --- Ці могуць здольнасці міс Грэнджэр зрабіць яе самай 
пагрозлівай студэнткай у класе? Ну? Што думаеце?

Падавалася, што ніхто нічога не думаў. Нават Гары не ведаў, што сказаць.

--- Ёсць толькі адзін спосаб праверыць, --- сказаў прафесар Квірэл. Ён павярнуўся 
да Герміёны, і абвёў рукой усю залу.

--- Выбярыце любога студэнта, і скастуйце на яго Шчучынскага Штурханца.

Герміёна замерла.

--- Ну, што такое, --- сказаў прафесар Квірэл. --- Вы выдатна каставалі яго 
болей за пяцьдзесят разоў. Ён не наносіць перманентнай шкоды, і не такой ужо балючы.
Проста як рэзкі штуршок, і доўжыцца ўсяго пару секунд, --- ягоны голас стаў жарстчэй.
--- Гэта прамы загад вашага настаўніка, міс Грэнджэр. Выбярыце студэнта і кастуйце
заклён.

На твары Герміёны панаваў ціхі жах, і палачка ў яе руцэ траслася. Гары 
спачувальна сціскваў сваю ўласную палачку. Нават калі ён разумеў, што 
намагаецца зрабіць прафесар Квірэл. Нават калі ён разумеў, што прафесар 
Квірэл хацеў гэтым сказаць.

--- Калі вы не паднімеце сваю палачку, і не стрэліце, міс Грэнджэр, вы 
страціце бал Квірэла.

Гары ўперыўся ў Герміёну, моцна жадаючы прыцягнуць яе ўвагу. Яго рука амаль незаўважна
паказвала на ягоныя грудзі.   \emph{Абяры мяне, я не баюся...}

Палачка Герміёны тузаналася; потым яе твар разняволіўся, і яна апусціла 
руку.

--- Не, --- сказала яна. Голас быў спакойны, і хаця ён быў нягучны, усе яго пачулі 
ў цішыні.

--- Тады я здымаю бал, --- сказаў прафесар. --- Гэты быў тэст, і вы яго не прайшлі.

Гэта быў удар для яе самой. Гары заўважыў. Але яна трымала галаву высока.

Спачувальны тон прафесара Квірэла запоўніў аўдыторыю.

--- Ведаць шмат --- недастаткова, міс Грэнджэр. Калі вы не можаце пашкодзіць 
камусьці на ўзроўні "зачапіў мезенцам нагі за мэблю", вы не здольныя абараніць 
сябе, і вы не пройдзеце экзамен па Абароне. Можаце далучыцца да сваіх 
аднакласнікаў.

Герміёна пайшла да свайго месца. Яе выраз быў спакойны, і Гары чамусьці 
захацелася пачаць пляскаць ёй. Нават, калі прафесар і быў \emph{правы}.

--- Так, --- сказаў прафесар Квірэл. --- Зразумела, што Герміёна Грэнджэр ---
не самы пагрозлівы студэнт у класе. Як думаеце, хто ў гэтай аўдыторыі
сапраўды самы пагрозлівы? Акрамя мяне, канешне.

Нават не паспеўшы падумаць, Гары паглядзеў у бок, дзе сядзелі слізэрыны.

--- Драко з Вялікага і Старажытнейшага Роду Малфоеў, --- сказаў прафесар Квірэл.
--- Мне падаецца, што шмат хто зараз глядзіць у вашым напрамку. Калі ласка, 
выйдзіце наперад.

Драко выйшаў з цяжка схаваным гонарам. Ён стаў каля прафесара на памосце,
і паглядзеў на яго з усмешкай.

--- Містэр Малфой, --- сказаў прафесар Квірэл, --- агонь.

Гары паспрабаваў бы спыныць яго, калі меў хаця б секунду, але адным 
плаўным і хуткім рукам Драко падняў палачку, павярнуўся у бок рэйвенкло, 
і сказаў "\emph{Mahasu!}", быццам гэта быў адзіны склад, Герміёна ахнула, 
і ўсё было скончана.

--- Трапны стрэл, --- сказаў прафесар Квірэл. --- Даю вам два бала Квірэла.
Але скажыце, чаму міс Грэнджэр?

Была пауза.

Нарэшце Драко сказаў:

--- Яна больш за ўсіх стракацела.

Вусны прафесара Квірэла расцягнуліся ва ўсмешцы.

--- І гэта сапраўдная прычына, чаму Драко Малфой пагрозлівы. Абяры ён каго-небудзь
іншага, той мог бы абурыццца, і містэр Малфой, імаверна, атрымаў бы новага ворага.
І нават калі некаторыя і так ловяць кожнае слова, якое кажа містэр Малфой,
рабіць новага ворага яму не прыгодна. І гэта значыць, што містэр Малфой ведае,
каго біць, а каго --- не быць, як знаходзіць сяброў і як не рабіць ворагаў.
Яшчэ два бала Квірэла вам, містэр Малфой. І таму што вы паказалі выдатныя здольнасці 
слізэрына, ваш факультэт таксама варты бала. Можаце вярнуцца да сваіх сяброў.

Драко крыху пакланіўся, і пайшоў да сваіх. Нехта стуль пачаў пляскаць у далоні, але 
адзін жэст Квірэла зноў вярнуў цішыню.

--- Падобна на тое, што гульня наша скончана, --- сказаў прафесар Квірэл. --- Але 
ёсць у гэтай аўдыторыі студэнт, які пагрозлівей за нашчадка рода Малфоеў.

І \emph{зараз} чамусьці падавалася, што вельмі шмат хто глядзеў на...

--- Гары Потэр. Падыдзіце.

Гэта не прадвяшчала нічога добрага.

Гары няўпэўнена пайшоў да прафесара Квірэла, які ўсё яшчэ абапіраўся аб свой стол.

Усхваляванне ад таго, што яго паставілі ў цэнтр увагі, сканцэнтравала Гарын розум,
пакуль ён падыходзіў да памоста, і ён ліхаманкава перабіраў варыянты, што можа 
прафесар запытаць яго зрабіць, каб паказаць яго пагрозлівасць.
Скаставаць небяспечны заклён? Адолець Цёмнага Лорда? Прадэманстраваць, што 
на Гары не спрацуе забівальны заклён? Вядома, прафесар Квірэл надта разумны для 
\emph{такога}...

Гары спыніўся на краі памоста, і прафесар Квірэл не запытаў падысці бліжэй.

--- Іронія ў тым, --- сказаў прафесар, --- што вы паглядзелі на правільнага 
чалавека па няправільнай прычыне. Вы думаеце, --- яго вусны зноў скрывіліся ва 
усмешку, --- што Гары Потэр перамог Цёмнага Лорда, і таму ён павінен быць вельмі
пагрозлівым. Ха. Яму быў адзін год. Што бы ні забіла Цёмнага Лорда, яно мае 
няшмат дачынення да байцоўскіх здольнасцяў містэра Потэра.
Але калі да мяне дашлі чуткі, як адзін рэйвенкло супрацьстаяў пяцёрым старэйшым 
слізэрынам, я апытаў некалькі сведкаў, і прыйшоў да высновы, што Гары Потэр можа
быць самым пагрозлівымі сярод вас.

Гары прасяк выбрас адрэналіну, прымусіўшы яго стаць раўней. Ён не ведаў, да якой 
высновы прыйшоў прафесар Квірэл, але яна не магла быць добрай.

--- Эм... прафесар Квірэл... --- пачаў казаць Гары.

Гэта забавіла прафесара Квірэла:

--- Вы думаеце, што мой адказ няправільны, ці не так, містэр Потэр? Вы 
хутка навучыцеся не недаацэньваць мяне. Містэр Потэр, усе рэчы маюць 
свае звычайнае карыстанне. Назавіце мне дзесяць рэчаў у гэтым класе, якія мы 
можам скарыстаць незвычайным чынам у бойке!

На імгненне Гары страціў мову, ад ашаламлення, што прафесар яго \emph{зразумеў.}

И потым да яго пачалі даходзіць ідэі.

--- Тут ёсць парты, якія будуць смяротнымі, калі кінуць іх з дастатковай вышыні. 
Ёсць крэсла з металічнымі ножкамі, якімі можна праткнуць цела, калі іх 
дастаткова моцна штурхнуць. Паветра можа быць смяротным праз сваю адсутнасць, 
бо людзі паміраюць у вакууме, або яго можна напоўніць смяротным газам.

Гары спыніўся, каб удыхнуць паветра, і ў гэтую паузу прафесар сказаў:

--- Гэта тры. Трэба дзесяць. Усе прысутнічаючыя думаюць, што вы назвалі ўвесь 
кантэнт аўдыторыі.

--- \emph{Ха!} Падлогу можна выдаліць, калі зрабіць яму з вострымі дошкамі і скідваць 
туды ворагаў; столь можна абрынуць ім на галаву; з матэрыялу сцен можна странсфігураваць
зброю, нажы, напрыклад...

--- Гэта шэсць. Няўжо ўсё?

--- Я нават не пачынаў! Паглядзіце на людзей! Прымусць грыфіндораў нападаць на людзей,
канешне, --- \emph{звычайны} спосаб карыстання, але...

--- Я гэта не залічу.

--- ... але ў іх крыві можна кагосьці ўтапіць; рэйвенкло вядомыя за іхнія мазгі,
але іх іншыя ограны можна прадаць на чорным рынку, і на тыя грошы наняць асасінаў;
слізэрыны не толькі самі добрыя асасіны, іх яшчэ можна і кінуць з дастатковай хуткасцю
ў ворага; кожны хафлпаф --- не толькі цэнны работнік, а тры-чатыры кілаграма 
касцей, якія можна з іх дастаць, завастрыць, і закалоць ворагаў. 

Людзі глядзелі на Гары жахліва. Нават слізэрыны выглядалі шакавана.

--- Гэта дзесяць, і гэта ўлічваючы маю шчодрасць за варыянт з рэйвенкло. Зараз,
у якасці бонуса,  даю вам па балу Квірэла за кожную рэч у гэтым пакоі, якую вы яшчэ не назвалі,
--- прафесар Квірэл адорыў Гары таварыскай ўсмешкай. --- Астатнія думаюць, што 
нарэшцэ вы папаліся, бо назвалі ўсё, акрамя мішэней, і ў вас няма ідэй, як іх скарыстаць.

--- Ха! Я назваў людзей, але не маю мантыю, якой можна задушыць ворага, калі абярнуць 
іх галаву некалькі разоў; або мантыю Герміёны Грэнджэр, якую можна парваць на стужкі,
зрабіць вяроўку і павесіць кагосьці; або мантыю Драко Малфоя, якую можна падпаліць...

--- Тры бала, --- сказаў прафесар Квірэл. --- Больш ніякага аддзення.

--- Маю палачку можна ўторкнуць ў мозг ворага праз вачніцу! --- нехта ў зале 
выдаў сціснуты крык.

--- Чатыры, больш без палачак.

--- Мой гадзіннік можна ўвапхнуць камусьці ў трахею, каб ён задахнуўся...

--- Пяць балаў, і дастаткова.

--- Хм, --- сказаў Гары, --- дзесяць балаў Квірэла за адзін бал факультэта, так? 
Калі вы бы мяне не спынілі, я бы выйграў Келіх Факультэтаў, бо я яшчэ не пачаў 
перабіраць рэчы ў маіх кішэнях, --- або ў махляскіне, і калі ён не мог сказаць 
пра часаварот і плашч нябачнасці, ён мог прыдумаць  \emph{нешта} на конт 
шароў-мішэняў...

--- \emph{Дастаткова,} містэр Потэр. Ну што, усе думаюць, што зразумелі, што робіць містэра Потэра 
самым пагрозлівым студэнтам у класе?

Па зале прайшоў згодны шэпт.

--- Скажыце ўголас, калі ласка. Тэры Бут, што робіць вашага суседа па спальне 
пагрозлівым?

--- Э... эм... ён крэатыўны?

--- \emph{Няправільна!} --- прагрымеў прафесар Квірэл, і ягоны кулак рэзка апусціўся
на стол з узмоцненым гукам, ад якога ўсе падпрыгнулі. --- Усе яго ідэі былі бескарыснымі!

Гары здрыгануўся ад нечаканасці.

--- Выдаліць падлогу каб зрабіць яму? Бязглузда! У бойке ў вас няма столькі часу
на падрыхтоўку, і калі б было --- ёсць сотня спосабаў скарыстаць падлогу лепей!
Трансфігураваць зброю са сцен? Містэр Потэр не здольны рабіць такую трансфігурацыю!
Увогуле, у містэра Потэра была толькі адна ідэя, якую ён мог прымяніць безадкладна 
і без доўгай падрыхтоўкі, або дапамогі з боку ворага, або магіі, якой ён не валодае.
Гэтая ідэя --- уторкнуць яго палачку ворагу ў вока. Ад чаго яна, хучэй за ўсё,
проста зломіцца. Карацей, містэр Потэр, баюся, вашыя прапановы былі даволі 
так сабе.

--- Што? --- сказаў Гары абурана. --- Вы \emph{запыталі} незвычайныя спосабы, а не 
практычныя! Гэта быў творчы падыход! Як \emph{вы} скарысталі бы рэчы ў гэтай зале,
каб забіць кагосьці?

Выраз прафесара быў няўхвальны, але вакол вачэй былі ўсмешкавыя маршчынкі.

--- Містэр Потэр, я не казаў \emph{забіць}. Ёсць час і месца, калі ворага варта браць 
у палон, жывым, і зала ў замку Хогвартс, звычайна, добра для гэтага падыходзіць.
Але адказ на вашае пытанне --- ударыць ворага па шыі стулам.

З боку слізэрынаў пачуўся смех, але яны смяяліся разам з Гары, а не над ім.

Астатнія глядзелі з жахам.

--- Але містэр Потэр прадэмантстраваў, чаму ён --- самы пагрозлівы студэнт у класе. 
Я спытаў пра незвычайныя спосабы карыстання рэчаў для бойкі. Містэр Потэр мог 
прапанаваць карыстаць парты, каб блакаваць праклёны, або закідаць крэсламі калідор,
каб ворагу было цяжка прабрацца, або прыматаць падручнік да рукі ў якасці 
імправізаванага шчыта. Замест гэтага, кожны спосаб, які ён прапанаваў,
быў прызначаны для атакі, не абароны, і кожны быў фатальны, або патэнцыйна фатальны.

Эм, што? Пачакайце, гэта ўсё не так... Гары раптам адчуў заварот галавы, пакуль ён
спрабаваў прыпомніць, што канкрэтна ён казаў... дакладна павінна было быць нешта,
каб запярэчыць...

--- І менавіта таму, --- працягваў прафесар, -- ідэі містэра Потэра былі дзіўнымі 
і бескаштоўнымі --- бо ён проста быў вымушаны пачаць разглядаць непрактычныя спосабы, каб
адпавядаць свайму высокаму стандарту \emph{забойства ворага}. Для яго ідэі, якія 
не гарантуюць гэтага выніку, не вартыя разглядання. Гэта адлюстроўвае якасць,
якую мы можам назваць \emph{смяротны намер}. У мяне ёсць такая якасць. У містэра Потэра ---
таксама. Дзякуючы ёй, ён перажыў сустрэчу са старэйшымі слізэрынамі. 
Драко Малфой не мае такой якасці --- пакуль што. Містэра Малфоя, канешне,
не спужаць думкай аб звычайным забойстве, але нават яго шакавала --- так, містэр 
Малфой, я назіраў за вамі, --- калі містэр Потэр прапанаваў скарыстаць целы 
аднакласнікаў як сыравіну для зброі. У мазгах усіх вас ёсць блок, які 
прымушае вас здрыгануцца ад такой думкі, але містэр Потэр думае \emph{толькі}
аб абсалютным знішчэнні супраціўніка, карыстаючы любыя наяўныя спосабы, і у яго 
няма такой блакіроўкі. 
Нават калі яго малады геній яшчэ недысціплінаваны, і выдае непрактычныя ідэі,
яго \emph{смяротны намер} і ёсць тое, што робіць Гары Потэра Самым Пагрозлівым 
Студэнтам у Класе. Фінальны бал яму... давайце зробім яго балам Рэйвенкло ---
за гэты незаменны рэквізіт сапраўднага баявога чараўніка.

Сківіца ў Гары адвалілася, пакуль ён у бязмоўным ступары сутаргава шукаў, што 
на гэта адказаць. \emph{Гэта абсалютна на мяне не падобна!}

Але ў позірках вучняў вакол яго было разуменне і згода. Гарын розум усё яшчэ 
мітусіўся ў спробах знайсці, што сказаць, чым запярэчыць, і не мог 
знайсці нешта, што магло супрацьстаяць аўтарытэтнаму голасу прафесара Квірэла.
Адзінае, што прыдумаў Гары ў сваю абарону было "Я не псіхапат, я проста вельмі
крэатыўны", але нават у яго галаве гэта гучала занадта пафасна. Ён павінен сказаць 
нешта нечаканае, нешта, што прымусіць іх усіх спыніцца і падумаць яшчэ раз...

--- А зараз, --- сказаў прафесар Квірэл, --- містэр Потэр, агонь!

Нічога, канешне, не адбылося.

--- Ну так, ну так, --- уздыхнуў прафесар Квірэл. --- Мяркую, усе мы павінны 
пачынаць з чагосьці. Містэр Потэр, абярыце любога студэнта ў якасці мэты для 
Штурханца. Вы \emph{зробіце} гэта да канца ўрока. Інакш, я пачну здымаць 
балы факультэта, і не перастану, пакуль вы не згадзіцеся. 

Гары асцярожна падняў сваю палачку. Бо прафесар Квірэл мог вырашыць адразу пачаць здымаць 
балы.

Павольна ён павярнуўся да слізэрынаў. 

І вочы Гары сустрэлі вочы Драко.

Драко Малфой выглядаў ніколькі не спужаным. Ён не паказваў ніякіх сігналаў, 
як Гары паказваў Герміёне, але гэта было чакана. Іншыя слізэрыны не ацэнілі бы.

--- Чаму вагаецеся? --- сказаў прафесар Квірэл. --- Відавочна, ёсць толькі адзін 
варыянт. 

--- Так, --- сказаў Гары. --- Толькі адзін \emph{відавочны} варыянт.

Ён махнуў палачкай і сказаў "\emph{Ma-ha-su!}"

У зале наступіла мёртвая цішыня. 

Гары памахаў левай рукой, спрабуючы справіцца з вострым болем.

Цішыня працягнулася.

Нарэшце прафесар Квірэл уздыхнуў. 

--- Так, даволі крэатыўна, але ў гэтым быў ваш урок, а вы ад яго ўхілілся.
Адзін бал з Рэйвенкло за паказуху хітрасці за кошт рэальнай мэты. 
Урок скончаны.

І да таго, як хтосьці паспеў нешта сказаць, Гары пракрычаў:

--- Проста пажартаваў! РЭЙВЕНКЛО!

На колькі імгненняў павісла ціша, пакуль людзі ўспрынялі гэта, потым пачаўся ціхі гоман,
які паступова ўзняўся да размаўляльнага грохату.

Гары павярнуўся да прафесара Квірэла, бо ім трэба было пагаварыць...

Квірэл абмяк і зноў паваліўся ў свае крэсла.

Не. Так не пойдзе. Ім  \emph{праўда} трэба было пагаварыць. Негледзячы на 
стан зомбі, прафесар, мабыць, прачнецца, калі Гары добра штурхане яго пару разоў.
Гары зрабіў крок наперад...

\emph{ДРЭННА}

\emph{СТОЙ}

\emph{ДРЭННАЯ ІДЭЯ}

Гары качнуўся і спыніўся, адчуваючы кружэнне галавы.

І потым да яго падышла група рэйвенкло і пачалася дыскусія.

