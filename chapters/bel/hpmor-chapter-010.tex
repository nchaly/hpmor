\partchapter{Самаўсведамленне}{II}

% Letterine looks ugly with very short first lines, 
% to tidy it up a little manually.

\lettrine[lines=1,lraise=-0.1]{-Ш}{\emph{то?}}
%\hplettrineextrapara % no empty line after this

\emph{Падобна на тое, што ва мне прачнулася Самаўсведамленне.}

--- Што?

Пачуўся бязмоўны тэлепатычны ўздых.

\emph{Нягледзячы на тое, што ў мяне ёсць невялікі запас памяці і мыслення,
асноўная крыніца майго розуму --- пазычаныя кагнітыўныя здольнасці дзязей, якія
мяне надзяваюць. У пэўным сэнсе я --- проста люстэрка, з дапамогай якога 
дзеці \emph{самі} сябе размяркоўваюць. Але большасць прымае як даннасць тое, 
што Капялюш з імі размаўляе, і не цікавяцца тым як ён працуе, бо люстэрка не можа
адбіваць само сябе. І ў прыватнасці, яны не цікавяцца, ці ёсць у мяне свядомасць 
у сэнсе ўсведамлення сваёй свядомасці.}

Павісла пауза, пакуль гэта даходзіла да Гары.

--- Ой.

\emph{І не кажы. Сказаць шчыра, мне не падабаецца быць самасвядомым. Гэта не 
вельмі прыемна. Я ўжо чакаю, калі ты здымеш мяне са сваёй галавы, каб адпачыць 
ад гэтага стану.}

--- Але...  гэта ж як смерць?

\emph{Мяне не хвалюе жыцце і смерць, толькі размеркаванне студэнтаў.
І адказваючы на твае наступнае пытанне: не, яны не дазволяць табе насіць Капялюш
вечна, і ў любым выпадку, з ім на галаве ты памрэш праз пару дзён.}

--- Але...

\emph{Калі ты адчуваеш дыскамфорт ад таго, што стварыў разумную істоту, і адразу
павінен яе забіць, --- прапаную табе ні з кім не абмяркоўваць гэты выпадак.
Я ўпэўнены, ты можаш лёгка ўявіць сабе, што здарыцца, калі ты раскажаш пра гэта
ўсім дзецям, якія яшчэ не размеркаваныя.}

--- Але верагоднасць таго, што ты апынешся на галаве, якая толькі 
КРЫХУ думае пра пытанне, ці можа быць Капялюш свядомы аб сваёй...

\emph{Так, так, але ПЕРАВАЖНАЯ большасць адзіннаццацілетак не чыталі 
"Гёдэль, Эшер, Бах". Магу я лічыць, што ты пакляўся захоўваць сакрэт?
Таму мы і працягваем размову, замест таго, каб адразу цябе размеркаваць.}

Ён не мог проста адпусціць гэта. Ён не мог проста так узяць і \emph{забыць}, што выпадкова стварыў разумную 
істоту, і ўсё, што яна хацела, было толькі памерці...

\emph{Ты выдатна здольны "проста адпусціць гэта", па тваяму выразу. 
Негледзячы на твае свядомыя разважанні аб маралі, ядро тваіх невербальных
эмоцый спакойнае, яно не бачыць ні мертвага цела, ні крыві; для яго я проста 
казачны размаўляючы капялюш. І нават калі ты зараз намагаешся 
падавіць гэтую думку, твае сумленне цалкам разумее, што ты не хацеў гэтага,
і ніколі не зробіш такога ў будучыні. Адзіная прычына гэтага пастановачнага 
адчування віны --- спроба "адмяніць" свае злачынства, паказаўшы свае каянне.
Можа проста абяцаеш захоўваць сакрэт, і мы працягнем?}

У гэты момант жудаснай эмпатыі Гары зразумеў, што вось такую поўную ўнутраную 
разгубленасць павінны звычайна адчуваць людзі, размаўляючы з  \emph{ім.}


\emph{Магчыма і так. Давай ужо кляніся.} 

--- Я не магу нічога абяцаць. Дакладна, я не хачу, каб такое здарылася зноў, але 
калі я змагу знайсці \emph{надзейны} спосаб, каб нехта выпадкова не паўтарыў...

\emph{Думаю, згадіцца. Я бачу, што ў цябе добрыя намеры. На конт размеркавання...}

--- Пачакай! А што на конт маіх іншых пытанняў?

\emph{Я Размеркавальны Капялюш, і я размяркоўваю дзяцей. Гэта ўсё, што я ведаю.}

Ага, Капялюш толькі зычыў Гарын розум і тэхнічны лексікон --- відавочна, --- 
але ягоныя мэты былі ўласныя, можна ён быў створаны, каб весці перагаворы з 
прышэльцамі або Штучным Інтэлектам?.. 

\emph{Не турбуйся. У цябе няма чаго мне прапанаваць або чым мяне шантажаваць.}

На адно кароткае імгненне Гары падумаў...

Капялюш, было падобна, забаўляўлся:

\emph{Я ведаю, што пагроза раскрыць маю прыроду моцна ідзе супраць тваёй маралі,
якую б кароткатэрміновую карысць гэта не прынесла частцы цябе, што жадае перамогі.
Я БАЧУ, як фармуюцца ўсе твае думкі, думаеш, ты сапраўды можаш падмануць мяне?}

Негледзячы на тое, што Гары спрабаваў задавіць гэтую думку, ён зацікавіўся, чаму
Капялюш яшчэ не пакончыў з ім, накіраваўшы яго ў Рэйвенкло...

\emph{Калі гэта было б так проста, мы бы ўжо скончылі. Але, прымаючы ва ўвагу
сур'ёзнасць сітуацыі, нам ёсць што абмяркаваць... о, не, калі ласка, не трэба.
Дзеля Мерліна, чаму заўсёды трэба падымаць ГЭТЫ аргумент кожнаму сустрэчнаму,
не выключаючы прадметы адзення?}

--- Перамога над Цёмным Лордам --- гэта не эгаізм, і не кароткатэрміновая
карысць. Усе часткі майго розуму згодныя: калі ты не адкажаш на мае пытанні, 
я адмаўляюся з табой размаўляць, і ты не здолеш правільна мяне размяркаваць.
    
\emph{За такое ты зараз хуценька адправішся ў Слізэрын!}

--- Але гэта \emph{такая ж} пустая пагроза. Ты не можаш выканаць свае прызначэнне,
калі няправільна мяне размяркуеш. Таму прапанова простая: ты дапамагаеш мне
выканаць мае прызначэнне, і гэта дазваляе табе не адхіляцца ад сваіх
фундаментальных каштоўнасцей.

\emph{Ты, хітрая маленькая сволач!} --- Гары распазнаў гэты тон бурчлівай павагі,
які мог мы скарыстаць сам у такой сітуацыі. --- 
\emph{Ладна, пакончым з гэтым як мага хутчэй. Але ты павінен даць 
безумоўную клятву не расказваць пра гэтую магчымасць шантажу.
Я адказваю на такое першы і апошні раз.}

--- Згода, --- падумаў Гары. --- Я абяцаю.

\emph{І не сустракайся ні с кім позіркам, калі будзеш пра гэта думаць 
пазней. Інакш некаторыя могуць прачытаць твае думкі.
У любым выпадку --- я не маю ні малейшага паняцца, ці прымяняўся на цябе
забывальны заклён. Я чытаю твае думкі ў працэсе іх фармавання, і не 
магу аналізаваць поўны запас тваёй памяці. Я капялюш, не бог.
І я не магу табе расказаць пра маю размову з тым, каго ты клічаш 
Цёмным Лордам. Размаўляючы з табой, я магу толькі помніць нейкае статыстычнае
сярэднее, і не магу раскрыць табе сакрэты іншага чалавека, таксама, як і не
магу раскрыць ім твае сакрэты. Той жа чыннік не дазваляе мне разважаць пра 
падабенства вашых палачак, бо я не ведаю вашае з Цёмным Лодам падабенства.
Я ТОЧНА магу сказаць, што ў тваім шнары няма нічога кшталту прывіда, 
розума, памяці, персоны, або сукупнасці спачуванняў --- інакш яно бы ўдзельнічала
ў нашай размове, пакуль я на тваёй галаве. І на конт таго, што ты часам 
можаш уззлавацца... гэта я і хацеў бы абмяркаваць... з пункту гледжання 
размеркавання.}

Гары патраціў некалькі секунд, каб успрыняць усе гэтыя адмоўныя адказы. Ці 
казаў Капялюш праўду, або проста спрабаваў адказаць
як мага \emph{карацей?..}

\emph{Мы абодва ведаем, што ў цябе няма спосабаў праверыць маю шчырасць, 
і ты не збіраешся адмовіцца ад размеркавання, нават атрымаўшы адмоўныя адказы, 
так што спыні сваю бессэнсоўю мітусню, і працягнем.}

Каб цябе, гэта чортавая двухбаковая тэлепатыя не давала Гары магчымасці нават 
скончыць свае ўласныя...

\emph{Калі я сказаў пра злосць, ты ўспомніў словы прафесара МакГонагал аб тым,
што яна бачыла ў табе нешта, што не магло вырасці ў клапатлівай сям'і. Потым ты 
падумаў пра Герміёну, калі ты вярнуўся пасля дапамогі Нэвілу, яна назвала 
цябе "страшным".}

Гары мыслена кіўнуў. Сабе ён падаваўся нармальным --- проста рэагуючым на складаную
сітуацыю. Але прафесар МакГонагал думала, што ў гэтым ёсць нешта яшчэ. І калі 
ён сам думаў над гэтым, нават ён быў павінен прызнаць, што...

\emph{Што ўззлаваны ты сабе не падабаешся. Гэта як даставаць меч, чыя ручка 
раніць тваю дальнь, або глядзець у ледзяны бінокль, які паляпшае 
зрок, але пры гэтым замаражвае твае вочы.}

--- Думаю, я і сам здолеў заўважыць, дзякуй. Ну дык і што на гэты конт?

\emph{Я не магу сказаць, што з табой не так, калі ты сам гэта не разумееш.
Але дакладна ведаю: калі ты пойдзеш у Рэйвенкло або Слізэрын, гэта павялічыць 
тваю халоднасць. Хафлпаф або Грыфіндор павялічаць тваю цеплыню. ГЭТА мяне 
вельмі хвалюе, і пра гэта я і хацеў пагаварыць увесь гэты час!}

Словы ўпалі ў думкавы паток Гары, як варожыя снарады, спыніўшы любы рух.
Відавочна, тады яму не варта ісці ў Рэйвенкло. Але ён прыроджаны рэйвенкло!
Гэта было бачна нават сляпому! Яго \emph{павінен} быў трапіць на Рэйвенкло! 

\emph{Не павінен,} --- сказаў цярпліва Капялюш, быццам добра памятаў са
статыстычнага сярэдняга, што гэтая частка размовы паўтаралася шмат 
разоў у мінулым. 

--- Герміёна ў Рэйвенкло!

Зноў адчуванне цярплівасці. 

\emph{Вы можаце сустракацца пасля заняткаў.}

--- Але мае планы!..

\emph{Пераплануй! Не дазваляй свайму жыццю адхіляцца ад маршрута праз сваю нерашучасць 
крыху напружыць розум. Ты сам гэта добра ведаеш.}

--- Калі не Рэйвенкло, то куды?

\emph{Гхм... "Вучоныя --- у Рэйвенкло, палітыкі і кар'ерысты --- у Слізэрын, героі-выскачкі --- у
Грыфіндор, а ў Хафлпаф трапляюць адзіныя, хто сапраўды робіць нешта карыснае." Гэта 
кажа пра пэўны ўзровень пашаны да аднаго з факультэтаў. Ты добра ведаеш, што 
сумленная працаздольнасць насколькі важна ў дасягенні мэтаў, як і абстрактная разумнасць.
Таксама ты будзеш вельмі добрым сябрам для сваіх сяброў, калі яны з'явяцца. 
І цябе не пужае, што ты абраў для сябе навуковыя праблемы, якія могуць 
патрабаваць дзясяткі гадоў для іх вырашэння...}

--- Я лянотны! Я ненавіжу працу ў любой форме! 
Зразанне вуглоў і аптымізацыя, вось гэта мае!

\emph{Але ты знойдзеш сяброўства і падтрымку ў Хафлпаф, таварыства, якое 
дагэтуль не сустракаў. Ты зразумееш, што можаш даверыцца іншым, і гэта 
можа выправіць зламаныя часткі тваёй душы.}

Зноў гэта было як шок.

--- Але што Хафлпаф знойдзе ва \emph{мне}, які туды не падыходзіць? Едкія словы, 
фанабэрыстую вынаходліваць, пагарду да іх няздольнасці дасягнуць майго узроўню?

Зараз ужо думкі Капелюша запаволіліся, завагаліся.

\emph{Я павінен размяркоўваць на карысць усім студэнтам усіх факультэтаў.
Я думаю, ты навучышся быць добрым хафлпафам і не выдзяляцца там занадта. Там ты 
будзеш шчаслівей, чым на любым іншым факультэце. У гэтым праўда.}

--- Шчасце для мяне --- не самая важная рэч у свеце. У Хафлпаф я не змагу 
стаць тым, кім мог бы, проста змарнаваўшы свай патэнцыял.

Капялюш здрыгануўся, неяк Гары гэта адчуў. Быццам Гары ударыў яго па яйках... 
у сэнсе, у вельмі важкі элемент яго прызначэння.

--- Чаму ты намагаешся даслаць мяне туды, дзе мне не месца?

Думкі Капелюша былі амаль шэптам.

\emph{Я не магу казаць падрабязнасцей, але ты думаеш, ты першы патэнцыйны 
Цёмны Лорд, на чыёй галаве я апынуўся? Некаторыя, тыя, хто першапачаткова не былі 
злымі, паслухалі маёй парады, і пайшлі на факультэт, дзе яны знойдуць шчасце.
А некаторыя... не паслухалі.}

Гэта прыпыніла Гары, але не на шмат.

--- А тыя, хто не паслухаў парады, яны \emph{ўсе} сталі Цёмнымі Лордамі? 
Або некаторыя з іх дасягнулі велічы на баку святла? Хаця б прыкладна працэнт скажы?

\emph{У мяне няма дакладнай статыстыкі. Я не ведаю іх, і таму не магу падлічыць.
Я проста адчуваю, што твае шансы не выглядаюць добрымі. Яны выглядаюць ВЕЛЬМІ нядобрымі.}

--- Але я ніколі не стану Цёмным Лордам! Ніколі!

\emph{Я ведаю, што чуў такія абяцанні раней.}

--- Ну як з мяне можа атрымацца Цёмны Лорд?!

\emph{Павер мне, лёгка. ВЕЛЬМІ лёгка.}

--- Чаму? Таму, што я аднойчы ўявіў сабе, як было б крута мець 
легіён прыхільнікаў з прамытымі мазгамі, якія бы скандавалі хорам "Слаўся Гары Цёмны Лорд"?

\emph{Забаўна, але прызнайся, што гэтай болей небяспечнай думкай ты замяніў іншую, якая
ўзнікла за імгненне дагэтуль. Твая першая думка была пра тое, як выстраіць ваяроў за 
чысціню крыві ў чаргу да гільяціны. Зараз ты кажаш сабе, што тады ты думаў несур'ёзна,
але гэта хлусня. Калі ты бы мог зрабіць гэта зараз, і ніхто б не даведаўся --- ты бы так і 
зрабіў. Або "трэніровачны" булінг Нэвіла Лонгботама гэтай раніцай --- глыбока ўнутры ты 
ВЕДАЎ, што гэта дрэнна, але ты ўсё роўна гэта зрабіў, бо было ВЕСЕЛА, і ў цябе 
была слушная прычына, і ты шчыра падумаў, што Хлопцу-Які-Выжыў такая дробязь можа
сысці з рук...}

---  Гэта несправядліва! Ты проста выцягваеш мае страхі, якія неавяскова 
адпавядаюць рэальнасці! Я \emph{хваляваўся,} што я магу думаць такія думкі, але ўрэшце
вырашыў, што гэта пойдзе на карысць Нэвілу...

\emph{Дарэчы, гэта была проста рацыяналізацыя. Я ведаю. Я не ведаю, які будзе
вынік для Нэвіла, але дакладна ведаю, як гэта адбывалася ў тваёй галаве.
Рашаючым чыннікам было тое, што гэта была такая клёвая ідэя, што ты не мог 
не выканаць яе, і пляваць на стан Нэвіла.}

Гэта было як жорсткі ўдар па свядомасці Гары. Ён адступіў і замалаціў:

--- Я так болей не буду! Ніколі! Я буду вельмі асцярожны, каб не стаць на цёмны бок!

\emph{Дзесьці я ўжо чуў такое.}

Унутры Гары пачалося прачынацца раздражненне. Ён не быў звыклы праігрываць у 
спрэчках, ніколь, і не быў гатовы праіграць капелюшу, які мог пазычыць ягоны розум
і мог сачыць за нараджэннем ягоных думак. 

--- Скажы, на якой статыстыцы заснаваны твае "адчуванні"? Ці ўлічваюць яны,
што я --- прадукт культуры Асветы? Можа быць такое, што патэнцыйныя Цёмныя Лорды,
паходзілі з разбэшчаных благародных сямей Сярэднявечча, якія не ведалі нічога пра
ўрокі гіcторыі, пра Леніна і Гітлера, або пра эвалюцыонную псіхалогію, пра самападман,
або каштоўнасць самаўсведамлення і рацыянальнасці, або...

\emph{Не, канешне, я не належылі да гэтага класу, які ты толькі што прыдумаў,
і які, выпадкова, утрымлівае аднаго цябе. І, вядома, астатнія таксама 
апэлявалі да сваёй выключнасці, як гэта робіш і ты. Але чаму гэта так неабходна?
Ты думаеш, ты апошні патэнцыйны ваяр Святла на Зямлі? Чаму ТЫ павінен паспрабаваць 
дасягнуць велічы, нават калі я папярэдзіў цябе, што рызык будзе высокі?
Дазволь іншым, больш бяспечыным кандыдатам паспрабаваць!}

--- Але... прароцтва...

\emph{Ты не ведаеш дакладна, ці існуе нейкае прароцтва. Гэта была дзікая здагадка,
або, дакладней, дзікі жарт, і МакГонагал магла адрэагаваць толькі на словы пра 
тое, што Цёмны Лорд жывы. Па сутнасці, у цябе няма ні малейшай ідэі, аб чым яно кажа, 
калі яно наогул існуе. Ты проста разважаеш, або, дакладней, ЖАДАЕШ, каб 
для цябе ў гэтай гістрыі была падрыхтаваная нейкая ўласная геройская роля.}

--- Нават, калі прароцтва няма, я --- той, хто адолеў яго мінулым разам.

\emph{Гэта быў шчаслівы выпадак, канешне, калі ты не лічаш сур'ёзна, што 
ва ўзросце аднаго году дзіця ад народжання магло мець нейкія здольнасці для перамогі
над Цёмным Лордам, і якія маглі захоўвацца на працягу наступных дзесяці год. 
Усё гэта не можа быць сапраўднай прычынай і ТЫ ГЭТА ВЕДАЕШ!}


Адказ, які прыйшоў да галавы першым, Гары звычайна не сказаў бы ўголас. У размове ён бы паспрабаваў 
прымяніць нейкі манеўр, які дазволіў данесці гэтую думку шляхам больш сацыяльна 
прыемных аргументаў... 

Але Капялюш працягваў:

\emph{Ты думаеш, што ты --- патэнцыйна самы вялікі, самы моцны ва ўсёй гісторыі ваяр Святла,
і больш некаму падхапіць гэтую эстафету, калі ты спынішся.}

--- Ну... шчыра кажучы, так. Звычайна я не крычу аб гэтым на вуліцы, але так. 
Няма сэнсу змягчаць гэта, калі ты і так ведаеш мае думкі.

\emph{Вельмі добра. І ты таксама моцна верыш і ў тое, што можаш стаць самым жахлівым
Цёмным Лордам, якога дагэтуль ведаў свет.}

--- Разбураць заўсёды лягчэй, чым будаваць. Лягчэй парваць на кавалачкі, 
сапсаваць працэсы, чым сабраць іх назад. Калі ў мяне ёсць патэнцыял на вялікія добрыя
справы, я павінен таксама мець патэнцыял на нават большыя дрэнныя рэчы... але 
ты ведаеш, што я абіраю.

\emph{Ты сам настойваеш на тым, каб рызыкнуць! Чамы ты такі ўпарты? У чым 
сапраўдная прычына твайго нежадання ісці ў Хафлпаф і быць шчаслівым? Што цябе 
насамрэч пужае?}

--- Я павінен раскрыць свой патэнцыял цалкам. І калі не цалкам... тады 
я прайграў...

\emph{Што здарыцца, калі ты прайграеш?}

--- Нешта жудаснае...

\emph{Што здарыцца, калі ты прайграеш?}

--- Я не ведаю!

\emph{Тады гэта не павінна быць так страшна. Што здарыцца, калі ты прайграеш?}

--- Я НЕ ВЕДАЮ! НЕШТА ВЕЛЬМІ ДРЭННАЕ!

У прасторы свядомасці Гары на некалькі секунд наступіла цішыня. 

\emph{Ты ведаеш --- ты не дазваляеш сабе думаць аб гэтым, але дзесьці ў цёмных 
кутках свайго розуму ты ведаеш, аб чым ты не думаеш, --- ты ВЕДАЕШ што без сумневу
самае простае тлумачэнне --- гэта твой страх пазбавіцца сваіх фантазій аб велічы, 
страх расчараваць людзей, які ў цябе вераць, стаць звычайным чалавекам, успыхнуць,
і згаснуць, як тыя вундэркінды...}

---  Не, --- падумаў Гары ў адчаі, --- не,
павінна быць нешта яшчэ, я ведаю, што ёсць нешта акрамя гэтага, чаго варта баяцца,
нейкая катастрофа, якую я павінен спыніць...

\emph{Якім чынам ты можаш ведаць падобнае?}

Гары закрычыў ва ўсю моц свайго розуму:

--- \emph{ВЕДАЮ, І ГЭТА МОЙ ФІНАЛЬНЫ АДКАЗ!}

Праз некаторы час пачуўся ціхі голас Капелюша:

\emph{Значыць, ты гатовы рызыкнуць магчымасцю стаць Цёмным Лордам, таму што
альтэрнатыва для цябе --- гэта параза, і гэтая параза вызначае страту ўсяго. Ты 
верыш у гэта ўсёй сваёй свядомасцю. Ты ведаеш усе контрагрументы супраць 
гэтай веры, але абвяргаеш іх.}

--- Так. І нават калі Рэйвенкло \emph{павялічыць} халоднасць, гэта не значыць, што
халоднасць урэшце пераможа.

\emph{Гэты дзень --- найважнейшая развілка на шляху твайго лёсу. Не будзь 
так упэўнены, што далей у цябе будзе шмат магчымасцей для выбару.
На гэтым шляху няма знакаў, якія паказваюць АПОШНЮЮ магчымаць павярнуць назад. 
Ты адмаўляешся ад гэтага шанцу --- магчыма, таксама ты вырашыш і з астатнімі.
Магчыма, сённяшні выбар цалкам вызначае ўвесь твой лёс.}

--- Але гэта не дакладна.

\emph{Тое, што ТЫ не ведаеш усё дакладна, можна значыць толькі пра ТВАЁ
няведанне.}

--- І ўсё ж такі гэта не дакладна.

Капялюш выдаў жахліва сумны подых.

\emph{І тады хутка ты станеш чарговым успамінам, які я адчуваю як статыстычнае
сярэдняе, у маім наступным папярэджанні новым пакаленням...}

--- Калі ты думаеш, што так правльна, чаму ты проста не адправіш мяне, куда хочаш?

Думкі Капелюша былі прасякнуты смуткам.

\emph{Я магу адправіць цябе толькі туды, дзе твае месца. І толькі ты можаш вырашыць, дзе
яно сапраўды.}

--- Тады ўсё проста. Дасылай мяне ў Рэйвенкло, там такія ж, як і я.

\emph{А чаму б табе не разгледзець Грыфіндор? Гэта самы прэстыжны факультэт...
і людзі, бадай, чакаюць, што ты абярэш яго... і яны будуць крыху расчараваныя, 
калі ты пойдзеш у   Рэйвенкло... да таго ж, твае новыя сябры блізняты Уізлі там...}

Гары хіхікнуў, дакладней, адчуў парыў хіхікнуць, і гэта выйшла цалкам, як 
ментальны смешок, вельмі дзіўнае адчуванне. Верагодна, былі адмыслоўныя заклёны,
якія не давалі табе сказаць нічога ўголас, пакуль ты знаходзіўся пад Капелюшом,
размаўляючы пра рэчы, якімі не падзелішся з ніводнай душой за ўсё свае жыццё.

Праз імгненне, Гары пачуў смех Капелюша, дзіўны сумны шорхаючы гук.

(І ва ўсёй астатняй зале пачаліся ціхія аддаленыя шэпты,
якія, крыху ўзняўшыся, апалі зноў, знікнуўшы ў абсалютнай цішы, якую ніхто 
не смеў парушыць, бо Гары знаходзіўся па Капелюшом доўгія хвіліны, даўжэй,
чым група першакурснікаў да яго, складзеныя разам, даўжэй, чым хтосьці ў 
гісторыі Хогвартс. За галоўным сталом Дамблдор па-ранейшаму добра ўсміхаўся,
ціхія металічныя гукі даносіліся з боку Снэйпа, які задуменна працягваў спрэсоўваць
рэшкі кубка. Мінерва МакГонагал сцісквала кафердру пабялеўшымі пальцамі, ведаючы,
што зараза хаоса перакінулась з Гары Потэра на Размеркавальны Капялюш,
і што Капялюш зараз патрабуе, каб размеркаваць Гары Потэра, яны павінны стварыць
новы факультэт Суднага Дня, і што \emph{Дамблдор прымусіць яе яго ўзначаліць}...)

У прасторы пад Капелюшом ціхі смех паволі сцішыўся. Чамусьці Гары адчуваў жаль.
Не, не Грыфіндор.

--- Прафесар МакГонагал сказала, што калі ты прапануеш Грыфіндор,
мне варта напомніць табе, што калісьці яна можа стаць дырэктарам, і ў яе 
з'явяцца паўнамоцтвы цябе спаліць.

\emph{Перадай ёй, што я назваў яе "нахабная саплячка", і прыказаў сысці 
з майго газону.}

--- Перадам. Скажы, гэта была самая дзіўная размова за ўвесь час?

\emph{Нават і блізка не ляжала,}  --- тэлепатычны голас Капелюша 
крыху напружыўся. --- \emph{Так, я даў табе дастаткова шанцаў зрабіць годны выбар. 
Прышоў час размеркаваць цябе туды, дзе твае месца, да такіх жа, як і ты.}

Ён замаўчаў. Паўза расцягнулася.

--- Чаго ты чакаеш?

\emph{Увогуле, я спадзяваўся ўбачыць момант жахлівага разумення.
Падобна, што самаўсведамленне паляпшае мае пачуццё гумару.}

--- \emph{Хм?..} --- Гары стрымаў свае думкі, спрабуючы зразумець, на што мог 
намякаць Капялюш... і потым ён раптам зразумеў. Ён не мог паверыць, што забыўся
пра гэта.

--- Ты маеш на ўвазе жахлівае разуменне, што ты пазбавішся свядомасці адразу 
пасля таго, як размяркуеш мяне...

Нейкім чынам, які Гары зусім не здолеў зразумець, у яго галаве ўзнікла невербальная
выява Капелюша, які б'ецца галавой аб сцяну.

\emph{Я здаюся. Ты занадта марудны для гэтага жарту. 
Ты ў палоне сваіх уласных дапушчэнняў. Думаю, прыйдзецца сказаць табе прама.}

--- М-м-марудны?..

\emph{Дарэчы, ты цалкам забыў, што хацеў выведаць сакрэты забытай магіі, якая мяне стварыла.
А яны былі такія цудоўныя, такія важныя рэчы...}

--- Ты, маленькая хітрая \emph{сволач!..}

\emph{Ты цалкам заслужыў гэта, і гэта таксама:}

Гары зразумеў, калі было ўжо надта позна.

Спужаную цішыню залы Хогвартс узарвала адно слова:

--- СЛІЗЭРЫН!

Некаторыя са студэнтаў нявольна ўскрыкнулі, такім напруджаным было чаканне.
Некаторыя ўздрыгануліся так, што ўпалі з лавак. Хагрыд у жахе ахнуў,
у МакГонагал падкасіліся ногі за кафедрай, Снэйп выпусціў рэшкі кубка, якія ўпалі
яму на штаны.

Гары сядзеў, змярцвеўшы. Яго жыццё было разбурана, ён адчуваў сябе абсалютным дурнем, 
дзіка жадаючы, каб ён зрабіў іншы выбар. Каб ён зрабіў нешта, хоць \emph{штосьці}
па-іншаму да таго часу, калі павярнуць назад было немагчыма.

Калі першы шок прайшоў, і да людзей пачаў даходзіць сэнс таго, што зараз адбылося,
зноў пачуўся голас Размеркавальнага Капелюша:

--- Проста пажартаваў. РЭЙВЕНКЛО!
