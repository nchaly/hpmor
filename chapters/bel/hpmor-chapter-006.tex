\chapter{Заблуда Планавання}
\begin{chapterOpeningQuote}
    Думаеш, твой дзень быў сюррэальны? Паспрабуй мой.
\end{chapterOpeningQuote}

\lettrine{І}{ншыя} людзі, напэўна, дачакаліся бы заканчэння свайго першага
паходу ў Дыягон-аллею. Але...

--- Мяшок элемента нумар семдзесят дзевяць, --- сказаў Гары, і дастаў пустую 
руку з махляскіна.

Іншыя людзі прынамсі дачакаліся бы сваёй першай чароўнай палачкі.

--- Мяшок \emph{оканэ,} --- сказаў Гары. Цяжкі мяшочак золата ўскочыў яму ў руку.

Гары дастаў яго з кашэля, паглядзеў на яго, і вярнуў на месца. Дастаў руку, засунуў
назад, і сказаў:

--- Мяшок сродкаў эканамічнага амену, --- пуста. --- Вярні мне мяшок, які я толькі што паклаў, ---
зноў у яго руцэ апынуўся мяшочак манет.

Гары Потэр атрымаў свой першы магічны прадмет --- навошта чагосьці чакаць?

--- Прафесар МакГонагал, можаце сказаць два словы, адно каб вызначала золата, а
другое --- нешта іншае, не грошы, на мове, якую я не ведаю? Але не кажыце мне, 
якое з іх што.

--- \emph{Ахава} і \emph{захаф,} --- сказала МакГонагал. --- Гэта іўрыт, іншае 
слова значыць "любоў".

--- Дзякуй. Мяшок \emph{ахава}, калі ласка.

Нічога.

--- Мяшок \emph{захаф,} --- і той апынуўся ў яго руцэ. --- Захаф --- золата? --- 
спытаў Гары. МакГонагал кіўнула.

Гары мысленна агледзеў яшчэ раз дадзеныя, якія ён сабраў.
Яго эксперымент быў грубы і прыблізны, але яго цалкам хапала прынамсі на папярэднюю выснову:

--- \emph{А-а-а, што за бязглудзіца!}

МакГонагал падняла бровы.

--- Праблемы, містэр Потэр?

--- Я толькі што абвергнуў усе свае гіпотэзы! Якім чынам яно разумее пра "сумку 
са ста пятнаццаццю галеонамі", але пры гэтым "дзевяноста плюс дваццаць
пяць галеонаў" не працуе? Яно можа \emph{лічыць}, але не ўмее \emph{дадаваць}?
Яно разумее назоўнікі, але не назоўнікавыя групы з тым жа самым значэннем? Чалавек, які
зрабіў кашэль, магчама, не ведаў японскую мову, а я не гавару на іўрыце, значыць
кашэль не карыстае ні мае веды, ні веды стваральніка... --- Гары бездапаможна
махнуў рукой. --- Правілы яго працы \emph{выглядаюць} як кансістэнтныя, але яны 
не маюць сэнсу! Я нават не буду пытаць, якім чынам кашэль абсталяваны сістэмамі
распазнавання голасу і апрацоўкі натуральнай мовы, калі найлепшыя ШІ-эксперты
не здолелі зрабіць гэта на супер-кампьютэрах за трыццаць пяць гадоў 
працы?\footnote{{} Нагадаем, што падзеі адбываюцца ў 1991 годзе.} ---
Гары глытнуў паветра, --- \emph{Да што за халера там унутры адбываецца?}

МакГонагал пацепнула плячыма.

--- Магія.

--- Гэта проста \emph{слова!} Ну добра, вы сказалі мне, што там магія, але я не
магу зрабіць ніякіх прадказанняў! Гэта тое ж самае, як вучоныя казалі "флагістон",
або "élan vital" або "эмерджэнтнасть". % Complexity... is it really so controversial?

Прафесар МакГонагал засмяялася.

--- Але гэта \emph{сапраўды} магія, містэр Потэр.

Гары выдахнуў і знясілена згорбіўся.

--- З усёй магчымай павагай, прафесар, я не ўпэўнены, што вы разумееце, што менавіта я
намагаюся зрабіць.

--- З усёй магчымай павагай, містэр Потэр, не разумею. Акрамя, канешне --- і
гэта толькі мая здагадка, --- вы плануеце захапіць весь свет?

--- Не! У сэнсе --- так... але, \emph{не!}

--- Мне ўжо пачынаць хвалявацца? Бо я назіраю ў вас нейкія цяжкасці з адказам на
прасцейшыя пытанні.

Гары змрочна ўспамінаў Дартмускі варкшоп па штучнаму інтэлекту 1956 года. Гэта
была першая ў гісторыі канферэнцыя на гэту тэму, там жа і прыдумалі тэрмін 
"штучны інтэлект". Яны вызначылі ключавыя пытанні гэтай вобласці, такія, як
распазнаванне мовы, самаадукацыя і камп'ютэрная творчасць.
Яны на поўным сур'ёзе заключылі, што група з дзесяці вучоных зможа дасягнуць
значнага прагрэсу па гэтым напрамкам прыкладна за два месяцы.

\emph{Не. Вышэй галаву. Ты толькі прыадчыніў дзверцы і стаіш у самым пачатку шляху
раскрыцця ўсіх магічных таямніц. Ты не можаш ведаць, колькі часу спатрэбіцца, каб
прайсці яго да канца.}

--- І вы \emph{праўда} не чулі, каб нехта задаваў такія пытанні, або праводзіў 
навуковыя эксперыменты? --- Гары не разумеў, як такое было магчыма. Яму гэта падавалася
такой відавочнай ідэяй.

З другога боку, маглаўскім вучоным спатрэбілася два стагоддзя пасля адкрыцця
навуковага падыходу, каб пачаць сістэматычна вывучаць, якія сказы 
можа або не можа зразумець \emph{чалавек ва ўзросце чатырох гадоў}. Гэта можна
было высветліць яшчэ ў пачатку васямнаццатага, але чамусьці ніхто нават і не 
здагадаўся гэта зрабіць да дваццатага стагоддзя. Таму нельга было вінаваціць 
значна малейшую магічную суполку ў тым, што яны не даследвалі заклён Аўта-дастаўкі.

МакГонагал на імгненне сціснула вусны, потым зноў пацепнула плячыма. 

--- Я ўсё яшчэ не разумею, што вы маеце на ўвазе пад "навуковымі эксперыментамі", 
містэр Потэр. Як я ужо казала, мне прыходзілася быць сведкай таго, як
дзеці маглаў спрабавалі прымусіць вашу навуку спрацаваць у Хогвартс. Да таго ж,
кожны год хтосьці прыдумляе новыя зачараванні і зёлкі.

Гары адмоўна пакачаў галавой. 

--- Тэхналогія --- гэта зусім не тое ж самае, што навука. І проста спрабаваць шмат
розных спосабаў зрабіць нешта --- адрозніваецца ад эксперымента, які павінен 
высветліць законы прыроды --- напрыклад, шмат людзей спрабавалі пабудаваць 
самалёт, проста перабіраючы ўсялякія варыянты "нешта з крыламі", але толькі
браты Райт здагадаліся пабудаваць аэрадынамічную трубу і вымераць пад'ёмную сілу. 
Эммм, дарэчы, \emph{колькі} дзяцей маглаў трапляе ў Хогвартс штогод?

МакГонагал на секунду задумалася.

--- Прыкладна дзесяць?

Гары спатыкнуўся і амаль не упаў.

--- \emph{Дзесяць?}

Насельніцтва маглаўскага свету налічвала шэсць мільярдаў. Калі ты быў адзін з мільёна,
то такіх, як ты, было дванаццаць у Нью-Ёрку, і болей за тысячу ў Кітаі. Немагчыма, каб
свет маглаў не мог парадзіць яшчэ некалькі адзінаццацілетак, якія валодалі матаналізам ---
Гары ведаў, што ён быў не адзіным такім. Ён сустракаў іншых вундэркіндаў на 
 алімпіядах па матэматыцы, дзе яго лёгка перамагалі дзеці, якія праводзілі літаральна
цэлыя дні, вырашаючы алімпіядныя задачы, і якія ніколі не чыталі фантастыку, і якія,
імаверна, цалкам выгараць, не паспеўшы дасягнуць палавой зрэласці, і нічога
не дасягнуць у сваім дарослым жыцці, бо яны карысталі толькі вядомыя падыходы замест
таго, каб вучыцца думаць \emph{крэатыўна}. (Звычайна Гары быў мізэрным лузерам на 
такіх спаборніцтвах).

Але... у магічным свеце...

Дзесяць чалавек у год, і ўсе спынілі сваю маглаўскую адукацыю ў адзінаццаць? І хаця
МакГонагал можа быць неаб'ектыўнай, яна лічыць Хогвартс самай вялікай і самай
выбітнай з усіх школ у свеце... і нават тут навучэнне ідзе толькі да сямнаццаці гадоў.

Прафесар МакГонагал безумоўна магла дасканала ведаць малейшыя тонкасці працэсу
пераўтварэння чалавека ў ката. Але падавалася, што яна \emph{ніколі} ў жыцці не чула
пра навуковы падыход. Для яе гэта была ўсяго ж маглаўская магія. І ёй нават 
не было \emph{цікава}, якія сакрэты могуць хавацца за разуменнем чалавечай мовы, якое
дэманстравала Аўта-дастаўка. 

Гэта пакідала толькі дзве магчымасці.

Магчымасць першая: магія была настолькі змрочнай, перакручанай і бязглуздай штукай,
што нават калі магі і намагаліся з усіх сіл зразумець яе, у ніх нічога не атрымалася,
і ў выніку яны здаліся; і ў Гары няма шанцаў на лепшы вынік.

\emph{Або...}

Гары рашуча хруснуў касцяшкамі пальцаў, але яны выдалі толькі ціхі, амаль нечутны гук,
замет таго, каб злавесна прагрымець, і адбіцца эхам ад сцен Дыягон-аллеі. 

Магчымасць два: ён здолее захапіць свет.

Можа не адразу. З цягам часу. У рэшце рэшт.

Такія рэчы часам займалі болей за два месяца. Маглы-вучоныя не дасягнулі Луны
на наступны тыдзень пасля вынаходства Галілеем тэлескопа.

Гары не мог спыніць вялізарную ўсмешку, якая расцягвала яго шчокі так, што 
яны крыху забалелі.

Яго заўсёды палохала верагоднасць скончыць як тыя вундэркінды, якія пражывалі
жыццё, казыраючы, якімі класнымі яны былі ў чацьвёртым класе. Але большасць 
геніеў-дарослых таксама нічога значнага не дасягала. Бо у гісторыі на кожнага 
рэальнага Эйнштэйна мабыць прыходзілася  
па тысячы настолькі ж разумных людзей.
І ўсё гэта таму, што ў іх не было нечага абсалютна неабходнага для дасягнення велічы.

Яны так і не знайшлі важнай задачы.

\emph{Цяпер ты мой,} думаў Гары, агледжваючы сцены Дыягон-аллеі, крамы і вітрыны, 
прадаўцоў і мінакоў, думаючы пра ўсе землі і ўсіх людзей магічнай Брытаніі,
і пра магічны свет планеты, і пра ўвесь навакольны сусвет, пра які маглаўскія
вучоныя ведалі значна менш, чым ім падавалася. 

\emph{Я, Гары Джэймс Потэр-Эванс-Верэс, у імя Навукі, абвяшчаю гэтую тэрыторыю сваёй!}

Гром з маланкай цалкам праігнаравалі гэтую магчымасць бліскнуць і прагрымець у 
чыстым небе.

--- Аб чым вы там усміхаецеся? --- спытала МакГонагал, стомлена і насцярожана.

--- Цікава, ці ёсць заклінанне, каб на фоне грымеў гром кожны раз, калі я раблю 
фатальныя рэзалюцыі? --- адказаў Гары. Ён старанна запамінаў дакладныя словы сваёй
фатальнай рэзалюцыі, каб будучыя кнігі па гісторыі маглі перадаць яе правільна.

--- У мяне адчуванне, што мне трэба было б нешта рабіць на гэты конт, --- уздыхнула
МакГонагал.

--- Ігнаруйце яго, і яно пройдзе. О-о-о, крута! --- Гары тэрмінова паставіў ідэю 
аб захопе свету на паузу, і пабяжаў да яскравай вітрыны. 

\later

Гары набыў інгрыдыенты для зёлкаварэння, кацёл, і так, яшчэ пару рэчаў. 
Рэчаў, якія было б добра мець ў сваёй Захавальнай Сумцы (таксама вядомай як
Махляскінавы Супер-Кашэль \abbrev{qx31} з зачаркамі Гумавага Хайла, Незаўважнага Пашырэння,
і Аўта-дастаўкі). Разумныя, узважаныя пакупкі.

Гары шчыра не мог зразумець, чаму МакГонагал выглядала такой \emph{падазронай}.

У гэты момант яны былі ў краме, якая знаходзілася на галоўнай вуліцы 
Дыягон-аллеі. У магазіна не было вітрыны, затое большасць тавараў была выстаўлена
на вуліцы на наклонных драўляных паліцах, абараняемая толькі слабым шэрым ззяннем
і дзеўчынай-прадаўшчыцай у вельмі скарочанай версіі мантыі, якая агаляла калені і 
локці. 

Гары вывучаў магічны эквівалент аптэчкі неадкладнай падамогі пад назвай
"Emergency Healing Pack Plus". Унутры былі два сама-закручваюшчыхся турнікета для
спынення крыві. Зёлка Стабілізацыі, якая запавольвала крываток, і прадухіляла
шок. Шпрыц з чымсьці, што выглядала, як вадкі агонь, які павінен быў 
значна панізіць цыкруляцыю крыві ў пашкоджанай вобласці, адначасова падтрымліваючы
высокую аксігенацыю да трох хвілін --- напрыклад, каб не даць атруце распаўсюджвацца
па целе. Кускі белай тканіны, якія можна было абярнуць вакол часткі цела для 
болесуцішальнага эфекту. Таксама было колькі рэчаў, які Гары не змог зразумець,
кшталту "Постдэментавір", якое выглядала, як звычайны шакалад.
Або "АнтыБафлснэр", якім звалося маленькае дрыжачае яйка, да якога была
прывязана інструкцыя, як увапхнуць яго у чыюсьці ноздру.

--- Салідны закуп за пяць галеонаў, ці не так? --- сказаў Гары МакГонагал,
і прадаўшыца, што стаяла побач, актыўна заківала.

Гары чамусьці чакаў ад МакГонагал нейкай ўхвальнай рэакцыі на конт яго  разважлівасці
і падрыхтаванасці.

Аднак адказны позірк можна было параўнаць хіба што з Вокам Саурона.

--- І магу я пацікавіцца, \emph{чаму менатавіта}, --- сказала МакГонагал з цяжкім
сарказмам, --- вы чакаеце, што вам патрэбіцца аптэчка неадкладнай дапамогі, малады
чалавек? (Пасля апошняга інцыдэнту ў краме зёлкаварэння яна намагалася не клікаць
яго "містэр Потэр", калі нехта быў побач.)

Рот у Гары ад здзіўлення нават сам раскрыўся.

--- Я не \emph{чакаю} нечага, каб ім карыстацца! Гэта проста на выпадак!

--- На які выпадак? 

Вочы Гары пашырыліся.

--- Вы думаеце, што я \emph{задумаў} нешта небяспечнае, і таму мне патрэбная
аптэчка?

Позірк, поўны змрочнага падазрэння і іранічнага недаверу быў для яго дастаткова 
зразумелым адказам.

--- Great Scott! --- усклікнуў Гары на мове Шэкспіра і Толкіна. (Гэту крутую фразу
ён пераняў у вар'ята-вучонага з фільма "Назад у Будучыню"). --- Тады вы,
напэўна, думаце тое ж самае пра зёлкі Падучай Пярыны, Жабразелле, і Вода-сілкавальныя
таблеткі? 

--- Безумоўна.

Гары здіўлена пакачаў галавой.

--- І што, вы мяркуеце, я запланаваў?

--- Не ведаю, --- сказала змрочна МакГонагал, --- але думаю, што ў выніку ўсё скончыцца
або дастаўкай тоны срэбра ў Грынготс, або сусветнай дыктатурай.

--- "Сусветная дыктатура" --- такая агідная фраза. Мне болей падабаецца 
называць гэта "глабальнай аптымізацыяй"

Гэтым ён не здолеў пераканаць прафесара МакГонагал, якая ўсё яшчэ глядзела на 
яго Позіркам Року.

--- Ох, --- сказаў Гары, зразумеўны, што яна сур'ёзна. --- Вы сапраўды так думаеце. 
Вы сапраўды думаеце, што я планую нешта вельмі небяспечнае.

--- Так.

--- І гэта адзіная прычына, чаму людзі набываюць аптэчкі? Не прыміце за крыўду, 
прафесар, але \emph{што за вар'яты прымусілі вас прыняць такую думку?}

--- Грыфіндорцы, --- праз зубы адказала МакГонагал. Гэтае кароткае слова несла ў 
сабе такі груз горычы і адчаю, што ён падаваўся вечным праклёнам усяму падлеткавану 
гераізму і ідэалізму.

--- Паважаная намесніца дырэктара прафесар МакГонагал, --- сказаў Гары сурова, 
паклаўшы рукі на пояс. --- Я не збіраюся ў Грыфіндор...

МакГонагал у гэты момант уставіла фразу на конт, што калі ён усё ж-такі збіраецца
ў Грыфіндор, яна абавяскова прыдумае, як забіць капялюш, а Гары пакінуў яе без
каментара, бо нічога не зразумеў, але на прадаўшчыцу быццам напаў прыступ кашля.

--- Я планую трапіць у Рэйвенкло. І калі вы думаеце, што я планую нешта небяспечнае,
то --- пры ўсёй паваге, --- вы ні кроплі мяне не разумееце. Я не люблю \emph{небяспеку},
я \emph{баюся} яе. Я проста разважлівы. Я проста асцярожны. Мне падабаецца адчуваць сябе
падрыхтаваным да непрадбачаных абставінаў. Як мае бацькі пелі мне ў дзяцінстве:
\emph{Будзь гатоў! То бой-скаутаў дэвіз! Будзь гатоў! Праз жыццё яго нясі!
І не бойся, не хвалюйся, проста да ўсяго рыхтуйся!}”\footnote{{}
Бацькі Гары пелі яму толькі першы куплет гэтай знакамітай 
і даволі непрыстойнай песні Тома Лерэра,
і ён знаходзіцца ў блажэнным няведанні астатняга тэксту.}

Выраз на твары МакГонагал крыху памягчэў, --- па большасці, пасля слоў Гары пра Рэйвенкло.

--- Да якіх менавіта абставінаў, вы думаеце, можа падрыхтаваць вас гэтая аптэчка, 
\emph{малады чалавек?}

--- Напрыклад, маю аднакласніцу ўкусіў жудасны монстр, і пакуль я ліхаманкава шукаю ў
махляскіне нешта, што можа ёй дапамагчы, яна сумна глядзіць на мяне і кажа
з апошнім уздыхам: \emph{‘Чаму ж ты не падрыхтаваўся?’} І потым яна памірае,
яе вочы заплюшчваюцца навечна, і я ведаю, што яна ўжо ніколі мяне не прабачыць...

Гары пачуў рзкі ўдых з боку прадаўшчыцы, і падняўшы вочы, убачыў яе спужаны позірк і 
моцна сціснутыя вусны. Потым яна павярнулася і ўбегла кудысці ўглыб крамы.

\emph{Што за?..}

Прафесар МакГонагал нахілілася, узяла Гары за руку, спакойна, але цвёрда, і 
пацягнула Гары ў бліжэйшы завулак, які быў брукаваны бруднай цэглай,
і канчаўся сцяной сплашной чорнай гліны.

Ведзьма накіравала сваю палачку ў бок галоўнай вуліцы, сказала "\emph{Quietus}",
і вэлюм цішыні апанаваў іх, заглушаючы ўсі вулічныя гукі.

\emph{Ну што я зноў не так зрабіў?...}

Потым яна павярнулася да Гары і кінула ў яго ледзяной позірк ва усю сваю моц.

--- Я буду ўдзячна, калі вы нарэшцэ запомніце раз і назаўсёды, што не далей,
як дзесяць гадоў таму ў магічнай Брытаніі скончылася вайна, і \emph{кожны} страціў 
кагосьці, і саркасцічныя размовы пра сяброў, якія паміраюць на тваіх
руках \emph{НЕ! У! ПАШАНЕ!}

--- Я... я не... не хацеў... --- сэнс ягоных слоў, быццам падаючы з гары камень,
запусціў лавіну жвавага Гарынага ўяўлення. Тая дзеўчына ў краме, падчас вайны ён было
восем або дзевяць, калі... калі...

--- Прабачце, я не хацеў... --- Гары падавіўся камком у горле, ён адвярнуўся ад 
халоднага позірка МакГонагал з моцным жаданнем убяжаць, але шлях быў заблакаваны
сцяной, а сваёй палачкі ў яго яшчэ не было. --- Мне жаль, жаль, \emph{жаль!}

Ззаду пачуўся цяжкі ўздых.

--- Я ведаю, што вам жаль, містэр Потэр.

Гары нясмела глянуў назад. Гнеў сышоў с яе твара. 

--- Прабачце, --- паўтарыў Гары, адчуваючы сябе спустошаным. --- Не варта было гэта
казаць. Нешта такое здарылася і з... --- тут Гары далонню зачыніў свой рот, каб
не сказаць яшчэ больш.

--- Вы павінны навучыцца думаць, перад тым як казаць нешта, містэр Потэр, ---
сказала яна сумна. --- Інакш у вашым жыцці будзе няшмат сяброў. Гэта 
часты лёс
для выпускнікоў Рэйвенкло, і я спадзяюся, вы не пойдзеце тым шляхам.

Гары хацеў збяжаць. Ён хацеў проста знікнуць з гэтага месца, хацеў махнуць 
палачкай, і сцерці ўсё з памяці МакГонагал, вярнуцца на дзесяць хвілін назад, 
і зрабіць, каб \emph{нічога не адбылося}...

--- На конт вашага пытання, --- працягвала МакГонагал, --- не, нічога \emph{такога}
са мной не здаралася. Дакладна, я была сведкай апошняга подыху сябра, аднойчы...
а можа і болей. Але ніхто з іх не праклінаў мяне перад смерцю, і я 
ніколі не думала, што яны мяне больш не прабачуць.
\emph{Напрамілы Мерлін, што за чорт дзёрнуў
вас сказаць гэта?} Як \emph{такое} наогул можна прыйсці ў галаву?

Слёзы цяклі па шчаках Гары.

--- Прабачце, мне не трэба было такога казаць... прабачце...

--- Я разумею, што вы адчуваеце віну. Чаго я не разумею, дык гэта таго, з якой прычыны 
адзінаццацілетні хлопец можа думаць такія рэчы. Вы сапраўды вырашылі набыць 
пяці-галеонную аптэчку і насіць яе ў пятнаццаці-галеонным кашалі, таму што
ўпэўнены, што інакш вашы сябры будуць \emph{праклінаць вас падчас смерці?}

--- Я... я... я... --- Гары зглынуў. --- Так было заўсёды. Я проста кожны раз
уяўляю самае дрэннае, што можа здарыцца.

--- \emph{Чаму?}

--- Каб перадухіліць гэта!

--- Містэр Потэр... --- с уздыхам сказала МакГонагал, і прысела перад ім. --- Містэр
Потэр, --- сказала яна мягчэй, --- гэта не ваш абавязак --- клапоціцца аб студэнтах 
у Хогвартс. Гэта мой абавязак. Я не дазволю здарыцца чамусьці дрэннаму. 
Хогвартс --- самае бяспечнае месца ў магічнай Брытаніі, а мадам Помфры ---
выбітны медык, і ў яе ёсць кіпа ўсіх магчымых лекаў. Вам проста не можа спатрэбіцца
ўласная аптэчка.

--- Я думаю, што \emph{можа}! --- зноў узняў голас Гары. --- Не існуе абсалютна
бяспечнага месца! А што калі ў маіх бацькоў адбудзецца сардэчны прыступ, або яны
трапяць у аварыю, калі я прыеду на Раство? Мадам Помфры там не будзе. 
Мне трэба уласная аптэчка. 

--- Да што ж такое, дзеля Мерліна... --- сказала МакГонагал. Яна ўстала і паглядзела
на Гары з выразам дзесьці паміж клопатам і раздражненнем. --- Няма ніякай прычыны
думаць такія жудасныя рэчы, містэр Потэр!

Выраз Гары перакруціўся ад горычы ад такіх слоў.

--- Ёсць! Калі не думаць, то потым не проста табе будзе балюча! Будуць пакутваць 
блізкія людзі!

Прафесар МакГонагал адчыніла рот і потым зачыніла. Яна пацёрла пераносіцу, 
аб нечым разважаючы. 

--- Містэр Потэр, калі я прапаную выслухаць вас, не перарываючы... ці ёсць нешта, 
аб чым вы хацелі бы мне расказаць?

--- Аб чым?

--- Аб тым, чаму вы так ўпэўнены, што трэба заўсёды быць наварце супраць 
жудасных падзей, якія павінны з вамі здарыцца.

Гары, азадачаны, ўтаропіўся ў яе. Гэта падавалася відавочнай аксіёмай.

--- Ну... --- сказаў ён павольна. Ён паспрабаваў прывесці свае думкі ў парадак.
Як можна было наогул растлумачыць нешта МакГонагал, калі яна не разумее нават
такіх базавых рэчаў?

--- Маглы-вучоныя высветлілі, што людзі заўсёды занадта аптымістычныя, напрыклад, 
калі яны 
кажуць, што нешта зойме два дня, то звычайна трацяць дзесяць, або кажуць --- два 
месяцы, а ім можа спатрэбіцца трыццаць пяць гадоў. У адным эксперыменце спыталі 
студэнтаў, колькі ім трэба часу на хатнюю працу з упэўненасцю 50, 75 і 99 адсоткаў,
толькі адпаведна 13, 19 і 45 адсоткаў студэнтаў уклаліся ў свае ацэнкі.
І потым вучоныя зразумелі, што аценкі ідэальнага сцэнару (калі ўсё ідзе
ідэальна і без перашкод) і ацэнкі рэалістычнага 
сцэнару (калі нешта заўсёды перашкаджае) \emph{статыстычна не адрозніваюцца}.
Разумееце, калі вы пытаеце людзей зрабіць план, яны выбіраюць у гр\'афе магчымасцей
лінейны пуць з мінімальным часам, без аніякіх сюрпрызаў і памылак.
Але, як вы памятаеце, большасць студэнтаў не паспела скончыць працу за час, калі
яны былі на 99\% упэўнены, значыць, што ў рэальнасці звычайныя рэзультаты будут 
крыху горш, чым ацэнка самага \emph{песімістычнага} сцэнару.  
Гэта называецца "заблуда планавання", і самае простае выйсце з яе --- спытаць,
колькі часу сышло на падобную задачу мінулым разам. Гэта значыць, карыстаць знешні
--- аб'ектыўны --- погляд замест унутраннага --- суб'ектыўнага. 
Але калі робіш нешта новае, такі падыход не працуе, і ты павінен быць 
вельмі, вельмі, вельмі песімістычным. Настолькі песімістычным, што 
насамрэч рэальнасць атрымліваецца \emph{лепей} тваіх планаў хаця б у палове
выпадкаў.
А гэта сапраўды складана --- перабіць рэальнае жыццё па песімістычнасці. Напрыклад,
вось я змрочна ўяўляю, што монстр пакусаў маіх аднакласнікаў, але ў рэальнасці
выжыўшыя Пажыральнікі Смерці зробяць напад на школу, і заб'юць усіх, каб 
дабрацца да мяне. Але трэба адзначыць добрую...

--- Стоп, --- сказала МакГонагал.

Гары спыніўся. Ён як раз збіраўся адзначыць добрую навіну, што прынамсі сам 
Цёмны Лорд не будзе ўдзельнічаць у нападзе, бо ён памёр.

--- Я думаю, я сказала недастаткова дакладна, --- сказала яна асцярожна. --- Ці
здаралася \emph{персанальна з вамі} нешта такое, што вельмі моцна вас пужала?  

--- Тое, што здаралася персанальна са мною, не можа быць доказам, --- паспрабаваў
растлумачыць Гары. --- Яно не нясе той вагі, якую маюць рэплікаваныя 
артыкулы з рэцэнзаваных часопісаў, якія апісваюць чотка спланаваны эксперымент
з мноствам удзельнікаў, вялікай колькасцю дадзеных і моцнай статыстычнай
значнасцю. 

МакГонагал пацерла пераносіцу, удыхнула, выдыхнула.

--- Усё ж, я бы хацела пра тое паслухаць.

--- Гхмм... --- Гары глыбока ўздыхнуў. --- У нашым раёне было некалькі выпадкаў
рабавання, а маці загадала схадзіць да суседзяў і вярнуць патэльню, якую яна
пазычыла. І я сказаў, што не пайду, бо я баяўся, што мяне таксама могуць
абрабаваць, а яна сказала "Гары, ну што ты такое кажаш!" Быццам, калі пра гэта
не размаўляць, усё будзе ў парадку. Я паспрабаваў растлумачыць, але яна мяне
ўрэшце прымусіла. Вядома, я быў надта маленькі для рабавання, але я быў дастаткова
дарослы, каб зразумець рэальныя прычыны падзей, і мне было вельмі страшна.

--- І ўсё? --- спытала МакГонагал пасля паузы, калі стала зразумела, што Гары скончыў
расказ. --- Больш нічога з вамі не зраралася?

--- Я ведаю, гучыць не вельмі ўразліва, --- пачаў абараняцца Гары. --- Але то 
быў адзін з паваротных момантаў жыцця, разумееце? Я маю на ўвазе, я \emph{ведаў}, 
што "не думаць" не спыняе нешта ад здарэння, але я зразумеў, што мая маці сапраўды
верыла ў гэта, --- Гары спыніўся, змагаючыся са злосцю, якая пачала ўздымацца,
калі ён успомніў той дзень. --- Яна не хацела нічога слухаць. Я спрабаваў 
пераканаць яе, я \emph{маліў} яе не пасылаць мяне на вуліцу, а яна проста над гэтым
пасмяялася. Усё, што я казаў, ёй падавалася нейкім вялікім жартам... ---
Гары зноў прыгасіў чорную злобу ўнутры. --- Тады я зрабіў выснову, што ўсе, 
хто павінны клапоціцца пра мяне --- бязглуздыя вар'яты, і што няважна, колькі я буду 
маліць, яны не будуць мяне слухаць, і што калі хочаш зрабіць нешта правільна,
на іх нельга спадзявацца.

Часам адных добрых намераў у жыцці не хапае. Часам трэба мець розум і карыстацца ім.

Доўгі час абодва маўчалі.

Гары выкарыстаў гэты час, каб глыбока падыхаць і супакоіцца. Не было сэнсу злавацца.
Не было сэнсу злавацца. \emph{Усе} бацькі такія. \emph{Ніводны} дарослы не згадзіцца
настолькі пазбавіцца свайго статусу, каб паставіць сябе на адзін узровень з дзіцём,
і яго генетычныя бацькі --- не выключэнне. Разумнасць была амаль нябачнай іскрай 
у цемры, знікальна маленькім выключэннем у царстве бязглусдасці, і таму не было
ніякага сэнсу злавацца.

Уззлаваным Гары сабе не падабаўся.

--- Дзякуй, што падзяліліся, містэр Потэр, --- сказала МакГонагал. На яе твары быў
крыху адстранёны выраз (амаль такі ж меў Гары падчас эксперыменту з махляскінам,
калі б ён толькі бачыў сабе ў люстэрка). --- Мне трэба над гэтым падумаць, ---
яна павярнулася да галоўнай вуліцы, падняла палачку...

--- Эммм, --- сказаў Гары, --- зараз мы можам вярнуцца да аптэчкі?

МакГонагал спынілася і цвёрда паглядядзела на яго. 

--- І калі я скажу, што яна надта каштоўная, і што яна вам не патрэбная, што тады?

Гары ажно скурчыўся ад горычы. 

--- Так я і ведаў. Тады я зраблю выснову, што вы наступны бязглузды дарослы, з якім
няма сэнсу размаўляць, і пачну распрацоўваць план, як атрымаць аптэчку ў любым
выпадку.

--- У гэтым паходзе я адказная за вас, --- адказала МакГонагал з адценнем пагрозы. ---
Я не дазволю сабой камандаваць.

--- Я разумею, --- сказаў Гары. Ён, як змог, прыбраў крыўду з голасу, і ўтрымаўся,
каб не сказаць яшчэ тое-сёе, што прыйшло яму да галавы. МакГонагал сказала яму 
думаць перад тым, як казаць нешта. Магчыма заўтра ён пра гэта забудзе, але 
прынамсі на пяць хвілін ён павінен гэта запомніць.

МакГонагал махнула палачкай, і гоман Дыягон-аллеі вярнуўся.

--- Ну што ж, малады чалавек, --- сказала яна, --- пойдзем за той аптэчкай.

Ад здзіўлення ў Гары сківіца адвалілася. Ён паспяшыў за ёю, амаль спатыкаючыся
ад нечаканасці.

\later

Крама была ў тым жа стане, як яны яе пакінулі: розныя рэчы ўсё таксама ляжалі на 
наклонных драўляных паліцах, шэрае ззянне ўсё таксама іх абараняла, і дзеўчына-прадаўшчыца
таксама была на сваёй пачатковай пазіцыі. Яна здзіўлена ўзняла галаву, калі яны падышлі. 

--- Прашу прабачэння... --- Прабачце калі ласка... --- сказалі яна і Гары адначасова. 

Яны перарваліся і паглядзелі адзін на аднаго, і прадаўшчыча хіхікнула. --- Я
не хацела, каб у вас былі непрыемнасці з прафесарам МакГонагал, --- яе голас
загаворшчыцкі сцішыўся. --- Я спадзяюся, яна не надта жостка вам выгаварыла?

--- \emph{Дэла!} --- абурылася МакГонагал.

--- Мяшок золата, --- сказаў Гары свайму махляскіну, адлічыў пяць галеонаў, і 
працягнуў дзеўчыне. --- Не хвалюйцеся, я разумею, што гэта проста таму, што яна
пра мяне клапоціцца. Адзін Emergency Healing Pack Plus, калі ласка, --- сказаў ён праз новыя абурэнні
з боку МакГонагал.

Ён крыху знервавана назіраў, як Гумавае Хайло пашырылася і праглынула аптэчку
памерам з дыпламат. Гары не мог не цікавіцца, што адбудзецца, калі ён сам
паспрабуе залесці ў махляскін, улічваючы, што даставаць прадметы назад мог 
толькі той, хто іх туды паклаў. 

Калі кашэль скончыў... есці?.. яго цяжка атрыманую пакупку, Гары мог 
паклясціся, што пачуў ціхі "коўць". Нехта \emph{павінны быў} зачараваць гэта
наўмысна. Альтэрнатыўная гіпотэза была настолькі жудаснай... шчыра кажучы, Гары
нават не мог сфармуляваць ніводнай альтэрнатыўнай гіпотэзы. Ён паглядзеў на
МакГонагал.

--- Куды далей?

Яна паказала на краму, якая, падавалася, не была збудавана з цэглы, а сама неяк
вырасла, і была пакрыта поўсцю замест фарбы. 

--- У Хогвартс студэнтам дазваляецца мець хатніх жывёл, напрыклад, вы маглі бы
набыць саву, каб пасылаць лісты...

--- А можна арэндаваць саву, калі мне гэта спатрэбіцца?

--- Так.

--- Тады, з пункту гледжання эмпатыі, не.

МакГонагал кіўнула, быццам робячы паметку ў спісе.

--- Магу я спытаць, чаму не?

--- У мяне быў аднойчы хатні камень. Ён памёр.

--- Вы лічаце, что няздольныя выхоўваць хатняга гадаванца кшталту савы?\footnote{{}Калі вы, як і МакГонагал, 
не зразумелі жарт, то на захадзе ў 
семідзесятыя гады дваццатага стагоддзя сапраўды прайшла хваля продажу жартоўных 
"хатніх" камянёў з адмысловай інструкцыяй па іх гадаванню.
Вы можаце лёгка знайсці пра гэта ў сеціве 
па запыту "Pet rock".}

--- Да здольны я, --- адказаў Гары, --- але скончыцца ўсё тым, што я буду бесперапынна
хвалявацца, ці не забыў я пакарміць яе, і што мабыць зараз яна ціха памірае з голаду
ў сваёй клетцы, і думае, дзе ж яе гаспадар падзяваўся, і чаму не нясе ежу...

--- Бедная сава, --- сказала МакГонагал мякка. --- Быць забытай сваім 
гаспадаром. Цікава, што бы яна зрабіла.

--- Калі яна канчаткова выгаладаецца, яна, магчыма, пачне працарапваць сабе шлях
на волю з клетцы або каробцы, або дзе яна там... хаця шанцаў у яе будзе
не вельмі шмат... --- Гары раптам спыніўся

МакГонагал сказала ўсё тым жа мяккім голасам:

--- І што адбудзецца з ёю пасля?

--- Прабачце, --- сказаў Гары, узяў яе за руку, спакойна, але цвёрда, і 
павеў яе ў бліжэйшы завулак. Гэта ўжо станавілася амаль незаўважнай руцінай для 
абодвух.

--- Калі ласка, скастуйце тую штуку, Quietus.

--- \emph{Quietus.}

--- Тая сава \emph{не} вызначае мяне, --- сказаў ён трасучымся голасам, --- і
мае бацькі \emph{ніколі} не зачынялі мяне ў кладоўцы, і \emph{ніколі} не марылі
мяне голадам, і у мяне няма страху, што мяне забудуць, і наогул: 
\emph{мне вельмі не падабаецца напрамак вашых думак, прафесар МакГонагал!}

--- Які менавіта напрамак вы маеце на ўвазе, містэр Потэр?

--- Вы думаеце, што я... что мяне... --- яму было цяжка гэта сказаць, ---
што мяне \emph{гвалтавалі}?

--- Вас гвалтавалі?

--- \emph{Не!} --- крыкнуў Гары. --- Ніколі! Вы думаеце, я нейкі \emph{дурань?} 
Мне вядома канцэпцыя дзіцячага гвалту, і калі нешта такое адбылося бы, я бы 
вызваў паліцыю! І расказаў бы дырэктару школы! І знайшоў бы тэлефон службы 
даверу ў даведніку! І расказаў бы дзядулі і бабуле, і місіс Фігг! 
Але мае бацькі \emph{ніколі} нічога падобнага не рабілі, \emph{ніколі!}
Як вы смееце наогул такое прапанаваць?

МакГонагал глядзела на яго.

--- Раследваць магчымыя адзнакі гвалту вучняў --- мой абавязак як намесніцы дырэктара.

Злосць Гары паграджала вырвацца з-пад кантролю і пераўтварыцца ў шалёную чорную
лютасць. 

--- І нават не думайце, каб сказаць нешта з гэтых... гэтых \emph{інсінуацый}
камусьці! Нікому, чуеце мяне, МакГонагал? Такое абвінавачванне ламае людзей,
разбурае сем'і, нават калі бацькі абсалютна не вінаватыя. Я чытаў пра такое 
ў газетах! --- голас Гары пачаў набліжацца на высокага крыку. --- Сістэма не
спыняецца, яна не верыць ні бацькам, ні дзецям, калі яны кажуць, што нічога не
было. \emph{Я не дазволю вам знішчыць маю сям'ю!}

--- Гары, --- сказала ціха МакГонагал і пацягнулася да яго.

Гары хутка зрабіў крок назад, і пляснуў па яе руцэ. 

МакГонагал застыла, потым прыбрала руку.

--- Гары, усё добра, --- сказала яна, --- я табе веру.

--- Ну канешне, --- агрызнуўся Гары. Лютасць усё яшчэ завіхалася ў ягонай крыві.
--- А можа проста чакаеце моманту, калі зможаце падаць кудысці заяву?

--- Гары, я бачыла твой дом і бацькоў. Яны любяць цябе, а ты --- іх. 
Я веру, калі ты кажаш, што вы --- шчастлівая сям'я. Але я павінна была спытаць,
таму што я бачу ў табе нешта дзіўнае.

Гары глядзеў на яе холадна.

--- Напрыклад?

МакГонагал глыбока ўздыхнула.

--- Гары, я бачыла ахвяр гвалту ў Хогвартс, і ты не паверыш наколькі шмат.
Твае паводзіны не падобныя да іх: ты ўсміхаешся незнаёмым, ты можаш абняць
чалавека, і калі я клала руку табе на плячо, ты не ўздрыгіваў. Але часам, вельмі
рэдка, ты можаш сказаць або зрабіць нешта такое, што мог бы зрабіць нехта... 
нехта, хто правёў першыя адзінаццаць гадоў жыцця, закрыты ў сутарэнні.
Не з той шчаслівай сям'ёй, што я бачыла, --- МакГонагал нахіліла галаву 
на бок, і яе выраз зноў стаў збянтэжаны. 

Гары раздумваў над яе словамі. Чорная лютасць пачала адыходзіць, і раптам
ён зразумеў, што яго толькі што ўважліва выслухалі, і што ягоная сям'я не была 
ў небяспецы.

--- І значыць, так вы растлумачылі свае назіранні, прафесар?

--- Магчыма, нешта магло здарыцца з табой, але ты забыў.

Гэта зноў нагадвала артыкулы пра знішчаныя сем'і. 

--- Выцісканыя ўспаміны --- гэта нешта псеўданавуковае. Людзі не выціскаюць з 
галавы траўматычныя моманты, наадварот, яны запамінаюць іх надта добра,
і пракручваюць у галаве ўсё жыццё!

--- Не, містэр Потэр. Існуе Забывальны Заклён.

Гары застыў.

--- Заклён, які выдаляе ўспаміны?

МакГонагал кіўнула.

--- Але не астатнія эфекты досвіду, калі вы разумееце, аб чым я, містэр Потэр.

Па хрыбту Гары прабяжаў халадок. \emph{Такую} гіпотэзу так проста не абвергнуць.

--- Але мае бацькі не маглі зрабіць такое!

--- Так, --- сказала МакГонагал, --- на тое быў патрэбны чараўнік. Але, наколькі
мне вядома, не існуе спосаба пацвердзіць факт яго ўжывання.

Рацыянальная частка Гары пачала прасынацца.

--- Прафесар, наколькі вы ўпэўнены ў сваіх назіраннях, і якія альтэрнатыўныя 
гіпотэзы вы можаце выдвінуць?

МакГонагал развяла рукамі.

--- Упэўнена? Я ні ў чым ужо не ўпэўнена, містэр Потэр. Калі разглядваць вас як
суцэльную асобу, то за ўсё свае жыццё я не сустракала нікога, падобнага на вас.
Часам падаецца, што вам шмат больш за адзінаццаць гадоў, а часам --- што вы нават не 
\emph{чалавек}.

Бровы Гары папаўзлі дагары.

--- Я прашу прабачэння, --- сказала хутка МакГонагал. --- Гэта прагучала не зусім
як тое, што я мела на ўвазе...

--- Насупраць, прафесар МакГонагал, --- сказаў Гары і паволі ўсміхнуўся. --- Для 
мяня гэта вялікі камплімент. Але, ці не будзеце вы супраць, калі я прапаную 
іншае тлумачэнне?

--- Я толькі за.

--- Дзеці не павінны быць разумнейшыя за сваіх бацькоў. Або быць больш 
рацыянальнымі, кшталту... ведаце, мой тата мог бы перамагчы мяне ў розуме,
калі б ён толькі шчыра \emph{паспрабаваў,} замест таго, каб выдумляць прычыны 
не змяняць свае дарослае меркаванне, --- Гары спыніўся. --- Я занадта разумны, 
прафесар. З дзецьмі майго ўзросту мне няма аб чым
гаварыць, дарослыя не ўспрымаюць мяне як роўнага. І, шчыра кажучы, мала хто іх
такі ж разумны, як Рычард Фейнман, таму мне прасцей пачытаць яго кнігі. 
Я \emph{ізаляваны}, разумееце? І ўсё свае жыццё я быў ізаляваны. Магчыма гэта 
спараджае такія ж эфекты, як і быць зачыненым у сутарэнні.
І я занадта разумны, каб успымаць бацькоў так, як тое задумана прыродай. Яны 
любяць мяне, але яны не адчуваюць, што абавязаны адказваць на мае 
разумныя пытанні або прапановы. Часам мне падаецца, што яны дзеці --- 
дзеці, якія маюць абсалютную ўладу над маім жыццём, і якія ніколі не будуць
мяне слухаць. Гэта даволі сумна, і я намагаюся не думаць пра гэта, але, шчыра
кажучы, мне даволі часта сумна. І яшчэ ў мяне праблема кантролю злосці, але 
над гэтым я працую. Гэта ўсё.

--- \emph{Гэта ўсё?}

Гары ўпэўнена кіўнуў. 

--- Гэта ўсё. Без сумневу, прафесар, нават у магічный Брытаніі, вы не адкідваеце
лагічныя тлумачэнні?


\later

Сонца пачало нахіляцца ніжэй, колькасць пакупнікоў на вуліцы парадзела.
Некаторыя крамы былі 
ўжо зачыненыя; Гары ледзьве паспеў набыць падручнікі ў "Флорыш і Блоц" да часу,
улічваючы невялікі выбух, калі ён, праходзячы міма паліцы з надпісам
"Арыфмансія" высветліў, што падручнікі для сёмага курса не ўтрымлівалі нічога 
складаней за планіметрыю.

Але ў гэты канкрэтны момант думкі Гары былі вельмі далёка ад праблем агранізацыі
даследавання магічнага свету.

У гэты канкрэтны момант Гары і МакГонагал адыходзілі ад крамы Алівандара, і 
Гары разгледжваў сваю чароўную палачку. Ён зрабіў узмах, і палачка выдала 
струмень блакітна-жоўтых іскраў. Гэта павінна было быць не такім шокам пасля
ўсяго, што ён сёння перажыў, але... 

\emph{Я магу рабіць магію.}

\emph{Я --- магічная істота, я --- чарадзей.}

Ён \emph{фізічна} адчуваў магію, якая лілася з яго пальцаў праз палачку, і ў 
гэты момант зразумеў, што гэта пачуццё заўседы было з ім, нешта акрамя зроку,
слыху, смаку, нюху і кранання --- проста пачуццё магіі.
Быццам як вочы, якія былі заплюшчаны ўсё жыццё, і частка мозга, адказная за 
іх, рэгістравала толькі цемру да таго моманту, пакуль вочы не расплюшчыліся
і не ўбачылі свет.

І яшчэ...

\emph{Даволі цікаўна, што вы прызначаны да гэтай палачкі... бо яе сястра надзяліла 
вас гэтым шнарам.}

Гэта \emph{ніяк} не магло быць супадзеннем. У той краме былі мабыць тысячы палачак.
Ну, добра, гэта магло быць супадзеннем, у свеце было шэсць мільярдаў чалавек, 
і супадзення разраду тысяча-да-аднаго здараліся кожны дзень. 
Але згодна з тэарэмай Байеса, любая слушная гіпотэза пра тое, што шанец атрымаць сястрынскую
палачку Цёмнага Лорда быў болей за тысячу-да-аднаго, была болей рэалістычнай за астатнія. 

МакГонагал проста сказала безвыразна \emph{"не можа быць"}, што вагнала Гары ў 
чарговы шок, ад бяскрайняй \emph{абыякавасці} чараўнікоў. Ні ў адным з 
магчымых паралельных міроў Гары не мог проста хмыкнуць і працягнуць 
свой шопінг, хаця б не паспрабаваўшы сфармуляваць гіпотэзу аб тым, што
адбываецца.

Яго левая рука паднялася і дакранулася да шнара.

Што... \emph{насамрэч...}

--- Нарэшце вы --- сапраўдны маг, --- сказала МакГонагал. --- Вітаю вас.

Гары кіўнуў.

--- Як вам магічны свет?

--- Ён вельмі дзіўны, --- сказаў Гары. --- Мне варта думаць пра ўсе магічныя 
феномены, што я бачыў, пра тое, ва што я верыў дагэтуль, і якім неверагодным
і цяжкім будзе мой шлях вывучэння таямніц магіі. Але мяне пастаянна адцягваюць
нейкія трывіяльныя рэчы, кшталту, --- Гары сцішыў голас, --- гэтай
мітусні вакол "хлопчыка-які-выжыў", --- і хаця вакол нікога не было, не трэба
было правакаваць лёс.

МакГонагал \emph{ахымкнула}.

--- Праўда? І не кажыце.

Гары кіўнуў.

--- Так, усё гэта... сюррэальна. Раптам высветліць, што ты быў часткай вялікіх
падзей, паразы жудаснага Цёмнага Лорда, і гісторыя ўжо \emph{прайшла.} Скончылася.
Перайшла ў кнігі па гісторыі. Быццам ты Фрода Бэгінс, і табе расказваюць, што 
ў дзяцінстве твае бацькі аднеслі цябе на Ракавую Гару, каб ты мог
бросіць Пярсцёнак ў прорву, і ты нічога з гэтага не памятаеш.

Усмешка МакГонагал крыху здранцавела.

--- Ведаеце, калі я быў бы кісмьці іншым, я, магчыма, ужо бы пачаў хвалявацца,
як жыць адпаведна гэтаму стандарту.  \emph{Божа, Гары, чаго ты дасягнуў 
з тых пор, як адалеў Цёмнага Лорда? Завёў кнігарню? Крута! А я ў твой 
гонар дзіця назваў.} 
Спадзяюся, што для рэальнага мяне гэта не стане праблемай, --- ён 
уздыхнуў. --- Але мне амаль \emph{хочацца}, каб у той гісторыя былі нейкія прабелы, 
нейкія незразумелыя часткі, якія я магу скончыць, і сказаць, што сапраўды,
разумееце, прымаў удзел.

--- О? --- сказала МакГонагал дзіўным тонам. --- Вы маеце на ўвазе нешта
канкрэтнае?

--- Напрыклад, вы згадвалі, што маім бацькам здрадзілі --- хто быў здраднік?

--- Сірыус Блэк, --- працэдзіла МакГонагал. --- Ён у Азкабане. Магічнай турме.

--- Наколькі верагодна, што ён збяжыць, і мне прыдзецца высочваць яго і пераадолеваць
у эпічнай бойцы? А лепей --- абвесціць узнагароду за яго галаву, і, пакуль
яго ловяць, хавацца ў Аустраліі?

МакГонагал міргнула. Двойчы. 

--- Малаверагодна. Ніхто ніколі не збягяў з Азкабана, і я сумняваюся, што ён 
будзе першым.

Гары скептычна ўспрыняў гэтае "Ніхто ніколі не збягяў". Хаця, мабыць з дапамогай
магіі і сапраўды можна было пабудаваць стоадсоткава ідэальную турму, асабліва, калі
ў цябе была палачка, а ў вязней --- не. Лепшай стратэгіяй будзе наогул туды 
ніколі не трапляць.

--- Ну то добра, --- сказаў Гары. --- Прыйдзецца вам паверыць на слова, ---
ён уздыхнуў, пачасаў патыліцу. --- Або, мабыць Цёмны Лорд не зусім памёр у 
тую ноч. Толькі часткова пашкодзіўся. А ягоны дух вандруе па свеце, ён 
прыходзіць на людзей ва сне і шэпча ім усялякія жахі, намагаючыся намацаць 
шчыліну ў свет жывых, і, згодна са старажытным прароцтвам, мы з ім звязяны
ў смяротнай дуэлі, дзе пераможца прайграе, а прайграўшы --- выйграе...

Галава МакГонагал тарганулася па баках, яе вочы мітусліва агледзелі вуліцу,
шукаючы сведкаў.

--- Я проста жартую, прафесар, --- сказаў Гары раздражнённа. Хосспадзі, чаго ж
яна заўсёды такая сур'ёзная...

Але дзесьці ў раёне Гарынага страўніка абуджалася дзіўнае адчуванне падзення.

МакГонагал глядзела на яго са спакойным выразам на твары. Вельмі, 
\emph{вельмі} спакойным выразам. А потым на ім з'явілася ўсмешка.

--- Ну канешне, містэр Потэр.

\emph{Ох, чорт.}

Калі Гары паспрабаваў бы выразіць словамі ту іскру азарэння, якая за імгненне
выбліснула ў ягонай галаве, то гэта гучала бы неяк так: "Калі бы мне прыйшлося
параўнаць імавернасць такіх паводзін МакГонагал з размеркаваннем 
імавернаці яе звычайных рэакцый на мае дрэнныя жарты, то я бы прыйшоў да высновы,
што такая рэакцыя была не натуральная, а хутчэй, --- вынікам вельмі моцнага і
акуратнага самакантролю, што з'яўляецца значным доказам таго, што яна 
намагаецца нешта ад мяне схаваць."

Але ў рэальнасці Гары толькі і паспеў падумаць "\emph{Ох, чорт.}"

Ён таксама паглядзеў па баках. Нікога.

--- Ён \emph{не} памер, ці не так? --- уздыхнуў Гары.

--- Містэр Потэр!..

--- Цёмны Лорд жывы. Ну канешне, ён жывы. Як наіўна аптымістычным з майго боку было спадзявацца
на гэта. Не магу ўявіць, што на мяне нашло... мабыць я крыху з'ехаў з глузду.
Паверыць проста таму, што нехта сказаў, што яго цела было знойдзена апаленае 
ўшчэнт! Як наогул можна прыняць такое ў якасці доказа смерці? Дакладна, мне 
яшчэ вучыцца і вучыцца сапраўднаму мастацству песімізма.

--- Містэр Потэр...

--- Прынамсі скажыце, што няма ніякага прароцтва... --- Але МакГонагал усё таксама дэманстравала яму гэтую застыўшую ўсмешку. --- Да ладна, \emph{вы сур'ёзна???}

--- Містэр Потэр, вам не варта выдумляць розныя жахі, каб потым пра іх 
хвалявацца...

--- Сур'ёзна, вы проста хаціце ігнараваць мяне? Уявіце сабе маю рэакцыю потым, калі
раптам высветліцца, што ўсё ж такі варта было хвалявацца.

Усмешка МакГонагал знікла.

Гары стомлена згорбіўся.

--- У мяне тут цэлы магічны свет, які патрабуе даследавання. На \emph{гэта} ў мяне няма часу.

Яны абодва памаўчалі, пакуль нехта ў аранжавай мантыі павольна прайшоў міма.
МакГонагал незаўважна сачыла за ім вачыма. Гары прыкусіў губу, і калі нехта 
стаяў бы побач, то заўважыў бы кроплю крыві.

Калі мінак адышоў дастаткова далёка, Гары прамармытаў ціха:

--- Вы раскажаце мне праўду зараз, прафесар? І не спрабуйце аджартавацца, 
я не зусім дурань.

--- Вам усяго адзінаццаць, містэр Поттэр! --- сказала яна громкім шэптам.

--- І таму я --- не чалавек. Выбачайце, на секунду я забыўся.

--- Гэта жудасныя і важныя рэчы! Гэта \emph{сакрэт,} містэр Потэр!
Гэта катастрофа, што вы, яшчэ дзіця, ужо ведаеце нават гэта! Вы павінны не 
расказваць пра гэта нікому, чуеце мяне? Абсалютна нікому!

Як часам здаралася, калі ў Гары атрымлівалася \emph{дастаткова} ўззлавацца, яго 
кроў стала халоднай, замест гарачай, і жахлівая цёмная яснасць апанавала ягоны
розум, хутка малюючы мапу магчымых тактык, і ацэньваючы іх наступствы з
жорсткім рэалізмам.

\emph{Указаць на твае права ведаць: правал. У вачах МакГонагал, адзінаццацілеткі
не маюць права ведаць.}

\emph{Прыгразіць, што вы больш не будзеце сябрамі: правал. Яна не цэніць твае
сяброўства так значна.}

\emph{Адзначыць пагрозу, якую павялічвае твае няведанне: правал. Усе іхнія планы
заснаваны на тваім няведанні. Пэўная нязручнасць змены падыходу больш непрыемная,
чым нявызначаная верагоднасць тваёй шкоды.}

\emph{Любыя агрументы да справядлівасці і розуму праваляцца. Трэба знайсці нешта, 
што ёй патрэбна, або чаго яна баіцца...}

Гхм...

--- Тады, прафесар МакГонагал, --- сказаў Гары нізкім ледзяным тонам, --- 
мне здаецца, што я валодаю даволі каштоўнымі звесткамі.
Вы можаце, калі хочаце, расказаць мне праўду, усю праўда, і ў адказ 
я буду захоўваць ваш сакрэт. Або вы можаце пасправабаць трымаць мяне ў няведанні,
і выкарыстоўваць, як пешку, і ў такім выпадку, я нічога вам не вінны.

МакГонагал уздрыгнула. Яе вочы бліснулі, і яна амаль прашыпела: 

--- Як вы смееце!

--- Як \emph{вы} смееце! --- прашыпеў ён у адказ.

--- Вы мяне шантажыруеце?

Вусны Гары скрывіліся.

--- Я \emph{прапаную} вам паслугу. Я даю вам шанец зберагчы ваш каштоўны сакрэт. Калі 
вы адмовіцеся, у мяне будзе поўнае права шукаць інфармацыю ў іншых месцах, не
каб вам назаляць, а таму што я \emph{павінны ведаць!}
Пакіньце свой бессэнсоўны гнеў! З чаго вы вырашылі, што я павінен слухацца 
дарослых, якія не карыстаюцца розумам? 
\emph{Гляньце з майго пункта гледжання! Як \scream{вы} бы сябе адчувалі?}

Гары глядзеў на МакГонагал, слухаў яе хрыплае дыханне. Ён падумаў, што надышоў 
добры момант, каб знізіць ціск, і дазволіць ён крыху абдумаць усё самастойна.

--- Вам не абавяскова прымаць рашэнне зараз, --- сказаў ён нармальным тонам. --- Я 
магу зразумець, калі вам трэба падумаць над маёй прапановай... але дазвольце вас
папярэдзіць, --- дадаў ён больш жорстка. --- Не спрабуйце на мне Забывальны Заклён.
Я прыдумаў адмысловае паведамленне для сябе, якое я даслаў сабе некаторы час таму.
Калі я атрымаю яго, і не буду памятаць, як даслаў яго... --- Гары дазволіў словам
шматзначна павіснуць у паветры.

Выраз МакГонагал змяніўся.

--- Я... не думала пра Забывальны Заклён, містэр Потэр... але з якой прычыны 
вы наогул прыдумалі сабе нейкае паведамленне, нават калі яшчэ не ведалі пра...

--- Ведаеце, у маглаўскай фантастыцы можна шмат чаму навучыцца. Я падумаў,
\emph{ну, на ўсялякі выпадак...} І не, я вам яго не скажу, я ж не дурань. 

--- Я не збіралася пытаць, --- сказала МакГонагал. Яна неяк паменшылася ростам, 
і стала выглядаць старэй, і вельмі, вельмі стомленай. --- Гэта быў цяжкі дзень,
містэр Потэр. Мы можам набыць вам куфар і паслаць вас дадому? Я верю, што вы
не будзеце ні з кім абмяркоўваць тое, аб чым даведаліся сёння, пакуль я буду думаць.
Улічвайце, што акрамя нас, толькі два чалавека ведаюць праўду --- дырэктар
Альбус Дамблдор і прафесар Северус Снэйп.

Цікава, новая інфармацыя. Было падобна на мірную прапанову. Гары згодна кіўнуў, 
павярнуўся да яе бокам, і яны закрочылі наперад.

--- Гхм, цяпер мне трэба будзе знайсці спосаб забойства бессмяротнага Цёмнага
Лорда, --- сказаў Гары і расчаравана ўздыхнуў. --- Як бы я хацеў, каб вы сказалі
мне \emph{да} пачатку нашага шопінгу.

\later

Крама куфараў была аздоблена значна багацей за астатнія, дзе быў Гары: шторы 
былі пышней і вышыты тонкімі ўзорамі, падлога і сцены былі пакрытыя інкруставаным
і паліраваным дрэвам, куфары займалі ганаровыя месцы на пастаментах са слановай
косці. Прадавец, апрануты ў мантыю, годную для Люцыуса Малфоя, размаўляў
з вытанчанай, алеістай ветлівасцю і з МакГонагал, і з Гары.

Гары пазадаваў свае пытанні, і абраў цяжкі на выгляд куфар, без упрыгожванняў,
але зролены з моцнага і цёплага дрэва, на якім быў выразаны цмок-ахоўнік, які 
сачыў вачыма за любым, хто падыходзіў блізка. Куфар быў зачараваны на лёгкасць,
мог сціскацца па камандзе, і мог выпускаць кароткія кіпцюрыстыя тэнтаклі, каб
клэпаць за гаспадаром. На кожным з чатырох бакоў было па дзве шуфлядкі, 
унутры кожная была глыбінёю з вышыню куфара. Крышка куфара зачынялася на чатыры замкі,
кожны з якіх адчыняў доступ да паасобнай прасторы ўнутры. І яшчэ была --- гэта 
была самая крутая фіча --- ручка знізу, якая адчыняла люк, пад якім была
лесвіца --- яна вяла ўніз, у маленькі, асвечаны пакой, які, па прыкдках Гары,
мог змясціць каля дванаццаці кніжных шафаў.   

Калі ў іх такія чэмаданы, думаў Гары, то не ведаю, навошта камусьці турбаваць
сябе забудовай дамоў.

Сто восем залатых галеонаў. Такі быў кошт добрага куфара, крыху б/к. Пры курсе ў
каля пяцідзесяці брытанскіх фунтаў за галеон, гэтага бы хапіла на невялікую машыну.
Гэта было значна даражэй за ўсё, што Гары калісьці купляў у жыцці, складзенае 
разам.

Дзевяноста сем. Столькі галеонаў заставалася ў мяшку золата, якому Гары дазволілі
ўзяць у банку.

У МакГонагал быў вельмі засмучаны выгляд. Пасля доўнага дня шопінгу яна не спытала
Гары, колькі ў яго засталося, калі прадавец сказаў кошт --- што значыла, што яна
была здольная да арыфметыкі без алоўка і бумагі. На будучыню, сказаў сабе Гары, 
\emph{не адукаваны} --- не тое ж самае, што \emph{дурны.}

--- Мне жаль, малады чалавек, --- сказала МакГонагал. --- Гэта цалкам мая віна.
Я бы прапанавала вярнуцца ў Грынготс, але банк ужо зачынены.

Гары зрабіў глыбокі ўдых. Дзеля таго, што ён запланаваў, яму было трэба крыху ўззавацца,
інакш яму не здолець. \emph{Яна мяне не слухала,} --- падумаў ён сабе, --- \emph{Я
бы ўзяў больш золата, але яна мяне не слухала...} Ён прыслухваўся да лютасці ўнутры,
клікаў яе. Ён уявіў сабе, \emph{якім ён павінны стаць,} і надзеў гэтую асобу на сябе,
як мантыю. Канцэнтруючы ўсю сваю моц на тым, каб схіліць МакГонагал на свой бок, 
ён загаварыў.

--- Дазвольце дапусціць, --- сказаў ён. --- Вы пакідалі сабе прастору для манеўра.
Вы думалі, што сотня галеонаў будзе больш, чым дастаткова, і таму вы не турбаваліся
папярэдзіць мяне, калі сума зменшылася да дзевяноста сямі?

МакГонагал заплюшчыла вочы ў пакоры. 

--- Так.

--- Я чакаў гэтага, прафесар. Я ведаў, што такое можа здарыцца. Існуюць даследаванні,
якія дэманструюць, што такое часта атрымліваецца, калі людзі думаюць, што пакідаюць
сабе прастору для манеўра. Я бы на вашам месцы ўзяў дзвесце галеонаў, на ўсякі выпадак.
Там было дастаткова грошаў у тым сховішчы, і астатак заўсёды можна вярнуць. Але вы не
дазволілі мне ўзяць больш. І я ведаў, што няма ніякага сэнсу пытаць. 
Я ведаў, што будзеце раздражняцца, або нават зліцца, калі я спытаю. Ці не так?

--- Я думаю, што павінна згадзіцца гэтым разам, --- сказала МакГонагал, 
--- вы правы, --- у яе голасе былі прабачальныя ноткі, але і разам нейкія рэзкавватыя, быццам Гары павінен 
быў адзначыць, які вялікі, вялізны гонар ён зараз атрымліваў, калі \emph{
прафесар МакГонагал} прасіла ў яго прабачэння. --- Але, малады чалавек...


--- Менавітая такія рэчы і ёсць прычына, чаму ў мяне праблемы 
з даверам дарослым, --- нейкім чынам ён трымаў голас роўным. --- Бо 
дарослыя злуюцца, калі ты нават \emph{паспрабуеш} размаўляць з імі
лагічна. Для іх гэта нахабства, гэта выклік вышэйшаму па статусу ў тваім
племені. Таму, калі мне трэба зрабіць нешта \emph{сапраўды важнае}, 
я бы не змог вам давяраць. Нават калі вы мяне будзеце ўважліва слухаць
і выказваць клопат --- бо гэта таксама частка ролі дарослых, --- вы ніколі
не зменіце свае планы і не зробіце нешта інашк проста таму, што я запытаў.

Прадавец назіраў за імі з непрыхаваным захапленнем. 

--- Я разумею ваш пункт гледжання, --- памаўчаўшы, адказала МакГонагал. ---
Калі я часам выглядаю занадта строгай, помніце, што я працую дэканам Грыфіндора
--- як мне падаецца --- некалькі тысяч гадоў.

Гары кіўнуў і працягваў:

--- Д-давайце ўявім, што ў мяне ёсць спосаб атрымаць некалькі галеонаў са 
сховішча, без неаходнасці вяртацца ў банк, але дзеля гэтага мне 
прыйдзецца выйсці з ролі паслухмянага дзіця. Ці здолееце вы ўбачыць карысць,
нават калі дзеля гэтага вам таксама прыйдзецца выйсці з ролі суровага 
дарослага? Ці магу я давяраць вам у гэтым?

--- \emph{Што?} --- сказала МакГонагал.

--- Іншымі словамі, калі я бы мог памяняць мінулае так, што ў банку мы бы 
ўзялі дастаткова грошаў, ці не пайшло яно на карысць, нават, калі рэтраспектыўна 
высветліцца, што дзіця нахабна не паслухалася дарослых загадаў?

--- Ну... напэўна... --- адказала яна, выглядаючы здзіўленай.

Гары дастаў махляскін, і сказаў:

--- Адзінаццаць запасных галеонаў, калі ласка, --- і манеты прыгнулі ў яго руку.

Вочы МакГонагал выбліснулі абурэннем.

--- \emph{Дзе вы іх ўзялі?}

--- У маім сховішчы, прафесар, як, я і сказаў.

--- \emph{Як?}

--- Магія.

--- Гэта не адказ! --- крыкнула МакГонагал, міргнула пару разоў, і не знайшла, 
чым працягнуць.

--- Чаму не? Я проста \emph{павінны} выдумаць, што я эксперыментальна высветліў
сапраўдны сакрэт махляскіна, і што ён можа дастаўляць рэчы скуль заўгодна,
калі правільна яго запытаць... але не буду хлусіць.
Калі я ўпаў на кучу манет, я схапіў некалькі ў кулак,
і схаваў у кішэні. Любы дурань разумее, што грошы --- рэч, якая можа 
спатрэбіцца хутка і без папярэджання. Зараз вы разумееце, што я чакаў нешта такога з самога пачатку? Ці будзеце вы злавацца на мяне, што пакрыўдзіў ваш 
аўтарытэт? Або будзеце рады, што дзень скончыцца поспехам нашай важнай міссіі?

Вочы прадаўца былі памерам са сподак.

МакГонагал некаторы час стаяла моўчкі.

--- Дысцыпліна ў Хогвартс \emph{павінна} падтрымлівацца, --- сказала яна пасля амаль
цэлай хвіліны. --- Дзеля \emph{ўсіх} студэнтаў. 
І гэта \emph{павінна} уключаць ветлівасць і паслухмянасць з вашага боку да
\emph{ўсіх} настаўнікаў.

Гары нахіліў галаву.

--- Я разумею, прафесар, --- ён падумаў, наколькі значным падавалася 
падтрыманне дысцыпліны, тым, хто быў на вяршыні піраміды... але вырашыў,
што будзе разумней не высветляць гэтае пытанне тут і зараз.

--- Тады... я віншую вас з вашай падрыхтаванасцю.

Гары хацеў заараць ад шчасця, або страціць прытомнасць, або ўсё адразу. Першы
раз у жыцці яго прамова спрацавала на дарослым. Першы раз у жыцці, калі \emph{любая}
з яго прамоў спрацавалі на \emph{кімсьці.} Магчыма таму, што першы раз у жыцці
ён меў сапраўдныя рычагі ціску на гэтага дарослага, аднак усё ж...

Мінерва МакГонагал, +1 бал.

Гары ўзяў мяшок з золатам, адзінаццаць дадатковых манет, і сунуў іх МакГонагал
у рукі. 

--- Калі ласка, расплаціцеся за мяне, мне трэба ў туалет... скажыце, дзе?...

Прадавец, зноў пачцівы, паказаў  у глыбіню залы на дзверы з залатой ручкай. Адыходзячы,
Гары паспеў пачуць, як прадавец спытаў МакГонагал сваім алейным голасам:

--- Магу я пацікавіцца, мадам МакГонагал, хто гэты малады чалавек? Я так разумею,
слізерын, не менш як за трэці курс, і з выбітнага роду... але я не здолеў 
распазнаць...

Дзверы туалета, бахнуўшы за спінай Гары, перарвалі яго. Гары, намацаўшы замок,
і зачыніўшы яго, паваліўся на падлогу, як мяшок з бульбай. Усё ягонае цела 
было пакрыта ліпкім потам, які прапітаў маглаўскае адзенне ўшчэнт, але, прынамсі,
не здолеў пранікнуць праз мантыю. Ён нахіліўся над аздобленым золатам унітазам,
і яго знудзіла.

\later

Яны зноў стаялі ў двары Дзіравага Катла, у маленькім, закінутым і занесеным старым
лісцем пераходзе паміж ўсім маглаўскім светам і магічнай Дыягон-алеяй,
месцам вельмі
ізаляванай эканомікі. На тым баку Гары быў павінен знайсці тэлефон-аўтамат
і патэлефанаваць бацьку. Аднак, яму можна было не хвалявацца, што яго багаж
могуць скрасці: гэта быў важны магічны артэфакт, і маглы яго проста не 
заўважалі. Вось такое можна было атрымаць у магічным свеце, калі ты быў 
гатовы выкласць суму, як за бэкашную машыну. Гары было цікава, ці заўважыць 
яго бацька куфар, калі яму паказаць. 

--- Тут нашыя шляхі разыходзяцца на пэўны час, --- сказала прафесар МакГонагал. 
Яна патраласа галавой. --- Гэта быў найдзіўнейшы дзень у маім жыцці за... шмат 
гадоў. Мабыць, з таго дня, калі я даведалася, што дзіця адолела Самі-Ведаеце-Каго.
Магчыма, гэта быў апошні нармальны дзень гэтага свету.

О, быццам ёй ёсць аб чым скардзіцца. \emph{Думаеш, твой дзень быў сюррэальны? 
Паспрабуй мой.}

--- Мы мяне сёння вельмі ўразілі, --- сказаў Гары. --- Варта было вас падзячыць 
уголас, бо я налічаў балы ў галаве, і ўсё такое...

--- Дзякуй, містэр Потэр. Калі вы былі бы ўжо размеркаваны, я бы зняла з 
вашага факультэта столькі балаў, што вашыя ўнукі працягвалі бы праігрываць
Кубак Хогвартс. 

--- Дзякуй вам, Мінерва, --- напэўна, было надта рана, каб клікаць яе
"Міні".

Гэтая жанчына было найбольш разумным дарослым з тых, каго сустракаў Гары,
нягледзячы на адсутнасць у яе навуковай адукацыі. Гары нават разважаў аб тым,
каб прапанаваць ён другое месца ў іерархіі групы, якую ён створыць дзеля
барацьбы з Цёмным Лордам, хаця яму хапіла розуму не казаць гэта ўголас.
\emph{Як бы назваць гэтую групу...? Пажыральнікі Пажыральнікаў Смерці?}

--- Мы зноў сустрэнемся вельмі хутка, калі пачнуцца заняткі, --- сказала
МакГонагал. --- І, містэр Потэр, ваша палачка...

--- Я ведаю, што вы скажаце, --- сказаў Гары. Ён дастаў сваю дарагую чароўную
палачку, і з уколам болю дзесцьі ўглыбыні свядомасці, працягнуў яе МакГонагал
ручкай наперад. --- Вазьміце яе. Я і не планаваў нічога, але я не хачу, каб
вы снілі кашмары, што мой дом пускаецца ў паветра.

МакГонагал адмоўна патрасла галавой. 

--- О, не, містэр Поттэр. Палачка ваша. Я проста хацела папярэдзіць вас не 
карыстаць яе па-за Хогвартс, бо за гэтым сочыць Міністэрства Магіі. Непаўналетнім
забаронена рабіць магію без нагляду.

--- А, --- сказаў Гары і ўсміхнуўся. --- Гэта гучыць як вельмі разумнае правіла.
Я рады, што магічны свет ставіцца сур'ёзна да гэтага пытання.

МакГонагал паглядзела на яго ўважліва. 

--- Вы сур'ёзна?

--- Так, --- сказаў Гары. --- Магія --- рэч небяспечная, і для падобных правілаў
ёсць важныя прычыны. Вы ж памятаеце, што я не дурань?

--- Малаверагодна, што я калісьці гэта забуду. Дзякуй, Гары Потэр. Мне будзе
спакайней. Да пабачэння.

Гары павярнуўся да Дзіравага Катла. Калі ён паклаў руку на ручку дзвярэй,
ён пачуў за спінай:

--- Герміёна Грэнджэр.

--- Што? --- спытаў Гары, трымаючы ручку.

--- Шукайце першакурсніцу па імі Герміёна Грэнджэр у цягніку на Хогвартс.

--- Хто яна?

Адказу не было, і калі Гары паглядзеў назад, МакГонагал знікла.


\subsection{Паследкі}

Дырэктар Дамблдор нахіліўся над сваім працоўным сталом. Яго пабліскваючыя вочы
разгледжвалі МакГонагал.

--- Ну што, Мінерва, як табе Гары?

Яна адчыліна рот. Потым зачыніла. Потым адчыніла яшчэ раз. Але так нічога і не
сказала.

--- Зразумела, --- змрочна сказаў Дамблдор. --- Дзякуй за справаздачу, Мінерва.
Ты можаш ісці.
