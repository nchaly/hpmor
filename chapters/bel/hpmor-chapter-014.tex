\chapter{Невядомае і неспазнанае}

\begin{chapterOpeningQuote}
Існуюць загадкавыя пытанні, але загадкавы адказ --- супярэчнасць па вызначэнню.
\end{chapterOpeningQuote}

\lettrine[lines=1,lraise=-0.1]{-У}{вайдзіце,} --- сказаў прыглушаны голас прафесара МакГонагал.

Гары ўвайшоў.

Кабінет намесніцы дырэкрата быў проста і добра ўладкаваны. Уздоўж бліжэйшай да
пісьмовага стала сцяны стаяла картатэка з неймаверным лікам ячэйкаў 
усіх магчымых памераў, з якіх тырчалі канцы пергаментных скруткаў, і было зразумела,
што прафесар МакГонагал даклада ведала, што ў якой ячэйцы ляжала, нават калі больш 
ніхто гэтага не разумеў. Стол, калі не лічыць разгарнутага перад ёю пергамента, 
быў абсалютна пусты. 
За яе спінаю былі дзверы, замкнутыя на некалькі вісячых замкоў.

Прафесар МакГонагал падняла галаву ад сваёй працы, з крыху здзіўленым выразам. 
Яе вочы пашырыліся, калі яна ўбачыла Гары, быццам чакаючы непрыемных навінаў.

--- Містэр Потэр? --- сказала прафесар МакГонагал. --- Чаго вы хацелі?

Гарын розум увайшоў у ступар. Атрымаўшы інструкцыю ад Гульні ісці сюды, ён чакаў,
што ёй было чым з ім падзяліцца...

--- Містэр Потэр? --- паўтарыла яна, пачынаючы выглядаць раздражнёна.

На шчасце, Гарын панікуючы розум успомніў, што ён як раз і планаваў абмеркаваць нешта 
з прафесарам МакГонагал. Нешта важнае і вартае яе часу.

--- Эммм... --- сказаў ён, --- ёсць нейкі заклён, каб ніхто не мог нас падслухаць?..

Прафесар МакГонагал паднялася са стула, шчыльна зачыніла дзверы, дастала сваю 
палачку і пачала каставаць заклёны.

У гэты момант Гары зразумеў, што гэта была неацэнная і магчыма незаменная магчымасць
прапанаваць прафесару МакГонагал выпіць Жарта-Колы, і --- ён не мог паверыць у тое,
што ён сур'ёзна пра гэта думае, --- і што ўсё ўрэшце скончыцца добра, і 
плямы знікнуць праз некалькі секунд... --- і ён рашуча сказаў гэтай сваёй частцы 
\emph{заткнуцца.}

Частка заткнулася, і Гары пачаў мыслена арганізоўваць, што ён скажа. Ён не планаваў
гэтую размову \emph{настолькі} хутка, але раз прыйшоў...

Прафесар МакГонагал скончыла серыю заклёнаў нейкай знайчна гучнейшай і старажытнейшай
латынню, і села на месца.

--- Гатова, --- сказала яна ціха. --- Ніхто не слухае, --- але яе 
выраз быў даволі напружаным.

\emph{А, зразумела! Яна чакае, што я буду шантажаваць яе на конт інфармацыі пра 
прароцтва.}

Эмм... ну Гары заўсёды мог вярнуцца да гэтай тэмы ў будучыні.

--- Гэта на конт Інцыдэнту з Размеркальным Капелюшом, --- сказаў Гары. 

Прафесар МакГонагал міргнула.

--- Эмм... --- працягваў ён, --- я думаю, што на яго накладзены дадатковы заклён,
нешта, аб чым сам Капялюш не ведае, нешта, што трыгерыцца, калі Капялюш кажа 
"Слізэрын". Я пачуў звестку, якую --- я даволі ўпэўнены ---
студэнты Рэйвенкло не павінны чуць. Гэта адбылося ў тый самы момант, калі 
я зняў Капялюш і адчуў разрыў сувязі. Яно гучала як неёкае шыпенне, --- МакГонагал тут 
гучна ўдыхнула, --- і яно казала \parsel{"Салют ад Слізэрына слізэрыну: калі хочаш раскрыць
мае сакрэты, шукай майго аспіда."}

Прафесар МакГонагал сядзела, раскрыўшы рот і ўтаропіўшыся ў Гары, быццам ён адрасціў 
сабе другую галаву.

--- І тады... --- сказала яна павольна, быццам не верыла, што словы вылятаюць з 
яе ўласнага рота, --- вы адразу вырашылі прыйсці да мяне і расказаць мне пра гэта?

--- Ну... так, канешне, --- сказаў Гары. Не было неабходнасці дзяліцца, колькі 
насамрэч гэтая думка заняла ў яго часу. --- У адрозненне, скажам, ад таго, каб 
паспрабаваць высветліць самому, або падзяліцца з іншымі студэнтамі.

--- Зразумела... --- сказала МакГонагал. --- І калі, магчыма, вы знайшлі бы ўваход 
у легендарную Таемную Залу Салазара Слізэрына, уваход, які вы і толькі вы маглі бы 
адчыніць...

--- Я бы яго зачыніў, і адразу абвясціў бы вас, прафесар, каб вы маглі сабраць 
каманду дасведчаных магаў-архелогаў, --- сказаў Гары хутка. --- Пасля я адчыніў бы
ўваход, і яны бы вельмі асцярожна праверылі бы, ці няма там нейкай пагрозы. Я бы мог
зайці туды пазней, проста агледзецца, або адчыніць нешта яшчэ, калі спратрэбіцца.
Але толькі пасля таго, як яны ўсё зафіксуюць, сфатаграфуюць і запішуць, як усё выглядала,
да таго, як турысты пачнуць брудзіць бескаштоўнае гістарычнае месца.

Прафесар МакГонагал глядзела на яго так, быццам гэта ён зараз пераўтварыўся ў котку.

--- Гэта само сабой разумецца, калі ты не грыфіндор, --- ветліва сказаў Гары.

--- Я думаю, --- сказала прафесар МакГонагал сціснутым голасам, --- што вы 
пераацэньваеце ўзровень разважлівасці ў Хогвартс, містэр Потэр.

--- Любы сярэдні хафлпаф сказаў бы тое ж самае.

МакГонагал задумалася.

--- Магчыма.

--- Размеркавальны Капялюш прапанаваў мне Хафлпаф.

Яна міргнула, быццам не веруючы сваім вушам. 

--- Праўда?

--- Праўда.

--- Містэр Потэр, --- сказала МакГонагал, яе голас стаў ніжэй, --- пяцьдзесят 
гадоў таму апошні раз памёр нехта са студэнтаў, і я зараз упэўнена, што пяцьдзесят 
гадоў таму нехта таксама чуў тыя словы.

Гары прабраў халадок.

--- Тады я буду вельмі асцярожны, і не буду пачынаць ніякія дзеянні на гэты конт без 
вашага зацвярджэння, прафесар, --- ён зрабіў паузу. --- І магу я прапанаваць,
каб вы сабралі каманду лепшых магаў, і знялі той заклён з Капелюша? Або прынамсі
наклалі на яго іншы заклён, напрыклад, сцішальны, які актывуецца на кароткі час, калі
Капялюш здымаюць з галавы студэнта --- можа спрацаваць як заглушка... Вось, і ніякіх
больш мёртвых студэнтаў, --- дадаў Гары задаволена.

Прафесар МакГонагал выглядала яшчэ больш уражанай, калі такое было магчыма.

--- Я не магу налічыць вам балы за гэтую справу, інакш прыйдзецца адразу аддаць Райвенкло
Келіх Факультэтаў гэтага году. 

--- Эммм... --- сказаў Гары, --- я бы адмовіўся ад  \emph{такой} колькасці балаў.

--- Чаму? --- МакГонагал глянула на яго дзіўна.

Гары з цяжкасцю мог сказаць гэтую ідэю ўголас.

--- Таму... што гэта будзе вельмі сумна, разумееце? Быццам, як... у маглаўскай школе,
кожны раз, калі ў нас былі групавыя праекты, я проста браў і рабіў усё сам, бо астатнія
мяне толькі марудзілі. Мне, канешне, хацелася бы заработаць балаў, а больш за астатніх ---
асабліва, але, калі я адзін зраблю столькі, каб адразу перамагчы... гэта быццам 
ташчыць увесь факультэт на сваёй спіне, і гэта --- вельмі сумна.

--- Зразумела, --- сказала, вагаючыся, МакГонагал. Было бачна, што такая думка ён 
ніколі дагэтуль не прыходзіла. --- Дапусцім, калі я вам дам пяцьдзесят балаў?

Гары зноў адмоўна пакачаў галавой. 

--- Гэта несправядліва да астатніх --- бо я атрымаю балы за асабістую рэч, такую, якую
больш ніхто астатні не можа паўтарыць. Вы можаце ўявіць сабе, што налічаце пяцьдзесят 
балаў Тэры Буту за тое, што ён пачуў таямнічы шэпт падчас размеркавання? Зусім 
не справядліва...

--- Бачна, чаму Капялюш прапанаваў вам Хафлпаф, --- сказала МакГонагал. Яна глядзела 
на яго з незвычайнай павагай.

Гары ўздыхнуў. Ён праўда думаў, што не варты Хафлпафа, і што Капялюш проста 
хацеў размеркаваць яго куды ўгодна, толькі не ў Рэйвенкло.

МакГонагал усміхнулася.

--- Скажам, калі я вам дам \emph{дзесяць} балаў?

--- А калі вас нехта спытае, як вы патлумачыце, скуль узяліся тыя дзесяць балаў?
Калі слізэрыны --- і я маю на ўвазе не мясцовых студэнтаў --- раптам дазнаюцца,
што нейкі важны заклён, зроблены самім Слізэрынам, быў зняты з Капелюша, і што 
ў гэтым быў замешаны я... Думаю, лепшай стратэгіяй у гэтай справе будзе 
абсалютная сакрэтнасць. І мне не трэба асаблівай падзякі, мэм, зрабіць 
добрую справу для мяне --- лепшая ўзнагарода. 

--- Як скажаце, --- сказала прафесар МакГонагал, --- але ў мяне і сапраўды ёсць нешта
асаблівае для вас. Я бачу, што сільна памылялася ў вас, містэр Потэр. Чакайце тут.

Яна паднялася, падышла да замкнутых дзярэй, махнула палачкай, і нешта падобнае на 
туманны вэлюм апанавала яе. Гары не бачыў і не чуў нічога, што там адбывалася.
Праз некалькі хвілін вэлюм знік, і за ім стаяла МакГонагал, а дзверы былі замкнутыя 
зноў, быццам ніколі не адчыняліся.

Яна трымала ў руцэ кар\'алі, тонкі залаты ланцужок, на якім вісела срэбнае кальцо,
унутры якога быў малюсенечкі пясочны гадзіннік. У другой руцэ ў яе была нейкая папера.

--- Гэта вам, --- сказала яна.

Ваў! Ён атрымае магічны прадмет за тое, што выканаў квест! Верагодна, гэты патэрн,
калі некалькі разоў адмаўляешся ад грошаў, і атрымліваеш важны артэфакт, працаваў 
не толькі ў камп'ютэрных гульнях.

Гары ўсміхнуўся і прыняў каралі. 

--- Што гэта такое?

--- Містэр Потэр, гэта прадмет, які часова выдаецца студэнтам, якія паказалі
сваю адказнасць, каб дапамагчы ім з занадта цяжкімі заняткамі, --- МакГонагал завагалася,
быццам успомніла нешта яшчэ. --- Я падкрэсліваю, містэр Потэр, сапраўдная прырода гэтага 
прадмета --- \emph{сакрэт}, і вы павінны нікому не расказваць пра яго, і ніхто не 
павінен бачыць, як вы яго карыстаеце. Калі гэтыя ўмовы недапушчальныя для вас, можаце
вярнуць яго зараз.

--- Я магу захоўваць сакрэты, --- сказаў Гары. --- Так што ён робіць?

--- Для астатніх вучняў, гэта --- Скрутнасцішны бардальён, і яго карыстаюць для 
лячэння рэдкай і незаразнай магічнай хваробы пад назвай "ідыядубліпатыя",
або простымі словамі "расстройства спантаннай дублікацыі".
Насіце яго пад адзеннем, і хаця не варта яго паказваць, рабіць страшную 
тайну вакол яго таксама не трэба. Бардальёны нікому не цікавыя. Вы мяне разуееце, 
містэр Потэр?

Гары кіўнуў і ягоная ўсмешка пашырылася. Ён адчуваў почырк \emph{здольнага}
слізэрына. 

--- А што ён сапраўды робіць?

--- Гэта часаварот. Кожны абарот гадзінніка вяртае вас на адну гадзіну назад. 
Таму вы будзеце кожны дзень адматваць дзве гадзіны, і будзеце спаць, як усё.

Гаць, якой Гары замацаваў свае адчуванне недаверу, нарэшце прарвала:

\emph{Праблемы cа cном? Атрымай машыну часу!}

\emph{Праблемы Са Сном? Атрымай Машыну Часу!}

\emph{ПРАБЛЕМЫ СА СНОМ? АТРЫМАЙ МАШЫНУ ЧАСУ!}

--- М... м... м... мэ... --- Гары трымаў 
медальён у выцягнутай руцэ, як мага далей ад сябе, быццам гэта была бомба. Дакладней,
лепей то была бы бомба. А так... у яго нават не хапала слоў, каб выказаць 
узровень пагрозы гэтага прыбору. Ён трымаў яго так, быццам гэта была сапраўдная 
машына часу.

\emph{Скажыце, прафесар МакГонагал, а вы ведалі, што звычайная матэрыя, калі яе 
рэверсіраваць у часу, выглядае зусім як антыматэрыя? Дык вось, зараз ведаеце!
А вы ведалі, што калі кілаграм антыматэрыі сустэне кілаграм звычайнай матэрыі,
абудзецца выбух эквівалентам ў 43 мегатоны? А калі мая вага 41 кілаграм, і я раптам
сустрэну рэверсіраванага сябе, у выніку выбуха \shout{ад Шатландыі застанецца 
толькі дымячыйся кратэр?}}

--- Прабачце, --- нарэшце здолеў сказаць Гары, --- але гэта гучыць сапраўды 
\emph{вельмі ВЕЛЬМІ НЕБЯСПЕЧНА!} --- яго голас ледзьве не сарваўся ў віск, але ўсё роўна
ён фізічна не мог бы выдаць гучнасць, адпаведную сур'ёзнасці сітуацыі.

Прафесар МакГонагал глядзела на яго з зацікаўленай цярплівасцю.

--- Я радая бачыць, што вы ставіцеся да гэтай справы так сур'ёзна, містэр Потэр,
але часавароты не  \emph{настолькі} небяспечны. Інакш мы бы не давалі іх вучням.

--- Праўда? --- сказаў Гары. --- Ах-ха-ха... Ну вядома, вы бы не давалі іх вучням,
калі яны былі бы небяспечнымі. Аб \emph{чым} я толькі думаў? То бок, калі я выпадковы чыхну на 
гэты прыбор, ён не пашле мяне ў сярэднявечча, дзе я знянацку пераеду цялегай Гутэнберга,
і адмяню век Асветы? Бо, ведаеце, я так не люблю, калі такое здараецца са мной.

Вусны МакГонагал падрыгвалі, як заўсёды, калі яна намагалася стрымаць усмешку. 
Яна працягнула яму тую іншую паперу, але абедзве рукі Гары былі заняты трыманнем 
часаварота, пакуль уся яго ўвага была накіравана на тое, каб сачыць, ці не 
збіраецца ён правярнуцца.

--- Не хвалюйцеся, --- сказала МакГонагал праз некалькі секунд, калі стала ясна, 
што Гары больш не зробіць ні адзінага руху ў сваём жыцці. --- Нічога падобнага 
немагчыма. Часаварот можа вярнуць вас не болей, чым на адзну гадзіну за раз, і яго 
можна скарыстать толькі шэсць разоў на дзень. 

--- О, добра, вельмі добра. І калі нехта выпадкова ў мяне ўрэжэцца на калідоры,
ён не разаб'ёцца і не зачыніць цалкам замак Хогвартс у бясконцым коле паўторных
чацьвяргоў?

--- Хм... яны \emph{крыху} крохкія... --- сказала МакГонагал. --- І мне падаецца,
я чула аб нейкіх дзіўных рэчах, калі яны разбіваліся. Але нічога пагрозлівага!

--- Магчыма, --- сказаў Гары, калі зноў мог гаварыць, --- вам варта ўсталяваць 
на вашыя машыны часу \emph{ахоўныя абалонкі}, каб не пакідаць крохкае шкло 
адкрытым, каб яны больш \emph{не разбіваліся!}

МакГонагал выглядала ўражанай.

--- Гэта выдатная ідэя, містэр Потэр. Я неадкладна дашлю ліст у міністэрства.

\emph{Вось так, цяпер усё афіцыйна і ратыфікавана ў Парламенце: магічны свет населены
дурнямі.}

--- І прабачце за надта \shout{філасоўскі} тон, --- Гары адчайна намагаўся знізіць
свой голас з узроўню віску, --- хтосьці думаў пра \shout{наступствы} магчымасці
вярнуцца на шэсць гадзін таму, і зрабіць там нешта такое, што  фактычна
\shout{знішчыць усіх далучаных} асоб і заменіць іх \shout{іншымі версіямі...}

--- О, Час немагчыма \emph{змяніць}! --- перарвала прафесар МакГонагал. --- Святыя 
нябёсы, містэр Потэр, вы праўда думаеце, што мы бы дазволілі студэнтам \emph{такое}?
Каб яны пастаянна мянялі свае адзнакі за кантрольныя?

Гары патрэбілася некалькі секунд, каб гэта ўсвядоміць. Яго рукі крыху разняволіліся,
прынамсі яго пальцы перасталі сціскваць ланцужок пабялеўшымі пальцамі. 
Быццам ён трымаў ужо не машыну часу, а проста ядравую баегалоўку.

--- Так... --- сказаў ён павольна, --- нейкім чынам людзі высветлілі, што сусвет
захоўвае сваю цэльнасць і згоднасць, нават з прыжкамі ў часе. І калі я  
ўзаемадзейнічаю з сабой з будучыні, абедзве маі версіі будут бачыць падзею 
аднолькава, нават калі мая будучая версія добра ведае, што для мяне мінулага 
падзея яшчэ не адбылася... --- Гары сціх, бо яму не хапала склонаў, каб выразіць
усё свае тэмпаральныя дзеянні.

--- Думаю, усё правільна, --- сказала прафесар МакГонагал. --- Хаця чараўнікам
\emph{рэкамендуецца} не кантактаваць яўна са сваімі мінулымі копіямі. Напрыклад, 
сустрэўшыся ў калідоры, будучы павінен заплюшчыць вочы, і сасітупіць дарогу. 
Пра гэта ўсё ёсць у правілах, --- яна ўсё яшчэ працягвала яму тую іншую паперу. 

--- А-а-а, правілы!.. А што здарыцца, калі нехта \emph{праігнаруе} правілы? 

МакГонагал падціснула вусны. 

--- Я разумею, гэта можа крыху бянтэжыць.

--- І гэта прыбор не можа стварыць, скажам, парадокс, які знішчыць увесь сусвет?

Яна зноў цярпліва ўсміхнулася.

--- Містэр Потэр, я думаю, я бы помніла, калі нешта такое здарырался бы.

--- \shout{Гэта не вельмі супакойвае! Вы што, ніколі не чулі пра Антропны Прынцып?
Якім ідыётам трэба быць, каб наогул стварыць магію для маніпуляцыі часам?}

Тут прафесар МакГонагал выдала сапраўдны смяшок. Гэты раптоўны гук задавальнення
ніяк не падыходзіў да яе строгага твара. 

--- У вас зноў прыступ на тэму "чалавек не можа стаць коткай", ці не так, містэр Потэр?
Магчыма, вы не хаціце такое чуць, але гэта даволі міла.

--- "Стаць коткай" нават і блізка нельга параўнаць з \emph{гэтым}. Бачыце, да 
гэтага моманту ў мяне было такое цяжка стрыманае адчуванне, што ўвесь 
гэты сусвет --- проста камп'ютэрная сімуляцыя, як у кнізе "Сімулякрон-3", але 
зараз нават гэта тлумачэнне можна выкінуць, бо гэтая 
вашая цацка \emph{НЕ ВЫЛІЧАЛЬНАЯ ПА Т'ЮРЫНГУ!} Машына Т'юрынга можа лёгка сымуляваць
вандроўку ў мінулае, праз вылічэнне новай галіны рэальнасці з таго моманту,
але калі верыць вам, то рэальнасць нейкім самазгодным чынам вылічае сабе 
за адзін праход, карыстуючы інфармацыю, якая... яшчэ... не існуе... 

Раптоўнае ўразуменне ўдарыла Гары, і рух ў яго свядомасці можна было 
параўць ці што з лавінай.

Усё стала зразумела. \emph{Урэшце} усё стала на свае месцы.

--- \shout{Дык вось як працуе Жарта-Кола!} Як проста! Вядома, яна не прымушае нечаканасці 
здарацца, яна выклікае ў цябе \emph{жаданне выпіць} перад тым, як яно само па сабе здарыцца!
Які ж я дурань, мог бы і здагадацца, бо я адчуў позыў выпіць перад другой прадмовай Дамблдора,
адмовіў гэты позыў, і падавіўся ўласнай слінай! Жарта-Кола не выклікае жарт, гэта жарт прымушае 
цябе адпіць Жарта-Колы! Спачатку я проста адзначыў карэляцыю гэтых падзей, і вырашыў, што
Кола павінна быць чыннікам, а жарт --- вынікам, праз абмежаванні іх тэмпаральнага парадку,
бо граф чынніка-выніковых сувязяў павінен быць ацыклічным.
\shout{Але ўсё становіцца на месца, калі дазволіць стрэлкам ісці \emph{назад у мінулае!}}

Новае ўразуменне ўдарыла Гары.

Ён здолеў стрымаць другі сход лавіны, выдаючы толькі ціхія гукі быццам прыдушанае кацяня,
бо ён зразумеў, хто быў аўтарам запіскі на ягоным прыгалоўі сённяшняй раніцай.

Вочы прафесара МакГонагал асвяціліся.

--- Пасля выпуску, містэр Потэр, а магчыма і раней, вам праўда  \emph{варта}
правесці курс некаторых гэтых вашых маглаўскіх тэорыях, яны гучаць проста цудоўна,
нават калі яны ўсе і бессэнсоўныя.

--- Бглеа-а-а....

Прафесар МакГонагал дадала яшчэ крыху хвальбы, загадала яшчэ некалькі абяцанняў, на 
якія Гары проста кіўнуў, сказала нешта пра тое, каб ён не размаўлял са змеямі там,
дзе яго нехта можа пачуць, нагадала яму прачытаць уважліва збор правілаў, і потым 
Гары нейкім чынам апынуўся ў калідоры, дзверы за яго спінай былі шчыльна зачынены. 

--- Гррааарр-бугр-маагррр-др... ---- сказаў Гары.

Ну, а што вы хацелі ад чалавека з выбухнуўшым мозгам?

Не у апошнюю чаргу яго ўражваў факт, што без гэтага пранка над сабою,
ён мог так ніколі і не атрымаць часаварот.

Або яна ўсё роўна дала бы яго пазней, калі Гары бы сабраўся спытаць яе на конт сродка 
ад расстройства сну? Або калі ён бы расказаў пра абвестку ад Капелюша? І магчыма,
тады ён бы пажадаў пранкануць сябе самога, каб атырмаць часаварот нават раней? 
І адзіная самазгодная ветка рэальнаці была тая, дзе ён распачаў пранк да свайго 
абуджэння?..

Гары ўпершыню ў сваім жыцці думаў над пытаннем, якое было літаральна \emph{непазнаваемым}.
Бо калі ягоны мозг складаўся толькі з нейронаў, якія ішлі наперад у часе, 
то мозг не меў фізічнай магчымасці, каб зрабіць хацяб адзіную аперацыю, якая 
ішла бы ў адным напрамкам з часаваротам.

Да гэтага дня Гары жыў адпаведна папярэджанню Э.~Т.~Джэйнса: калі 
феномен для цябе чымсьці незразумелы, гэта кажа толькі пра становішча твайго розуму,
а не пра сам феномен; твая няўпэўненасць кажа больш пра цябе, чым пра аб'ект 
тваёй няўпэўненасці; няведанне існуе толькі ў тваёй галаве, не ў рэальнасці;
пустая карта не вызначае пустую тэрыторыю. Існуюць загадкавыя пытанні, але 
загадкавы адказ --- супярэчнасць па вызначэнню.
Феномен можа казацца загадкавым толькі канкрэтнаму чалавеку; феноменаў, загадкавых 
па сваёй прыводзе не бывае. Пакланяцца сакральнай таямніцы азначала пакланяцца 
ўласнаму невуцтву.

Таму Гары глядзеў магіі прама ў вочы, і адмаўляўся пужацца.
Сучасныя людзі не разумелі, што калісьці хімія, біялогія, астраномія ---
тое, што ўсе ўспрымалі як сапраўдную жалезабетонную наувуку --- таксама 
калісьці былі загадкавымі. Зоркі былі загадкавымі. Лорд Кельвін аднойчы 
назваў прыроду жыцця --- падпарадкаванне мускулаў воле чалавека, або фармаванне 
дрэва з насення --- загадкай, "бясконца недасягальнай" для навукі (Не проста 
крыху недасягальнай, а  \emph{бясконца} недасягальнай. У Лорда Кельвіна 
няведанне адказаў, без сумневу, выклікала моцны эмацыйны водгук).
Кожная вырашаная загадка да моманту свайго вырашэння была таямніцай спакон веку.

А цяпер, упершыню ў жыцці ён сустрэў загадку, якая пагражала застацца 
\emph{назаўсёды}. Калі Час не быў уладкаваны ў форме ацыклічнага накіраванага графа,
тады паняцця "чынніка" і "выніка" страчвалі свой сэнс для Гары; а калі 
ён не разумеў, што такое чыннік, што --- вынік, сувязі паміж імі не было,
як яму зразумець, што за рэальнасць яго абкружала? І было цалкам магчыма, што 
чалавечы розум ніколі не здолее зразумець, бо ён зроблены са \emph{старамодных
нейронаў} --- тых, што ішлі ў адным напрамку ў часе, то бок, 
былі значна лімітаванымі ў параўнанні з асяроддзем.

Але былі і добрыя навіны. Жарта-Кола, якая калісьці выглядала неверагодна ўсемагутнай,
атрымала простае тлумачэнне. Якое не прыйшло яму да галавы \emph{проста} 
таму, што яно было за межамі прасторы яго гіпотэз, або за межамі 
таго, што ягоны чалавечы мозг мог успрымаць як рэальнасць. Але зараз ён 
зразумеў. Магчыма. Што давала пэўную надзею. Кшталту таго.

Ён паглядзеў на свой гадзіннік. Было амаль 11 гадзін, і калі прошлай ноччу ён 
заснуў каля першай гадзіны ночы, то гэтай ноччу ў яго нармальным рытме ён 
засне каля трох гадзін. Каб прачнуцца раніцай а сёмай, ён павінен пакласціся каля 22:00,
а дзеля гэтана ён павінен вярнуцца ў мінулае на пяць гадзін. І калі ён хацеў 
вярнуцца ў сваю спальню да 6:00 сёння, пакуль усе спалі, яму было лепей 
паспяшацца...

Нават гледзячы рэтраспектыўна, Гары не разумеў, \emph{як} ён здолеў раздабыць усе 
тыя рэчы, якія карысталіся ў пранку. Скуль маглі ўзяцца \emph{пірагі?}

Спадарожжа ў часе сур'ёзна пачыналі яго палохаць.

З другога боку, вартра прызнаць, што магчымасць не паўторыцца: 
гэты пранк ты мог зрабіць на сябе самога толькі раз у жыцці, і толькі ў прамежку шасці гадзін
пасля атрымання часаварота.

Але, калі задумацца, яшчэ больш турбавала іншае. Час выдаў гэты пранк як 
\emph{fait accompli}\footnote{{} фр. фетакомплі, закончаны факт.}, і ўсё ж было відавочна,
што гэта ўласнае тварэнне Гары --- канцэпцыя, выкананне, стыль запісак. 
Кожная дэталь несла Гарын почырк, нават тыя, якіх ён сам не канца разумеў.

Але чаго марнаваць час на гэтыя думкі? Тым больш, што ў сутках было ўсяго трыццаць гадзін.
Гары \emph{часткова} ведаў, што ён павінны зрабіць, а рэшткі, кташлту пірагоў, ён 
зможа прыдумаць па ходу справы. Адно было ясна як дзень:
пакуль ён заграс тут, ў \emph{будучыні}, нічога цудоўнага не адбудзецца.


\later

Пяць гадзін раней Гары пракрадваўся ў сваю спальню, накінуўшы мантыю на галаву, у 
якасці якой-ніякой маскіроўкі, на выпадак, калі нехта ўжо прачнуўся і мог убачыць 
яго адначасова ўваходзячым і спячым у сваім ложку. Ён не гарэў жаданнем 
тлумачыць камусьці пра сваю праблему з ідыя... ідыё... са спантаннай дублікацыяй.

На шчасце, усе яшчэ спалі.

І яшчэ, каля яго ложка ляжала святочнкая каробка, загарнутая ў чырвона-зялёную
бліскучую паперу, і перавязаная залатой стужкай; тыповы падарунак пад яліну, за 
тым, канешне, выключэннем, што да Раства было далёка.

Гары як мага цішэй --- на выпадак, калі хтосьці не ўключыў свай Сцішальнік ---
прайшоў праз пакой.

На падарункавай каробке ляжаў канвэрт, запячатаны простым, без пячаткі, воскам.

Гары дастаў ліст і прачытаў:

\begin{writtenNoteCursive}

Гэта Плашч Нябачнасці Ігнотуса Певерэла, які праз пакаленні дайшоў да яго нашчадкаў 
Потэраў. У адрозненне ад меней магутных прадметаў і заклёнаў, ён мае здольнасць
\emph{схаваць} цябе, а не проста зрабіць нябачным. Твой бацька пазычыў мне яго перад
сваёй смерцю, дзеля вывучэння, і прызнаюся, за гэтыя годы я карыстаў яго на поўную катушку.

Баюся, у будучыні мне прыйдзецца абыходзіцца простай Дызілюзорнасцю. Надышоў час 
вярнуць Плашч яго законнаму гаспадару. Я планаваў зрабіць гэта падарункам на Раство,
але ён пажадаў вярнуцца да цябе раней. Падобна, што табе, па яго меркаванню, 
патрэбна ягоная дапамога. Ужывай яго на карысць.

Без сумневу, ты ўжо думаеш пра ўсемагчымыя цудоўныя пранкі, кшталту тых, што ўчыняў 
твой бацька ў свае часы. Калі хаця б частка іх стала вядома, кожная жанчына Грыфіндора
здейсніла бы паломніцтва, каб надругацца над ягонай магілай. Я не маю намер 
спыніць гісторыю ад паўтарэння, але калі ласка --- \emph{вельмі} пастарайся не 
выдаць сябе. Калі Дамблдор атрымае магчымасць завалодаць адным са Скарбаў Смерці,
ён не выпусціць яго са сваіх збялелых пальцаў да сканчэння вякоў.

Вельмі вясёлага Раства табе.

\end{writtenNoteCursive}

Подпісу не было.

\later


--- Пачакайце, --- сказаў Гары як раз, калі хлопцы збіраліся пакінуць спальню 
першакурснікаў Рэйвенкло. --- Выбачайце, мне трэба зрабіць тое-сёе ў маім куфары.
Я прыпазнюся на сняданак на пару хвілін.

Тэры Бут паглядзеў падазрона.

--- Спадзяюся, ты не плануеш шныпарыцца ў нашых рэчах.

Гары падняў раскрытую далонь.

--- Я клянуся, што не маю ніякага намеру кранаць вашыя рэчы, я планую карыстацца толькі
рэчамі, якія належаць мне, у мяне няма планаў пранкануць любога з вас, і 
я не чакаю, што мае намеры зменяцца да таго часу, калі я дасягну Вялікай Залы.


Тэры нахмурыўся.

--- Чакай, гэта...

--- Не хвалюйся, --- сказала Пенелопа Кліруотэр, якая павінна была давесці іх да Залы.
--- Шчылінаў няма. Добра вярзеш, Потэр, табе варта ў юрысты.

Гары міргнуў. Ах, так, гэта ж прэфектэса Рэйвенкло.

--- Дзякуй, --- сказаў ён. --- Напэўна...

--- Калі паспрабуеш знайсці дарогу да Вялікай Залы, ты згубішся, --- Пенелопа 
сказала гэта як бясспрэчы факт. --- Як толькі згубішся, спытай бліжэйшы партрэт,
як прайсці на першы паверх. У той \emph{момант}, калі падазруеш, што зноў 
згубіўся, зноў пытай партрэт.  \emph{Асабліва}, калі табе падаецца, што ты падымаешся
вышэй і вышэй. Калі ты вышэй, чым замак павінен быць, \emph{спыніся}, і чакай
ратавальны атрад. Інашк мы зноў убачым цябе толькі праз некалькі месяцаў, а ты 
будзеш апрануты ў набедраную павязку і пакрыты снегам, і \emph{таму
трэба заўсёды заставацца ў замку.}

--- Зразумела, --- сказаў Гары, цяжка зглынуўшы. --- Эмм... а вы не павінны расказваць 
гэта студэнтам адразу?

Яна ўздыхнула.

--- \emph{Усё} адразу? Гэта зойме некалькі тыдняў. Па ходу справы навучыцеся, --- 
яна павярнулася, каб пайсці, астатнія паследвалі. --- Калі я не ўбачу цябе 
на сняданку праз паўгадзіны, Потэр, я пачынаю пошукі.

Калі ўсе сышлі, Гары прымацаваў запіску да свайго прыгалоўя --- ён ужо падрыхтаваў яе
і ўсе астатнія запіскі, працуючы ў склепе свайго куфара, пакуль усе спалі. Потым ён 
ціхенька працягнуўся і зняў Плашч Нябачнасці са спячага Гары-1.

І, проста ад агульнага свавольства, Гары паклаў Плашч у махляскін Гары-1 з задавальненнем
ведаючы, што той імгненна апынецца і ў ягоным уласным кашалі.

\later

--- Я прасачу, каб зветку перадалі Карнэліану Флаберволту, --- сказаў партрэт 
чалавека арыстакратычнага выгляду, але з цалкам звычайным носам. --- Але 
магу я пацікавіцца аб \emph{арыгінальнай} яго крыніцы?

Гары пацепнуў плячыма з дасканалай бездапаможнасцю. 

--- Мне сказалі, што яе абвясціў загробны голас, які пачуўся з разлому ў прасторы нашага свету,
з разлому, які адчыняўся напрамкі ў вогненую бездань!


\later

--- Гэй! --- абурана крыкнула Герміёна, якая сядзела насупраць. --- Гэта \emph{агульны}
дэсерт! Ты не можаш узяць цэлы пірог і пакласці ў свой кашэль!

--- Я не бяру цэлы пірог. Я бяру два цэлых пірага. Выбачайце, трэба бяжаць! --- 
Гары праігранаваў крыкі абурэння, і пакінуў Вялікую Залу. Ён быў павінен дабрацца 
да кабінета Гарбалогіі крыху раней.

\later

Прафесар Спраут паглядзела на яго рэзка. 

--- І скуль  \emph{вы} ведаеце, што плануюць слізэрыны?

--- Я не магу раскрываць свае крыніцы, --- сказаў Гары. --- Дарэчы, я вымушаны 
запытаць вас зрабіць выгляд, што гэтай размовы ніколі не было. Проста ўдавайце, 
што вы выпадкова прайшлі там, па сваіх справах, кшталту таго. Я пабягу наперад,
як толькі наш урок скончыцца. Думаю, я змагу адцягваць іх, пакуль вы не прыйдзеце.
Мяне цяжка напужаць або забуліць, і думаю, яны не пасмеюць сур'ёзна пашкодзіць 
Хлопца-Які-Выжыў. Я, кашешне, не пытаю вас бегаць, але я буду вас вельмі
ўдзячны, калі вы не будзеце марнаваць час па дарозе.

Прафесар Спраут некаторы час глядзела на яго, потым яе выраз памягчэў.

--- Калі ласка, не рызыкуйце, Гары Потэр. І... дзякуй.

--- Проста пастарайцеся не спазніцца, --- сказаў Гары. --- І памятайце, калі 
прыйдзеце --- вы не чакалі мяне ўбачыць, і гэтай размовы не было.


\later

Насамрэч, было жудасна назіраць, як ён штурхнуў Нэвіла з кола слізэрынаў. 
Нэвіл быў правы, Гары прымяніў занадта сілы, вельмі занадта.

--- Дароўкі, --- сказаў Гары халодна. --- Я --- Хлопец-Які-Выжыў.

Восем першакурснікаў, амаль аднолькавага росту. Той, што са шнарам на ілбе,
дзейнічаў не так, як астатнія.


\vskip 0pt plus 4\baselineskip\settowidth{\versewidth}{Калі б гасподзь даў нам дар бачыць нас,}
\begin{verse}[\versewidth]
%Oh wad some power the giftie gie us\\
%To see oursel’s as others see us!\\
%It wad frae monie a blunder free us,\\
%And foolish notion—\footnote{{} }
\itshape
Калі б гасподзь даў нам дар бачыць нас,\\ 
Як бачуць усе навакольныя нас,\\
Пазбавіла б гэта бязглуздасці нас\\
І ішных глупстваў...\footnote{{} Верш Р. Бёрнса "Вошы, якую я заўважыў на шляпцы адной дамы ў царкве", у 
арыгінале "To A Louse, On Seeing One on a Lady's Bonnet at Church".}
\end{verse}


Прафесар МакГонагал была права. Размеркавальны Капялюш быў правы. Калі ты бачыў яго 
з боку, гэта было відавочна з першага позірку. 

З Гары Потэрам нешта было не так.
