\chapter{Кантроль імпульсаў}


\begin{chapterOpeningQuote}
Цікава, а з \emph{ім} што не так?
\end{chapterOpeningQuote}

% Letterine looks ugly with very short first lines, 
% to tidy it up a little manually.
\lettrine[lines=1,lraise=-0.1]{-Ц}{ёрпін,} Ліза!
%\lettrinepara[ante=—]{Ц}{ёрпін,} Ліза!

%\hplettrineextrapara
Шур-шур  шур-шур гары потэр шур-шур слізэрын шур-шур сур'ёзна шур якога чорта шур-шур-шур

--- РЭЙВЕНКЛО!

Гары далучыўся да апладысментаў у гонар дзяўчыны, якая саромеючыся ішла да стала Рэйвенкло,
і пакуль яна ішла, аблямоўка яе мантыі змяніла колер на блакітны. Было заўважна, што Лізу Цёрпін
радзірала паміж жаданнем сесьці як мага далей ад Гары, і жаданнем ўціснуцца побач і пачаць
выбіваць з яго адказы.

Быць у цэнтры гэтых неверагодных падзей, а потым яшчэ і трапіць у Рэйвенкло можна было 
параўнаць з тым, быццам цябе палілі соусам барбекю, і потым кінулі ў яму з 
галоднымі кацянятамі.

--- Я абяцаў Капелюшу не расказваць, --- прашаптаў камусьці Гары ў стопіццоты раз. --- 
Да, праўда... Не, я праўда даў клятву Капелюшу ні пра што не казаць... Ладна, 
я даў клятву не казаць пра \emph{большасць} таго, але астатняя частка --- 
\emph{прыватная}, таксама, як і твая, таму хопіць мяне пытаць... 
Проста хочаш ведаць, што адбылося? Добра! Вось, што адбылося! Я сказаў Капелюшу,
што МакГонагал паграджала яго спаліць, а ён сказаў перадаць ёй, што яна 
"нахабная саплячка", і што ён прыказаў ёй сысці з газону!.. 
Калі ты ўсё роўна не паверыш, \emph{навошта пытаеш?}... Не,
я таксама не ведаю, як я адолеў Цёмнага Лорда. Я бы расказаў, калі б ведаў!

--- \emph{Цішыня!} --- крыкнула прафесар МакГонагал са сваёй кафедры. --- 
Не шумець да канца Размеркавання! 

Гучнасць у зале рэзка панізілася, быццам усе чакалі ад яе нейкіх канкрэтных пагрозаў,
але, калі іх не паследвала, шэпт узняўся зноў.

Срэбнабароды старык падняўся са свайго трону, дабразычліва ўсміхаючыся.

Адразу наступіла ціша. Хтосьці пхнуў локцем Гары, які ўсё яшчэ працягваў 
свой шэпт, і Гары перарваўся на сярэдзіне сказа.

Старычок з добрым выразам на твары сеў назад.

\emph{На будучыню: Не дурыць з Дамблдорам.}

Гары ўсё яшчэ спрабаваў перастрававаць тое, што адбылося падчас Інцыдэнту з Размеркавальным
Капелюшом. Не апошняе месца ў яго думках займала тое, што, калі ён зняў Капялюш,
ён пачуў ціхі шэпт прама на вуха, дзіўны, быццам шыпячы: 

\parsel{"Салют ад Слізэрына слізэрыну: калі хочаш раскрыць мае сакрэты, шукай майго аспіда."}

Гары падазраваў, што гэта не было часткай афіцыяльнай працэдуры размеркавання.
І што тое была нейкая дадатковая магія, накладзеная Салазарам Слізэрынам. І што
сам Капялюш пра яе не ведаў. І хучэй за ўсё, трыгерам для яе было 
вымаўленне Капелюшом слова "СЛІЗЭРЫН". І што студэнты Рэйвенкло, такія як ён, 
\emph{ніколі, ніколі не павінны былі гэта чуць.} І калі знойдзецца сапраўды добры 
спосаб прымусіць Драко да сакрэтнасці, і Гары зможа з ім пра гэта пагаварыць,
тое будзе добры час, каб распіць крыху Жарта-Колы.

\emph{Што і гаварыць. Ты вырашаеш не ісці шляхам Цёмнага Лорда, і сусвет 
пачынае спакушаць цябе, як толькі ты здымаеш Капялюш. У некаторыя дні проста
бессэнсоўна бароцца з лёсам. Магчыма, варта адкласці пачатак "не быць Цёмным Лордам"
на заўтра.}

--- ГРЫФІНДОР!

Рон Уізлі атрымаў шмат апладысментаў, і не толькі з боку Грыфіндора.
Відавочна, сям'ю Уізлі тут шырока любілі. Гары праз момант усміхнуўся, і 
пачаў пляскаць з астатнімі.

Але, магчыма не трэба адкладаць адмаўленне ад Цёмнага боку на заўтра,
калі можаш адмовіцца сёння. К чорту лёс, і к чорту сусвет. Ён пакажа гэтаму
Капелюшу.

--- Забіні, Блэйз!

Пауза.

--- СЛІЗЭРЫН! --- крыкнуў Капялюш.

Гары папляскаў і Забіні, ігнаруючы здзіўленыя позіркі ад усіх вакол, не выключаючы 
Забіні.

Болей імён не было, і Гары толькі зараз зразумеў, што "Забіні, Блэйз!" было
даволі блізка да канца алфавіта. Цудоўна, выглядае так, быццам ён павітаў  \emph{толькі} 
Забіні. Ох, чорт...

Дамблдор падняўся яшчэ раз, і накіраваўся да кафедры. Відавочна, дзеля прамовы...

Тут Гары прасякла геніяльная ідэя \emph{бліскучага} эксперымента.

Герміёна казала, што Дамблдор --- самы магутны чараўнік у свеце, так?

Гары сунуўся ў махляскін, і прашаптаў: "Жарта-Кола".

Каб Жарта-Кола спрацавала, Дамблдор павінен быў сказаць нешта 
\emph{настолькі} бязглуздае, што Гары, нават у стане сваёй 
ментальнай падрыхтаванасці, павінен быў папярхнуцца.
Нешта кшталту "усе студэнты павінны хадзіць без адзення ўвесь год", або што 
"усе павінны пераўтварыцца ў кацянят".

Але калі \emph{нехта ў гэтым свеце} і мог супраціўляцца сіле Жарта-Колы, гэта 
быў Дамблдор. Таму, калі спрацуе, значыць, Жарта-Кола была літаральна 
\emph{непераможнай}.

Гары расчыніў банку пад сталом, намагаючыся зрабіць гэта незаўважна, але 
яна ўсё роўна выдала ціхі свісцячы гук. Некалькі галоў павярнуліся глянуць, 
але раздалося:

--- Вітаю! Вітаю ўсіх у новым годзе ў Хогвартс! --- сказаў Дамблдор,
шырока усміхаючыся і раскідваючы рукі, быццам нічога не магло задавальніць яго 
так, як гэтая сустрэча.

Гары набраў поўны рот газіроўкі. Ён будзе праглатываць яе пакрысе, і будзе намагацца 
не закашляцца, што бы Дамблдор ні сказаў...

--- Перад тым, як пачаць вячэру, я бы хацеў сказаць некалькі слоў. Вось яны:
Happy happy boom boom swamp swamp swamp! Дзякуй!\footnote{{} Тут мае паўнамоцтвы ўсё.
Чытаецца так: "х\'апі-х\'апі, бум-бум, свомп, свомп, свомп". Ну, а што вы чакалі ад Дамблдора?}

Усе запляскалі ў далоні і засмяяліся, і Дамблдор сеў на месца.

Гары сядзеў здранцавелы, з вугалкаў рота ў яго цяклі два тонкіх зялёных ручайка. 
Прынамсі, ён здолеў папярхнуцца \emph{ціха}.

Яму праўда, праўда, \emph{праўда} не варта было гэтага рабіць. Дзіўна, 
як відавочна яно стала праз секунду пасля таго, як стала надта позна.

Рэтраспектуўна ён павінен быў, канешне, заўважыць нешта незвычайнае, калі ён 
думаў пра пераўтварэнне вучняў у кацянят... або ўспомніць свой ранейшы стан, 
калі ён падумаў не дурыць з Дамблдорам... або калі ён думаў пра тое, што не 
будзе Цёмным Лордам... калі б ў ім наогул была б хаця б кропля ўважлівасці і розуму!..

Усё было безнадзейна. Ён быў дрэнны да мозгу касцей. Слаўся, Гары, Цёмны Лорд. 
Немагчыма адолець лёс.

Хтосьці побач спытаў у Гары, ці ўсё з ім добра. (Астатнія пачалі накладваць сабе ежу, якая
цудоўным чынам з'явілася на стале.)

--- Я нармальна, --- сказаў Гары. --- Скажыце... эмм... Гэта \emph{нармальна} для
дырэктара, казаць такія прамовы? Вы...  не выглядалі надта... здзіўленымі...

--- О, Дамблдор звар'яцелы, безумоўна, --- сказаў старэйшы рэйвенкло, які сядзеў побач,
і раней прадставіўся, але Гары не запоўніў яго імя. --- Вельмя вясёлы, неверагодна 
моцны чарадзей, але цалкам з'ехаўшы з глузду... Дарэчы, я бы хацеў калісьці дазнацца,
чаму ў цябе з рота цякла зялёная вадкасць, хаця, думаю, ты  абяцаў Капелюшу
не казаць пра гэта таксама.

Значным напруджаннем волі Гары прымусіў сябе не кінуць позірк на банку Колы ў
сваёй руцэ.

Калі думаць лагічна, не магла ж газіроўка проста рандомна  \emph{матэрыялізаваць}
загаловак у Куіблеры  пра яго і Драко? Драко таксама сведчыў, што ўсе, што адбывалася
вакол, выглядала натуральна. Быццам Жарта-Кола рэтраспектыўна 
\emph{перарабіла мінулае}, каб усё схадзілася?

Гары ўявіў, як б'ецца галавой аб стол.  \emph{Бам, бам, бам} бахала ў яго розуме.

Іншы студэнт сказаў сцішаным голасам:

--- А я чуў, што Дамблдор сакрэтная кантралюе шмат чаго, і карыстуецца вар'яцтвам, каб 
ніхто не западозрыў.

--- Я таксама чуў гэта, --- прашаптаў трэці, і шмат хто вакол іх згодна заківалі.

Гэта не магло не прыцягнуць Гарыну ўвагу.

--- Зрадумела, --- прашаптаў Гары, --- дык усе тут упэўнены, што Дамблдор сакрэтна 
ўзначальвае сакрэтную арганізацыю?

Большасць людзей вакол Гары кіўнула. Некалькі з іх выглядалі задуменна, уключаючы
старэйшага суседа Гары.

\emph{Гэта точна стол Рэйвенко?} --- Гары здолеў не спытаць гэта уголас.

--- Цудоўна, --- прашаптаў Гары. --- Калі ўсе ведаюць, ніхто не западозрыць, што 
гэта сакрэт!

--- Дакладна, --- адказаў нехта, і нахмурыўся. --- Чакай, тут нешта не так...

\emph{На будучыню: 75-я персэнтыль Хогвартс, таксама вядомая як "факультэт Рэйвенкло"
--- не "самая выдатная ў свеце школа для таленавітых дзяцей".}

Але прынамсі ён сягодня высветліў важны факт: Жарта-Кола была ўсемагутнай. І гэта значыла...

Гары здзіўлена міргнуў, калі ягоны розум заўважыў відавочную сувязь. 

... гэта значыла, што як толькі ён вывучыць заклён, які часова можа адключыць яго 
пачуццё гумару, ён можа прымусць Жарта-Колу зрабіць \emph{што заўгодна}, калі гэтая
падзея будзе з разраду настолькі дзіўных, каб прымусіць яго разліць газіроўку. 

\emph{Гэта была даволі кароткая вандроўка на Светлы бок. Нават я чакаў, што 
яна працягнецца даўжэй за мой першы школьны дзень.}

Да таго ж, ён яшчэ здолеў перавярнуць Хогвартс дагары нагамі, нават не скончыўшы 
свае размеркаванне.

Гары крыху шкадаваў аб гэтым --- Мерлін ведае, як мог адпомсціць яму звар'яцелы дырэктар
за наступныя сем гадоў, але ён не мог не адчуваць і пэўны гонар.

Заўтра. Не далей як заўтра ён канчаткова сыйдзе са шляху, які вядзе да Гары Цёмнага Лорда.
З кожнай хвілінай думкі аб тым, што было ў канцы таго шляха, былі ўсё страшней.

Але і неяк прывабней, таксама. Часта яго розуму ўжо пачала прыдумляць уніформу для
сваіх міньёнаў.

--- Еж, --- сказаў старэшы студэнт побач, і пхнуў Гары пад робры. --- Не думай. Еж.

Гары аўтаматычна ўзяў сабе на талерку то, што стаяла на стале непасрэдна
перад ім: блакітныя сасіскі са святлістымі ўкрапінамі, якая розніца.

--- Аб чым ты задумаўся, аб Размер... --- пачала казаць Падма Паціл, таксама адна з 
першакурсніц Рэйвенкло.

--- Як ем, дык глух і нем! --- хорам сказалі як мінімум трое. --- Правіла Рэйвенкло, 
--- дадаў яшчэ адзін. --- Інакш мы тут усе з голаду памрэм.

Гары заўважыў, што пачынае вельмі, вельмі спадзявацца, што яго ідэя не спрацуе. 
Што Жарта-Кола не мела ўсемагутнай сілы мяняць рэальнасць і перапісваць мінулае.
Не тое, каб ён не хацеў, каб яна не была ўсемагутнай... ён проста не мог 
прыняць думку, што яму прыдзецца жыць такім сусвеце. Было нешта 
\emph{недастойнае} ва ўсталяванні міравой дыктатуры 
праз нестандартнае выкарыстоўванне газірованых напоеў.

Але ён дакладна збіраўся праверыць гэта эксперыментальна.

--- Ведаеш, --- сказаў яго сусед вельмі ветліва, --- у нас ёсць сістэма, каб прымусіць
есці такіх, як ты; хочаш дазнацца, як яна працуе?

Гары адкінуў думкі і пачаў есць блакітную сасіску. Яна была даволі добрай на смак,
асабліва святлістыя ўкрапіны.

Вячэра прайшла на здзіўленне хутка. Гары спрабаваў пакаштаваць пакрыху ўсёй 
магічнай ежы, якую заўважаў. Яго зацікаўленасць не магла выцярпець неведання,
як яно было на смак. Дзякуй богу, гэта не было як у рэстаране, дзе ты мог 
загадаць толькі адну страву, і ніколі не дазнацца пра смак усяго астатняга 
меню. Гары \emph{ненавідзеў} такое, гэта было як камера катавання для кожнага,
хто меў хоць кроплю цікаўнасці:  \emph{Табе дазволена разгадаць толькі
адну таямніцу з гэтага спісу, ха ха ха!}

Потым быў час дэсертаў, на якія Гары цалкам забыў пакінуць месца. Ён, канешне,
паспрабаваў, але здаўся пасля першага кавалачка пірага з патакай. Без сумневу,
усе гэтыя дэсерты сустрэнуцца яшчэ прынамсі раз за навучальны год.

Так, што было ў яго спісе важных справаў, акрамя звычайных школьных?

\begin{em}
Справа №1. Даследаваць заклёны, якія ўплываюць на розум, каб праверыць гіпотэзу 
пра Жарта-Колу, ці сапраўды яна была сродкам усемагутнасці. Нават лепей, 
даследваць любую магію, якая датычыцца розуму. Розум --- падмурак нашай
сілы як чалавецтва, і любая магія, якая на яго ўплывае --- самая важная магія сярод
астатняй. 

Справа №2. На самой справе, гэта справа №1, а дагэтуль ішла справа №2. Прабяжацца
па ўсіх полках бібліяіэк Хогвартс і Рэйвенкло, пазнаёміцца з класіфікацыяй,
упэўніцца, што прачытаў прынамсі назвы ўсіх кніг. Гэта за першы праход.
Другі праход: прачытаць змест усіх кніг. Кааперацыя з Герміёнай, яе памяць
значна лепей за тваю. Дазнацца, ці існуе між-бібліятэчны абмен у Хогвартс,
і ці можаце вы двое, асабліва Герміёна, наведаць тыя бібліятэкі. 
Калі ўласныя бібліятэкі ёсць на іншых факультэтах, высветліць ці можна 
трапіць туды легальна, або нелегальна.


Опцыя №3, пункт А: Узяць з Герміёны клятву маўчання і пачаць даследаваць 
"ад Слізэрына слізэрыну: калі хочаш раскрыць мае сакрэты, шукай майго аспіда".
Праблема: гучыць як нешта, што датычыцца вельмі канкрэтнай таямніцы, і можна 
заняць шмат часу, каб знайсці рэлевантную падказку, проста рандомна 
прабягачы па кнігах. 

Справа №0 (ноль): Выведаць, якія існуюць заклёны, дапамагаючыя з індэксацыяй і пошукам
інфармацыі, калі такія ёсць наогул. Бібліятэчная магія не такая важная, 
як магія розуму, але зараз мае большы прыарытэт.

Опцыя №3, пункт Б: Знайсці заклён, які прымусіць Драко трымаць сакрэт, або
знайсці спосаб магічна праверыць шчырасць яго абяцання трымаць сакрэт (Верытасерум?),
і потым спытаць яго пра звестку Слізэрына... 
\end{em}

Хаця... у Гары было даволі дрэннае прадчуванне на конт опцыі 3Б.

А падумаўшы, Гары пачаў сумявацца і ў опыцыі 3А.

Яго думкі вярнуліся да магчыма самага жудаснага моманту ў яго жыцці, тыя доўгія
некалькі секунд пад Капелюшом, калі ён зразумеў, што прайграў. Тады ён жадаў
вярнуцца ўсяго на некалькі хвілін назад, і змяніць нешта, пакуль было яшчэ не позна...

А потым высветлілася, што было зусім не позна. 

Яго жаданне ажыццявілася.

Нельга змяніць гісторыю. Але можна пачаць думаць перад тым, як зрабіць нешта незваротнае.

І гэтая ўся справа з пошукам сакрэта Слізэрына... падавалася страшна падобнай на тое,
аб чым ты праз шмат гадоў скажаш: "Вось \emph{тут} усё пайшло не туды". І тады
ён таксама будзе роспачна жадаць вярнуцца ў мінулае ў зрабіць іншы выбар...

А пакуль, яго жаданне ажыццявілася. Што далей?

Гары павольна ўсміхнуўся.

Крыху \emph{контр-інтуытыўная} думка... але... 

Ён проста мог \emph{прыкінуцца} --- не было правіла, якое казала наадварот, --- 
што ніякага ціхага шэпту не было. Дазволіць сусвету ісці тым шляхам, быццам 
гэтага крытычнага моманту так і не адбылося. Праз дваццаць гадоў ёй будзе маліць,
каб менавіта так ён і зрабіў дваццаць гадоў таму (то бок --- зараз). Змяніць даўняе 
мінулае было магчыма, проста трэба было абраць правільны час.

Або... гэтая думка была нават  \emph{больш} контр-інтуытыўная... ён мог праінфармаваць,
скажам, прафесара МакГонагал, замест Драко або Герміёны. А яна збярэ некалькі
добрых людзей, і яны знімуць таямнічы заклён з Капелюша.

Ну, а чаму не? Калі Гары \emph{прагаварыў} ідэю цалкам у галаве, яна прагучала
выдатна.

Рэтраспектыўна гэта было так відавочна, але чамусці абодва варынта опцыя №3
прыйшлі да галавы першымі, і падаваліся лепшымі варыянтамі. 

Гары ўручыў сабе +1 бал за поспехі ў праграме анты-Гары-Цёмны-Лорд.  

\emph{Справа №4: Папрасіць прабачэння ў Нэвіла Лонгботама.}

Пранк, якому ён падверг Нэвіла, быў вельмі жорсткі, 
хаця з пункту гледжання кансеквенцыяліста спрачацца з вынікам было складана.
Досвід безумоўна даваў і Гары паняцце аб тым, што адчувае ахвяра.

Раз пайшла такая п'янка...

\emph{Кожны дзень, любым магчымым спосабам, я буду станавіцца ўсе Святлей і Святлей...}

Людзі навокал Гары ўжо ў асноўным скончылі вячэраць, талеркі пачалі знікаць са стала.

Калі сталы цалкам спусцелі, Дамблдор зноў падняўся са свайго трона.

Гары не мог не адчуць імпульс выпіць яшчэ Жарта-Колы. 

\emph{СУР'ЁЗНА?..} падумаў Гары пра гэты свой бок.

Але эксперымент не лічыцца, калі яго не паўтабылі хаця б раз, ці не так? 
І самае страшнае ўжо здарылася, ці не так? 
Ці не цікава яму было, што здарыцца \emph{гэтым} разам?
Ні кропельку ні цікава? А што калі ён атрымае іншыя рэзультаты? 


\emph{Б'юся аб заклад, ты --- та частка майго розуму, якая падштурхнула мяне да
пранка над Нэвілам.}

Э-э-э...  можа і так?

\emph{Табе што, не ВІДАВОЧНА, што я пашкадую аб гэтым праз секунду пасля таго, 
як будзе надта позна?}

Хм...

\emph{Вось-вось. Таму --- НЕ.}

--- Агхм, --- сказаў Дамблдор з кафедры, прыгладзіўшы сваю доўгую сівую бараду. ---
Яшчэ некалькі слоў, калі мы ўжо наеліся і напіліся. Некалькі аб'яў перад
пачаткам новага навучальнага года. Першакурснікі павінны запомніць, што 
лес на тэрыторыі школы забаронены для наведвання любымі вучнямі. Таму ён 
і называецца "Забаронены Лес". Калі б ён быў дазволены, ён бы зваўся 
"Дазволены Лес".

Лагічна. \emph{На будучыню: Забаронены Лес --- забаронены.}

--- Таксама, па просьбе містэра Філча, наглядчыка, я нагадваю, што на перапынках 
забаронена карыстацца любой магіяй. Нажаль, мы ўсе ведаем, што так \emph{павінна быць}.
Але тое, як павінна быць, і тое, яе ёсць --- дзве вялікія розніцы. Загадзя вам удзячны,
што не забываецеся пра гэта.

Эммм...

--- Адборныя выпрабаванні новых гульцоў у Квіддзіч пройдуць на наступным тыдне.
Усе зацікаўленыя гуляць за каманду свайго факультэта, звяртайцеся да мадам Хуч.
Усе зацікаўленыя ў татальным рэфармаваніі Квіддзіча як нацыянальнага спорту,
звяртайцеся да Гары Потэра. 

Гады папярхнуўся ўласнай слінай, і закашляўся як раз у момант, калі ўсе вочы
павярнуліся да яго. Да якога \emph{чорта!} Ён не сустракаўся позіркамі з Дамблдорам
дагэтуль... прынамсі, яму так падавалася. І нават калі так, ён дакладна не 
думаў пра Квіддзіч з сустрэчы з Ронам. І наўрад ці Рон мог камусьці расказаць...
або, што, ён адразу пабяжаў скардзіцца настаўнікам?.. \emph{Як? Як?!}

--- Хачу таксама дадаць, што сёлета трэці калідор правага крыла зачынены для ўсіх,
хто не хоча памерці вельмі балючай смерцю. У ім усталявана мудрагелістая 
чарга пагрозлівых і патэнцыйна летальных лавушак, і хутчэй за ўсё, у вас 
не атрымаецца прайсці іх усе, асабліва, калі вы толькі на першым курсе.

Гары нават занямеў.

--- І нарэшце, дазвольце выразіць нашу найвялікшую падзяку Кв\'ірынесу Кв\'ірэлу за
геройскае пагаджэнне заняць пасаду прафесара Абароны ад Цёмных Майстэрстваў у Хогвартс, ---
пранізлівы позірк Дамблдора агледзеў студэнтаў у зале. ---
Я спадзяюся, усе студэнты будуць праяўляць толькі ветлівасць і \emph{талерантнасць} 
за такую эксраардынарную паслугу нашай школе, і не будуць \emph{назаляць} нам 
сваімі \emph{дакучлівымі скаргамі} на яго, калі яны не маюць намер заняць яго месца.

Аб \emph{чым} гэта ён?

--- А зараз я саступаю каферду нашаму новаму настаўніку --- прафесару Квірэлу, які 
хоча сказаць некалькі слоў.

Малады, худы, знерваваны малады чалавек, ягога Гары бачыў у Дзіравым Катле, марудна
пайшоў да кафедры, спужана гледзячы па баках. Гары заўважыў адзін раз яго патыліцу, і
падавалася, што прафесар, нягледзячы на малады ўзрост, ужо пачаў хутка лысець.

--- Цікава, а з \emph{ім} што не так? --- прашаптаў Гарын сусед. Падобныя прыглушаныя
пытанні чуліся па ўсёй зале. 

Прафесар Квірэл дайшоў нарэшце да кафедры, і спыніўся за ёю, міргаючы.

--- Э... --- сказаў ён. --- Э... --- на гэтым, падавалася, яго смеласць яго канчаткова
пакінула, і ён проста стаяў моўчкі, час ад часу ўздрыгваючы. 

--- Ну, цудоўна... --- прашаптаў старэйшы студэнт, --- выглядае, што нас чакае яшчэ адзін 
\emph{доўгі} год Абароны...

--- Салютаванні, мае юныя вучні, --- раптам сказаў прафесар Квірэл сухім упэўненым
тонам. --- Як мы ўсе ведаем, Хогвартс пераследуе дзіўная \emph{няўдача} ў 
пошуке кандыдатаў на гэтаую пасаду, і без сумневу, многія з вас ўжо гадаюць, якое 
ліха чакае мяне сёлета. Хачу вас запэўніць, што гэтым ліхам будзе не мая кампетэнтнасць,
--- ён крыху ўсміхнуўся. --- Хаціце верце, хаціце не, але гэта мае даўняе жаданне ---
пасправаць сябе ў ролі прафесара Абароны ад Цёмных Майстэрстваў тут, у школе 
вядзьмарства і чарадзейства Хогвартс. Першым настаўнікам па гэтай дысцыпліне быў 
сам Салазар Слізэрын, і не пазней як за чатырнаццае стагоддзе з'явілася традыцыя, 
што самыя вялікія магі-байцы свайго часу абавяскова спрабавалі свае сілы ў навучанні 
тут. У спісе былых настаўнікаў Абароны быў не толькі легендарны вадруючы герой 
Гаральд Шы, але і Баба Яга па мянушцы "Неўміручая", і я бачу, як некаторыя 
здрыгануліся ад гэтага імя, нават калі яна сама ўжо шэсцьсот лет як памерла. 
То быў цікавы час, каб наведваць Хогвартс, што скажаце?

Гары моцна зглынуў, намагаючыся справіцца з хваляй эмоцый, якая накрыла яго, калі 
прафесар Квірэл пачаў сваю прамову. Точныя фармулёўкі нагадалі яму оксфардскіх
лектараў, ён успомніў дом, і яшчэ --- што ён ўбачыць маму і тату не раней за Раство.

--- Вы звыкліся, што гэтую пасаду займаюць дылетанты, няўдачнікі, і махляры. 
Аднак любы, хто ведае гісторыю, з вамі не згадзіцца. Не ўсе, хто тут вучыў, былі 
найлепшымі, але ўсе найлепшыя магі свайго часу былі калісьці прафесарамі Хогвартс.
У такой славутай кампаніі, пасля майго доўгага чакання гэтага дня, будзе сорамна,
калі я задам  узровень маіх стандартаў ніжэй, чым "дасканаласць". Я маю намер
зрабіць так, каб кожны з вас запомніў гэты год як \emph{найлепшы} курс Абароны, які 
вы мелі за ўсё жыццё. Тое, што вы вывучыце сёлета, будзе для вас заўсёды 
надзейным падмуркам у мастацтвах Абароны, няважна, хто вас вучыў да мяне, або пасля.

Прафесар Квірэл стаў вельмі сур'ёзным.

--- Нам трэба \emph{шмат} чаго прайсці, і ў нас вельмі мала часу. Таму я маю намер адысці 
ад усталяваных у Хогвартс практык навучання, і таксама ўвесці шэраг пазакласных
актыўнасцяў, --- ён зрабіў паузу. --- Калі гэтага будзе не дастаткова, магчыма, 
я прыдумаю новыя спосабы матывацыі. Вы --- мае доўгачаканыя студэнты, і вы 
\emph{будзеце} рабіць \emph{усё} магчымае ў маім доўгачаканым курсе Абароны.
Я бы мог дадаць нейкую страшную пагрозу кшталту "Інакш вы ўсе будзеце пакутваць",
але гэта будзе проста клішэ, ці не так? Я лічу свае ўяўленне больш здольным. Дзякуй.

Жывасць і ўпэўненасць быццам сышлі з прафесара Квірэла. Яго рот раскрыўся,
быццам ён нечакана апынуўся перад незнаёмай аўдыторыяй. Ён дзёрнуўся, павярнуўся, і 
пайшоў на свае месца, згорбіўшыся, быццам жадаў сціснуцца ў маленькі і незаметны кавалачак.

--- Нейкі ён дзіўны, --- прашаптаў Гары.

--- Мдэ, --- сказаў яго сусед, --- мы і не такое бачылі.

Дамблдор вярнуўся за кафедру.

--- А зараз, --- сказаў ён, --- перад тым, як адпраўляцца спаць, давайце спяем 
школьны гімн! Кожны павінен абраць сваю любімую мелюдыю, любімыя словы, і,
галоўнае, пець як мага грамчэй. І, раз... два... раз, два, тра, чатыры!...
