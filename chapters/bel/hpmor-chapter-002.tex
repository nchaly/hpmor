
\chapter{Усё, у што я веру - хлусня}


\begin{chapterOpeningQuote}
    Вядома, гэта мая віна. Больш няма нікога, хто можа несці адказнасць
    за усё на свеце.
\end{chapterOpeningQuote}


\lettrine{-Я}{шчэ} раз, для яснасці, --- сказаў Гары, --- калі настаўніца здолее падняць
цябе ў паветра, тата, і ты ведаеш, што да цябе не прымацаваны ніякія правады, гэта будзе 
дастатковым доказам. І ты не сказаш, што гэта такі магічны трук. Гэта будзе несумленна.
Калі ты плануеш нешта такое, скажы \emph{зараз}, і тады мы прыдумаем іншы эксперымент, які 
зможа цябе задаволіць.

Бацька Гары, прафесар Майкл Верэс-Эванс, закаціў вочы. 

--- Згодны, Гары.

--- А ты, мама, твая тэорыя ў тым, што настаўніца здолее падняць тату ў паветра. І калі раптам
у яе не атрымаецца, тады ты прызнаеш, што памылялася. І не будеш казаць, што магія
не працуе, калі ставіцца да яе скептычна, або нешта кшталту таго.

Намесніца дырэктара Мінерва МакГонагал глядзела на Гары з зацікаўленым здзіўленнем. Яна выглядала
даволі па-вядзьмарскі, у сваёй чорнай мантыі і востраканечным капялюшы. На першы позірк падавалася,
што яна проста павінна злобна і скрыпуча смяяца, запіхваючы немаўлят у кіпячы кацёл, але гэты
вобраз зруйнаваўся, як толькі яна адкрыла рот. Яна гучала разумна, фарамальна, і з
шатладнскім акцэнтам.

--- Ці будзе гэта дастаткова, містэр Потэр? --- спытала яна. --- Магу я пачаць... дэманстрацыю?

--- \emph{Дастаткова?} Магчыма не, --- сказаў Гары. --- Але, прынамсі, гэта дапаможа нам выйсці з
крызісу. Калі ласка, пачынайце, намесніца дырэктара.

--- Можаце звяртацца да мяне проста "прафесар", --- сказала яна. --- \emph{Wingardium Leviosa.}

Гары глядзеў на бацьку.

--- Хм, --- сказаў Гары.

Яго бацька пагдзяў на яго. 

--- Хм, --- эхам паўтарыў ён за сынам.

Прафесар Верэс-Эванс паглядзеў на прафесара МакГонагал. 

--- Добра, можаце апусціць мяне назад.

І ён акуратна вярнуўся на падлогу.

Гары запусціў рукі ў свае валасы і моцна за іх схапіўся. Мабыць гэта гаварыла ў ім тая частка,
якая ўжо была перакананая, але...

--- Неяк... крыху.. расчаравальна, --- сказаў ён. --- Можна было б чакаць болей драматычнай
ментальнай рэакцыі на пацверджанне гіпотэзы праз падзею бясконца малой імавернасці... --- Гары
замоўкнуў. Мама, МакГонагал, і нават тата глядзелі на яго \emph{тым позіркам}. --- Я маю на ўвазе,
рэакцыю, калі высвятляееца, што ўсё, у што я веру --- хлусня.

Сур'ёзна, гэта павінна было быць болей драматычна. Яго мозг быў павінен ужо быць у працэсе
сцірання свайго запасу тэорый аб сусвеце, якія супярэчылі магіі. Але замест таго яго мозг
бясконца паўтараў сам сабе: \emph{Добра, прафесар з Хогвартса прымусіў бацьку левітыраваць,
махнуўшы чароўнай палачкай, што далей?}

Пажылая ведзьма з райнешай зацікаўленасцю спытала:

--- Ці жадаеце вы яшчэ нейкай дэманстрацыі, містэр Потэр?

--- Дастаткова, --- сказаў Гары. --- Мы правялі даволі пераканаўчы эксперымент. Але... --- Гары
завагаўся. Ён не мог стрымацца, але ў дадзеных абставінах ён меў поўнае права на інтарэс. --- Што
яшчэ вы можаце зрабіць?

Прафесар МакГонагал пераўтварылася ў котку. Звычайную маленькую паласатую котку.

Гары адскочыў назад, і зачапіўшыся за нечаканы стосік кніг, прызямліўся на азадак з гучным
\emph{чмякам.} 

Котка імгненна змянілася назад у жанчыну ў мантыі і капялюшы. 

--- Прабачце, містэр Потэр, --- сказала МакГонагал. Голас гучаў шчыра, хаця на вуснах яе гуляла 
ледзьве заўважная ўсмешка. --- Трэба было вас папярэдзіць.

Гары дыхаў цяжка, кароткімі ўздыхамі. Ён сказаў сціснута:

--- \emph{Гэта \shout{немагчыма}!}

--- Гэта проста трансфігурацыя, --- сказала МакГонагал. --- А дакладней, трансфармацыя анімага.

--- Вы сталі коткай! \shout{Маленькай} коткай! Вы парушылі закон захавання энергіі! Гэта не 
проста эмпірычнае правіла, яго фундаментальнасць мяркуецца формай гамільтаніяна!
Яго парушэнне вядзе да адмены унітарнасці і перадачу інфармацыі хутчэй за свет!\footnote{
{}Гары згадвае аксіомы квантавай фізіцы. Гл. таксама тэарэму Нётэр.
} І каты ---
істоты \shout{складаныя!} Чалавечы розум не можа проста візуалізіраваць цалкам анатомію коткі, яе 
біяхімію... і што на конт \emph{неўралогіі}? Як вы наогул можаце \emph{думаць} мозгам такога
памеру?

Усмешка заўважней прабівалася на вусны МакГонагал. 

--- Магія.

--- Магіі проста \emph{не дастаткова} для такога парушэння! Вы павінны быць богам!

МакГонагал міргнула.

--- \emph{Так} мяне яшчэ не клікалі.

Зрок Гары затуманіўся, калі яго розум пачаў спробы зразумець маштаб наступстваў. Уся стрункая
ідэя суцэльнага сусвету з матэматычна точнымі законамі была толькі што змыта ва ўнітаз. Уся 
\emph{фізіка} за ёй. Тры тысячы гадоў скрупулёзнага раздзялення складаных вялікіх праблем
на простыя ясныя кавалачкі, адкрыццё таго, што музыка планет была той жа мелодыі, што і ў падаючага
яблыка\footnote{
{}Відавочна, Гары вывучаў гісторыю касмалагічных ідэй, ад старажытных грэкаў да Кеплера і Ньютона.
}, што галоўныя законы сусвету былі ўніверсальныя, не мелі выключэнняў, і мелі форму
простых ураўненняў, \emph{не кажучы} нават пра тое, што розум быў у мазгу, які складаўся з
нейронаў, а розум як раз і быў \emph{сутнасцю} чалавека...

А потым настае дзень, калі жанчына пераўтвараецца ў котку, і адмяняе навуку.

Сотня пытанняў біліся ў галаве Гары за кантроль над яго вуснамі, і потым з іх вырваўся пераможца:

--- І што гэта за заклён такой --- \emph{Wingardium Leviosa?} Хто выдумяе такія інкантацыі,
першакласнікі?

--- Дастаткова, містэр Потэр, --- сказала МакГонагал строга, але вочы яе блішчэлі вясёлым 
здзіўленнем. --- Калі вы жадаеце вывучаць магію, я прапаную запоўніць бумагі, і даслаць
ваш адказ у Хогвартс.

--- Дакладна, --- сказаў Гары, усё яшчэ як у трансе. Ён паспрабаваў сабраць думкі разам. Рацыянальны
падыход яму дагэтуль не здраджваў. Трэба прымяніць яго яшчэ раз, вось і ўсё. --- Тады як можна 
трапіць у Хогвартс?

Сціснуты смех вырваўся з МакГонагал, быццам яго дасталі з яе абцугамі.

--- Пачакай, Гары, --- сказаў бацька. --- Памятаеш, чаму ты не хадзіў у школу дагэтуль? Што на конт
твайго становішча?

МакГонагал павярнулся да Майкла. 

--- Якое становішча?

--- Я сплю не як усе, --- сказаў Гары і бездапаможна развёў рукамі. --- Мае біялагічныя суткі 
доўжацца дваццаць шэсць гадзін, і таму я кожны дзень засынаю на дзве гадзіны пазней. 
Я фізічна не магу заснуць раней, і таму кладуся ў 10, 12, 2, 4 гадзіны і гэтак далей па кругу.
Калі я спрабую прачынацца рана, я проста як зомбі цэлы дзень. Таму я і не хадзіў у звычайную
школу да гэтага часу.

--- Гэта адна з прычын, --- сказала маці. Гары гнеўна на яе зірнуў.

МакГонагал выдала доўгае \emph{хмммммм.}

--- Не магу прыпомніць, каб я чула нешта падобнае ў мінулым... --- сказала яна павольна. --- Я
праверу з мадам Помфрі, можа яна ведае лекі для гэтага "становішча", --- потым яе твар прасвятлеў.
--- Не, я ўпэўнена, што праблем не будзе --- мы знойдзем, як гэта пераадолець. Так, ---
і яе позірк зноў стаў пранізлівым, --- а што былі за іншыя прычыны?

Гары паглядзеў на бацькоў.

--- Я катэгарычна адмаўляюся трываць пакуты праз тое, што пастаянна дэградуючая школьная сістэма
не можа забяспечыць вучняў настаўнікамі або навучальнымі матэрыяламі нават мінімальна адэкватнай
якасці.

У адказ на гэта абодва бацькі Гары заліліся дзікім смехам, быццам тое быў неверагодны жарт.

--- Ох, --- сказаў бацька, --- і менавіта таму ты укусіў настаўніцу па матэматыцы ў трэцім класе.

--- \emph{Яна не ведала, што такое лагарыфм!}

--- Ну канешне, --- дадала маці. --- Пакусаць яе было вельмі дарослым адказам на гэта.

Бацька кіўнуў:

--- Вельмі прадуманая палітыка для вырашэння агульнай праблемы настаўнікаў, якія не разумеюць лагарыфмаў.

--- Мне было ўсяго \emph{сем}! Колькі вы яшчэ будзеце пра тое згадваць?

--- І не кажы, -- сказала маці спачувальна, --- варта толькі адзін раз ўкусіць настаўніка, і табе
гэтага ніколі не забудуць, так?

Гары павярнуўся да МакГонагал.

--- Бачыце? І вось з гэтым мне прыходзіцца жыць кожны дзень!

--- Прабачце, --- сказала Петунія, і выбежала праз сеткавыя дзверы на ганак, скуль яе дзікі
смех быў усё яшчэ вельмі добра чутны.

--- У нас, ммм, в Хогвартс, --- па нейкай прычыне МакГонагал таксама было цяжка гаварыць, ---
я не пацярплю ніякіх кусаній настаўнікаў, гэта зразумела, містэр Потэр?

Гары нахмурыўся.

--- Цудоўна, абяцаю не кусаць нікога, хто не ўкусіць мяне першым.

Прафесару Майклу Верэс-Эвансу пасля гэтых слоў таксама прыйшлося тэрмінова панікуць пакой.

--- Ну што ж, --- уздыхнула МакГонагал, калі бацькі Гары крыху супакоіліся і вярнуліся
назад. --- Думаю, што ўлічваючы ўсё, што мы сёння бачылі, нам лепей адкласці паход за 
падручнікамі і матэрыяламі на трыццатага жніўня.

--- Што? Чаму? Астатнія вучні ўжо ведаюць магію, ці не? Каб іх дагнаць, мне трэба пачачь
як мага хутчэй!

--- Не хвалюйцеся, містэр Потэр, --- адказала прафесар МакГонагал. --- Хогвартс цалкам здольны
навучыць вас асновам. І ў мяне ёсць стойкае падазрэнне, містэр Потэр, што калі я пакіну вас на
два месяцы са школьнымі кнігамі, нават без палачкі, то да трыццатага жніўня я знайду на месцы 
вашага дома дымячыйся пурпуровым дымам кратэр, спусцелы закінуты горад вакол, і бясконцыя гурты
вогненых зебраў, тэрарызуючых рэшткі краіны.

Бацька і маці Гары адначасова кіўнулі.  

“\emph{Мама! Тата!}”
