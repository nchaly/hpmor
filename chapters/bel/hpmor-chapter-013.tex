\chapter{Бязглуздыя Пытанні}


\begin{chapterOpeningQuote}
Гэта самая простая загадка, якую я калісьці чула.
\end{chapterOpeningQuote}

\lettrine{Я}{к} толькі Гары расплюшчыў вочы наступнай раніцай 
ў спальне першакурснікаў Рэйвенкло, ён зразумеў, што нешта здарылася.

Было ціха.

\emph{Надта} ціха.

А, дакладна... на ложках быў усталяваны заклён цішыні, якім можна было кіраваць 
маленькім слайдэрам над прыгалоўем --- толькі дзякуючы яму студэнты Рэйвенкло наогул маглі 
спаць.

Гары сеў і агледзеўся, чакаючы ўбачыць астатніх хлопцаў, рыхтуючыхся да...

Спальня: пустая.

Ложкі: скамячаныя і незапраўленыя.

Сонца: яскрава і высока.

Сцішальнік на яго ложку выкручаны на максімум. Гарын механічны будзільнік працаваў,
але гук быў выключаны.

Калі верыць гадзінніку, ён праспаў да 9:52. Нягледзячы на спробу сінхранізаваць 
свае 26-гадзінныя суткі з часам прыбыцця ў Хогвартс, у яго атрымалася заснуць 
толькі каля адной гадзіны ночы. Ён планаваў падняцца а сёмай раніцы з усімі 
студэнтамі, і першы дзень ён мог бы дазволіць сабе крыху не выспацца, бо яму
абяцалі нейкі магічны сродак. Але ён праспаў сняданак. І яго самы першы
ўрок у Хогвартс, Гарбалогія, пачаўся дваццаць дзве хвіліны таму.

Павольна, вельмі павольна ў ім абуджалася злосць. О, які цудоўны пранк. 
Выключыць ягоны будзільнік. Уключыць ягоны Сцішальнік. І дазволіць гэтаму 
Містэру Выскачцы Гары Потэру спазніцца на яго першы ўрок, набыўшы ганаровае 
званне першага соні Хогвартс. 

Калі Гары высветліць, хто гэта зрабіў...

Не, гэта можна было зрабіць толькі праз дамову сярод усіх дванаццаці 
жыхароў спальні Рэйвенкло. Усе яго бачылі, і ўсе яго пакінулі.

Разгубленасць і нечаганае абурэнне ў ім выціснулі злосць. Яны жа \emph{любілі} яго.
Так ён думаў. Учора. Учора ўвечары ён думаў, што народ яго любіць. Чаму ж?..

Калі Гары ўстаў з ложка, то заўважыў лісцік паперы, прымацаваны да яго 
падгалоўя.

На паперцы было напісана:

\begin{writtenNote}
\letterAddress{Мае дарагія аднакурснікі,}  % рэйвеклоўцы? бе

Гэта быў вельмі доўгі дзень. Калі ласка, дайце мне выспацца, і не хвалюйцеся 
за прапушчаны сняданак. Я \raisebox{0.15em}{\rotatebox[origin=c]{-3}{\Large не забыў}} пра заняткі.

\letterClosing[Ваш]{Гары Потэр.}
\end{writtenNote}

Гары стаяў некалькі секунд, гледзячы на запіску, і адчуваў, як па ягоных 
венах пачынае цекчы ледзяная вада.

Запіска была напісаная механічным алоўкам, ягонай уласнай рукой.

І ён гэтага не памятаў.

І... Гары прышчурыўся на паперку. Калі ён не ўяўляў сабе гэтага, словы
"не забыў" былі напісаны крыху ішакш, быццам ён хацеў штосьці сабе сказаць?

Ён што, ведаў, што на яго прыменяць забывальны заклён? Напрыклад, калі ён 
учора ўчыніў нейкае злачынства, і потым... але сам ён не ведаў заклён, значыць
нехта іншы...

Потыў у яго ўзнікла новая думка. Дапусцім, ён \emph{ведаў}, што яго чакае 
забывальны заклён...

Усё яшчэ ў піжаме Гары падбяжаў да свайго куфара, прыціснуў адзінец да замка,
дастаў махляскін, засунуў у яго руку, і сказаў "Запіска для самога сябе".

Невялікі кавалак паперы апынуўся ў ягонай руцэ.

на ім таксама была напісана Гарынай рукой:

\begin{writtenNote}
\letterAddress{Дарагі я,}
Калі ласка, не адмаўляйся ад гульні. У яе можна згуляць толькі раз 
у жыцці. Гэтая магчымасць не паўтарыцца.

Код распазнавання 927, я бульба.

\letterClosing[Твой]{ты.}
\end{writtenNote}

Гары задуменна кіўнуў. "Код распазнавання 927, я бульба" сапраўды быў 
кодам, які мог ведаць толькі ён. Гары прыдумаў яго некалькі гадоў таму,
пакуль глядзеў серыял па тэлевізару. На выпадак, калі яму прыйдзецца праверыць
ці сапраўды ён --- гэта ён, або нешта такое. Будзь падрыхтаваны.

Ў прысутнасці магіі Гары не мог давяраць гэтай звестке... але, для прастаты
згадзімся, што ён сапраўды напісаў запіску і не памятаў пра гэта.

Паглядзеўшы на яе яшчэ раз, ён заўважыў, што чарнілы праступаюць і з іншага боку.

Ён перавярнуў запіску.

Адвароты бок казаў:

\begin{writtenNoteGame}
\underline{Інструкцыя па Гульне}

ты не ведаеш правілы гульні\\
ты не ведаеш стаўкі ў гульне\\
ты не ведаеш мэты гульні\\
ты не ведаеш, хто вядучы гульні\\
і ты не ведаеш, як яе скончыць.

Ты пачынаеш Гульню са ста баламі.

Старт!
\end{writtenNoteGame}

Гары ўтаропіўся ў гэтыя "інструкцыі". Тэкст гэтага боку не быў напісаны рукой, літары былі 
абсалютна роўнымі, значыць, штучнымі. Напрыклад, яны маглі быць зробленымі з
дапамогай Прыткага Пяра; Гары набыў такое для хуткага запісу лекцый.

Ён не меў \emph{абсалютна} аніякага паняцця, што адбывалася.

Ладна... усё роўна першым крокам трэба было апрануцца і паесці. Магчыма, спачатку паесці.
Ягоны страўнік адчуваўся вельмі пустым.

Ён прапусціў сняданак, канешне, але ён быў Падрыхтаваны да такой акалічнасці.
Гары сунуў руку ў махляскін, сказаў "Мюслі", чакаючы атрымаць у руку каробку
батончыкаў, якія ён купіў яшчэ дома.

Тое, што прыгнула ў яго руку, не было падобна на каробку.

Дастаўшы руку з махляскіна, Гары ўбачыў яшчэ адну запіску, напісанаю тым жа шрыфтом,
што і "інструкцыі", да якой былі прыклеены дзве пліткіх шакаладу, 
такія маленькія, што гэта нельга было назваць нават перакусам.

Запіска казала:

\begin{writtenNoteGame}
Спроба правалена
\begin{tabular}{rl}
Вынік: & -1 бал\\
Усяго балаў: & 99\\
Фізічны стан: & Усё яшчэ галодны\\
Ментальны стан: & Разгублены
\end{tabular}
\end{writtenNoteGame}

--- Хххалера, --- Гарыны вусны сказалі без яніякага ўдзелу яго свядомасці.

Ён пастаяў так яшчэ з хвіліну.

Праз хвіліну ён \emph{ўсё яшчэ} 
не меў ніякага паняцця, што адбывалася, і ягоны мозг нават 
не мог \emph{сфармуляваць} ніякую гіпотэзу, быццам яго ментальныя рукі  
былі апрануты ў баксёрскія пальчаткі, і не маглі набраць тэкст на 
ментальнай клавіятуры. 

У страўніка былі сваі прыарытэты, што падказала напрамак для эксперыменту.

--- Эмм... --- сказаў Гары ў пусты пакой. --- Мабыць... я магу патраціць адзін 
бал, каб атрымаць маю каробку, дзе павінны ляжаць мюслі?

Цішыня.

Гары сунуў руку ў махляскін і сказаў: 

--- Каробка, дзе павінны ляжаць мюслі.

Каробка, якая апынулася ў яго руцэ, была на вомбацак правільнай формы... але яна была
занадта лёгкая. Насамрэч яна была адчыненая і пустая, калі не лічыць запіскі ўнутры:

\begin{writtenNoteGame}
\begin{tabular}{rl}
Балаў страчана: & 1\\
Усяго балаў: & 98\\
Атрымана: & Каробка, дзе павінны ляжаць мюслі
\end{tabular}
\end{writtenNoteGame}

--- Я бы хацеў патраціць яшчэ адзін бал, і атрымаць \emph{батончык мюслі}, 
--- сказаў Гары.

Зноў цішіня. 

Гары сунуў руку ў махляскін і сказаў:

--- Батончык мюслі.

Нічога.

Гары роспачна паціснуў плячыма, і пайшоў да сваёй тумбачкі, каб апрануцца.

На падлозе каля тумбачкі, пад яго мантыяй, ляжалі батончыкі мюслі і запіска:

\begin{writtenNoteGame}
\begin{tabular}{rl}
Балаў страчана: & 1\\
Усяго балаў: & 97\\
Атрымана: & 6 батончыкаў мюслі\\
Ты ўсё яшчэ апрануты ў: & піжаму\\
\end{tabular}

Не еш, пакуль апрануты ў піжаму\\
Атрымаеш Піжамны Штраф
\end{writtenNoteGame}


\emph{І цяпер я ведаю, што Вядучы Гульні --- вар'ят.}

--- У мяне ёсць здагадка, што Вядучы Гульні --- Дамблдор, --- сказаў уголас Гары.
Магчыма, ён можа паставіць рэкорд па прахаджэнню гульні ў Хогвартс?

Цішыня

Але Гары зразумеў патэрн: запіска павінна быць у наступным месцы, у якім ён
будзе шукаць. Ён заглянуў пад ложак.

\begin{writtenNoteGame}
Ха! Ха ха ха ха ха!\\
Ха ха ха ха ха!\\
Ха! Ха! Ха! Ха! Ха! Ха!\\
Дамблдор не Вядучы\\
Дрэнная здагадка\\
Вельмі дрэная здагадка\\
Мінус 20 балаў\\
І ты ўсё яшчэ ў піжаме\\
і гэта твой чацьвёрты ход\\
але ты ўсё яшчэ ў піжаме

\begin{tabular}{rl}
Піжамны Штраф: & -2 бала\\
Усяго балаў: & 75
\end{tabular}
\end{writtenNoteGame}


А вось гэта было ўжо складана... Гэта быў яго першы дзень у Хогвартс, і, калі 
выдаліць са спіса Дамблдора, ён не ведаў, хто яшчэ мог параўнацца з ім 
у вар'яцтве.

На аўтапілоце Гары сабраў у кучу свае аддзенне, выцягнуў лесвіцу ў склеп свайго 
куфара (на выпадак, калі нехта зойдзе ў спальню), сышоў у склеп, пераапрануўся, і вярнуўся ў спальню,
каб схаваць піжаму. 

Гары прыпыніўся, перад тым, як адчыніць шуфлядку, куды ён хацеў пакласці
піжаму. Патэрн павінен быў спрацаваць і тут.

--- Як я магу заработаць балы? --- сказаў ён громка.

І выцягнуў шуфлядку.

\begin{writtenNoteGame}
Магчымасці зрабіць дабро --- паўсюль\\
але цемра там, дзе павінна быць святло

\begin{tabular}{rl}
Кошт пытання: & 1 бал\\
Усяго балаў: & 74
\end{tabular}

Клёвыя труселя\\
Твая маці абірала?
\end{writtenNoteGame}

Гары змяў паперку ў кулацэ, яго твар палаў чырвоным. Да галавы прыйшла 
лаянка Драко. \emph{Бруднакроўны суччын сын...}

Зараз ён ужо ведаў, што лепей не казаць такое ўголас. Магчыма, існаваў
Штраф за Лаянку.

Гары павесіў на пояс кашэль, узяў у рукі палачку. Потым разарваў запакоўку
з аднаго батончыка мюслі, і кіную яе ў сметніцу, дзе яна прызямлілася паверх
практычна з'едзенай Шакаладнай Жабы, змятага канвэрту, і чырвона-зялёнай 
паперы. Астатнія батончікі ён паклаў у махляскін.

Ён яшчэ раз агледзеў пакой у адчайнай у дарэмнай спробе знайсці яшчэ падказкі. 

Гары пакінуў спальню, жуючы па ходу, у пошуках падзямеддя Слізэрына. 
Прынамсі, ён падумаў, што там было шмат цемры.

Хадзіць па калідорам Хогвартс было як... магчыма \emph{не} насколькі дрэнна, як 
бадзяцца па выяве Эшэра, такія рэчы звычайна кажуць дзеля рытарычнага эфекту.

Праз некаторы час Гары пачаў думаць, што выявы Эшра мелі свае плюсы і мінусы.
Мінусы: шматкірункавая гравітацыя. Плюсы: лесвіцы не змяняюцца, \emph{пакуль ты па 
ім ідзеш}.

Учора вечарам Гары прайшоў чатыры пралёты па лесвіцы да спальні першакурснікаў.
Да дванаццатым пралёце спуску ўніз без аніякіх слядоў падзямелля Гары зрабіў 
выснову, што 1) выява Эшера па складанасці проста \emph{бегавой дарожкай}
у параўнанні з Хогвартс; 2) нейкім чынам ён зараз быў \emph{вышэй}, чым яго спальня; і 3)
ён так глыбока заблукаў,  што не здзівіцца, калі, выглянуўшы ў вакно, убачыць на небе
дзве луны.

Відавочным планам Б было спыніцца і спытаць у кагосьці дарогу, але
людзі, бадзяючыяся без справы па калідорах Хогвартс чамусьці былі ў дзікім
дэфіцыту.

План В...

--- Я заблукаў, --- сказаў ён уголас. --- Ці можа... э... дух замка Хогвартс
дапамагчы мне, або як?

--- Не думаю, што ў гэтага замка ёсць дух, --- заўважыла старая ледзі з аднаго з 
патрэтаў на сцяне. --- Жыццё, магчыма, але не дух.

Некаторы час было ціха.

--- А вы... --- пачаў Гары, і потым хутка прыкрыў рот. Ён \emph{не будзе} пытаць 
выяву, ці была яна свядомая ў сэнсе ўсведамлення сваёй свядомасці.

--- Я Гары Потэр, --- сказаў ён больш-менш на аўтапілоце. І таксама аўламатычна,
ён працягнуў выяве руку.

Жанчына на партрэце кінула позірк на Гарыну руку, і бровы яе падняліся.

Павольна рука апусцілася ўніз

--- Прабачце, --- сказаў Гары, --- я тут, кшталту, новенькі.

--- Я так і зразумела, малады рэйвенкло. Куды ты хочаш трапіць?

Гары завагаўся.

--- Я не вельмі ўпэўнены.

--- Тады, магчыма, ты ўжо на месцы?

--- Хм... куды бы я ні намагаўся прысці, я не думаю, што гэтае месца --- яно... --- 
гучала даволі бязглузда. --- Дазвольце мне пачаць с пачатку. Я ўдзельнічаю ў 
гульне, але не ведаю правілаў гэтай гульні... --- гэты варыянт быў не лепей. ---
Акей, трэцяя спроба. Я шукаю магчымасць зрабіць дабро, каб атрымаць балы, і
яшчэ ў мяне ёсць таямнічая падказка пра тое, што цемра там, дзе павінна быць святло,
таму я хацеў спусціцца наніз, але нейкім чынам я паднімаюся...

Жанчына з партрэта глядзела на яго скептычна. 

Гары ўздыхнуў.

--- Мае жыццё заўсёды даволі дзіўнае.

--- Ці можам мы сказаць, што ты не ведаеш ні куды ідзеш, ні чаму ты павінен туды
трапіць?

--- \emph{Цалкам} згодны. 

Жанчына кіўнула.

--- Я не ўпэўнена, што тое, што ты заблукаў --- твая самая вялікая праблема, 
малады чалавек.

--- Так, але ў адрозненне ад самай вялікай праблемы, я магу зразумець, як 
яе вырашыць... \emph{стойце}, а наша размова, што, становіцца метафарай аб
чалавечым жыцці? Я толькі зараз заўважыў.

Ледзі паглядзела на Гары з пахвалой.

--- А ты разумны малады рэйвенкло, ці не так? На секунду я засумнявалася.
Ну, тады агульнае правіла такое: калі ты ўвесь час паварочваеш налева,
то будзеш спускацца.

Гэта гучала дзіўна знакома, але Гары не мог прыпомніць, дзе ён мог такое чуць.

--- Эмм... вы падаецеся мне вельмі разумным чалавекам... прынамсі, выявай разумнага
чалавека... у любым выпадку, вы чулі пра загадкавую гульню, у якую можна згуляць
толькі раз, і табе не расказваюць правілы?

--- Жыццё, --- адказала яна адразу. --- Гэта самая простая загадка, якую я калісьці чула.

Гары міргнуў.

--- Н-не, --- сказаў ён павольна. --- Я атрымаў сапраўдную запіску, якая кажа, што 
я павінен гуляць, але правілы мне не скажуць, і нехта пакідае мне па ходзе гульні 
новыя запіскі, дзе кажацца, колькі балаў я згубіў за парушэнні, кшталту,
мінус два за тое, што быў у піжаме. Вы ведаеце кагосьці ў Хогвартс, 
хто дастаткова звар'яцелы і магутны, каб зрабіць нешта такое?
У сэнсе, акрамя Дамблдора?

Яна ўздыхнула.

--- Я ўсяго толькі партрэт, малады чалавек. Я памятаю Хогвартс такім, якім ён быў --- не
такім, які ён зараз. Усё, што магу сказаць, што калі б тое была загадка, то адказ ---
жыццё, і калі мы не кантралюем усе правілы, то прынамсі той, хто дае або забірае балы ---
гэта мы самі. Калі гэта не загадка, то не ведаю.

Гары нізка пакланіўся выяве. 

--- Дзякуй вам, міледзі

Ледзі зрабіла рэверанс у адказ.

--- Я бы хацела сказаць, што запомню тваю ветлівасць, --- сказала яна, --- 
але магчыма я забуду цябе адразу. Удачы табе, Гары Потэр.

Ён пакланіўся яшчэ раз, і пачаў спускацца па бліжэйшай лесвіцы.

Праз чатыры павароты ён апынуўся ў калідоры, які раптам скончыўся насыпам з 
вялікіх камянёў --- быццам, у выніку абвалу, але сцены і столь былі непашкоджаныя.

--- Ну, добра, --- сказаў Гары ў пустату. --- Я здаюся. Мне трэба яшчэ адна падказка. 
Як мне трапіць туды, куды я хачу трапіць?

--- Падказка! Падказка! Хто пытаў пра падказку?!

Узбуджаны голас ішоў з выявы на сцяне недалёк да Гары: то быў партрэт чалавека
сярэдняга ўзросту ў неймаверна яскравай ружовай мантыі. На галаве ў яго была стары 
востраканечны капялюш з рыбай (не выявай, а з сапраўднай рыбай).


--- Так! --- сказаў Гары. --- Падказка! Я пытаў пра  падказку! І не проста 
абы-якую падказку, я пытаўся пра вельмі \emph{канкрэтную} падказку, 
на конт гульні, у якую я гуляю...

--- Так, так! Падказка ў гульню! Ты ж Гары Потэр, ці не так? Я Карнэліан Флаберволт!
Я пачуў ад Аляксандры Вялікай, якой расказаў Лорд Уізелноуз, якому расказаў... ну няважна.
Але гэта была звестка для мяне, каб я перадаў табе! Для мяне! Я не быў патрэбны нікому 
бог ведае колькі, мабыць вечнасць, бо я заграс тут, у нікому не патрэбным цёмным калідоры!..
Дык гэта! Падказка! У мяне ёсць падказка для цябе! Гэта будзе каштааваць табе толькі
тры бала! Бярэш?

--- Неверагодная ўдача! --- сказаў Гары з  сакрыстычным энтузіязмам, які варта было б
не паказваць, але ён не мог стрымацца. --- Бяру!   

--- Ты знойдзеш цемру паміж зялёным пакоем самападрыхтоўкі і кабінетам трансфігурацыі МакГонагал!
Гэта падказка! І давай ногі ў рукі! Хопіць валаводзіць валадоды! Мінус дзесяць балаў
за маруднасць! У цябе 61 бал! Гэў канчатак звесткі!

--- Дзякуй, --- сказаў Гары. Ён і сапраўды патраціў шмат часу. --- Эмм... я 
мяркую, ты не ведаеш, ад каго першапачаткова прыйшла гэтая звестка?

--- Атож! Яе абвясціў загробны голас, які пачуўся з разлому ў прасторы нашага свету,
з разлому, які адчыняўся напрамкі ў вогненую бездань! Так мне казалі!

Гары не быў ўпэўнены, ставіцца да гэтага скептычна, або проста прыняць на ўвагу. 

--- Як знайсці той зялёны пакой?

--- Проста! Павяртай назад, і ідзі налева, направа, уніз, уніз, направа, налева, направа,
уверх і зноў налева, і прыйдзеш на зялёнага пакоя. Калі зойдзеш у пакой, выходзь 
з яго праз дзверы ў супрацьлеглым канцы, апынешся ў доўгім змяістым калідоры, на 
першым скрыжаванні --- направа, потым прама, і там будзе кабінет трансфігурацыі! ---
ён задумаўся. --- Прынамсі, так было, калі я вучыўся ў Хогвартс. Сёння ж 
панядзелак няцотнага году, ці не так?

--- Алову і механічны паперак, --- сказаў Гары свайму махляскіну. --- Э...
адмена. Паперу і механічны аловак, --- ён глянуў на партрэт. --- Можаце гэта паўтарыць?

Пасля яшчэ дзвюх спроб згубіцца, Гары пачаў адчуваць, што разумее асноўнае
правіла арыентацыі ў гэтым бясконца зменлівым лабірынце пад назвай "Хогвартс",
а менавіта: \emph{пытай дарогу ў выяваў}. Галі гэта вызначала нейкій невыразна
глыбокі жыццёвы ўрок, ён не мог уявіць сабе, чаму ён павінен быў вучыць.

Зялёны пакой для самападрыхтоўкі апынуўся вельмі ўтульным месцам. Сонечнае святло 
струменямі падала праз вокны з зялёным мазаічным шклом, якія складалі выявы драконаў
сярод спакойных пастаральных краявідаў. Тут былі вельмі камфортныя на выгляд фатэлі,
і сталы, якія добра бы падышлі, каб вывучаць нешта ў кампаніі пары-тройкі сяброў.

Гары не мог проста прайсці праз пакой да выхаду, бо там былі кніжныя паліцы, 
і ён быў павінен падысці і прачытаць прынамсі назвы некаторых кніг, каб 
не страціць права насіць фамілію Верэс. Але, памятаючы пра штраф за маруднасць,
ён зрабіў гэта даволі хутка, і выйшаў з пакою.

Ён ішоў уздоўж "змяістага" калідора, калі пачуў, як дзіцячы голас нешта крыкнуў.

У такія часы Гары меў апраўданне бегчы з усіх ног, не думаючы пра захаванне энергіі,
правільную размінку, або магчымасць у штосьці ўрэзацца. Яго раптоўная прабежка
скончылася таксама раптоўна, калі ён амаль не ўбег у групу з шасці 
хафлпафаў-першакурснікаў...

...якія збіліся ў кучу, і выглядалі даволі напужанымі, і так, быццам яны хацелі 
нешта зрабіць, але не ведалі што --- верагодна, гэта было звязана з групай 
пяці старэйшых слізэрынаў, якія абкружылі сёмага першакурсніка. 

Гары раптам адчуў дзікую злосць.

--- Зрабіце ласку! --- крыкнуў Гары з усёй моцы сваіх лёгкіх.

Гэта, шчыра кажучы, было не абавяскова. Усе і так ужо на яго ўтаропіліся.
Але пасля яго слоў усе яшчэ і замерлі.

Гары прамінуў  гурт хафлпафаў і наблізіўся  да слізэрынаў.

Тыя глядзелі на яго ў дыяпазоне выразаў ад гневу да забаўкі да захаплення.

Частка Гарынага мозу вішчэла ў паніцы, што гэтыя старэйшыя і больш вялікія хлопцы
зараз яго тут і размажуць.

Другая частка заўважыла суха, што тых, каго паймаюць з размазваннем Хлопца-Які-Выжыў,
чакаюць вялікія праблемы, асабліва, калі яны былі кіпкай старэйшых слізэрынаў і
сем хафлпафаў былі сведкамі. Шанцы, што яны свядома сур'ёзна яго пашкодзяць, імкнуліся
да нуля. Адзінай сапраўднай зброяй слізэрынаў у гэтай бойке быў яго ўласны страх,
калі ён гэта дазволіць.

Потым Гары зразумеў, што хлоцам у сярэдзіне кола быў Нэвіл Лонгботам.

Гэта ўсё вырашыла. Калі Гары вырашыў папрасіць прабачэння ў Нэвіла, то Нэвіл 
належыў \emph{яму}, як яны наогул \emph{пасмелі?}

Гары схапіў Нэвіла за руку і \emph{выдзернуў} яго з цэнтра кола слізэрынаў, 
і адным плаўным рухам увапхуў сябе на ягонае месца.

Ён стаяў на месцы Нэвіла, сярод слізэрынаў, гледзячы ўверх на старэйшых і мацнейшых 
за яго хлопцаў.

--- Дароўкі, --- сказаў Гары. --- Я --- Хлопец-Які-Выжыў.

Пасля гэтага наступіла даволі недарэчная пауза. Падавалася, што ніхто не ведаў,
куды гэтая размома павінна была ісці зараз.

Гары глянуў уніз і ўбачыў там раскіданыя па падлозе кнігі і паперы. Ну так, 
стары трук, калі ахвяра павінна збіраць свае кнігі, якія можна выбіць адразу, калі ён
скончыў. Гары не мог прыпомніць, ці быў ён калісьці аб'ектам гэтага труку, але 
ён меў добрае ўяўленне, і тое, што ён уяўляў, вельмі яго злавала. Ну, калі ён будзе 
адцягваць увагу слізэрынаў дастаткова доўга, у Нэвіла будзе шанец вярнуць свае кнігі.

На жаль, ягоныя позіркі не прайшлі незаўважна.

--- О-хо-хо, --- сказаў самы вялікі з хлопцаў, --- нашая малеча хоча пагуляцца з 
кні-ігамі...

--- Зяпу забі, --- скзаў Гары холадна. \emph{Збівай іх з раўнавагі.
Не рабі тое, чаго яны чакаюць. Не паўтараў патэрнаў, які выклікае жаданне
новага булінгу.} --- Гэта ў вас такі вельмі хітры спосаб атрымаць новыя факультэтныя балы?
Бо выглядае як 
бязглуздая ганьба імі Салазара Слізэрына, як я...

Той моцна піхнуў Гары, так, што ён разцягнуўся на цвёрдай каменнай
падлозе, выляцеўшы з іх кола.  

Слізэрыны зарагаталі.

Гары падняўся з рухам, які падаваўся яму жудасна замедленай кіназдымкай. 
Ён яшчэ не ведаў, як карыстацца сваёй палачкай, але ўлічваючы абставіны,
гэта было недастатковай прычынай не пасправаць сваю магію.

--- Я бы хацеў заплаціць \emph{столькі балаў, сколькі спратрэбіцца}, каб 
пазбавіцца ад гэтага чалавека, --- сказаў Гары, указваючы пальцам на верхавода 
слізэрынаў.

Пасля гэтага ён падняў іншую руку.

--- АБРАКАДАБРА! --- сказаў ён і цокнуй пальцамі.

На гуках "абра" двое хафлпафаў, уключаючы Нэвіла, закрычалі, трое слізэрынаў 
адпрыгнулі ў бок ад напрамку, куды указваў палец Гары, іх верхавода з шакаваным 
выразам адступіў на некалькі крокаў назад, і тут з гучным чмякам ў яго твар прыляцела нешта,
пакінушы чырвоныя плямы на шыі і грудзях. 

\emph{Такога} Гары не чакаў.

Павольна галоўны слізэрын падняў руку і адарваў са свайго твара талерку, на якой
да прылёту, відавочна, знаходзіўся вішнёвы пірог. Некаторы час ён ашалела глядзеў на яе,
і потым кінуў на падлогу.

Гэта быў самы дрэнны час у свеце, які нейкі хафлпаф мог абраць для смеху, але так 
менавіта і здарылася.  

Гары заўважыў, што на дне талеркі была запіска.

--- Чакай, --- сказаў ён, і кінуўся наперад узяць яе. --- Мне падаецца, там 
ёсць запіска...

--- \emph{Ты,} --- зарычэў верхавода слізэрынаў, --- \emph{зараз... я... цябе...}

--- Не, ну ты толькі \emph{глянь!} --- крыкнуў яму Гары, размахваючы запіскай. --- 
Я сур'ёзна, глядзі сюды! Ну як можна!? Вось як можна спісаць трыццаць балаў за 
дастаўку і прымяненне аднаго чортавага пірага? 30 балаў? Гэта нават выратаваўшы 
непавіннага хлопца ад булінга? І... "плата за захоўванне"?! Няўстойка? Растаможка?..
Навошта пірагу \emph{растаможка?} 

Зноў павілса нягеглая пауза. Гары пачаў думаць смяротныя думкі на конт хафлпафа,
які ўсё не мог спыніць свой дурны смех, гэты ідыёт наклікае на сябе ліха.

Гары кінуў у галоўнага слізэрына свай самы грозны позірк. 

--- Заваз валіце адсюль, або я буду рабіць вашае існаванне ўсё больш і больш 
сюррэальным. Зразумеў?

Адзіным жудасна хуткім рухам слізэрын выкінуў руку ў бок Гары, і ў той самы момант 
у яго твар прыляцеў новы пірог, гэтым разам з чарніцамі.

Запіска на дне гэтага пірага была вялікая і добра чытаемая. 

--- Табе варта прачытаць гэтую запіску, --- заўважыў Гары. --- Думаю, гэтым разам 
яна для цябе.

Слізэрын зняў талерку, перавярнуў яе, ад чаго рэшткі пірага з вільготнымі гукамі ўпалі
на зямлю, і прачытаў:

\begin{writtenNoteGame}
\MakeUppercase{\underline{Заўвага!}}

\MakeUppercase{Ніякая} магія не можа выкарыстоўвацца\\
на гэтым удзельніку, пакуль ідзе Гуьня\\
Далейшае ўмяшальніцтва ў Гульню\\
\MakeUppercase{будзе} даведзена да ведама Адказных Органаў
\end{writtenNoteGame}

Выраз неверагоднай разгубленасці на яго твары быў мастацкай вяршыняй. 
Гары пачынаў падабацца Вядучы гэтай Гульні.  

--- Слухай, --- сказаў Гары, --- чаму б нам не разысціся мірна? Я думаю, рэчы
ўжо пачынаюць выходзіць з-пад кантролю. Прапаную вам пайсці ў свае падзямелле, а я
пайду ў сваю вежу, і мы ўсе крыху паастынем, што скажаш?

--- У мяне ёсць ідэя лепей, --- прасычэў слізэрын. --- Што скажаш на конт 
таго, што ўсе твае пальцы выпадкова пераламаюцца?

--- Як, дзеля Мерліна, можна зымітаваць няшчасны выпадак пасля таго, як ты 
ўголас паграджаеш мне перад тузінам сведкаў, ты, \emph{ідыёт...}

Верхавода слізэрынаў павольна, паказальна без спешцы, узяў руку Гары, які застыў 
на месцы, частка яго розуму білася ў істэрыцы  \scream{Якогачортаяраблю?}

--- Стой, --- сказаў адзін з іншых слізэрынаў, у ягоным голасе былі панічныя ноткі, --- 
не трэба!..

Верхавода, праігнараваўшы гэта, узяў моцна руку Гары ў сваю левую руку, а 
правай ахапіў ягоны паказальнік.

Гары глядзеў яму прама ў вочы. Тая частка Гары, што крычала, працягвала кшталту,
такое не павінна адбывацца, такому \emph{нельга} было адбывацца, дарослыя ніколі 
не \emph{дазволяць}, каб нешта такое... 

Паволі слізэрын пачаў адгібаць палец Гары.

\emph{Ён яшчэ не зламаў мой палец, і гэта ніжэй мяне --- тузацца да таго моманту. 
Пакуль нічога не здарылася, гэта чарговая спроба мяне спужаць.}

--- Гэта вельмі дрэнная ідэя! --- крыкнуў той слізэрын, што пратэставаў і дагэтуль.

--- І я вельмі з гэтым пагаджаюся, --- сказаў новы голас. Ледзяны жаночы голас.

Той, што трымаў Гары, кінуў ягоную руку і адскочыў, быццам апекшыся.

--- Прафесар Спраут! --- закрычаў адзін з хафлпафаў, так задаволена, як Гары ніколі 
не чуў у жыцці.

У Гарыным палі зроку з'явілася тоўсценькая маленькая жанчына з бязладна 
закручанымі сівымі валасамі і з плямамі зямлі на мантыі. Яна абвінаваўча
тыкнула пальцам у бок слізэрынаў. 

--- Я патрабую тлумачэння, --- сказала яна. --- Што вы тут робіце з маімі хафлпафамі
і... --- яна кінула позірк на яго, --- маім выдатынм студэнтам, Гары Потэрам?

\emph{Ой-ёй. Дакладна, гэта яе заняткі я прапусціў раніцай.}

--- Ён пагражаў нас забіць! --- крыкнуў адзін са слізэрынаў.

--- Што? --- сказаў Гары разгублена. --- Я не пагражаў! Калі я бы хацеў вас забіць,
я бы не казаў пра гэта пры сведках!

Трэці слізэрын знервавана засмяяўся, і потым раптам спыніўся, убачыўшы як на яго
паглядзелі астатнія.

Прафесар Спраут паглядзела на іх скептычна. 

--- І якую менавіта смяротную пагрозу вы маеце на ўвазе?

--- Забівальны заклён! Ён удаваў, што кастуе забівальны заклён на нас!

Прафесар Спраут павярнулася, каб паглядзець на Гары.

--- Ммм... даволі жахлівая пагроза з боку адзінаццацілетняга першакурсніка.
Аднак, не варта нават \emph{думаць}, каб удаваць такое ў будучыні, Гары Потэр. 

--- Я нават не ведаю слоў забівальнага заклёна, --- адразу адказаў Гары. ---
І ўвесь час у мяне не было ў руках палачкі...  

Зараз Гары атрымаў ад прафесара скептычны позірк.

--- Значыць, гэты вучань \emph{сам} кінуў сабе ў твар два пірага.

--- У яго не было палачкі! --- падтрымаў Гары адзін з хафлпафаў. --- Не ведаю, 
як ён гэта зрабіў... ён проста цокнуў пальцамі, і пірог сам прыляцеў!

--- Да няўжо, --- сказала прафесар Спраут пасля паузы. Яна дастала сваю палачку. ---
Гэта не загад, калі вы тут у ролі ахвяры, але вы не супраць, калі я 
праверу вашу палчку?

Гары дастаў палачку.

--- Што мне?...

--- \emph{Prior Incantato,} --- сказала Спраут, і потым нахмурылася. --- Дзіўна,
выглядае, што гэтую палачку ніколі не карысталі.

Гары паціснуў плячыма. 

--- Не карысталі, так. Я набыў яе разам з падручнікамі пару дзён таму.

Яна кіўнула:

--- Гэта бясспрэчны прыклад выпадковай магіі з боку хлопца, які адчуваў пагрозу.
І правілы чотка кажуць, што вашай віны ў гэтым няма. А на \emph{ваш} конт... ---
яна павярнулася да слізэрынаў. Яе позірк дэманстратыўна ўпаў на кіпу кніг Нэвіла, 
якія ўсё яшчэ ляжалі на падлозе.

Пауза расцягнулася надоўга, пакуль яна разгледжвала слізэрынаў.

--- Тры бала са Слізэрына \emph{на кожнага}, --- нарэшце сказала яна. --- І шэсць 
з яго, --- яна паказала на хлопца з пірагамі. --- \emph{Ніколі} больш не смейце чапляцца
да маіх падапечных, і майго файнага студэнта Гары Потэра. Свабодны!

Слізэрыны павярнуліся і хутка пайшлі прэч.

Нэвіл падышоў і пачаў збіраць свае кнігі. Падавалася, што ён плача, але нясільна.
Магчыма, ад сышоўшага шоку, або ад падзякі, што астатнія пачалі яму дапамагаць.

--- Вялікі вам дзякуй, Гары Потэр, --- сказала яму прафесар Спраут. --- Сем 
балаў да Рэйвенкло, адзін за кожнага студэнта, якога вы абаранілі. І больш мне няма 
чаго сказаць.

Гары міргнуў. Ён чакаў больш працяглай лекцыі на тэму, як важна не лезці ў праблемы,
або строгага пакарання за тое, што прапусціў яе першы ўрок. 

Можа яму варта было абраць Хафлпаф. Спраут была крутая.

--- \emph{Scourgify,} --- сказала яна, паказаўшы палачкай на рэшткі пігароў на падлозе,
якія хутка іспрарыліся.

Пасля гэтага яна сышла ў кірунку зялёнага пакоя.

--- \emph{Як} ты зрабіў гэта? --- з падазрэннем спытаў адзін з хафлпафаў, калі яна знікла з вачэй.

Гары замазадаволена ўсміхнуўся.

--- Я магу зрабіць што заўгодна, проста цокнуўшы пальцамі.


--- Праўда? --- вочы хлопца пашырыліся.

--- Не, --- сказаў Гары. --- Але калі будзеце расказваць пра гэта, абавяскова 
падзяліцеся гэтым з Герміёнай Грэнджэр, рэйвенкло першага курса, у вас будзе шмат 
што абмеркаваць, --- нягледзячы на тое, што сам не разумеў ні кроплі, што адбываецца,
ён не мог упусціць магчымасць дадаць імпульс сваёй лягендзе. --- О, і 
чаму ўсе так перапалашыліся на конт забівальнага заклёна?

Хлопец паглядзеў на яго дзіўна.

--- Ты праўда не ведаеш?

--- Калі б я ведаў, то не пытаў бы.

--- Словы для забівальнага заклёну, --- хлопец зглынуў, ягоны голас сцішыўся да шэпта, 
і ён развёў рукі ў бакі, ясна паказваючы, што ў яго няма намеру хапацца за палачку, --- 
\emph{Avada Kedavra.}

\emph{Ну, канешне!}

Гары дадаў гэта да спіса рэчаў, пра якія ён ніколі не раскажа свайму бацьку, прафесару
Майклу Верэс-Эвансу. Горш за байку, што ты быў адзіным чалавекам у гісторыі, які перажыў
забівальны заклён, было прызнанне таго, што забівальны заклён гучаў як "абракадабра".

--- Зразумела, --- сказаў Гары. --- Дагаварыліся, гэта апошні раз, калі я казаў такое, перад тым як
цокнуць пальцамі, --- хаця яно і зрабіла тактычна выгадны эфект.

--- Чаму ты...

--- Я вырас з магламі, а ў іх гэта такі жарт. Сур'ёзна, я хацеў проста пажартаваць.
Прабач, але можаш мне напомніць свае імя?

---  Эрні Макмілан, --- сказаў хафлпаф і працягнуў руку. --- Вялікі гонар пазнаёміцца з табой.

Раптам і астатнія стаўпіліся вакол, каб пазнаёміцца.

Калі з гэтым было скончана, Гары з цяжкасцю зглынуў. Зараз яго чакала сапраўды цяжкая справа.

--- Эм... калі ласка, прабачце... Мне трэба сказаць пару слоў Нэвілу.

Усе павярнуліся да Нэвіла, які зрабіў крок назад, на яго твары спалоханы выраз. 

--- Думаю, --- сказаў Нэвіл ціха, --- ты скажаш, што мне трэба было быць смялей...

--- Не, нічога падобнага, --- сказаў хутка Гары. --- Нічога такога. Гэта проста...
Размеркавальны Капялюш сказаў мне...

Усе вакол адарзу навастрылі вушы, акрамя Нэвіла, якому стала яшчэ страшней.

Нешта блакавала горла Гары. Ён ведаў, што такія рэчы лепей выпальваць адразу, 
але падавалася, што ён праглынуў вялкую цэглу, якая застрагла ў горле і перашкаджала
гаварыць.

Гары прыйшлося ўзяць свае вусны пад кантроль і прымусіць іх сказаць кожны склад 
паасобку.

--- Пра бач мя не, --- ён выдахнуў і зноў удыхнуў. --- За тое, што... эм... я зрабіў...
учора. Ты не павінен быць удзячны, або нешта... і я зразумею, калі ты будзеш мяне проста
ненавідзець. І не тое, каб я намагаўся выглядаць крута, бо папрасіў прабачэння. То было 
дрэнна.

Пауза.

Нэвіл сільней прыціснуў кнігі да сябе.

--- Чаму ты гэта зрабіў? --- пачуўся ягоны тонкі дрыжачы голас. Ён міргнуў,
намагаючыся ўтрымаць слёзы. --- Чаму ўсё такое робяць са мной, нават Хлопец-Які-Выжыў?

Гары адчуў сябе такім нікчэмным, як ніколі ў жыцці. 

--- Прабач, --- паўтарыў ён хрыпла. --- Проста... ты выглядаў такім спалоханым, 
і гэта як плакат з надпісам "ахвяра"... і я хацеў паказаць, што рэчы не заўсёды 
канчаюцца дрэнна... што монстры часам могуць пачаставаць цябе... я думаў, што 
пасля ты зразумееш, што няма было чаго \emph{настолькі} баяцца...

--- Але \emph{ёсць} чаго баяцца, --- прашаптаў Нэвіл. --- Ты сам зараз бачыў, ёсць чаго!

--- Яны  не зрабілі бы нічога сапраўды дрэннага перад сведкамі. Іх галоўная зброя ---
твой страх. Таму яны і нападаюць на цябе, бо яны бачаць, што ты баішся. 
Я хацеў паказаць, што твой страх горш за саму падзею... прынамсі я так казаў сабе.
Але Капялюш сказаў, што падманываў сам сябе, і што насамрэч я гэта зрабіў, там 
што было весела. Таму я і прашу прабачэння...

--- Ты мне руку выкруціў, --- сказаў Нэвіл, --- калі пацягнуў мяне... --- 
ён падняў руку і паказаў на прыдалонне, дзе Гары яго схапіў. --- Яно баліць больш за 
тое, што рабілі слізэрыны, а яны проста таўкаліся.

--- Нэвіл! --- ахнуў Эрні. --- Ён намагаўся цябе выратаваць!

--- Выбачай, --- прашаптаў Гары. --- Калі я цябе ўбачыў там... я так...
уззлаваўся...

Нэвіл глядзеў на яго ўважліва.

--- І таму ты амаль не зламаў мне руку, заняў мае месца і завёў свае "Я Хлопец-Які-Выжыў".

Гары кіўнуў.

--- Думаю, аднойчы ты будзеш сапраўды круты, --- сказаў Нэвіл, --- але зараз ты 
зусім не круты.

Гары праглынуў раптоўны камок у горле, павярнуўся і пайшоў ад іх. Ён ішоў, пакуль
не дайшоў да скрыжавання, павярнуў налева, і працягнуў ісці, не вельмі шмат чаго 
заўважаючы вакол.

Што яму трэба было зрабіць, каб усё было правільна? Не зліцца? Ён не быў упэўнены,
што нешта можна было зрабіць без гэтага, і хто ведае, чым бы скончылася справа 
з Нэвілам і ягонымі кнігамі. Акрамя таго, Гары чытаў шмат прыгод і фантастыкі, 
каб ведаць, як сканчваліся такія сцэнары. Ён паспрабуе прыціснуць свой гнеў,
і не зможа, і ягоны гнеў будзе вяртацца. І толькі скончыўшы доўгі шлях 
самапазнання ён у самым канцы гысторыі зразумее, што гнеў --- ягоная 
натуральная частка, і што толькі прыняўшы яго, чалавек можа навучыцца 
ўжываць яго на карысць. Адзіным сусветам, дзе героі былі павінны цалкам 
адмежаваць сябе ад негатыўных эмоцый, былі "Зорныя Войны", недарэмна Гары
адчуваў моцную пагарду да гэтай зялёнай краказябы Ёды. 

Таму відавочным спосабам эканоміі часу было цалкам прапусціць  доўгі шлях 
самапазнання, і адразу перайсці да часткі, дзе ён прыймае свой гнеў і можа ім
кіраваць.

Праблема была ў тым, што ён \emph{не адчуваў,} што губляе кантроль, калі злуецца.
Халодны гнеў даваў яму яснасць, даваў адчуванне, што ён цалкам валодае сітуацыяй.
І толькі калі ён глядзеў назад, калі ўсё скончылася, \emph{сітуацыя цалкам}
выглядала як дзікі звер, якая толькі і мроіла, што вырвацца з-пад кантролю.

Было цікава, наколькі такія рэчы хвалявалі Вядучага Гульні, і ці мог Гары 
заработаць або згубіць на гэтым балы. Сам Гары адчуваў, што згубіў, і 
ён быў упэўнены, што старая ледзі з партрэта сказала б, што толькі яго меркаванне 
і мае сэнс урэшце рэшт. 

І яшчэ Гары думаў, ці мог Вядучы паслаць прафесара Спраут. Думка была лагічная:
адна з запісак пагразіла абвясціць "Адказныя Органы", а потым з'явілася настаўніца.
Магчыма яна і была Вядучай... ---  \emph{главу факультэта Хафлпаф} будуць падазраваць 
у апошнюю чаргу, і адно гэта прымушала Гары памесціць яе на самы верх спіса падазраваных.
Ён калісці таксама прачытаў некалькі дэтэктыўных гісторый.

--- Ну дык, як у мяне статус у Гульне? --- сказаў Гары ўголас.

Кусок паперы праляцеў над яго галавой --- быццам нехта кінуў яго з-за спіны Гары, ---
Гары рэзка абярнуўся, але там нікога не было...

Новая запіска казала: 


\begin{writtenNoteGame}
\begin{tabular}{rl}
Балаў за стыль: & 10\\
Балаў за разважлівасць: & -3,000,000\\
Бонус за балы для факультэта: & 70\\
Усяго балаў: & -2,999,871\\
Засталося хадоў: & 2
\end{tabular}
\end{writtenNoteGame}

--- \emph{Мінус тры мільёна?} --- сказаў Гары з абурэннем пустому калідору. --- 
Гэта ўжо занадта! Я хачу падаць скаргу Адказным Органам! Якім чынам я павінен 
адрабіць тры мільёны балаў за два хады?

Яшчэ адна запіска пераляцела цераз яго галаву.

\begin{writtenNoteGame}
\begin{tabular}{rl}
Зварот да Адказных Органаў: & Правалена\\
Бязглуздыя Пытанні: & -1,000,000,000,000 балаў\\
Усяго балаў: & -1,000,002,999,871\\
Засталося хадоў: & 1
\end{tabular}
\end{writtenNoteGame}

Гары здаўся. За адзіны застаўшыся ход ён мог выдаць толькі сваю найлепшую здагадку,
хоць і гучала крыху пафасна:

--- Я думаю, што гэтая Гульня прадстаўляе сабой Жыццё.

Прыляцеўшая фінальная запіска казала:

\begin{writtenNoteGame}
Спроба правалена\\
Правал Правал Правал\\
Ппрррааааааааааавввваааааалллллл\\
Усяго балаў: мінус бясконцасць\\
\MakeUppercase{Ты памёр}

\hpFontFasthand{}Апошні загад:\\
\emph{ідзі ў кабінет МакГонагал}
\end{writtenNoteGame}

Апошняя строчка была напісна ягонай рукой.

Некаторы час Гары глядзеў на апошнюю строчку, потым паціснуў плячыма. Цудоўна.
Кабінет МакГонагал, так кабінет МакГонагал. Калі гэта \emph{яна} была Вядучай...

Шчыра кажучы, ён не меў абсалютна ніякага паняцця, як ставіцца да таго, калі 
МакГонагал была Вядучай. Ягоны розум быў цалкам пусты. А яму падавалася, што 
такое немагчыма.

Праз некалькі парадаў ад партрэтаў --- шлях быў не вельмі далёкі, бо яе кабінет быў
побач з кабінетам для заняткаў па трансфігурацыі (прынамсі па панядзелкам няцотных
гадоў), --- Гары стаяў перад патрэбнымі дзвярыма.

Ён прагрукаў.

--- Увайдзіце, --- сказаў прыглушаны голас прафесара МакГонагал.

Гары ўвайшоў.
