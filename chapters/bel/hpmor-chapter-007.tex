\chapter{Узаемадапамога}

\begin{chapterOpeningQuote}
    Твой бацька амаль такі ж круты, як і мой.
\end{chapterOpeningQuote}


\lettrine{В}{усны} Петуніі Эванс-Верэс дрыжалі, а ў вачах стаялі слёзы, пакуль
Гары абдымаў яе за сярэдзіну тулава на платформе №9 вакзала Кінгз-Крос.

--- Ты ўпэнены, што не хочаш, каб я пайшла з табой, Гары?

Гары паглядзеў уверх на яе, кінуў позірк на бацьку, які выглядаў стэрэатыпова
сурова-ганарыста, і потым назад на маці, якая выглядала даволі... разгубленай.

--- Мам, я ведаю, што ты не вельмі любіш іх свет, і ты не павінна ісці са мной.
Я гэта шчыра...

Петунія шмыгнула носам.

--- Гары, не хвалюйся за мяне. Я твая маці, і калі табе патрэбны хтосьці...

--- Мама, я еду ў Хогвартс на некалькі \emph{месяцаў}. Калі я самастойна не 
адолею нейкую платформу, то лепей высветліць гэта адразу, і наогул нікуды не ехаць, ---
ён сцішыў голас да шэпту: --- І дарэчы, мам, я там знакамітасць. Калі ў мяне 
будуць праблемы, мне трэба толькі зняць павязку, --- Гары пахлопаў па спатрыўнай
павязцы, якая закрывала яго шнар --- кшталу тых, што носяць тэнісісты, --- і 
ў мяне будзе памочнікаў вышэй вушэй. 

--- О, Гары, --- прашаптала Петунія. Яна стала на калена, і моцна яго абняла, 
прыціснуўшыся шчакой да шчакі. Гары адчуваў яе няроўны подых, і потым
пачуў, як усхліп сляцеў з яе вунснаў, сціснуты, ціхі, але ўсхліп. 

--- О, Гары, мілы, я люблю цябе, памятай пра гэта.

\emph{Быццам яна баіцца, што болей ніколі мяне не ўбачыць,} --- нечаканая
думка ўсплыла ў Гарынай галаве. Гэта, напэўна, так і было, але ён не разумеў,
чаму яна была так напужана.

Таму ён пасправаў угадаць.

--- Мам, ты жа ведаеш, што я не стану як твая сястра? Дзеля цябе я зраблю любую 
магію --- калі змагу, --- або калі ты не 
хочаш ніякай магіі, я таксама паслухаю. Я абяцаю, што магія ніколі не 
стане паміж намі...

Моцныя абдымкі перервалі яго словы.

--- У цябе добрае сэрца, --- прашаптала маці яму на вуха. --- Вельмі добрае 
сэрца, сынок.

Тады і ў Гары ўзнік камок у горле.

Маці адпусціла яго, і паднялася. Яна дастала з кішэні насоўку, і 
дрыжачай рукой пачала прамакаць вочы і паплыўшы макіяж.

Пытанне на конт яго бацькі і не стаяла. Яму было складана нават паглядзець
на Гарын куфар ва ўпор. Калі магія была бы нечым кшталту прамахаджэння, то
Майкл Верэс-Эванс мог бы толькі ціха поўзаць.

Таму яго бацька проста сказаў:

--- Гары, удачы табе ў школе. Ну як, дастаткова кніг я набыў?

Гары змог-такі патлумачыць яму пра свае ідэі на конт зрабіць нешта сапраўды 
важнае, а магчыма і рэвалюцыйнае, і прафесар Верэс-Эванс кіўнуў, і высвабадзіў
свой вельмі загружнаны расклад заняткаў на цэлых два дні ўрад, каб здейсніць 
Найвялікшы Набег на Кнігарні ў Гісторыі, які пакрыў чарыры гарады і скончыўся
дастаўкай  \emph{трыццаці} каробак з навуковымі кнігамі, якія зараз знаходзіліся 
ў склепе Гарынага куфара. Большасць кніг каштавала фунт-два, але некаторыя 
дакладна абышліся ў капеечку, напрыклад свежае выдане  
\emph{Дапаможніка па Хіміі і Фізіцы} або поўны набор \emph{Encyclopaedia Britannica}
1972 году выдання. Бацька паспрабаваў схаваць цэннікі ад Гары, але Гары 
прыкінуў, што на ўсё сышло не менш за тысячу фунтаў. Гары сказаў, што 
абавяскова верне яму доўг, калі высветліць, як перавесці магічнае золата ў 
маглаўскія грошы, а той параіў Гары выпіць яду.  

І вось, пасля гэтага, ён задае пытанне \emph{ці дастаткова кніг я набыў?} 
Было ясна, як дзень, якога адказу ён чакаў.

Чамусьці Гары было складана гаварыць.

--- Кніг не бывае дастаткова ніколі, --- паўтарыў ён іх сямейны дэвіз, і бацька
стаўшы на калена, крэпка яго абняў. --- Але ты падышоў вельмі блізка, 
--- Гары праглынуў камок. --- Гэта была самая лепшая спроба ў свеце.

Бацька ўстаў.

--- Так, --- сказаў ён, --- мабыць, \emph{ты} бачыш платформу Дзевяць і Тры Чвэрці?

Вакзал Кінгз-Крос быў агромным і мітуслівым месцам, на сценах і падлозе была
звычайная, запэцканая брудам кафліна, і паўсюль вакол была агромная колькасць
звычайных людзей, якія паспяшаліся па сваім звычайным справам, размаўляючы
свае звычайныя размовы, і пры гэтым генеруючы шмат, шмат звычайнага шуму.
Тут была платформа №9 (на якой Гары з бацькамі зараз стаялі) і платформа
№10 (побач), і паміж імі не было абсалютна нічога, акрамя тонкай, нічога не 
абяцальнай перагародкі. Праз шкляную столь высока над галавой сонца добра
асвятляла поўную адсутнасць платформы Дзевяць і Тры Чвэрці.

Гары моцна ўглядваўся ў асяроддзе, паўтараючы ў галаве \emph{
ну давай, магічны зрок, давай прачынайся...},
пакуль яго вочы не пачалі слязіца, але нічога новага перад ім так і не з'явілася.
Ён, натуральна, падумаў, што палачка б дапамагла, але МакГонагал забараніла
карыстацца ёю. Да таго ж, калі палачка працягнула бы выдаваць снапы іскраў,
яму паграджаў бы арышт за спробу запуску фейерверка ў будынку вакзала.
І гэта пры тым, што палачка не вытварыла бы нешта іншае, напрыклад, 
падняла бы ўвесь Кінгз-Крос у паветра. 
У Гары быў час толькі хутка пралістаць свае новыя падручнікі перад набегам
на кнігарні, і гэта толькі выклікала ў яго яшчэ больш пытанняў.

Прынамсі, у яго была --- Гары кінуў позірк на гадзіннік --- цэлая гадзіна, каб
вырашыць гэтую задачу, бо цягнік сыходзіў у адзінаццаць. Можа гэта быў
своеасабалівы \abbrev{iq} тэст, каб дурныя дзеці не маглі стаць чараўнікамі.
(І дадаковы час, які ты сабе абраў, будзе скарыстаны для вызначэння
пачатковага ўзроўню тваёй Сумленнасці, якая будзе другім па важнасці фактарам
школьнага поспеху.)

--- Я змагу гэта вырашыць, --- сказаў ён бацькам. --- Гэта проста нейкі тэст.

Яго бацька нахмурыўся.

--- Хмм... можа знайсці сляды на зямлі, якія вядуць кудысьці, і потым раптам
абрываюцца...

--- \emph{Тата!} --- сказаў Гары з абурэннем. --- Перастань! Я нават не \emph{паспрабаваў} 
вырашыць задачу самастойна! --- прапанова была добрай, ад чаго рабілася яшчэ 
горш.

--- Выбачай, --- сцішыўся бацька.

--- Я не думаю, што яны адмыслова нешта такое задумалі, --- сказала маці. ---
Прафесар МакГонагал дакладна табе нічога не казала?

--- Можа яе увага была нечым адцягнута... --- сказаў Гары, не падумаўшы.

--- \emph{Гары!} --- громкім шэптам прашыпелі абодва бацькі, --- што ты там 
утварыў?

--- Я... эммм, --- Гары зглынуў, --- слухайце, у нас няма на гэта часу...

--- \emph{Гары!}

--- Я сур'ёзна! У нас няма на гэта часу! Бо гэта праўда доўгая гісторыя, а мне
трэба зразумець, як трапіць у школу!

Яго маці закрыла твар рукой.

--- Натстолькі дрэнна?

--- Я, э-э, --- (\emph{не магу расказаць па меркаванням Нацыянальнай Бяспекі,}) ---
ну... прыкладна на 50\% дрэннасці ад Інцыдэнту з Праэктам на Навуковай Выстаўцы?

--- \emph{Гары!}

--- Эмм, о-о-о! Глядзіце! Там у людзей сава ў клетцы, я пайду і спытаю ў іх! --- 
і Гары пабяжаў напрамкі ад сваіх бацькоў да вогненна-рудавалосай сям'і. Яго 
куфар аутаматычна заклэпаў услед. 

Пухленькая жанчына кінула на яго хуткі позірк, калі ён узнік побач.

--- Добры дзень, даражэнькі. Першы раз у Хогвартс? 
Наш Рон таксама... --- і тут яна застыла, і паглядзела на яго вельмі ўразліва.
--- \emph{Гары Потэр?}

Чатыры рудавалосых хлопца, адна рудавалосая дзяўчына, і адна палярная сава 
імгенна аглянуліся і таксама застылі.

--- Ой, \emph{да ладна!}, --- запратэставаў Гары. Ён планаваў прадстаўляцца людзям
як "містэр Верэс", прынамсі, пакуль яны не даедуць да Хогвартс. --- У мяне ж
адмысловая павязвка і ўсё такое! Ян вы мяне распазналі?

--- У мяне тое ж пытанне, --- пачуў Гары бацькін голас, які падышоў 
ззаду, --- \emph{скуль} вы ведаеце, хто ён такі? --- 
у голасе адназначна былі спужаныя ноткі. 

--- Твая фотка была ў газэтах, --- сказаў адзін з двух ідэнтычных блізнятаў.

--- \shout{Гары!}

--- \emph{Тата!} Гэты не тое, што ты думаеш! Гэта таму, што я адолеў Цёмнага Лорда
Самі-Ведаеце-Каго ва ўзросце аднаго году!

--- \shout{Што?}

--- Мама табе потым растлумачыць.

--- \shout{Што?}

--- Эм... Майкл, ёсць некаторыя рэчы... я падумала, лепей не турбаваць цябе,
пакуль...

--- Прабачце, --- сказаў Гары рудавалосай сям'і, якая стаяла, усё яшчэ ўтаропіўшыся
на яго, --- але гэта б вельмі добра, калі б вы показалі мне, як трапіць
на платформу Дзевяць і Тры Чвэрці \emph{неадкладна!}

--- А-а, --- сказала жанчына. Яна паказала на сцяну паміж платформамі. --- 
Трэба проста прасці праз бар'ер паміж платформамі дзевяць і дзесяць. Не спыняйся
і не пужайся, што ўрэжашся, гэта вельмі важна. Калі нярвуешся, лепей крыху
разбяжацца.

--- І, што бы ні здарылася, не думай пра слана.

--- \emph{Джордж!} Не слухай, мой любы, няма ніякай прычыны не думаць аб сланах.

--- Я Фрэд, мам, не Джордж...

--- Дзякуй! --- сказаў Гары і пачаў разбягацца ў бок сцяны.

Пагодзь, яно што, не спрацуе, \emph{калі я ў гэта не паверу?}

У такія часы Гары ненавідзеў свой розум за хуткасць, якая выклікала 
"рэзананс сумневу", калі, напрыклад, ён зараз пачне думаць пра паспяховы пераход
праз бар'ер, аднак, ён быў крыху ўсхваляваны, што, магчыма, ён  \emph{недастаткова}
верыць у гэта, і гэта значыць, што ён \emph{сапраўды} баяўся ўрэзацца ў яго...

--- \emph{Гары, а ну вярніся! Ты павінен нешта мне патлумачыць!} --- то быў бацька.

Гары заплюшчыў вочы і пастараўся забыць усё, што ён ведаў пра тлумачэнні, пра
фізіку, і пра рэлігію, і паспрабаваў паверыць \emph{усім розумам}, што ён 
пройдзе праз бар'ер і...

...гукі вакол раптам змяніліся.

Гары спыніўся і расплюшчыў вочы, адчуваючы лёгкую агіду ад таго, што ён зрабіў
свядомы высілак, каб прымусіць сябе паверыць у нешта.

Ён стаяў на яскрава асвечанай адкрытай платформе побач з агромным цягніком:
чатырнаццаць доўгіх вагонаў за масіўным чырвоным паравозам з комінам,
які прарочыў смерць якасці паветра. На платформе ўжо было досыць народу
(нават улічваючы, што да адпраўлення была яшчэ добрая гадзіна),
групы вучняў і іх бацькоў таўпіліся вакол лавак, сталоў, разносчыкаў і 
прадаўцоў.

Не было ніякага сумневу, што такога месца на вакзале Кінгз-Крос не існавала, 
як і прасторы, каб гэта ўсё схаваць.

\emph{Ну ладна, значыць ці (а) я тэлепарваўся кудысьці ў іншае месца, ці
(б) чараўнікі маглі складваць прастору на скрут галавы,
ці (в) яны проста інгаравалі ўсе магчымыя правілы.}

Ззаду пачуўся гук дрэва па бятону, Гары абярнуўся, і --- сапраўды, ---
яго куфар перайшоў бар'ер услед за ім на сваіх кіпцюрыстых тэнтаклях. Відавочна,
раз магія спрацавала, ягоны багаж таксама здолеў дастаткова моцна паверыць у 
поспех пераходу. Гэтая думка чамусці была даволі трывожнай для Гары.

Праз секунду, малейшы з рудавалосых хлопцаў праляцеў праз чугунную арку 
(чугунную арку?) на бягу, цягнучы за сабою на павадку свой куфар, і амаль не 
сутыкнуўся з Гары. Гары, адчуваючы сябе няёмка, хутка пачаў рухацца далей ад месца
пераходу, і рудавалосы хлопец пайшоў за ім, моцна тузаючы павадок, каб прымусць
свой куфар ісці таксама. Яшчэ праз некалькі секунд белая сава праляцела праз
арку, і села хлопцу на плячо.

--- Гэй, --- сказаў хлопец, --- ты \emph{праўда} Гары Потэр?

\emph{Толькі не зноў.} 

--- У мяне няма рацыянальнага спосабу гэта праверыць. Мае бацькі навучылі мяне
верыць, што я Гары Потэр, і шмат хто казаў мне, што я падобны на сваіх бацькоў...
на маіх іншых бацькоў, але... --- Гары нахмурыўся, раптам зразумеўшы, ---
але нехта з лёгкасцю мог прымяныць заклён для таго, каб памяняць мне внешні выгляд
у маленстве...

--- Э-э... што?

\emph{Я так разумею, не кандыдат у Рэйвенкло.}

--- Так, я --- Гары Потэр.

--- Я --- Рон Уізлі, --- сказаў высокі худы рабаціністы даўганосы хлопец, і 
працягнуў руку, якую Гары ветліва паціснуў. Сава таксама вухнула
нешта ветлівае ў бок Гары (вуханне больш гучала як "э-э-эх", што было нечакана).

У гэты момант Гары зразумеў, што стаіць на парозе немінучай катастрофв, і прыдумаў
спосаб яе перадухіліць. 

--- Я на секунду, --- сказаў ён Рону, адчыніў адну з шуфлядак свайго куфара,
у якой павінна была ляжаць яго зімовая вопратка, калі ён не памыляўся (і ён
не памыляўся), знайшоў самы тонкі са сваіх шалікаў. Потым адным хуткім
рухам зняў сваю павязку, разгануў шалік, і абматаў ім сваю галаву і твар. 
Было крыху горача, --- лета жэ ж --- але Гары мог гэта цярпець.

Потым ён зачыніў шуфлядку (у якой зараз ляжала павязка, хаця і не была зімовым
аддзеннем), расчыніў іншую, дастаў --- раз яны ўжо былі на тэрыторыі чараўнікоў --- 
сваю чорную мантыю, і апрануў яе на сябе чераз галаву.

--- Гатова, --- сказаў Гары задаволена. Шалік на твары крыху прыглушаў яго голас. 
Ён павярнуўся да Рона. --- Як я выглядаю? Недарэчна, я ведаю, але ці можна
зразумець, што я --- Гары Потэр? 

--- Э-э... --- сказаў Рон, і некалькі секунд прастаяў з адчыненым ротам.
--- Не вельмі, Гары.

--- Цудоўна, --- сказаў Гары. --- Аднак, каб не ставіць пад пагрозу ўвесь сэнс маёй 
маскіроўкі, ты будзеш з гэтага моманту звяртацца да мяне як... --- "Верэс" магло
ўжо не спрацаваць, --- "містэр Плю".

--- Акей, Гары, --- сказаў неўпэнена Рон.

\emph{The Force is not particularly strong in this one,} --- падумаў Гары словамі 
з Зорных Войн. 

--- Клікай... мяне... містэр... Плю. 

--- Добра, містэр Плю... --- Рон спыніўся. --- Але я адчуваю сябе бязглуда!

\emph{Гэта не проста адчуванне.}

--- Ладна. Давай ты прыдумаеш мне імя.

--- Містэр К\'энан, --- адразу адказаў Рон. --- Як "Ч\'адлі К\'энанз".

--- Эммм... --- у Гары было вельмі вострае адчуванее, што ён пашкадуе аб гэтым
пытанні, --- і хто або што такое гэтыя "Чадлі Кэнанз"?

--- \emph{Хто такія "Чадлі Кэнанз"?} Ды проста самая найлепшая каманда за ўсю гісторыю 
квіддзіча! Яны, канешне, скончылі прошлы год у канцы лігі, але...

--- Што такое квіддзіч?

Гэтае пытанне таксама было памылкай.

--- Так, ці правільна я разумею, --- сказаў Гары, калі яму падалося, што моўны паток
Рона, суправаджаемы размахіваннем рукамі, пайшоў на спад, --- што за злоў Зніча
каманда атрымлівае \emph{сто пяцьдзесят ачкоў?}

--- Угу...

--- Колькі дзесяці-ачковых галоў забівае каманда ў сярэднім за гульню?

--- Э-э... мо пятнаццаць-дваццаць... ў прафесійных гульнях...

--- Гэта проста бессэнсоўна. Гэта парушае ўсе магчымыя правілы гейм-дізайна.
Слухай, усё, што не датачыцца Зніча, выглядае ў цэлым нармалёва, у сэнсе, 
як нармальная спартыўная гульня, але ты кажаш, што злавіць Зніч
каштуе столькі ж, колькі сярэдняя колькасць звычайных ачкоў. Пакуль усе 
намагаюцца забіваць галы, два лаўца лятаюць абы-дзе ў пошуках Зніча, і 
\emph{ніяк} не ўзаемадзейнічаюць са сваімі камандамі, і першым заўважыць 
Зніч --- гэта, па большасці, пытанне ўдачы...

--- Гэта не проста ўдача! --- запратэставаў Рон. --- Ты павінен 
правільна рушыць вачыма...

--- Такая гульня --- не \emph{інтэрактыўная,} Лаўцы не ўдзельнічаюць у камбінацыях
каманды, і наогул --- наколькі цікава назіраць за экспертам у валоданні сваімі 
вачыма? І потым першы Лавец, якому пашанціла заўважыць Зніч, кідаецца ўніз, 
ловіць яго, і ўсё, удзел у гульні ўсіх астатніх не лічыцца! Выглядае, быццам
нехта ўзяў добрую гульню, і проста дадаў да яе новую бязглуздую пазіцыю, каб 
можна было стаць Найважнейшым Гульцом без неабходнасці вучыцца гуляць у яе.
Хто быў першы Лавец, нейкі разяваты сынок караля, які хацеў славы, але так і 
не здолеў зразумець правілы? --- і дарэчы, гэта падавалася даволі зграбнай
гіпотэзай. Пасадзі ідыёта на мятлу, і скажы яму лавіць бліскучую краказябу...  

На твары Рона з'явілася грымася.

--- Калі табе не і падабаецца Квіддзіч, не варта з яго смяяцца!

--- Не можаш крытыкаваць --- не зможаш апрымізаваць. Я толькі прапаную 
\emph{палепшыць гульню.} І гэта вельмі проста зрабіць. Проста выкінь Зніч.

--- Ніхто не зменіць правілы проста таму, што \emph{ты} сказаў!

--- \emph{Я} --- хлопец, які выжыў, калі ты забыў. Мяне людзі паслухаюць. І калі 
я пераканаю іх змяніць правіла ў Хогвартс, гэта пачне распаўсюджвацца.

Рон, з выглядам абсалютнага жаху, сказаў:

--- Але, але, але калі ўбраць Зніч, як зразумець, калі гульня канчаецца?

--- \emph{Па гадзінніку.} Дарэчы, гэта будзе шмат слушней, бо зараз у вас 
напэўна гульні займаюць то дзесяць хвілін, то дзесяць гадзін, і дакладны
раскладе будзе рабіць гульню больш прадказальнай таксама і для гледачоў, ---
Гары ўздыхнуў. --- О, спыні гэты жахлівы выраз, я думаў, у мяне проста не 
\emph{хопіць часу} займацца поўным знішчэннем гэтай нікчэмнай бязглуздзіцы,
якую магічны свет кліча "нацыянальным спортам", і ягонай перабудовай 
на моцных прынцыпах, як я сабе яго ўяўляю. У мяне ёсць шмат, шмат, \emph{шмат}
важейшыя рэчы, пра якія варта турбавацца, ---  яго позірк стаў задуменным. ---
Але, шчыра кажучы, можна знайсці пару хвілін, каб запісаць мой Маніфест 
Рэфармацыі Квіддзіча, прыбіць яго да дзвярэй...

--- По-отэр, --- працягнуў голас побач, --- \emph{што} за хрэнь у цябе на галаве,
і \emph{што} за хрэнь стаіць побач з табой?

Жахлівы выгляд Рона змяніўся выразам нянавісці.

--- \emph{Ты!}

Гары павярнуў галаву; гэта і сапраўды быў Драко Малфой, які, відавочна, 
кампенсаваў абавязак насіць стандартную школьную мантыю сваім куфарам, які 
выглядаў такім жа магічным, як і Гарын, але больш элегантным, быў аздобленым 
срэбрам і смагардамі, з --- як гэта зразумеў Гары --- сямейным гербам Малфоеў:
прыгожы змей, які паказваў іклы над скрыжаванымі чароўнымі палачкамі.  

--- Драко! --- сказаў Гары. --- Эммм, або Малфой, калі табе так больш падабаецца,
хаця гэта больш пасуе Люцыусу. Я рады бачыць, што ты выглядаеш так добра пасля...
хм... нашай апошняй сустрэчы. Гэта Рон Уізлі. І я намагаюся быць інкогніта, 
таму кліч мяне, э... --- Гары агледзеў сябе, --- містэр Блэк.

--- \emph{Гары!} --- прашыпеў Рон, --- нельга браць \emph{гэтае} імя!

Гары міргнуў. 

--- Чаму не? \emph{Гучыць} даволі змрочна і таямнічна...

--- Гэта \emph{цудоўнае} імя, --- сказаў Драко, --- але Вялікі і Старажытнейшы
Род Блэкаў можа запярэчыць. Што на конт "містэр Сільвер"?

--- \emph{Знікні} з вачэй... з вачэй... містэра Голда, --- сказаў Рон холадна, 
і зрабіў крок наперад. --- Ён не будзе апускацца да размовы з такімі як ты!

Гары прыміральна падняў руку.

--- Тады я буду містэр Бронз, дзякуй за схему, Рон\footnote{{} Спадзяюся, вы без 
праблем зразумелі схему называння.
}, і дарэчы, эммм... --- 
Гары напруджана намагаўся знайсці словы, --- мне радасна бачыць, што праяўляеш
такі... энтузіязм у маёй абароне, але я зусім не супраць размовы з Драко...

Гэта, відавочна, стала апошняй капляй для Рона, бо ён рэзка павярнуўся да
Гары, і яго вочы гарэлі абурэннем.

--- \emph{Што?} Ты ведаеш, \emph{хто} ён такі?

--- Ведаю, --- сказаў Гары. --- Ты напэўна памятаеш, я назваў яго "Драко", і ён 
не адчуў неабходнасці прадстаўляцца.

Драко задаволенна хмыкнуў. Потым ён заўважыў саву на плячы Рона.

--- О божа, што \emph{гэта?} --- працягнуў ён. У яго словах была слаба прыкрытая
пагроза. --- А дзе ж знакаміты сямейны пацук Уізлі? 

--- Памёр, --- сказаў прахалодна Рон.

--- Які жаль. Пот... эм... містэр Бронз, я павінен адзначыць, што адназначна ў 
сям'і Уізлі --- найлепшая ў свеце гісторыя пра хатняга гадаванца. Не хочаш расказаць,
Уізлі?

Рон скурчыўся. 

--- Табе не было б так весела, калі гэта здарылася бы ў тваёй сям'і!

--- О, --- амаль прамурчэў Драко, --- але такое не можа здарыцца з Малфоямі.

Далоні Рона сціснуліся ў кулакі...

--- Хопіць, --- сказаў сказаў Гары, намагаючыся дадаць як мага больш ціхай
аўтарытэтнасці ў свой голас. --- Калі Рон не хоча пра тое размаўляць, ён і не 
павінны, і я бы папрасіў цябе таксама знайсці іншую тэму.

Драко здзіўлена паглядзеў на яго, а Рон кіўнуў.

--- Так яго, Гары! У сэнсе --- містэр Бронз. Бачыш, які ён гад? Скажы яму, каб 
ішоў адсюль!

Гары палічыў да дзесяці ў галаве, што ён умеў рабіць вельмі хутка --- \emph{12345678910}
--- яго старая звычка з пяцілетняга ўзросту, калі маці ўпершыню навучыла яго рабіць 
так, і яшчэ тады Гары вырашыў, што яго спосаб быў хутчэйшы і павінен быць больш
эфектыўным. 

--- Рон, --- сказаў ён спакойна, --- я не маю ні малейшага намеру праганяць Драко.
Ён можа размаўляць са мной, калі захоча.

--- Ну тады я не маю намеру завісаць з тымі, хто завісае з Малфоем, --- халодна 
адказаў Рон.

Гары паціснуў плячыма.

--- Гэта твая справа. Дарэчы, у мяне таксама няма намеру дазваляць камусці ўказваць
мне, з кім завісаць, а з кім --- не, --- адказаў Гары, паўтараючы ў думках
\emph{калі ласка, уйдзі, калі ласка, уйдзі...}

Рон ад нечаканасці не знайшоў, што адказаць. Падавалася, што ён сапраўды 
спадзяваўся, што яго апошняя фраза спрацуе. Ён павярнуўся, тузануў павадок куфара,
і хутка закрочыў прэч.

--- Калі ён табе не падабаўся, --- сказаў Драко зацікаўлена, --- чаму ты проста 
не паслаў яго?

--- Эммм... яго маці дапамагла мне знайсці шлях сюды з Кінгз-Крос, таму мне 
было складана сказаць яму такое. І не тое, каб я \emph{ненавідзеў} гэтага Рона, 
--- сказаў Гары, --- я проста... проста...

--- Проста не бачыш прычын для яго існавання?

--- Кшталту таго.

--- У любым выпадку, Потэр... калі цябе і праўда вырасцілі маглы... --- 
тут Драко зрабіў паузу, быццам чакаючы пратэстаў, але Гары маўчаў, ---
тады ты наўрад ці дакладна разумееш, што значыць --- быць знакамістасцю.
Людзі будуць намагацца заняць ўсю тваю ўвагу. Ты павінен
навучыцца казаць "не".

Гары кіўнуў і задумаўся.

--- Гучыць, як слушная парада.

--- Калі будзеш з імі па-добранькаму, скончыцца тым, што ты будзеш марнаваць час 
на самых нахабных з іх. Вырашы, з кім ты будзеш бавіць час, а ўсіх астатніх праганяй.
Людзі \emph{будуць} меркаваць аб цябе па тваёй кампаніі, і ты не захочаш, каб 
у тваю кампанію ўваходзілі тыпы кшталту Уізлі.

Гары зноў кіўнуў.

--- Калі я магу спытацца, як ты мяне распазнаў?

--- \emph{Містэр Бронз}, --- працягнуў Драко, --- калі ты памятаешь, я ўжо 
сустракаў цябе. Я гэта запомніў вельмі добра. Я ўбачыў чалавека з шалікам
на галаве, вы гэта выглядала \emph{неверагодна бязглузда}. І ў мяне з'явілася
\emph{дзікая здагадка.}

Гары нахіліў галаву, прыймаючы камплімент. 

--- Мне \emph{жудасна} шкада за тое, --- сказаў ён. --- У сэнсе за нашу 
першую сустрэчу. Я не хацеў цябе зняважыць пераю Люцыусам.

Драко толькі махнуў рукой.

--- Аднак, хацеў бы я, каб бацька зайшоў, калі ты мне ліслівіў... --- ён 
засмяяўся. --- Але дзякуй за тое, што ты яму сказаў. Калі б не ты, мне б 
прыйшлося туга.

Гары пакланіўся зноў.

--- І \emph{табе} дзякуй за ўзаемадапамогу з прафесарам МакГонагал.

--- Калі ласка. Падаецца, адна з асістэнтак у краме камусьці ўсё ж такі 
разбалматала нешта, бо бацька казаў, што ходзяць \emph{дзіўныя слухі}, 
быццам мы з табой пабіліся, або нешта такое.

--- Ой, --- сказаў Гары і зморшчыўся. --- Мне праўда вельмі шкада...

--- Мы да гэтага звыклыя. Мерлін ведае, колькі слухаў пра Малфоеў і без таго 
ходзіць па свеце.

Гары кіўнуў.

--- Ну тады я рады, што ўсё абышлося.

Драко ўсміхнуўся.

--- У бацькі даволі \emph{выкшталцонае} пачуццё гумару, але ён разумее 
сябраванне. Гэта ён разумее \emph{вельмі} добра. Настолькі, што я быў 
павінны целы месяц перад сном казаць уголас "Я буду заводзіць сяброў у Хогвартс".
Калі я ўсё яму растлумачыў, ён зразумеў, што я рабіў насамрэч, і не толькі 
папрасіў прабачэння, але і купіў мне марозіва.

У Гары сківіцы адваліліся.

--- \emph{Ты здолеў раскруціць гэта ў марозіва?}

Драко кіўнуў, выглядаючы задаволеным да кончыкаў валос, бо дасягненне таго
заслугоўвала. 

--- Ну як, бацька, канешне, разумеў, што я раблю, але ён навучыў мяне гэтаму:
калі я па-асабліваму ўсміхаюся падчас важнай размовы, гэта пераўтвараецца ў 
спецыяльны момант паміж бацькай і сынам, і ён павінен пачаставаць мяне марозівам,
інакш я зраблю па-асабліваму сумны выраз, быццам я ўпэўнены, што моцна яго
расчараваў.

Гары глянуў на яго ўважліва, адчуваючы прысутнасць іншага майстра.

--- У цябе былі \emph{заняткі} па маніпуляцыі людзьмі?

--- Колькі я сябе памятаю, --- сказад ганарыста Драко. --- Бацька
нанімаў мне рэпетытараў.

--- Ух ты, --- сказаў Гары. Тое, што ён калісьці прачытаў кнігу Чалдзіні
\emph{Уплыў: тэорыя і практыка}\footnote{{Robert Cialdini, \emph{Influence: 
Science and Practice.}}}, напэўна не магло параўнацца з такім (хоць кніжка і 
была даволі складанай).

--- Твой бацька амаль такі ж круты, як і мой.

Бровы Драко падняліся ўверх.

--- О? І што твой бацька для цябе робіць?

--- Купляе мне кнігі.

Драко секунду падумаў.

--- Гучыць не вельмі ўразліва.

--- Табе проста трэба было прысутнічаць. У любым выпадку, рады за цябе. Люцыус на
цябе глядзеў так, што я думаў ён цябе збіраецца з... збіраецца укрыжаваць.

--- Бацька сапраўды мяне любіць, --- цвёрда сказаў Драко. --- Такога ён ніколі не
зробіць.

--- Эммм... --- сказаў Гары. Ён успомніў белавалосую постаць у чорнай мантыі, 
дасканалаую да дробязей, якая зайшла ў краму мадам Малкін, узброеная тым смяротным
кіем са срэбрай ручкай. Было вельмі складана ўявіць гэтага прыроджанага 
забойцу клапатлівым бацькам. --- Не прымі за крыўду, але скуль ты гэта ведаеш?

--- А? --- было зразумела, што такое пытанне Драко ставіў сабе нечаста.

--- Гэта фундаментальнае пытанне рацыянальнасці: \emph{Чаму мы верым у тое, у што верым?}
Што, па твайму меркаванню, ты ведаеш, і як ты, па твайму меркаванню, пра тое даведаўся? 
Ці быў у цябе пераканаўчы досвід, што Люцыус не ахвяруе цябе пры
першым зручным выпадку, як і любую другую фігуру ў гульне?

Драко кінуў у Гары дзіўны погляд.

--- Да што ты наогул ведаеш пра бацьку?

--- Эммм... чалец Уізенгамота, чалец Папячыцельскай Рады Хогвартс, 
неверагодна заможны, мае доступ да вушэй міністра Фаджа, і таксама мае яго давер,
і магчыма мае некалькі вельмі няёмкіх фотаграфій міністра, самы знакаміты 
ваяр за чысціню крыві пасля Цёмнага Лорда, былы Пажыральнік Смерці з бліжэйшага
кола, мае Цёмную Метку, але не быў пакараны праз тое, што --- як ён сцвярджаў ---
быў пад уздзеяннем заклёна Імперыус, што гучала да смешнага непраўдападобна,
і ўсе гэта ведалі... злы з вялікай літары "З", і прыроджаны забойца... я думаю,
усё.

Вочы Драко сціснуліся да невялікіх шчылін.

--- МакГонагал расказала, ці не так?

--- Не, яна адмовілася расказваць мне нешта пра Люцыуса, акрамя парады трымацца 
ад яго далей. Таму падчас Інцыдэнту ў краме зёлак, пакуль МакГонагал суцяшала 
прадаўшыцу і прыбіралася, я распытаў аднаго з наведвальнікаў, і ён расказаў мне 
ўсё, што ведаў.

Вочы Драко зноў сталі шырэй.

--- Не хлусіш?

Гары паглядзеў на Драко здзіўлена.

--- Думаеш, спытаўшы двойчы, ты прымусіш мяне казаць праўду?

Паследвала пэўная пауза, пакуль сэнс яго слоў даходдзіў да Драко.

--- Ты абсалютна стоадсоткава будзеш у Слізэрыне.

--- Я стоадсоткава будзеш у Рэйвенкло, дзякуй. Улада мне патрэбна толькі каб 
набываць кнігі.

--- Ага, ага, --- хіхікнуў Драко, --- але, калі адказаць на твае пытанне... ---
ён глыбока ўдыхнуў, і твар яго стаў сур'ёзным. --- Бацька аднойчы дзеля мяне 
прапусціў пасяджэнне Уізенгамота. У той дзень я лятаў на мятле, упаў, і пераламаў
робры. Больна было --- жах. Я думаў, што памру. І бацька замест галасавання ў 
Уізенгамоце сядзеў каля майга ложка ў Сант-Манга, трымаючы мяне за руку, і 
абяцаючы, што ўсё будзе добра...

Гары няёмка адвёў вочы, потым, з цяжкасцю, прымусіў сябе зноў паглядзець на Драко.

--- Навошта ты мне расказваеш \emph{такое?} Гэта падаецца... кшталту, прыватным.

Драко паглядзеў на Гары сур'ёзна. 

--- Адзін з маіх настаўнікаў казаў, што людзі становяцца сябрамі праз тое, што 
ведаюць прыватныя рэчы адзін аднаго, і што большасць людзей не можа 
заводзіць сяброў як раз таму, што ім сорамна дзяліцца чымсьці важным пра сябе, ---
ён працягнуў наперад расчыненыя далоні. --- Твая чарга.

Гары падазраваў, што шчырасць і спадзяванне на твары Драко былі
ўбіты ў яго цяжкімі шматмесячнымі трэніроўкамі, аднак гэта не змяньшала эфектыўнасць
яго тактыкі. Дакладней, гэта \emph{змяньшала} эфектыўнасць, але не знішчала 
эфект цалкам. Таксама, ён заўважыў яшчэ адну тэхніку, пра якую чытаў у кнігах
па псіхалогіі: расказ Драко стварыў ціск "узаемадапамогі" на Гары. (Адзін з
эксперыментаў паказаў, што людзі, якім проста так даць пяць даляраў,
ў два разы часцей згаджаюцца прайсці апытальнык, чым тыя, якім абяцалі пяцьдзесят
даляраў пасля прахаджэння.) Драко нечакана падзяліўся сакрэтам, і чакаў сакрэта 
ў адказ... і праблема была ў тым, што Гары адчуваў ціск. Адмова, ён быў упэўнены,
выклікае глыбокае расчараванне, і, магчыма, пэўную порцыю пагарды з боку Драко.

--- Драко, --- сказаў Гары, --- проста дзеля яснасці: я дакладна разумею, што 
ты зараз намагаешся зрабіць. У маіх кнігах гэта называлася "узаемаадказнасць" або
"узаемадапамога", напрыклад, калі даць камусьці два сыкля, і спытаць нешта зрабіць, 
то шанец на поспех будзе ў два разы больш, чым калі ты паабяцаеш чалаеку дваццаць 
сыкляў пасля таго, як ён зробіць тое, што ты спытаў...  

Твар Драко паказваў жаль і расчараванне. 

--- Гэта не лавушка, Гары. Гэта сапраўдны спосаб стаць сябрамі.

--- Я не сказаў, што адмаўляюся адказваць. Проста мне патрэбны час, каб 
абраць нешта прыватнае, але не небяспечнае. Скажам... мне хацелася паказаць табе,
што мяне нельга вось так прымусіць зрабіць нешта.

Дарэчы, зрабіць паузу і паразважаць ---
гэта добрая абарона, якая пазбаўляе сілы шматлікія спосабы маніпуляцыі, 
калі ты толькі навучышся распазнаваць іх.

--- Акей, --- сказаў Драко, --- пачакаю, пакуль ты даспееш. І, калі ласка, 
здымі гэты шалік, калі будзеш расказваць.

\emph{Проста, але эфектыўна.} Гары не мог не заўважыць, якой недарэчнай і няўдалай
была яго спроба супраціву маніпуляцыі / збрагчы гонар / выпендрыцца была ў параўнанні 
з Драко. \emph{Мне патрэбныя яго настаўнікі.}

--- Добра, --- сказаў Гары пасля некаторага часу. --- Вось мая гісторыя, --- ён 
паглядзеў вакол, потым узняў шалік, адкрыўшы твар, але пакінуўшы шнар схаваным. ---
Эммм... выглядае, быццам, ты можаш давераць свайму бацьку. У сэнсе...
калі ў цябе ёсць што сказаць, ён заўсёды цябе сур'ёзна паслухае.

Драко кіўнуў.

--- Часам... -- сказаў Гары і зглынуў. Гэта была на здзіўленне няпроста, але 
ў гэтым і был задума, --- часам мне бы хацелася, каб мой бацька быў, як твой, ---
ягоны позірк аўтаматычна таргануўся  ўбок, і Гары прымусіў вярнуць вочы да Драко. 

Потым, як удар, у галаве ўзнікла думка \emph{што такое я зараз сказаў?!}, і Гары паспешна дадаў:

--- Канешне, не бездакорным інструментам смерці, як Люцыус,
а толькі каб успрымаў мяне сур'ёзна...

--- Разумею, --- сказаў Драко з усмешкай. --- Ну як... адчуваеш, што мы сталі 
бліжэй да сяброўства?

Гары кіўнуў. 

--- Ага... На самой справе -- так. Эммм... калі ты не супраць, я вярну маскіроўку.
Я \emph{праўда} не хачу... 

--- Я разумею.

Гары зноў вярнуў шалік на твар.

--- Бацька ўспрымае ўсіх прыхільнікаў сур'ёзна, --- сказаў Драко. --- Таму 
ў яго шмат прыхільнікаў. Можа табе варта з ім пазнаёміцца.

--- Я падумаю, --- сказаў нейтральна Гары. І потым здзіўлена пакачаў галавой. 

--- Выходзіць, ты ягонае адзінае слабое месца. Ха.

Тут Драго паглядзеў на Гары \emph{сапраўды} дзіўна.

--- Ты не хочаш знайсці чаго папіць і дзесьці прысесці?

Гары зразумеў, што ён стаяў нерухома даволі доўга, і пацягнуўся, спрабуючы хруснуць 
спінай. 

--- Анягож.

Платформа пачала пакрысе запаўняцца, але ў канцы цягніка ўсё яшчэ былі ціхія месцы.
На шляху туды яны мінулі прадаўца газет, лысага, але барадатага мужчыну, побач з якім
стаяла тачка, застаўленая неонава-зялёнымі бляшанымі банкамі. 

Дарэчы, сам прадавец сядзеў, адкінуўшыся ў крэсле, і пацягваў з зялёнай банкі 
ў тый самы момант, калі ён убачыў выкшталцонага Драко Малфоя, які набліжаўся 
да яго ў кампаніі таямнічага хлопца, які выглядаў бязглузда з шалікам на твары,
што вызвала ў прадаўца моцны прыступ кашлю, ад чаго ён падавіўся напоем і заліў ім
сваю бараду.

--- Прабачце, --- сказаў Гары, --- што гэта за напой, дакладна?

--- Жарта-Кола, --- адказаў прадавец. --- Калі вып'еш яго, павінна здарыцца нешта
дзіўнае, што прымусіць цябе праліць яго на сябе або на кагосьці побач. Але ён зачараваны
знікаць праз некалькі секунд... --- і сапраўды, зялёныя плямы ўжо пачалі знікаць з 
яго барады.

--- Неверагодна, --- сказаў Драко безвыразна, --- неверагодна пацешна. Хадзем, містэр Бронз, 
знойдзем...

--- Чакай, --- сказаў Гары.

--- \emph{Да ладна!} Гэта ж проста \emph{дзяцінства!}

--- Выбачай, Драко, але я \emph{павінны} даследаваць гэта. Што адбудзецца, калі
мы будзем старанна весці сур'ёзную размову, і вып'ем гэтую Колу?

Прадавец усміхнуўся і загадкава паціснуў плячыма.

--- Хто ведае? Магчыма ваш знаёмы выйдзе з-за рогу ў касцюме жабы?
У любым выпадку, \emph{нешта} смешнае і нечаканае здарыцца абавязкова...

--- Не. Прабачце, але я не веру. Гэта парушае мае шматпакутлівае пачуццё недаверу на
столькіх узроўнях, што ў мяне нават няма тэрмінаў, каб гэта апісаць.
Не існуе, разумееце, \emph{не існуе} ніякага магчымагу спосабу, каб нейкі
чортавы \emph{напой} мог змяняць рэальнасць, ствараючы \emph{камічныя сітуацыі}, 
або я проста здаюся і ўматваю на Багамы...

--- Мы \emph{праўда} зоймемся вось гэтым? --- прастагнаў Драко.

--- Ты можаш не піць, але я \emph{абавязаны} правесці хаця б адзін эксперымент. 
Які кошт?

--- Пяць кнатаў за банку.

--- \emph{Пяць кнатаў?} Вы прадаеце газіроўку, якая мяняе рэальнасць, па пяць кнатаў
за банку? --- Гары засунуў руку ў махляскін, загадаў "чатыры сыкля, чатыры кната",
і брацнуў манеты на прылавак. --- Два тузіна, калі ласка.

--- Я таксама адну вазьму, --- сказаў Драко, і пацягнуўся ў кішэнь.

Гары моцна замахаў галавой. 

--- Не, я купляю ўсё. І гэта не падарунак, а частка эксперыменту. Праверым, ці
спрацуе мой напой на табе, --- ён кінуў адну банку Драко, і пачаў скармліваць астатнія
махляскіну. Ціхія гукі "коўць-коўць", якія суправаджалі працэс, ніяк не павялічвалі
веру Гары ў тое, што ён калісьці здолее знайсці разумнае тлумачэнне ўсёй гэтай
бязглуздіцы. 

Дваццаць два "коўць" пазней, Гары трымаў апошнюю банку ў руцэ. Драко паглядзеў на
яго пытальна, і яны адначасова расчынілі банкі Жарта-Колы. Гары закатаў шалік, 
каб раскрыць рот, і яны зрабілі па глытку. Нейкім чынам, на \emph{смак} напой быў
кіслотна-зялёна-шыпучым, і болей лаймавы за сапраўдны лайм.

Нічога не адбылося.

Гары паглядзеў на прадаўца, які добразычліва глядзеў у адказ.


\emph{Ну ладна, калі ён скарыстаў выпадковае здарэнне, каб прадаць мне два тузіна 
зялёнай газіроўкі, я пахлопаю яго энтэрпрэнёрскім здольнасцям, а потым яго заб'ю.}

--- Яно не заўсёды здараецца адразу, --- сказаў прадавец. --- Але я гарантую адно 
здарэнне на банку, або вяртаю грошы.

Гары зрабіў яшчэ адзін вялікі глыток.

Зноў нічога не адбылося.

\emph{Можа, трэба заглынуць усю банку як мага хутчэй?.. і спадзявацца, што мой
страўнік не выбухне ад усяго гэтага вуглякіслага газу, або што ў мяне потым не 
будзе адрыжка на паўгадзіны...}

Не, ён мог дазвольць сабе \emph{трохі} цярплівасці. Але, шчыра кажучы, Гары
не ўяўляў, як яно можа спрацаваць. Нельга было да кагосьці падысці і сказаць:
"Зараз будзе сюрпрыз", або "Слухай, раскажу табе жарт". Гэта пазбаўляла эфекту 
нечаканасці. Гары быў настолькі падрыхтаваны, што, нават калі б Люцыус Малфой
прашпацыраваў па вуліцы ў касцюме балярыны, Гары  мог стрымацца.
На якую неймавернаю махінацыю сусвет быў гатовы \emph{зараз?}
 
--- Давай прысядзем, --- сказаў Гары. Ён сабраўся зрабіць чарговы глыток, і павярнуўся,
каб пайсці ў кірунку лавачак, і гэта паставіла яго ў падыходзяшчае становішча, 
каб кінуць позірк ўбок, на тую частку газэтнага прылавка, дзе ляжага газэта 
пад назвай \emph{Куіблер}\footnote{The Quibbler --- прыдзіра}, і прачытаць
вялікі загаловак:

\headline{Драко Малфой зацяжарыў\nopagebreak\\ад Хлопца-Які-Выжыў}

--- \emph{БГХА!} --- пачулася з боку Драко, калі яго дасягнуў спрэй яскрава-зялёнай
вадкасці, які выплюнуў Гары. --- Ты, бруднакроўны суччын сын! Я таксама магу плявацца, ---
ён набраў поўны рот Жарта-Колы ў тый самы момант, калі яму ў вочы кінуўся загаловак.
Рэфлекторна Гары паспрабаваў закрыць твар ад зялёнага патока, які паляцеў у яго напрамку.
Нажаль, ён скарыстаў дзеля гэтага руку, якая трымала банку, і рэшткі напою 
вылілілся яму на плячо.

Не веруючы сваім вачам, Гары, кашляючы, глядзеў на банку ў сваёй руцэ.
Зялёныя плямы ўжо паціху пачалі знікаць з мантыі Драко.

Ён падняў вочы і зноў убачыў загаловак.


\headline{Драко Малфой зацяжарыў\\ад Хлопца-Які-Выжыў}

Гары толькі і змог сказаць:

--- Шт... шт... шт...

Праблема была ў тым, што у яго было настолькі шмат аргументаў для запярэчання. Калі ён спрабаваў
сказаць "Але нам толькі адзінаццаць!", пярэчанне "Але мужчына не можа зацяжарыць!" 
патрабавала пралезці на першае месца, і потым была выціснута фразай 
"Але паміж намі нічога няма, праўда!".

Гары зноў глянуў на банку ў сваёй руцэ.

Дзесьці глыбока ў ім працыналася жаданне закрычаць і не спыняцца, пакуль ён не 
зваліцца ад недахопу паветра. Адзіная рэч, якая ўтрымлівала яго, быў успамін 
пра артыкул, дзе казалася, што раптоўны прыступ панікі быў знакам   
\emph{сапраўды} важнай навуковай праблемы.

Гары фыркнуў, кінуў банку ў бліжэйшую ўрну для смецця, і вярнуўся да прадаўца.

--- Адзін Куіблер, калі ласка, --- ён заплаціў яшчэ чатыры кната, вывудзіў
яшчэ адну банку Жарта-Колы з махляскіна, і вярнуўся да лаўкі, дзе Драко ў гэты
момант ашалела разгледжваў сваю зялёную банку.

--- Бяру свае словы назад, --- сказаў ён, --- гэта крутая хрэнь.

--- Ведаеш, Драко, што лепей за агульныя сакрэты для сяброўства? Агульнае забойства.

--- Адзін мой настаўнік казаў такое, --- згадзіўся Драко. Ён засунуў руку 
пад мантыю і пачасаўся з 
абсалюстна разняволеным выглядам. --- Каго ты задумаў?

Гары жорстка ляпнуў Куіблерам па століку, які стаяў перад лаўкай. 

--- Пісаку, які прыдумаў гэты загаловак.

--- Не які, а якая. Пэўная дзесяцігадовая шмара, калі я не памыляюся. У яе 
моцна паехаў дах, калі маці памерла, і яе бацька, уладальнік газэты,
\emph{перакананы}, што ў яе --- дар прадбачання, таму, каб праведаць навіны,
ён пытае Л\'юну, і верыць у любую яе лухту. 

Не думаючы аб тым, што ён робіць, Гары расчыніў сваю банку Жарта-Колы і 
прыгатаваўся адпіць. 

--- Сур'ёзна? Гэта нават жудасней за маглаўскіх журналістаў, а я думаў, быць хужэй
за іх фізічна немагчыма.

Драко фыркнуў:

--- Таксама у яе нейкая перакручаная апантанасць Малфоямі, ды яе бацька ---
наш палітычны супраціўнік, таму друкуе кожнае яе слова. Як толькі вайду ва ўзрост ---
абавязкова яе згвалтую.

Зялёная вадкасць шухнула праз ноздры Гары, запэцкаўшы шалік, які прыкрываў тую
вобласць. Лёгкія і Жарта-Кола --- дрэннае спалучэнне, таму ён 
правёў некалькі секунд, дзіка адкашліваючыся.

Драко рэзка глянуў на яго.

--- Штосьці не так?

У гэты самы момант Гары раптоўна зразумеў, што: а) гукі асяроддзя сталі нейкім
роўным ціхім шумам у момант, калі Драко засунуў руку пад мантыю, і б) 
у гэтым абмяркоўванні забойства ў якасці спосаба завясці сяброўства --- адзін 
з іх зусім не жартаваў пра гэта.

\emph{Дакладна. Бо спачатку ён падаваўся такім нармальным. І ў цэлым, ён настолькі 
нармальны, наколькі можна чакаць ад хлопца, якога гадаваў жудасны памочнік 
Цёмнага Лорда, і пры гэтым --- клапотлівы бацька.}

--- Ну... так, --- Гары кашляуў, \emph{ну і як ён выбярэцца} з гэтай размоўнай пасткі? --- 
Я проста дзіўлюся, што ты гатовы так адкрыта гэта абмяркоўваць, і не хвалюешся, 
што цябе паймаюць, кшталту таго.

--- Жартуеш? --- зноў фыркнуў Драко. --- Слова  \emph{Люны Лавгуд} супраць майго?

Чорт пабірай.

--- Я так разумею, што магічнага дэтэктара праўды не існуе? --- \emph{або
\abbrev{днк}-тэсту... пакуль.}

Драко паглядзеў па баках, яго вочы звузіліся.

--- Ну да, ты ж не ў курсе. Я раскажу табе, як тут усё працуе --- у сэнсе, 
як яно насамрэч працуе, быццам ты ўжо на Слізэрыне, і задаў мне такое пытанне.
Але ты павінны даць клятву не гаварыць ні з кім пра гэта.

--- Але я магу гаварыць на гэтую тэму, проста не згадваць, што ты мне гэта расказаў, 
так? Я маю на ўвазе, калі іншы малады слізэрын калісьці спытае мяне аб гэтым.

Драко падумаў.

--- Ну-ка, паўтары.

Гары паўтарыў.

--- Акей, гэта не гучыць, як нейкі падман, тады можаш. Проста трымай у галаве, 
я заўсёды буду ўсё адмаўляць. Кляніся.

--- Клянуся, --- сказаў Гары.

--- Суд карыстае Верытас\'ерум, але гэта проста смех, бо ты кастуеш на сябе
забывальны заклён, і настойваеш, што ў ахвяры заменены ўспаміны. 
Калі ў цябе ёсць думніца, то можаш спачатку захаваць свае ўспаміны, а пасля 
суду --- вярнуць назад. Звычайна суд вырашае, што быў прыменены забывальны 
заклён, бо фэйкавыя ўспаміны --- рэч болей складаная. Але суддзі ---
таксама людзі. І калі \emph{я} буду замешаны ў чымсьці, што можа кінуць 
цень на гонар высокага роду, то справа ідзе ва Уізенгамот, дзе ў бацькі
шмат сяброў. Пасля таго, як мяне апраўдаюць, Лавгуды павінны будуць заплаціць 
штраф за паклёп. Самае смешнае, што ім таксама вядома, што так і будзе,
і таму будуць трымаць рот на замку з самага пачатку.

Халодны пот прашыб Гары, але ён разумеў, што твар і голас не павінны яго выдаць.
\emph{На будучыню: Скінуць урад магічнай Брытаніі пры першай магчымасці.} Ён 
кашлянуў зноў, каб прачысціць горла. 

--- Драко, калі ласка, не прымі няправільна --- я ж пакляўся, --- але, як ты сказаў,
уяві, што я слізэрын, і я хацеў цябе справакаваць. Што --- тэарэтычна --- здарыцца, 
калі я пасведчыў бы на судзе, што я чуў твае планы на гэты конт?

--- Нехта не-Малфой атрымаў бы вялізныя праблемы, --- самазадавалена адказаў Драко.
--- А калі я Малфой... бацька пераможа ва Уізенгамоце. А потым ён зоймецца табой...
ну... з канкрэтна табой будзе нялёгка, бо ты Хлопец-Які-Выжыў, але бацька сапраўдны 
прафесіянал у такіх справах, --- Драко нахмурыўся. --- Дарэч, ты перада мной адкрыта 
прапанаваў забойства --- ці не хвалюе цябе, што я магу пасведчыць, калі яе
раптам знайдуць мёртвую?

\emph{Якім, о божа, чынам, 
мой дзень настолькі пайшоў не туды?} Гарыны вусны пачалі рухацца, нават калі 
розум яшчэ не знайшоў адказу.

--- Так, але гэта ўсе было, калі я думаў, што яна дарослая. 
Не ведаю, як у вас тут, але ў маглаў забойства дзіця --- гэта 
шмат сур'ёўнейшае злачынства...


--- Маеш рацыю, --- сказаў Драко, усё яшчэ крыху падазрона. ---
Але ў любым выпадку, лепей не даводзіць да аўрораў.
Калі мы будзем акуратна рабіць только тое, што можна пафіксаць 
лякарным зачараваннем, можна зрабіць забывальны заклён, і паўтарыць
праз тыдзень, --- ён хіхікнуў па-дзіцячы высокім голасам. ---
Ты толькі ўяві сабе спробу Люні Лавгуд абвінаваціць Драко Малфоя \emph{і Хлопца-Які-Выжыў},
такому нават Дамблдор не паверыць!

\emph{Я раздзяру вашае маленькае нікчэмнае магічнае Сярэднявечча на кавалачкі,
малейшыя за атамы, з якіх яно складаецца.}


--- Прытрымай каней, Драко. Калі я дазнаўся, што аўтарам была дзеўчына, на год 
маладзейшая за мяне, я прыдумаў нешта жудасней за забойства \emph{або} гвалт.

--- А? Выкладвай, --- сказаў Драко, і паднёс да рота банку Жарта-Колы.

Гары не ведаў дакладна, ці можа зачараванне спрацаваць болей аднаго раза на банку,
але ён заўсёды мог зваліць віну на Колу, калі правільна падгадаць час:

--- Я запланаваў, што \emph{ажанюся з гэтай дзеўчынай.}

Драко, натуральна, выдаў нейкі індустрыяльны гук, і зяленыя ручаі пацяклі з вугалкоў 
яго рота, як тасол з радыятара пасля аварыі.

--- Ты з глузду з'ехаў?

--- Насупраць. Я настолькі глузды, што ажно інеем пакрыўся.

Драко выдаў высокі хохат. 

--- Густ у цябе нават дзіўней за Лестрэнжа. І я так разумею, ты хочаш 
яе для сябе цалкам?

--- Так. Калі ты меў на яе планы,
то давай лічыць, што я табе абавязаны...

Драко махнуў рукой:

--- Не, бяры бясплатна. Вакол цьма дзевак, якія гэтага заслугоўваюць.

Гары глядзеў на банку ў сваёй руцэ, і яго кроў халадзела. Вясёлы, прывабны, 
шчыры і шчодры да тых, каго ён лічыў сябрамі, Драко не быў псіхапатам.
Гэта было сумна і страшна --- улічваючы веды Гары па чалавечай псіхалогіі -- 
\emph{ведаць}, што Драко не быў монстрам. Такая размова магла быць  нормай у 
дзясятках тысяч культур у гісторыі. Да, наш свет быў бы зусім іншым, калі б
для таго, што сказаў Драко, трэба было быць злым і жорсткім. Гэта быў вельмі 
вядомы ў гісторыі падыход: для Драко яго ворагі не былі людзьмі.

І ў гэты спакойны час у гэтай спакойнай краіне, у гэтыя самыя цёмныя, як перад
росквітам, часы, сын дастаткова магутнага высокага роду ўспрымаў як дадзенае,
што ён вышэй за закон. Прынамсі, калі справа тычылася крыху гвалту час ад часу.

У маглаўскім свеце былі краіны, дзе такія ж высокія роды існавалі да гэтых часоў,
і стаўленне да жыцяя ў іх было такое ж. Або болей дзікія краіны, дзе гвалт быў нормай
для ўсіх. Тыя месцы, якія, падавалася, абмінулі
Адраджэнне, і Цёмныя Стагоддзі ў іх не проста працягваліся --- яны рабіліся горш.
Але да магічнай Брытаніі, напэўна, такое не дайшло, бо максімумам культурнага 
ўплыву падаваліся толькі бляшаныя банкі з газіроўкай. 

\emph{Калі Драко не перадумае помсціць, і я знайду шчасце з кімсці ішным, і не ажанюся
з беднай звар'яцелай дзеўчынай, то я проста адцягваю жудасны момант, і не 
на вельмі доўгі час...}

І то --- толькі для адной дзяўчыны. Не для астатніх.

\emph{Цікава, колькі будзе каштаваць зрабіць спіс усіх ваяроў за чысціню крыві, 
выстраіць іх у чаргу, і забіць аднаго за другім?}

Людзі такое ўжо спрабавалі падчас Французскай Рэвалюцыі: выдаліць ўсім 
ворагам Прагрэсу усё вышэй за шыю, і, наколькі Гары памятаў, гэта не тое, каб
спрацавала. Магчыма, трэба прагледзіць кнігі па гісторыі, набытыя бацькам падчас
Вялікага Набегу, і праверыць, ці лёгка выправіць тое, што падчас Рэвалюцыі пайшло не так.

Гары паглядзеў у неба і на бледную луну, якая была добра бачная гэтай яснай раніцай.

\emph{Увесь свет зламаны, карумпаваны, звар'яцелы, злы, жорсткі, крывавы і цёмны. 
Знайшоў мне навіну. Ты і дагэтуль добра гэта ведаў...}

--- Нешта ты сур'ёзны стаў, --- сказаў Драко. --- Дай здагадаюся, твае бацькі-маглы
казалі табе, што так рабіць --- дрэнна.

Гары кіўнуў, не вельмі давяраўчы свайму голасу.

--- Бацька часта кажа, што факультэтаў чатыры, але ўрэшце рэшт, кожны належыць або
Слізэрыну, або Хафлпафу. І, кажучы шчыра, ты зусім не падобны на хафлпафа. 
Калі ты вырашыш быць на баку Малфоеў... наш уплыў і твая рэпутацыя...
табе бы магло сысці з рук нават немагчымае для мяне... Можаш проста паспрабаваць
спачатку, каб адчуць --- як гэта.

\emph{Ну ці ты не хітры змяёныш? Адзінаццаць гадоў, і ўжо можаш убалбатаць
мыш залесці табе ў рот. Можа ўжо занадта позна спрабаваць цябе выратаваць?}

Гары памаўчаў, падумаў, і абраў новуў тактыку.

--- Драко, дарэчы, у мяне да цябе пытанне. Можаш растлумачыць гэтую рэч з 
чысцінёй крыві? Я тут у вас новенькі, і не вельмі ведаю дэталі.

Шырокая ўсмешка асвяціла твар Драко.

--- Табе праўда варта пазнаёміцца з маім бацькам, і спытаць яго, бо, ведаеш,
ён наш лідэр.

--- Дай мне элеватар-пітч. У сэнсе, трыццацісекундную версію.

--- Акей, --- сказаў Драко. Ён набраў у грудзі паветра, яго голас стуаў ніжэй,
размераней: --- Нашыя сілы з кожным годам слабеюць, пакаленне за пакаленнем, а
бруднакроўная зараза ўсё шырэй. Калісьці Салазар і Годрык і Равена і Хельга 
адной сваёй сілай узнялі Хогвартс, стварылі Медальён, Дыядэму, Кубак і Капялюш,
але з тых пор не нарадзілася ніводнага чараўніка, здольнага кінуць ім выклік.
Мы знікаем, расчыняемся ў маглах, пакуль нашыя дзеці жэняцца з іх атроддзем,
пакідаючы пасля сябе сквібаў. Калі гэтую заразу не спыніць, хутка
нашы палачкі зламаюцца, нашае мастацтва знікне, род Мерліна 
вымрэ, і кроў Атлантыды сыдзе ў зямлю. Нашы нашчадкі будуць павінны дзеля 
выжывання корпацца ў гразі, як звычайны маглы, і цемра ўпадзе 
на ўвесь сусвет навечна, --- Драко, с задаволеным выглядам, зрабіў глыток з 
банкі. І гэтая прамовы, судзячы па яму, павінна было ўсё патлумачыць.

--- Пераканаўча, --- сказаў Гары, маючы на ўвазе фігуральны сэнс. Гэтаму патэрну
было некалькі тысяч гадоў. Заняпад, сагнанне з Раю, усё добрае --- у мінулым, 
а будучыня нясе толькі дрэннае. І заўсёды неабходнасць абараняцца ад заразы...
Але ў жыцці былі прыклады адваротнага развіцця падзей!

--- Але адзін факт я павінен паправіць. Ваша інфармацыя на конт маглаў крыху
састарэлая. Мы болей не корпаемся ў гразі.

Галава Драко тузанулася.

--- \emph{Што?} У сэнсе \emph{мы?}

--- Мы. Вучоныя. Род Фрэнсіса Бэкана, кроў Адраджэння. Можа табе падаецца,
што маглы толькі і робяць, што сумуюць, што ў іх няма чароўных палачак,
але гэта не так. У нас ёсць \emph{свая} сіла. Калі раптам магія згасне, чалавецтва
згубіць нешта вельмі цэннае, бо яна частка таго, як працуе сусвет, --- але мы
не станем пячорнымі людзьмі. Нашыя дамы будуць цёплымі ўзімку, і прахалоднымі летам,
доктары будуць лячыць людзей. Дзякуючы навуцы, людзі будуць жыць, нават калі і без
чарадзейства. Гэта, канешне, будзе трагэдыя, але не канец свету. Проста, 
каб ты ведаў.

Драко ўскочыў і адшоў на пару шагоў. На яго твары змагаліся страх і недавер.

--- \emph{Што, дзеля Мерліна, ты такое вярзеш, Потэр?}

--- Гэй, я слухаў тваю гісторыю, можа паслухаеш маю да канца? --- \emph{нязграбная
спроба,} Гары кляў сябе, але Драко, падавалася, спыніўся і вырашыў паслухаць.

--- У любым выпадку, --- працягнуў Гары, --- падаецца мне, што вы не вельмі сачылі за 
тым, што адбываецца ў маглаўскім свеце, --- магчыма таму, што жыхары магічнага свету 
лічылі астатнюю планету чымсьці кшталту трушчобы, якая заслугоўвае сколькі ж 
увагі, сколькі \emph{Файнэншыял Таймс} удзяляе праблемам Бурундзі. --- Добра,
хуткі тэст. Ці лятаў нехта з магаў калісьці на Луну? Ведаеш, вунь тую? --- Гары паказаў 
на далёкую бледную паўсферу.

--- \emph{Што?} --- сказаў Драко. Падобна, такая думка ніколі яго не турбавала.
\emph{Паляцець} на... гэта ж... --- яго палец паказаў у тым жа напрамку. ---
Немагчыма апар\'ыраваць у месца, дзе дагэтуль ніхто не быў. Якім чынам першы чалавек
туды дабярэцца?
 
--- Чакай, --- сказаў Гары. --- Хачу паказаць табе адну кнігу, і я думаю, я
нават памятаю каробку, дзе яна ляжыць, --- ён устаў, адчыніў люк свайго куфара,
бегам спусціўся ў склеп, адкінуў адну з каробак --- гэта было пагрозліва блізка да
мяжы зняважальнага стаўлення да кніг, --- расчыніў наступную, быстра дастаў кіпку кніг...

(Гары ўспадкаваў амаль чароўную здольнасць роду Верэсаў запамінаць з аднаго позірку,
дзе знаходзіцца якая кніга, што было даволі загадкава па прычыне адсутнасці 
малейшай генетычнай сувязі з імі.)

Потым ён узбяжаў уверх па лесвіцы, зачыніў люк пяткай, і, цяжка дыхаючы, пралістаў
старонкі кнігі, пакуль не знайшоў выяву, якую ён хацеў паказаць Драко.

Гэта была фотаграфія. Белая, сухая, пабітая кратэрамі зямля, людзі ў скафандрах, 
чорнае неба, і блакітна-белы шар над гарызонтам.

Тая самая фотаграфія.

\emph{Адзіная} фотаграфія, калі толькі адной выяве будзе дазволена застацца на Зямлі.

--- \emph{Вось так}, --- яго голас дрыжэў, бо ён не мог стрымаць гонар, --- выглядае Зямля
з Луны.

Драко павольна адкінуўся на лаўцы, на яго твары было падазрэнне.

--- Калі гэта \emph{сапраўдная} фотаграфія, чаму яна не рухаецца? 

\emph{Не рухаецца?} Божа мой.

--- Маглы могуць рабіць рухаючыюся выявы, але такое не памяшчаецца на адну старонку,
дзеля гэтата ім патрэбная скрыня паболей.

Драко тыкнуў пальцэм у скафандр.

--- Гэта што такое? --- голас яго дзіўным чынам змяніўся.

--- Гэта людзі. На іх адмысловыя касцюмы, яны пакрываюць усё цела, бо на Луне няма
паветра.

--- Немагчыма, --- прашаптаў Драко. У яго вачах быў страх, поўная разгубленасць.
--- Маглы?... Як?...

Гары ўзяў кнігу, перавярнуў некалькі старонак, і паказаў:

--- Гэта ўзлятае ракета. Агонь таўкае яе вышэй і вышэй, пакуль яна не дасягне Луны, 
--- яшчэ пару старонак. --- Гэта ракета на зямлі. Вось гэты мураш побач з ёю --- чалавек,
--- Драко ахнуў. --- Каб трапіць на Луну, трэба патраціць... можа нешта каля 
двух мільярдаў галеонаў, --- Драко папярхнуўся, --- і ў запуске ракеты ўдзельнічае...
болей людзей, чым ва ўсёй магічнай Брытаніі.

\emph{Прыляцеўшы, яны пакінулі там таблічку з надпісам "Мы прыйшлі з мірам ад 
усяго чалавецтва". Ты яшчэ не гатовы ўслыхаць гэтыя словы, Драко, але я спадзяюся,
калі-небудзь...}

--- Ты кажаш праўду, --- сказаў павольна Драко. --- Ты не змог бы падрабіць целую кнігу
дзеля гэтага... і я чую гэта ў тваім голасе. Але як... як?..

--- Як яны здолелі гэта без палачак і заклёнаў? То доўгая гісторыя, Драко. Навука
не карыстаецца палачкамі і заклёнамі. Наша мэта --- разумець, як праце сусвет на 
такім глыбокім узроўні, каб прымусіць сусвет дапамагаць нам. У магіі, напрыклад,
ты можаш скаставаць \emph{Imperius} на кагосьці, і прымусіць яго зрабіць нешта.
Але ў навуцы гэта быццам ведаць чалавека так добра, што пасля размовы з табой ён
будзе ўпэўнены, што зрабіць тое нешта было цалкам ягонай ідэяй. Гэта шмат складаней,
чым махаць палачкай, але яно працуе заўседы, напрыклад, калі ты згубіў палачку, 
і не можаш карыстаць  \emph{Imperius}, у цябе застаецца опцыя проста пераканаць
чалавека. І галоўнае --- навука расце з кожным пакаленнем. Каб займацца навукай,
ты павінен вельмі добра разумець сваю справу, а калі ты добра нешта разумееш, то 
можаш растлумачыць гэта камусьці яшчэ, і так веды толькі прыбаўляюцца. 
Калі ўзяць найвялікшых вучоным сто год таму, іх сіла \emph{нішто} ў параўнанні 
з сучаснымі найвялікшымі вучонымі. У навуцы не бывае "страчанага майстэрства",
як у заснавацеляў Хогвартс. У навуке сіла толькі павялічваецца. І мы толькі пачалі 
разумець і разгадваць сакрэты жыцця і спадкавання. Мы з табой можам 
паглядзець на гэтую кроў, што ты кажаш, зразумець, што робіць чараўніка чараўніком,
і праз некалькі пакаленняў зрабіць дзяцей сільнейшымі за бацькоў. Карацей, 
праблема, што ты казаў, не такая і дрэнная, і праз гадоў дваццаць навука зможа 
вырашыць яе для вас. 

--- Але... --- голас Драко ўжо заўважна дрыжэў. --- Калі ў \emph{маглаў} ёсць 
такая моц... тады... што застаецца \emph{нам?}

--- Не, Драко, ты не разумееш. Сіла навукі --- у розуме, у чалавечым жаданні разумець
прыроду, як яна працуе. Гэтую сілу немагчыма адабраць. Нават калі законы 
прыроды зменяцца, я проста зразумею іх усіх па-новай. Гэта не \emph{маглаўская} сіла,
гэта \emph{чалавечая} сіла, і яна ўзрастае кожны раз, калі ты бачыш нетша незразумелае,
і пытаеш сябе "Чаму?" Ты слізэрын, Драко, ці ты не бачыш, якія магчымасці гэта дае?

Драко подняў вочы ад кнігі да Гары, у іх прачыналася разуменне.

--- Мы, магі, можам навучыцца карыстаць гэтую сілу.

А зараз вельмі акуратна... ён праглынуў нажыўку, можна рыхтаваць кручок.

--- Ёсць нюанс. Ты можаш развіваць свае чалавечыя здольнасці, толькі калі навучышся ўспрымаць
сябе як \emph{чалавека}, а не проста \emph{мага}.

І калі ніводны падручнік у свеце не ўтрымліваў такую параду, Драко не трэба было 
гэтага ведаць, ці не так?

Позірк Драко стаў задуменным.

--- Ты... ты ўжо зрабіў гэта?

--- Часткова, --- сказаў Гары. --- Маё навучэнне яшчэ далёка ад заканчэння. У адзінаццаць
надта рана. Але мой бацька таксама нанімаў мне настаўнікаў, разумееш? --- канешне, 
яны былі галодныя студэнты, і ўсё праз тое, што ў Гары былі 26-гадзінныя суткі...
дарэчы, а што \emph{менавіта} прафесар МакГонагал збіралася зрабіць з гэтым...
але пра гэта потым... 

Паволі Драко кіўнуў. 

--- І ты думаеш, што можаш авалодаць сілай з боку і маглаў і чараўнікоў, скласці іх разам, і... ---
Драко ўтаропіўся ў Гары, --- стаць уладаром абодвух міроў?

Гары выдаў злобны смех, вельмі натуральна падыходзячы да моманты.

--- Ты павінен зразумець, Драко, што ўвесь свет, які ты ведаеш, уся магічная Брытанія ---
толькі адна клетка на вялікай шахматнай дошцы, а яшчэ на ёй ёсць месцы кшталту Луны,
зорак у начным небе, і ўсе яны --- як наша сонца, але неверагодна далёка, і галактыкі,
якія настолькі вялікія, што толькі вучоныя могуць іх зразумець, а ты нават і не думаў 
пра іх існаванне. І ў рэшце рэшт, \emph{я} --- рэйвенкло, разуееш, не слізэрын.
Я не хачу быць уладаром сусвету. Я проста думаю, што яго можна арганізаваць крыху
лепей.

--- Чаму ты расказваеш гэта \emph{мне}?

--- О... калі ідзеш па незнаёмай тэрыторыі, дзе цябе ўсё прымушае, што ты з'ехаў 
з глузду --- дапамога правадыра была бы вельмі да-ладу.  

У Драко род адкрыўся.

--- Але я павінны папярдзіць цябе, Драко. Сапраўдная навука --- не магія. Ты не можаш
проста карыстаць яе, быццам вывучыць новы заклён. Сіла навукі мае свой кошт, 
і гэты кошт настолькі вялікі, што большасць людзей адмаўляецца яго плаціць.

Драко кіўнуў, быццам нарэшце ён пачуў нешта зразумелае.

--- І што за кошт?

--- Здольнасць прыняць сваю неправату.



--- ...хм, --- сказаў Драко, калі драматычная пауза надта расцягнулася. --- І што гэта значыць?


--- Калі ты спрабуеш нешта зразумець, першыя дзевяноста дзевяць тваіх спроб 
праваляцца. Магчыма, сотая атрымаецца. Ты павінен прыймаць сваю неправату, 
раз за разам за разам. Гэта гучыць дуравата, але гэта так цяжка, што большасць
людзей не можа займацца сапраўднай навукай. Крытычна ставіцца да сябе, 
заўсёды правяраць рэчы, якія падаваліся відавочнымі, --- як Зніч у квіддзічы, ---
і з кожным новым адказам мяняешся сам. Але я забягаю наперад, вельмі далёка наперад.
Я проста хачу, каб ты ведаў... Я проста прапаную падзяліцца ведамі. Калі хочаш.
І толькі пры адной умове.

--- Ой-ёй, --- сказаў Драко. --- Ведаеш, бацька вучыў мяне, калі чуеш такія словы,
то нічога добрага не чакай.

Гары кіўнуў.

--- Я не хачу ўбіць клін паміж табой і тваім бацькам, але мне хацелася бы
мець справы з кімсьці майго ўзросту, а не з Люцыусам напрамкі. Думаю, твой бацька
не будзе супраць, бо калісьці ты зоймеш яго месца, але свае хады ў гэтай гульне 
ты павінны рабіць сам. Гэта мая ўмова --- мы вядзем справы з табой, не з тваім бацькам. 

--- Хопіць, --- сказаў Драко, і падняўся. --- Занадта шмат інфармацыі. Мне трэба 
падумаць. Не кажучы, што ўжо час садзіцца на цягнік.

--- Канешне, падумай, --- сказаў Гары. --- Толькі памятай, што гэта не эксклюзіўная
прапанова. Нават калі ты згадзішся, сапраўдная навука часам патрабуе больш за аднаго 
чалавека.

Прыглушаны белы шум вакол іх змяніўся на мармытанне, калі Драко адышоў.

Гары паглядзеў на гадзіннік у сябе на прыдалонні, вельмі просты механічны гадзіннік,
які бацька набыў, спадзяючыся, што ён буддзе працаваць у прысутнасці магіі.
Гадзіннік цікаў і, калі ён не зламаўся, то да адзінаццаі было даволі часу.
Можна было зайсці ў цягнік і пачаць пошукі як-яе-там, але яму небыло неабходна 
патраціць пару хвілін на дыхальную гімнастыку, каб разагрэць кроў.

Падняўшы вочы ад гадзінніка, ён убачыў, што да яго набліжаюцца дзве
постаці, выглядаючы неверагодна бязглузда, бо іх твары былі замотаны зімовымі
шалікамі.

--- Дабрыдзень, містэр Бронз, --- сказаў адзін з таямнічых візітораў. --- Ці было 
б вам цікава далучыцца да Ордэну Хаоса?


\subsection{Паследкі:}

Увечары, калі ўся мітусня гэтага дня нарэшце суціхла, Драко сядзеў за сваім 
сталом з пяром у руцэ. У падзямеллі Слізэрына ў яго быў асобны пакой, з пісьмовым 
сталом і камінам --- і хаця яму, нават з яго сувязямі, не далі доступа да 
Лятучага сеціва, --- прынамсі на Слізэрыне абралі свае правілы, і не заганялі 
ўсіх падрад у агульныя спальні. Прыватных пакояў было няшмат, і атрымаць такую маглі 
толькі лепшыя вучні на факультэце, і ты аўтаматычна пападаў у гэтую катэгорыю, 
калі ты быў з рода Малфоеў.

{\Large\hpFontCursive Дарагі бацька,} напісаў Драко.

І потым ён спыніўся.

Кропля чарнілаў паволі павісла на пяры, і потым упала, зрабіўшы кляксу на пергаменце.

Драко не быў дурным. Ён быў яшчэ хлопец, але настаўнікі добра навучылі яго 
распазнаваць пэўныя патэрны. Драко падазраваў, што Потэр, верагодна, больш 
схіляўся да партыі Дамблдора, чым ён паказваў... хаця Драко думаў, што яго можна 
спакусіць. Але было ясна бачна, што пакуль Драко намагаўся спакусіць Потэра, той
спрабаваў спакусіць Драко.

Было таксама ясна, што Потэр быў вельмі разумны, і значна больш чым 
"крыху зглуздануты", і веў вельмі крупную гульню, якую сам практычна не разумеў,
імправізуючы на поўнай хуткасці з грацыяй слана ў фарфуровай краме.
Але тактыка, якую 
здолеў прыдумаць Потэр, не дазваляла Драко яго ігнараваць. Ён прапанаваў частку 
сваёй сілы, відавочна, робячы стаўку на тое, што Драко, карыстаючы яе, 
стаўне больш падобным да Потэра. Бацька адносіў такое да прадвінутых тэхнік, і 
папярэджваў, што яно часта не працавала. 

Драко ведаў, што зразумеў не ўсё, што адбылося... але Потэр прапанаваў \emph{яму}
удзел у гульне, і зараз быў яго ход. І калі ён усё раскажа бацьку, гульня
перайдзе ад Драко да бацькі. 

Гары Потэр у сваёй вар'яцкай імправізацыі ўгадаў
ключ да душы Драко. Таму зараз, упершыню ў жыцці, у яго з'явіўся сапраўдны сакрэт ад бацькі.
Ён гуляў у сваю асабістую гульню. План на гульню быў даволі туманны, але Драко
ведаў, что ўрэшце бацька будзе ім ганарыцца.

Пакінуўшы кляксю на паперы --- яна таксама несла ў сабе паведамленне, якое бацька
зразумее, бо ён вучыў Драко тонкасцям сакрэтнай камунікацыі, --- Драко напісаў пытыннне,
якое не давала яму спакою, якое, падавалася, ён павінен быў зразумець сам, але так і не змог.

\begin{writtenNoteCursive}
\letterAddress{Дарагі бацька,}
Уяві сабе, што я сустрэў студэнта ў Хогвартс, --- які яшчэ не ўваходзіць у 
кола нашых знаёмых, --- які абмаляваў цябе як "бездакорны інструмент смерці",
і сказаў, што я --- твае адзінае "слабое месца". Што бы ты мог сказаць аб ім?
\end{writtenNoteCursive}

У савы не заняло шмат часу, каб прынесці адказны ліст.

\begin{writtenNoteCursive}
\letterAddress{Мой улюбённы сын,}
Я бы сказаў, што табе пашанціла сустрэць чалавека, які карыстаецца блізкім даверам у
нашага сябра і цэннага звязніка Северуса Снэйпа. 
\end{writtenNoteCursive}

Пэўны час Драко задумённа глядзеў на ліст, і потым кінуў яго ў агонь.
