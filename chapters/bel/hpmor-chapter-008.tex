
\chapter{Пазітыўная схільнасць}

\begin{chapterOpeningQuote}
  Парярэджваю: кідаць выклік маёй вынаходлівасці~--- ідэя небяспечная,
  бо твае жыццё адразу стане неймаверна сюррэальным.
\end{chapterOpeningQuote}

\lettrine{Н}{ікому} не была патрэбна яе дамамога, вось у чым праблема.
Усе проста сядзелі, елі, глядзелі па баках, пакуль іхнія бацькі пляткарылі.
Чамусьці ніхто нічога не чытаў, што значыла, яна не магла проста так 
сесьці побач з кімсьці, раскрыць кнігу і пачаць чытаць. І нават калі яна вырашыла сесьці 
паасобку, раскрыць \emph{Гісторыю Хогвартс}, і пачаць чытаць, ніхто не гарэў 
жаданнем да яе далучыцца. 

Яна не ведала, як знакоміцца з людзьмі, акрамя як дапамагаць ім з заняткамі, хаця 
і не адчувала сябе сарамлівай дзяўчынай. Яна адчувала сябе як дзяўчына, якая бярэ
ініцыятыву ў свае рукі. Аднак, пакуль ніхто яе не пытаў аб чымсьці кшталту 
"як падзяліць вялікі лік вугалком?", ёй было няёмка падысці да каго-небудзь і 
сказаць... што? Яна ніколі не магла прыдумаць, што. І для гэтага не існавала 
нейкай тлумачальнай схема, вось што было дзіўна. Канцэпцыя знаёмства наогул 
не мела для яе сэнсу. Чаму \emph{яна} павінна была браць адказнасць, калі сустракаліся
два чалавека? І чаму дарослыя ніколі не дапамагалі? Ён было б значна лягчэй, калі
іншая дзяўчына падышла да яе са словамі: "Герміёна, настаўнік даў заданне нам пасябраваць".

Але гэта ніякім чынам не значыла, што Герміёна Грэнджэр, --- якая сядзела адна ў пустым 
купэ апошняга вагону, з расчыненымі дзярыма, на выпадак, калі нехта па нейкай прычыне
захоча з ёю паразмаўляць, --- \emph{не была} сумная, змрочная, у дэпрэсіі, або нейкім іншым 
чынам перажывала аб гэтых праблемах. Яна з задавальненнем перачытвала \emph{Гісторыю Хогвартс}
ў трэці раз, і толькі лёгкая цень раздражнення на нерацыянальнасць гэтага свету
лунала дзесьці на фоне яе свядомасці. 

Пачуўся гук дзвярэй паміж вагонамі, якія расчыніліся, і потым зачыніліся, потым
шагі, і вельмі дзіўны гук, быццам па падлозе цягнулі нешта цяжкае. Герміёна паклала 
\emph{Гісторыю Хогвартс} на сядзенне, паднялася і выглянула ў калідор --- проста на 
выпадак, калі камусьці патрэбіцца яе дапамога, --- і ўбачыла хлопца ў мантыі, 
судзячы па росце --- перша- або другакурсніка, які выглядаў даволі бязглузда, бо яго твар
быў замотаны шалікам. За ім стаяў невялікі куфар. Хлопец прагрукаў у зачыненыя дзверы суседняга
купэ, адчыніў іх і спытаў: "Прабачце, можна пытанне?".

Яна не чула адказу з купэ, але яго пытанне было --- калі яна
нічога не пераблытала: "Хто-небудзь
можа назваць шесць тыпаў кваркаў, або дзе я магу знайсці першакурсніцу па імі Герміёна Грэнджэр?"

Калі ён зачыніў дзверы, Герміёна спытала:

--- Магу я нечым дапамагчы?

Твар, пакрыты шалікам, павярнуўся ў яе напрамку:

--- Толькі калі ты можаш назваць мне шесць тыпаў кваркаў, або ведаеш, дзе знайсці
першакурсніцу па імі Герміёна Грэнджэр.


--- Уверх, уніз, дзіўны, чароўны, сумленны, прыгожы,\footnote{{} Дэфакта стандартныя
імёны тыпаў кваркаў --- ангельскія: up - down - strange - charm - truth/top - beauty/bottom.}
і навошта табе патрэбная першакурсніца па імі Герміёна Грэнджэр?

Было складана сказаць, але ёй падалося, што хлопец шырока ўсміхнуўся пад 
шалікам.

--- А-а, дык гэта  \emph{ты}  першакурсніца па імі Герміёна Грэнджэр, --- сказаў 
прыглушаны голас. --- І ты адзназначна на цягніке, які ідзе на Хогвартс, --- хлопец
працягваў набліжацца, мармытаючы сабе пад нос: --- ...тэхнічна, усё, што я павінен
быў зрабіць --- гэта проста \emph{шукаць} цябе, але лагічна меркаваць, што мелася на
ўвазе паразмаўляць з табой, або запрасіць цябе далучыцца да маёй каманды, або 
атрымаць ад цябе ключавы магічны прадмет, або высветліць, што Хогвартс быў 
пабудаваны на развалінах старажытнага храма, і гэтак далей. \abbrev{pc} або \abbrev{npc},
вось у чым пытанне.

Герміёна адчыніла рот, каб адказаць, але так і не здолела прыдумаць любы \emph{магчымы}
адказа на гэта...  \emph{нешта}, што яна зараз пачула, а ён гэтым часам падышоў, паглядзеў 
унутр купэ, задаволена кіўнуў, і сеў на сядзенне насупраць. Ягоны куфар запоўз за ім, 
павялічыўся ў памеры ў тры разы, і прыстроіўся побач з куфарам Герміёны самым непрыстойным 
чынам.

--- Калі ласка, садзісь, --- сказаў хлопец, --- і калі ласка, зачыні за сабою дзверы,
калі табе не складана. Не хвалюйся, я не кусаю нікога, хто не ўкусіў мяне першы, ---
і ён пачаў разматваць шалік.

Намёк на тое, быццам яна \emph{пужалася} быў дастатковым, каб яе рука моцна штурхнула
дзверы так, што яны ляпнулі аб сцяну з нечаканай сілай. Яна абярнулася і ўбачыла 
твар хлопца, вясёлыя зялёныя вочы, злосны цёмна-чырвоны шнар на яго ілбе, які
аб чымсьці ёй адразу нагадаў, але зараз у яе былі больш важныя справы. 

--- Я не сказала, што я Герміёна Грэнджэр!

--- \emph{Я} не казаў, што ты \emph{сказала}, што ты Герміёна Грэнджэр, я сказаў,
што ты і ёсць Герміёна Грэнджэр. Калі табе цікава, скуль я ведаю, дык гэта таму, што 
ведаю ўсё. Добры вечар, ледзі і джэнтльмены, мяне клічуць Гары Джэймс Потэр-Эванс-Верэс,
або скарочана Гары Потэр, і я думаю, для разнастайнасці, што табе гэта ні аб чым не кажа...

Розум Герміёны нарэшце зразумеў сувязь. Шнар на яго ілбе, у форме маланкі.

--- Гары Потэр! Пра цябе напісана ў \emph{Сучаснай гісторыі магіі}, і ва 
\emph{Усходзе і заходзе цёмных майстэрстваў} і ў 
\emph{Вялікіх чароўных падзеях дваццатага стагоддзя}, --- упершыню ў жыцці яна 
сустрэла кагосьці  \emph{з кнігі,} і гэта было вельмі дзіўнае адчуванне. 

Хлопец мігрнуў тры разы ўрад. 

--- Пра мяне ---  \emph{у кнігах?} Стой... ну канешне так і павінна... якая дзіўная 
думка.

--- Божачкі, ты не ведаў? --- спытала Герміёна. --- На тваім месцы я бы паспрабавала
высветліць усё пра сябе.

Хлопец сказаў сухавата:

--- Міс Грэнджэр, прайшло меней за 72 гадзіны з майго першага візіту ў 
Дыягон-аллею, дзе я дазнаўся пра тое, што я --- знакамітасць. Я патраціў 
апошнія два дні, набываючы навуковыя кнігі. \emph{Можаш мне паверыць},
я даклана збіраюся высветліць усё, што я магу, --- ён завагаўся. --- І 
\emph{што} пішуць пра мяне ў кнігах?

У Герміёны ў галаве быццам замігралі мігалкі з сірэнамі: яна і не чакала, што 
ў яе будзе тэст па \emph{тых} кнігах, бо яна паспела прачытаць іх 
толькі па аднаму разу, але з таго часу яшчэ і месяца на прайшло, і яна памятала 
добра:

--- Ты адзіны чалавек у свеце, хто выжыў пасля забівальнага заклёна, таму цябе 
клічуць Хлопец-Які-Выжыў. Ты нарадзіўся ў сям'і Джэймса Потэра і Лілі Потэр (у
дзявоцтве --- Лілі Эванс) 31 ліпеня 1980 года. 31 кастрычніка 1981 года 
Цёмны Лорд Той-Якога-Нельга-Называць --- але я не зразумела чаму --- напаў на 
ваш дом, знаходжанне якога выдаў яму Сірыус Блэк, але там не сказана, як дазналіся,
што выдаў менавіта ён. Цябе знайшлі жывым, са шнарам на ілбе, у развалінах дома, 
побач былі згарэлыя рэшткі цела Самі-Ведаеце-Каго. Найвышэйшы Чарнакніжнік Альбус Персіваль
Вулфрык Брайан Дамблдор дзесьці цябе схаваў, і ніхто не ведае, дзе.  
\emph{Усход і заход цёмных майстэрстваў} кажа, што ты выжыў праз любоў сваёй маці,
і што твой шнар утрымлівае ўсю магічную сілу Цёмнага Лорда, і што цэнтаўры цябе 
баяцца, але \emph{Вялікія чароўныя падзеі дваццатага стагоддзя} не згадваюць
нічога падобнага, \emph{Сучасная гісторыя магіі} папярэджвае, што 
на твой конт існуе неверагодна шмат бязглуздых тэорый.

У хлоца ажно сківіца адвісла. 

--- Табе нехта казаў чакаць Гары Потэра на цягніку ў Хогвартс? Або нешта такое?

--- Не, --- адказала Герміёна, --- хто сказаў табе пра мяне?

--- Прафесар МакГонагал, і я думаю, разумею, чаму. У цябе фотаграфічная памяць, Герміёна?

Герміёна пакачала галавой:

--- Не фотаграфічная, хаця мне заўсёды хацелася. Мне проста прыходзіцца перачытваць
кнігі па пяць разоў, каб іх запомніць.

--- Нічога сабе, --- сказаў хлопец крыху сціснутым голасам. --- Я спадзяюся, 
ты не супраць, калі мы гэта праверым? Не тое, каб я табе не верыў, 
але прымаўка кажа " давярай, але правярай". Навошта гадаць, калі можна проста 
правесці эксперымент.

Герміёна ўсміхнулася крыху самазадаволена. Яна так любіла тэсты. 

--- Давай.

Хлопец засунуў руку ў кашэль на поясе і сказаў "Магічныя настоі і зёлкі Арсенія Джыгера".
Калі ён дастаў руку, у ён была кніга, якую ён загадаў.

Імгненна Герміёне захацелася такой кашэль болей за ўсё, што ёй калісьці хацелася.

Хлопец раскрыў кнігу дзесьці ў сярэдзіне і паглядзеў на старонку. 

--- Калі ты робіш \emph{алей уважлівасці}...

--- Ведаеш, я таксама \emph{бачу} старонку!

Ён падняў кнігу, каб ён не было бачна, і перагарнуў некалькі старонак. 

--- Калі ты варыш \emph{зёлкі павучынай спрытнасці,} што ты павінна дадаць пасля 
шоўка Акрамантулы?

--- Пасля шоўка чакай, пакуль зёлкі не стануць дакладна колеру 
бязвоблачнага росквіту, восем градусаў і восем хвілін ад гарызонта, як толькі 
край сонца стане бачны. Памяшаць восем разоў супраць гадзіннікавай стрэлкі, 
адзін раз --- па стрэлке, потым дадаць восем драм бугераў\footnote{{}Нават не пытайце мяне,
што гэта за халера.} адзінарога. 

Хлопец закрыў кнігу з рэзкім ляпам, і схаваў яе назад у кашэль, які праглынуў 
яе з ціхім "коўць". 

--- Ну і ну і ну і \emph{ну} і ну! Я бы хацеў зрабіць табе прапанову, міс Грэнджэр.

--- Прапанову? --- сказала падазрона Герміёна. Руплівай дзяўчыне не варта чуць 
такія рэчы ад хлопцаў. 

У гэты момант Герміёна зразумела яшчэ адну рэч --- сярод ішных, канешне, ---
высвятлялася, што чалавек \emph{з кнігі} і размаўляў, \emph{як кніга}.
Гэта таксама было даволі дзіўнае адкрыццё.

Хлопец залез у кашэль, сказаў "банка газіроўкі", дастаўшы яскрава-зялёны цыліндр,
і працягнуў ёй:

--- Магу я прапанаваць табе прахладжальны напой?

Герміёна ветліва прыняла банку. Дарэчы, яна ўжо адчувала пэўную смагу.

--- Вялікі дзякуй, --- сказала яна, адчыняючы газіроўку. --- Гэта і была 
твая прапанова?

Хлопец кашлянуў. 

--- Не, --- сказаў ён. Калі Герміёна пачала піць, ён дадаў: --- Я бы хацеў, 
каб ты дапамагла мне захапіць сусвет.

Герміёна скончыла піць, і апусціла банку. 

--- Не, дзякуй. Я не злодзей.

Хлопец глядзеў на яе здзіўлена, быццам ён чакаў нейкі іншы адказ. 

--- Ну... я казаў крыху рытарычна, --- сказаў ён. --- Гэта нешта кшталту Бэканаўскага
праекту, ведаеш, не ў сэнсе палітычнай улады. "Ажыццяўленне ўсяго магчымага", і гэтак
далей\footnote{{} Гары цытуе ўтапічны твор Ф. Бэкана "Новая Атлантыда": 
"Мэта нашага грамадства --- разуменне прычын і таямнічых сіл свету; і пашырэнне
межаў чалавечай улады [над прыродай] да ажыццяўлення ўсяго магчымага."}. 
Я хачу эксперыментальна даследаваць заклёны, высветліць, якія законы імі 
кіруюць, прынесці магію ў навуку, аб'яднаць магічны і маглаўскі свет, 
павысіць стандарты жыцця на ўсёй планеце, прасунуць прагрэс на стагоддзі наперад,
знайсці сакрэт бяссмерця, каланізаваць сонечную сістэму, даследаваць галактыку, 
і самае галоўнае --- што за халера тут сапраўды адбываецца, бо ўся гэтая магія 
--- проста неймаверная бязглудзіца.

Гучала ўжо цікавей.

--- І?

Хлопец нахмурыўся.

--- \emph{І?} Табе гэтага \emph{не хапіла?}

--- І што ты хочаш ад мяне? --- сказала Герміёна.

--- Ну канешне, я хачу, каб ты дапамагла мне ў даследаванні. З тваёй энцыклапедычнай
памяццю і маім розумам і  рацыянальнасцю мы скончым гэта праект вельмі хутка, і пад "вельмі хутка" 
я маю на ўвазе прынамсі трыццаць пяць гадоў.

Хлопец пачынаў яе раздражняць.

--- Я не бачыла, каб ты зрабіў нешта разумнае. Можа гэта \emph{я} дазволю
\emph{табе} дапамагаць у \emph{маім} даследаванні. 

Пэўны час у купэ было ціха.

--- Дак ты заклікаеш мяне прадэманстраваць май розум, --- сказаў хлопец пасля доўгай паузы.

Герміёна кіўнула.

--- Парярэджваю: кідаць выклік маёй вынаходлівасці --- ідэя небяспечная,
бо твае жыццё адразу стане неймаверна сюррэальным.

--- Не вельмі ўразліва, --- сказала яна. Яе рука з банкай у руцэ пачала 
падымацца.

--- Ну, магчыма \emph{гэта} цябе ўразіць, --- сказаў хлопей. Ён падаўся наперад і 
ўтаропіўся ў яе. --- Я ўжо правёў некалькі эксперыментаў, і высветліў, што 
мне не патрэбная палачка, і што я магу зрабіць што заўгодна, проста цокнуўшы пальцамі.

Герміёна як раз піла ў гэта момант, яна падавілася, кашлянула, і выплюнула 
паток зялёнай вадкасці.

На сваю новую, ні разу не ношаную мантыю --- у першы школьны дзень! 

Яна ўскрыкнула, гэты пранізлівы гук у маленькім купэ ударыў на вушах мацней 
за супрацьпаветраную сірэну.

--- \emph{А-а-а! Мая мантыя!}

--- Спакуха! --- сказаў хлопец. --- Я магу ўсё выправіць, --- ён падняў руку, і 
цокнуў пальцамі.

--- Ты... ты... --- яна перавела позірк з яго на сябе.

Залёныя плямы яшчэ былі там, але на вачах пачалі знікаць, і праз некалькі секунд 
яе мантыя выглядала, як новая.

Герміёна ўтаропілася ў хлопца, на твары якога была вельмі самазадаволеная ўсмешка. 
Беспалачкавая магія! У  \emph{яго} ўзросце? Пры тым, што ён атрымаў свае падручнікі
тры дні таму? Потым яна ўспомніла, пра што казала раней, ахнулі, і прасунулася далей ад яго.
\emph{Уся сіла Цёмнага Лорда! У яго шнары!}

Яна хутка ўскочыла. 

--- Мне, мне, мне, мне трэба ў туалет, нікуды не сыходзь... --- яна была павінна 
знайсці некага з дарослых, расказаць ім аб... аб...

Усмешка  сышла з яго твару.

--- Гэта проста фокус, Герміёна. Выбачай, калі ласка, я не хацеў цябе так спужаць...

Яе рука замерла на дзвярной ручцы.

--- \emph{Фокус?}

--- Фокус, --- сказаў хлопец. --- Ты спытала паказаць маю разумнасць, і я зрабіў
нешта немагчымае, што заўсёды добры спосаб выпендрыцца. Насамрэч, я нічога не магу
зрабіць, цокнуўшы пальцамі... Прынамсі, \emph{мне падаецца}, што не магу, я ніколі
не спрабаваў сур'ёзна, --- ён падняў руку і зноў цокнуў. --- Не, нічога ніякага.

Герміёна ў жыцці не адчувала сябе так разгублена. 

Убачыўшы гэта, хлопец зноў усміхнуўся. 

--- Я парярэдзіў, кідаць выклік маёй вынаходлівасці робіць тваё жыццё сюррэальным.
Будзь уважлівай, калі я папярэджваю аб чымсьці.

--- Але... але... --- Герміёна пачала заікацца, --- Тады што \emph{менавіта} ты зрабіў?

Позірк хлопца стаў ацэніваючым, вагаючым, чаго яна ніколі не бачыла ў 
кагосьці гэтага ўзросту. 

--- Ты думаеш, што маеш здольнасці вучонага, і можаш займацца навукай 
са мной або без мяне. Давай паназіраем, як \emph{ты} будзеш даследваць гэты 
заблытаны феномен.

--- Я... --- розум Герміёна быццам раптам апусцеў. Ёй падабаліся экзамены і тэсты, 
але такой праверкі яна ніколі не прахадзіла. Ліхаманкава яна намагалася знайсці 
ў памяці хоць нешта пра тое, што павінны рабіць навукоўцы. Яе розум пахрыбусцеў
шэсцернямі, потым завёўся, і выпаліў ёй спіс інструкцый па школьным навуковым праектам:

\begin{enumerate}[Крок 1:]\firmlist
\item Выказаць гіпотэзу.
\item Правесці эксперымент, каб праверыць гэтую гіпотэзу.
\item Сабраць дадзеныя.
\item Зрабіць плакат.
\end{enumerate}


Першым крокам было выказаць гіпотэзу. Гэта значыць --- прыдумаць нейкае \emph{магчымае}  
тлумачэнне таму, што адбылося. 

--- Добра. Мая гіпотэза такая: ты скаставаў зачараванне на маю мантыю, якое
прыбірае плямы ад нечага разлітата на яе.

--- Добра, --- сказаў хлопец, --- гэта твой адказ?

Першы шок пачаў праходзіць, і розум Герміёны пачаў працаваць нармальна. 

--- Стой, гэта не вельмі добрая ідэя. Я не заўважыла, каб ты кранаўся
чароўнай палачкі, або казаў заклён, таму як ты мог каставаць нешта?

Ён маўчаў, трымаючы нейтральны выраз.

--- Але давай уявім, што мантыя ўжо прадавалася зачараваная аўтаматычна чысціцца, --- 
што было бы вельмі добра, дарэчы, --- і ты выпадкова выліў на сябе штосьці дагэтуль, і так
дазнаўся пра гэта.

Бровы хлопца крыху тузануліся ўверх:

--- \emph{Гэта}  твой адказ?

--- ...не, бо я не зрабіла крок №2. Эксперымент, каб праверыць гіпотэзу.

Хлопец усміхнуўся.

Герміёна паглядзела на банку газіроўкі ў сваёй руцэ, і высветліла, што вадкасці
ў ёй яшчэ адна трэць.

--- Значыцца так, --- сказала Герміёна, --- эксперымент такі: я вылью крыху на сваю
мантыю, і я чакаю, што пляма знікне сама. Але калі гэта не спрацуе, то мая мантыя 
стане брудная, а я гэтага не хачу.

--- Лей на маю, --- сказаў Гары, --- і не будзеш хвалявацца пра сваю вопратку.

--- Але... --- сказала яна. З гэтым планам было нешта не так, але яна не магла 
сказаць дакладна, што.

--- У мяне ёсць запасная ў куфары, --- сказаў хлопец.

--- Але табе недзе пераапрануцца, --- запратэставала яна, потым падумала, і сказала:
--- Хаця, я магла бы выйсці і зачыніць дзверы...

--- Я магу пераапрануцца таксама ў куфары.

Герміёна глянула на яго куфар, які, як яна пачала падазраваць, моцна адрозніваўся
ад яе ўласнага.

--- Ну добра, --- сказала яна, --- як скажаш... --- і яна асцярожна наліла крыху 
газіроўкі на край яго мантыі, і пачала назіраць, спрабуючы ўспомніць, за колькі 
яна знікла мінулым разам...

І плямы зніклі!

Герміёна з палёгкай выдахнула, не ў апошнюю чаргу таму, што яна не мела справы
з усёй сілай Цёмнага Лорда.

Так, крок №3 быў сабраць дадзеныя, але тут даденыя былі простыя: плямы зніклі.
І, магчыма, можна было абысціся без шага №4. 

--- Мой адказ, што мантыі зачараваныя на чысціню.

--- Няправільна.

Яна адчула ўкол расчаравання. Ён моцна хацелася не адчуваць сябе \emph{так}, хлопец
не быў настаўнікам, але гэта быў тэст, і яна дала няправільны адказ, што 
заўсёды адчувалася як невялікі ўдар пад дых.

(Людзі кажуць, што ўсё, што ты трэба ведаць пра Герміёну Грэнджэр --- што нішто на 
свеце не можа спыніць яе ад праверкі яе ведаў.)

--- Сумная часка ў тым, --- сказаў хлопец, --- што ты, магчыма, зрабіла ўсё, што 
было напісана ў нейкай кнізе. Ты прадказала розніцу паміж зачараванай і звычайнай 
мантыяй, праверыла гіпотэзу, абвергла нулявую гіпотэзу (што мантыя была звычайнай).
Але толькі лепшыя кнігі вучаць цябе як займацца навукай \emph{дасканала}. У сэнсе ---
каб сапраўды вырашыць праблему, а не проста даслаць чарговы артыкул у часопіс,
пра якія бацька заўсёды скардзіцца. Таму дазволь мне патлумачыць --- не кажучы
табе пакуль правільны адказ, --- што ты зрабіла не так, і пасля мы паспрабуем яшчэ раз.

Яна адчула абурэнне, бо хлопец вёў сябе занадта пагардліва для такога ж як і яна,
адзінаццацілетняга хлопца, але патрэбнасць зразумець памылку перавагала.

--- Давай.

Яго выраз ажывіўся:

--- Гэта гульня, заснаваная на вядомым эксперыменце, называецца "задача 2/4/6", і
правілы ў яе такія. Існуе нейкае правіла --- вядомае толькі мне, --- якое 
адзначае пэўныя тройкі лікаў. Напрыклад, 2/4/6 --- тройка, якая задавальняе правілу.
Дарэчы... давай я гэтае правіла запішу і дам табе, каб ты была ўпэўнена, што я не
мяняю правіла па ходу гульні. Не падглядвай, бо я так разумею з прошлага разу, што ты здольная чытаць
перавёрнуты тэкст. 

Ён сказаў "папера" і "аловак" свайму кашалю, і яна дэманстратыўна зажмурыла вочы,
пакуль ён пісаў.

--- Гатова, --- сказаў ён, і працягнуў ёй складзеную паперку, --- пакладзі ў 
кішэнь, --- што яна і зрабіла.

--- Працэс гульні выглядае так: ты даеш мне тры ліка, а я адказваю "Так", калі 
яны задавальняюць правілу, і "Не", калі не задавальняюць. Я ў гэтай гульне ---
Прырода, і правіла --- адзін з маіх законаў, а ты --- навукоўца. Ты ўжо ведаеш, што
2/4/6 добрая камбінацыя. Калі ты зробіш усе неабходныя эксперыменты --- у сэнсе,
спытаеш мяне ўсе камбінацыі, якія хочаш, --- ты спыняешся і абвяшчаеш правіла,
і тады ты можаш дастаць з кішэні паперку, і праверыць рэзультат. Ты разумееш гульню?

--- Ну канешне, я разумею, --- сказала Герміёна.

--- Паехалі.

--- 4/6/8, --- сказала Герміёна.

--- Так.

--- 10/12/14?

--- Так.

Герміёне падавалася, што яна правяла ўжо ўсе неабходныя эксперыменты, але 
гэта ж не магло быць настолькі проста?

--- 1/3/5.

--- Так.

--- Мінус 3 / мінус 1 / 1.

--- Так.

Яна не магла прыдумаць больш нічога. 

--- Правіла ў тым, што кожны наступны лік на два больш папярэдняга.

--- Дарэчы, --- сказаў хлопец, --- гэты тэст складанеей, чым падаецца, і толькі 
20\% дарослых яго праходзяць. 

Герміёна нахмурылася. Што ж яна ўпусціла? Потым раптам яна прыдумала, што яшчэ праверыць.

--- 2/5/8! --- пракрычала яна трыўмфальна.

--- Так.

--- 10/20/30! 

--- Так.

--- Тады сапраўдны адказ такі: лікі павялічваюцца кожны раз аднолькава. Не толькі на два.

--- Цудоўна, --- сказаў хлопец, --- а зараз дастань паперу і правер свой адказ.

Герміёна дастала паперку і разгарнула яе.

\emph{Тры рэчаісных ліка ў парадку іх павялічэння.}

У Герміёны ўпала сківіца. У яе было выразнае адчуванне, што яе зараз жудасна падманулі,
што хлопец быў брудны і гадкі брахун, але крыху падумаўшы, яна не ўспомніла 
ніводнага хлуслівага адказу з яго боку.

--- Ты толькі што адкрыла феномен пад назвай "пазітыўная схільнасць", --- сказаў хлопец.
--- У цябе было нейкае правіла ў галаве, і ты працягвала называць толькі тройкі, на
якія чакала пазітыўны адказ. Але ты не спрабавала свядома праверыць некалькі 
адмоўных варыянтаў. Дарэчы, ты ніводнага разу не чула "Не" у адказ, і правіла 
проста магло быць "любыя тры лікі". Вось так людзі ўяўляюць
сабе эксперымент для пацверджання сваёй гіпотэзы, замест таго, каб абвергнуць яе.
Гэта не страшная, але ўсё ж такі памылка. Табе трэба навучыцца бачыць негатыўны
бок рэчаў, глядзець у цемру. Як я казаў, толькі 20\% дарослых праходзяць гэты тэст.
Астатнія прыдумляюць неверагодна складаныя гіпотэзы і моцна вераць у свае 
памылковыя дадзеныя, бо яны ж зрабілі так шмат эксперыментаў, і ўсё сходзіцца.

Герміёна маўчала.

--- А зараз, --- працягнуў ён, --- ці хочаш ты паспрабаваць яшчэ раз з 
першай праблемай?

Ён ажывіўся і глядзеў на яе пільна, быццам толькі зараз пачынаў \emph{сапраўдны}
тэст.

Герміёна заплюшчыла вочы і  паспрабавала сканцэнтравацца. Пад мантыяй яе прашыб пот.
Гэта быў самы складаны тэст... а магчыма і наогул першы сапраўдны 
экзамен у яе жыцці, дзе прыходзілася \emph{думаць}.

Якія іншыя эксперыменты яна магла зрабіць? У яе была шакаладная жаба, 
яе можна было пацерці аб мантыю і паглядзець, ці знікне пляма.
Але гэта ўсё роўна не выглядала як перакручанае негатыўнае стаўленне, пра ягое 
казаў хлопец. Быццам яна ўсё яшчэ чакала "Так" у адказ на сваю перашапачатковую 
гіпотэзу, замест "Не".

Але, па яе гіпотэзы... у якім выпадку газіроўка  \emph{не} знікне?

--- Яшчэ адзін эксперымент, --- сказала яна. --- Я налью крыху газіроўкі на 
падлогу, і пагляжу, ці знікне яна. У цябе ёсць папяровыя ручнікі, або нешта такое,
каб працерці, калі не спрацуе?

--- Ёсць насоўкі, --- сказаў ён, выглядаючы нейтральна.

Герміёна ўзяла банку і наліла крыху на падлогу.

Праз некалькі секунд сляды газіроўкі зніклі.

--- Эўрыка, -- сказала яна ціха. Гэта было рэфлекторна, і яна павінна была пракрычаць
гэта, але яна адчувала сябе занадта прыгнечана. Разуменне ўдарыла ёй у галаву, і ёй
самой захацелася сябе стукнуць. --- Ну канешне! \emph{Ты} даў мне газіроўку! 
Не мантыі былі зачараваныя, а газіроўка!

Хлопец падняўся і сур'ёзна ёй пакланіўся. Цяпер ён шырока усміхаўся. 

--- Тады, магу я дапамагаць табе ў тваім даследаванні, Герміёна Грэнджэр?

--- Я... э-э... --- яна не была ўпэўнена, як на такое адказаць.

Ёй дапамог слабы, ціхі,  \emph{няўпэўнены} стук у дзверы.

Хлопец адвярнуўся да акна, і сказаў: 

--- Я пакуль без маскіроўкі, ты можаш адчыніць?

Тут да яе дайшло, чаму хлопец... не --- Хлопец-Які-Выжыў --- быў замотаны ў шалік, 
і адчула сябе няёмка за тое, што не падумала пра гэта раней. Насамрэч, 
гэта падавалася дзіўна, бо  чалавек кшталту Гары Потэра хутчэй быў павінны
ганарыста паказвацца ўсяму сусвету; і тады яна падумала, што ён, магчыма,
крыху сарамлівей, чым выглядае.

Калі Герміёна адкаціла дзверы, яе сустрэў дрыжачы хлопчык, які выглядаў 
дакладна як ягоны стук.

--- Прабачце, --- сказаў ён ціхі голасам. --- Я Нэвіл Лонгботам. Я шукаю сваю 
жабу, я... я... не магу знайсці яе... можа вы бачылі жабу?

--- Не, --- сказала Герміёна, і тут яе імкненне дапамагаць уключылася на поўную
моц. --- Ты правяраў іншыя купэ?

--- Правяраў, --- прашаптаў хлопчык.

--- Тады мы проста абшукаем астатнія вагоны, --- сказала рашуча Герміёна. --- 
Я дапамагу табе. Мяне завуць Герміёна Грэнджэр, дарэчы.

Здавалася, што ён страціць прытомнасць ад удзячнасці.

--- Пачакай, --- пачуўся голас першага хлопца --- Гары Потэра. --- Мне 
падаецца, што гэта не лепшы спосаб.

Герміёна абярнулася да яго, уззлаваная. Калі Гары Потэр мог кінуць 
маленькага хлопчыка ў бядзе проста таму, што не хацеў, каб яго перарывалі...

--- Што? Чаму не?

--- Ну, --- сказаў Гары, --- трэба шмат часу, каб абшукаць цягнік цалкам, і мы 
ўсё роўна можам прапусціць яе недзе, а калі мы не знойдзем да прыбыцця ў 
Хогвартс, будзе дрэнна. У нас больш шанцаў, калі мы пойдзем у першы вагон,
да прэфектаў, і запытаем іх дапамагчы. Я так і зрабіў, калі шукаў цябе, Герміёна,
хоць яны і не ведалі. Але на конт жабы --- яны могуць ведаць нейкія заклёны, якія
могуць паскорыць працэс. А мы яшчэ нават не першакурснікі.

Гэта... гэта было вельмі разумна.

--- Як думаеш, ты здолееш самастойна дабрацца да вагона прэфектаў? --- спытаў 
Гары Потэр. --- У мяне ёсць прычыны не паказвацца на людзях.

Раптам Нэвіл ахнуў і адышоў на крок назад. 

--- Я памятаю твой голас! Ты адзін з Лордаў Хаоса!  \emph{Ты той, хто даў мне цукерку!}

Што? Што? Што-што-\emph{што?}? 

Гары Потэр павярнуўся да іх і драматычна падняўся на ногі. 

--- Я \emph{ніколі!} --- сказаў ён голасам, поўным абурэння. --- Я што, 
выглядаю як злодзей, які можа даць дзіцёнку \emph{цукерку?} 

Вочы Нэвіла пашырыліся. 

--- \emph{Ты} Гары Потэр? \emph{Той самы} Гары Потэр? \emph{Ты?}

--- Не, я проста \emph{нейкі} Гары Потэр, тут на цягніке яшчэ тры мяне...

Нэвіл піскнуў і ўбяжаў прэч. Яны чулі яго хуткія крокі ў калідоры, потым як хлопнулі 
дзверы ў суседні вагон.

Герміёна рэзка села на сядзенне. Гары зачыніў дзверы і сеў побач.

--- Растлумач мне, калі ласка, што тут адбываецца? --- сказала яна слабым голасам.
Цікава, ці заўсёды разам з ім яна будзе адчуваць сабе так разгублена.

--- Ну... ладна. Здарылася вось што: Фрэд, Джордж і я ўбачылі на платформе
маленькага няшчаснага хлопчыка --- дама, якая была з ім, кудысьці адышла,  і ён
выглядаў вельмі спалоханым, быццам на яго зараз нападуць Пажыральнікі Смерці.
Прымаўка кажа, што страх чагосьці заўсёды горш за гэтае чагосьці, і я падумаў,
што вось пацан, якому сустрэць свой самы страшны жах можа пайсці на карысць, 
і ён зразумее, што яно не такое, як ён баяўся...

Герміёна сядзела, адкрыўшы рот. 

--- ...вось, і Фрэд і Джордж кастанулі заклён, каб зрабіць шалікі ў нас на тварах
цёмнымі і расплыўчатымі, быццам мы былі ажыўлённыя каралі-чарнакніжнікі, а тое былі 
нашыя магільныя накідкі...

Ёй зусім не падабалася, куды ішоў яго разсказ.

--- ...і калі мы аддалі яму ўсе цукеркі, што я купіў, мы пакалі гукаць кшталту 
"Давай дазім яму крыху грошай! Ха-ха-ха! Трымай пару кнатаў, хлопча! Трымай сыкль!" 
і танчылі вакол яго, і злосна смяяліся, і гэтак далей. Думаю, некаторыя людзі вакол
спачакту хацелі ўмяшацца, але "сведчыцкая апатыя" утрымлівала іх да часу, пакуль мы
не пачалі частаваць яго цукеркамі, і тады ўсё стала канчаткова незразумела.
Нарэшце ён сказаў сваім ціхім танюсенькім голасам "Уйдзіце", і мы ўсе закрычалі 
нешта кшталту, як свет нас апаляе, і ў паніке ўбяжалі. Спадзяюся, наступным разам
ён будзе крыху смялей падчас булінга. Дарэчы, гэта называецца "тэхніка дэсенсібілізацыі".

Ох, як жа яна  \emph{адразу} не дагадался, да чаго ўсё прыйдзе. 

Апаляючы агонь абурэння, які быў адным з асноўных унутраных рухавікоў Герміёны, 
адразу завёўся на поўныя абароты (нават калі яна часткова разумела, што Гары 
спрабаваў здзейсніць). 

--- Гэта агідна! \emph{Ты} агідны! Гэты бедны хлопчык! Тое, што зрабіў, было... 
злобна і... дрэнна!

--- Думаю, слова, якое ты шукаеш --- \emph{здзекліва,} але ў любым выпадку, 
спытай сябе, чаго ў выніку мы зрабілі больш --- добрага або дрэннага?  
Калі ў цябе ёсць чаго дадаць на гэтую тэму, я буду рады паслухаць, але 
я адмаўляюся прымаць любую іншую крытыку, пакуль гэтае пытанне не вырашана.
Я, канешне, згодны, што яно ўсё  \emph{выглядае} як жудасны булінг, 
бо тут ёсць спужаны хлопчык і страшныя монстры, але 
гэта не галоўная праблема, ці не так? І дарэчы, гэты падыход называецца
\emph{кансеквенцыялізм,} калі цэннасць учынка не залежыць ад таго, што ён 
\emph{выглядае} злым або дрэнным, адзінае пытанне толькі фінальны вынік, 
наступствы ўчынка.

Герміёна раскрыла рот, каб сказаць нешла \emph{з'едлівае}, але, нажаль, яна 
забыла, што трэба мець штосьці, што сказаць, перад тым як адкрываць рот. Усё, 
што яна здолела выдавіць, было:

--- Што, калі яму будуць сніцца \emph{кашмары?}

--- Шчыра кажучы, ён не выглядае, быццам яму патрэбна знешняя дапамога, каб 
сніць кашмары. А калі ў яго будуць кашмары пра нас, дык ён будзе бачыць, як 
монстры частуюць яго цукеркамі, і ў гэтым і ёсць увесь сэнс.

Мозг Герміёны разгублена шчоўкаў і спыняўся кожны раз, калі яна спрабавала 
сапраўды ўззлавацца. 

--- Твае жыццё заўсёды такое дзіўнае?

Гары Потэр асвяціўся гонарам.

--- Я \emph{раблю} яго дзіўным. Перад табой стаіць прадукт цяжкай шматгадовай працы.

--- Дык... --- сказала Герміёна, і гэта так і павісла ў паветры.

--- Дык вось, --- сказаў Гары Потэр, --- што дакладна ты ведаеш з навукі? Я 
ведаю матаналіз, і крыху тэорыі Байеса, крыху тэорыі прыняцця рашэній, 
і шмат чаго з розных кагнітыўных навук, я чытаў  \emph{Фейнманаўскя лекцыі}
(прынамсі першы том), і \emph{Прыняцце рашэнніяў ва ўмовах неакрэсленнасці}, 
і \emph{Мова ў думках і ўчынках}, 
і \emph{Уплыў: тэорыя і практыка}, 
і \emph{Рацыянальны выбар у невызначаным свеце}, 
і \emph{Гёдэль, Эшер, Бах}, 
і \emph{Крок далей наперад}, 
і\footnote{{} Арыгінальныя назвы кніг, каб вы маглі самі знайсці і прачытаць іх:
\par{}The Feynman Lectures on Physics.
\par{}D. Kahneman, P. Slovic, A. Tversky --- Judgment Under Uncertainty: Heuristics and Biases.
\par{}S. Hayakawa, A. Hayakawa  --- Language in Thought and Action.
\par{}R. Cialdini --- Influence: Science and Practice (яе Гары ўжо згадваў).
\par{}R. Hastie, R. Dawes --- Rational Choice in an Uncertain World: The Psychology of Judgment and Decision Making
\par{}D. Hofstadter --- Godel, Escher, Bach.
\par{}J. Pournelle --- A Step Farther Out.
}...

Паследавала чарга апытанняў і контр-апытанняў, якая цягнулася некалькі хвілін, пакуль
яе не перарваў чарговы нясмелы стук у дзверы.

--- Заходзьце! --- яна і Гары Потэр сказалі амаль адначасова, і дзверы слізганулі ў бок.
За імі стаяў Нэвіл Лонгботам.

Гэтым часам ён  \emph{сапраўды} плакаў. 

--- Я п-пайшоў у першы вагон, і знайшоў п-п-прэфекта, але ён с-сказаў мне 
што п-п-прэфектаў не цікавяць такія д-дробязі, як нейкія жабы...

Твар Хлопца-Які-Выжыў раптам змяніўся. Яго вусны сціснуліся ў тонкую палоску. 
Калі ён загаварыў, яго голас быў халодны і змрочны. 

--- Якія былі яго колеры? Срэбна-зялёныя?

--- Н-не, яго знак быў чырвоны і залаты.

--- \emph{Чырвоны і залаты!} --- выбухнула Герміёна. --- Але гэта колеры Грыфіндора!

--- Якая \emph{паскуда,} --- кшталту змяі \emph{прасіпеў} Гары Потэр, ажно Герміёна і 
Нэвіл уздрыгануілся, --- вырашыла, што 
знайсці жабу першакурсніка не дастакова \emph{гераічная} справа, каб быць вартай 
прэфекта Грыфіндора? Хадзем, Нэвіл, гэтым разам я пайду з табой, і мы паглядзім,
ці варты Хлопец-Які-Выжыў іх увагі. Мы знойдзем прэфекта, які ведае заклён,
а калі не атрымаецца, знойдзем такога, які не баіцца запэцкаць рукі, а калі гэта 
не атрымаецца, я пачну набіраць армію з сваіх фанатаў, і мы рязбярэм гэты
цягнік на вінцікі.

Ён ускочыў, схапіў Нэвіла за руку, і Герміёна недзе на фоне свядомасці адзначыла, 
што яны амаль аднаго росту, хаця іншая частка свядомасці настойвала, што Гары Потэр
павінен быў быць прынамсі на фут вышэй.

--- \emph{Застанься!} --- кінуў Гары ў яе... не! у бок ягонага куфара, --- і ён 
зачыніў за сабою дзверы і яны сышлі.

Магчыма, ёй было трэба пайсці з імі, але ў тый момант, калі Гары Потэр стаў 
такім страшным, яна кшталту была радая, што ён яе не паклікаў.

Розум Герміёны быў у такім разгубленым стане, што наўрад ці ў яе атрымаецца 
ўважліва дачытаць "Гісторыю Хогвартс". Яна адчувала сябе, быццам па ёй праехаўся
каток і раскатаў яе ў тонкі блін. Яна не была ўпэўнена ў тым, што яна думае,
або што адчувае, і чаму. Яна проста сядзела і глядзела на рухаючыся далягляд за
акном.

Прынамсі, яна разумела, чаму яна адчувала сябе крыху сумна.

Магчыма, Грыфіндор не быў такім цудоўным факультэтам, як яна думала.
