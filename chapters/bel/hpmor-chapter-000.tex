\chapter*{Прадмова}
% from http://www.hpmor.com/chapter/1
% This is not a strict single-point-of-departure fic—there exists a primary point of departure, at some point in the past, but also other alterations. The best term I’ve heard for this fic is “parallel universe”.

У гэтай кнізе шмат падказак: простых, складаных, вельмі таемных падказак ---
якія, да майго вялікага здзіўлення, некаторыя чытачы такі здолелі знайсці, --- і проста
горы доказаў, раскіданых навідавоку. Гэта гісторыя аб рацыянальнасці, і ўсе яе 
таямніцы можна вырашыць, бо яны так і задуманы.

Стыль расказу падобны да сэрыяла, у тым сэнсе, што ў ім ёсць пэўная колькасць
сезонаў, і кожны эпізод задуман, каб мець уласную гісторыю, але і дадаваць 
да развіцця агульнага апавядання, накіроўваючы яго да фінальнай развязкі.

Усе  навуковыя рэчы ў кнізе ўзяты з рэальнага жыцця. Але ў тым, што не тычыцца навукі,
погляды персанажай могуць адрознівацца ад поглядаў аўтара. Не ўсе, што робяць 
пратаганісты --- урок мудрасці, а парада больш цёмных герояў можа быць 
нядобранадзейнай або наогул пагрозлівай.

\chapter*{Ад \enspace аўтара}
% from http://www.hpmor.com/chapter/22

\section*{Нештаа дзесьці калісьці магло пайсці інакш...}

\begin{itemize}
\item \textsc{Петунія Эванс} 
выйшла замуж за Майкла Верэса, прафесара біяхіміі ў Оксфардзе.
\item \textsc{Гары Джэймс Потэр-Эванс-Верэс}
вырас ў доме, па завязку набітым кнігамі. У патачковай школе ён укусіў настаўніцу матэматыкі,
якая не ведала, што такое лагарыфм. Ён прачытаў кнігу \emph{Гёдэль, Эшер, Бах},
\emph{Прыняцце рашэнніяў ва ўмовах неакрэсленнасці} і першы том 
\emph{Фейнманаўскіх лекцый па фізіцы}. І нягледзячы на агульны страх тых, хто 
яго сустракаў, ён не мае намер стаць наступным Цёмным Лордам. 
Ягонае гадаванне не дазваляе такіх дробязей. Ён хоча зразумець законы магіі і стаць
богам. 
\item \textsc{Герміёна Грэнджэр} атрымлівае лепшыя за Гары адзнакі па ўсіх дысцыплінах,
акрамя палётаў на мятле.
\item \textsc{Драко Малфой} абсалютна такі, якім вы чакаеце бачыць адзінаццацілетняга
хлопца, калі б Дарт Вейдэр быў яго клапатлівым бацькам. 
\item \textsc{Прафесар Квірэл} ажыццявіў сваю даўнюю мару --- навучаць 
Абароне ад Цёмных Майстэрстваў, або, як ён называе гэта --- Баявой Магіі. 
Усе яго студэнты гадаюць, што пойдзе не так з гэтай пасадай гэтым разам.
\item \textsc{Дамблдор} або звар'яцелы, або вядзе складанейшую гульню, дзе
яму прыходзіцца паліць курэй. 
\item \textsc{Мінерве МакГонагал} час ад часу трэба пабыць адной і паараць у падушку.
\end{itemize}

% \begin{center}
% Presenting:
%
% \textsc{Harry Potter and the Methods of Rationality}
%
% You ain't guessin' where this one's going.
% \end{center}

\section*{І яшчэ...}
 
Калі вы яшчэ не ведаеце пра сайт \url{https://hpmor.com}, не забудзьцеся 
калісьці туды зазірнуць. Вы знойдзеце там фан-арт, парады, як навучыцца таму, што
ведае Гары, і шмат іншага.


% NNnno
%If you don’t just enjoy this fic, but learn something from it, then please consider blogging it or tweeting it. A work %like this only does as much good as there are people who read it.

%  LocalWords:
