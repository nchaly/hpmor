\chapter{Омакэ I, II \& III}

\begin{chapterOpeningAuthorNote}
Слаўся, Роўлінг, Цёмны Лорд.

Дадзеныя "омакэ" з'яюцца не-кананічнымі дадаткамі.
\end{chapterOpeningAuthorNote}

\section{Омакэ I: 72 гадзіны да перамогі}

Дамблдор паглядзеў праз свой пісьмовы стол на Гары, яго вочы дабразычліва 
паблісквалі. На дзіцячым твары хлопца быў жахліва напружаны выраз --- Дамблдор 
спадзяваўся, што ў чым бы ні была справа, яна не была \emph{занадта} сур'ёзнай.
Гары быў яшчэ малы, каб пачаць праходзіць праз свае жыццёвыя праблемы.

--- Аб чым ты хацеў пагаварыць, Гары?

Гары Джэймс Потэр-Эванс-Верэс падаўся наперад у крэсле, змрочна ўсміхнуўшыся.

--- Дырэктар, падчас Размеркавальнай Вячэры я адчуў рэзкі боль у сваім шнары. 
Улічваючы, як я атрымаў гэты шнар, я падумаў, што гэта не тое, што варта 
ігнараваць. Спачатку я падумаў, што гэта з-за прафесара Снэйпа, але, прымяніўшы 
Бэканаўскі эксперыментальны метад --- параўнаць умовы ў прысутнасці і адсутнасці
феномена, --- я зразумеў, што шнар баліць толькі, калі я бачу патыліцу прафесара 
Квірэла, або, прынамсі, што ў яго там пад турбаном. Гэта, канешне, можа быць 
нечым бяскрыўдным, але я думаю, у дадзенай сітуацыі, мы можам дапусціць найгоршы
варыянт, што там хаваецца Самі-Ведаеце-Хто, не глядзіце так усхвалявана, бо 
гэта неацэнная магчымасць...




\section{Омакэ II: Не баюсь Цёмных Лордаў}

\subsection{Ад аўтара}


Гэта арыгінальная версія главы №9. Яна была заменена, бо --- нягледзячы на тое,
што яна шмат каму  падабалася, --- высветлілася, што шмат у каго яшчэ развілася
моцная алергія на песні ў фанфіках, па прычынах, не вартых асаблівага разжоўвання.
Я не хацеў спугнуць чытачоў прынамсі да дзесятай главы.

Лі Джордан, згодна з канонам, спадарожнік па пранкам Фрэда і Джорджа.
Ягонае імя для мяне гучыць вельмі як маглаўскае, і гэта тлумачыць, што ён
мог навучыць блізнят вядомай Гары песне. Некаторым чытачам гэта было не так
відавочна, як аўтару.


\later


Драко быў размеркаваны на Слізэрын, і Гары з палёгкай выдахнуў. 
Гэта падавалася, што яно само сабой разумеецца, але ніколі не ведаеш, якое
малейшае нечаканае здарэнне можа разбурыць твой стратэгічны план.

І яны пакрысе падбірался да літары П...

За сталом Грыфіндора веўся напруджаным шэптам дыэялог. 


--- \emph{А што, калі ён не спадабае?}

--- \emph{У яго няма права не спадабаць...}

--- \emph{...пасля таго, як ён пранкануў...}

--- \emph{...Нэвіла Лонгботама, калі я не памыляюся...}

--- \emph{...ён настолькі справядлівая справядлівая мішэнь, наколькі толькі можа быць справядліва...}

--- \emph{Добра. Усе ўсё памятаюць?}

--- \emph{Мы добра адрэпетыравалі...}

--- \emph{...у нас была цэлая гадзіна.}


У гэты момант Мінерва МакГонагал на подыуме ў галоўнага стала паглядзела на 
наступнае імя ў сваім спісе. \emph{Калі ласка, толькі не Грыфіндор,
калі ласка, толькі не Грыфіндор, \shout{калі ласка,} толькі не Грыфіндор...}
Яна глыбока ўдыхнула і сказала:

--- Потэр, Гары!

Раптам ва ўсёй зале натупіла цішыня.

Роўна праз тры секунды гэтую цішыню разазваў жудасны бразгатаючы гук,
які, падавалася, здзекваўся над папулярнай сярод маглаў мелодыяй.

Галава Мінерва шакавана тузанулася, вызначыла, што крыніцай шума быў 
Грыфіндорскі стол, дзе Гэтыя \emph{стаялі на стале} і дудзелі ў нейкія 
дзіўныя дудкі. Яе рука пацягнулася да палачкі, каб зрабіць на Гэтых 
\emph{Silencio}, але ішны гук яе спыніў.

Дамблдор хіхікаў.

Вочы Мінервы вярнуліся да Гары Потэра, які толькі пачаў пакідаць строй, але 
нібыта спатыкнуўся.

Пачуўшы "музыку" Гэтых, хлопец пачаў рухацца, вельмі дзіўна слізгаючы нагамі,
махаючы рукамі наперад і назад, цокаючы пальцамі ў такт.
\begin{center}

\emph{Галоўная тэма "Паляўнічых за прывідамі"}

\emph{Выканана на казу Фрэдам і Джорджам Уізлі, вакал: Лі Джордан}

\medskip

\emph{Нейкі Цёмны Лорд\\
Напужаў народ?\\
Дапаможа нам...}
\end{center}

% 1 та та ці та цім
% 2 та та ці та цім
% 3 ці та та та цім
% 4 ЦІ ЦІ ТАМ

--- \shout{Гары Потэр!} --- крыкнуў Лі Джордан, і блізняты Уізлі выканалі трыумфальны прыпеў.

\begin{center}
    \emph{Адаб'е заклён\\
    Адной левай ён.\\
    Дапаможа нам...}
\end{center}

--- \shout{Гары Потэр!}  --- гэтым разам крыкнулі шмат больш галасоў.

Кашмарныя Уізлі перайшлі ў сола, а да іх далучыліся некалькі старэйшых магланароджаных
са сваімі дудкамі, не інакш, як трансфігураваных з відэльцаў. Калі мелодыя дасягнула
піка сваёй немузыкальнасьці, Гары Потэр крыкнуў:
    
\begin{center}
    \emph{Не баюсь Цёмных Лордаў!}
\end{center}


У адказ зала прыхільна загарлапаніла, асабліва стол Грыфіндора, і яшчэ болей 
вучняў далучылася да выканаўцаў са сваімі імправізаванымі антымузыкальнымі інструментамі. 
Жахлівае бразгатанне падвоіла сваю гучнасць і ў сваім аглушальным крэшчэнда
дайшла да чарговага:

\begin{center}
    \emph{Не баюсь Цёмных Лордаў!}
\end{center}

Мінерва са страхам паглядзела на Галоўны Стол.

Трэлоні ліхаманкава абмахвалася веерам, Флітвік глядзеў на выканаўцаў зацікаўлена, 
Хагрыд пляскаў у далоні ў такт, Спраут глядзела сурова, Квірэл глядзеў на Гары Потэра 
з саркасцічным задавальненнем. Дамблдор падпяваў сабе пад нос, а Снэйп сціскаў у 
палябеўшымі пальцамі пусты кубак для віна так моцна, што тоўстае срэбра павчалі паволі згібацца.


\begin{center}
    \emph{Пажыральнік тут\\
    Знойдзе свой капут!\\
    Бо да нас прыйшоў\\
    \shout{Гары Потэр}!}
    
    \emph{Стрыгі і кляшчы?\\
    Кажаны ў плашчы?\\
    Пераможа ўсіх\\
    \shout{Гары Потэр}!}
\end{center}

Вусны Мінервы сціснулся ў тонкую паласу. Ёй трэба абмяркамаць з Кашмарнымі Уізлі
гэтую апошнюю частку. Калі яны думалі, што ў першы школьны дзень яшчэ не было 
факультэтных балаў, каб іх сняць, і калі іх не хвалявалі звычайшыя школьныя пакаранні,
яны прыдумае нешта яшчэ.

З лёгкім прыступам панікі яна глянула на Северуса, які дакладна павінен зразумець, аб кім гэта Уізлі...


Твар Северуса перайшоў ад стану раз'юшанасці да спакойнай абыякавасці. Лёгкая 
усмешка хадзіла па яго вуснах. Ён глядзеў на Гары Потэра, не ў бок стала Грыфіндора,
у яго руцэ былі скамечаныя рэшкі кубка...

І Гары Потэр прасоўваўся наперад, рухачы рукамі і нагамі ў танцы Паляўнічых за 
Прывідамі, трымаючы ўсмешку на твары. Задумка была крутая і вельмі неспадзяваная.
Ён толькі і мог зрабіць, што падыграць ім, каб не сапсаваць усё.

Усе яго віталі, і ад гэтага ён адчуваў ўнутры адначасова 
цеплыню і жах.

Агульныя апладысменты за тое, што ён зрабіў ва ўзросце аднаго году. І зрабіў не да канца.
Цёмны Лорд усё яшчэ быў жывы. Ці віталі б яго, калі б гэта было вядома?

Але аднойчы сіла Цёмны Лорд пала.

І Гары Потэр абароніць іх зноў. Прынамсі, калі пра гэта ёсць прароцтва. А, чорт з 
ім, з прароцтвам. Усё роўна абароніць.

Усе гэтыя людзі, якія віталі ў яго і верылі --- Гары не мог гэтаму здрадзіць.
Успыхнуць і згаснуць, як тысячы вундэркіндаў. Стаць расчараваннем. Не дайсці 
ў жыцці да высокага стандарту, які ён сам жа і ўсталяваў, як сымбал Святла ---
нягледзячы, як ён гэта зрабіў. Ён абавяскова, няважна, колькі часу гэта зойме, 
і нават калі гэта яго заб'е, апраўдае іх чаканні. А потым \emph{перавысіць}
іх чаканні, каб яны потым здзіўляліся, як мала ад яго калісьці чакалі...

І ён выкрыкнуў прыдуманую ім ману, таму што яна падыходзіла ў рытм, і 
песня проста яе патрабавала:


\begin{center}
    \emph{Не баюсь Цёмных Лордаў!\\
Не баюсь Цёмных Лордаў!}
\end{center}

Гары зрабіў апошнія крокі да Размеркавальнага Капелюша, калі музыка скончылася.
Абярнуўшыся, ён пакланіўся ў бок
Ордэна Хаоса за сталом Грыфіндора, потым --- у другі бок залы, і 
пачакаў, пакуль апладысменты і гоман сціхнуць...


\section{Омакэ III: Альтэрнатыўныя канцоўкі "Самаўсведамлення"}

\subsection{Ад аўтара}

Прапанова раскрыць увесь сюжэт тым, хто здагадаецца, чаго "ніколі такога не было" 
спарадзіла шмат цікавых ідэй\footnote{{} Першапачаткова
аўтар публікаваў па некалькі глаў на тыдзень, і ён сапраўды абяцаў такое
ў эпіграфы да дзявятай главы.}
(Далей ідуць імёны кантрыбутараў гэтых ідэй, і сведкі, якія не маюць 
аніякага сэнсу, калі толькі вы не былі актыўным чытачом fanfiction.net у
2010-м годзе. Калі вам вельмі цікава, шукайце ў ангельскай версіі кнігі.)

\later
У глыбіні сваёй свядомасці ён думаў, ці сапраыўды Капялюш \emph{свядомы},
у сэнсе ўсведамлення сваёй свядомасці, і калі так, ці задавальняла яго,
што яму дазволена размаўляць толькі з адзіннаццацілеткамі, і толькі раз на год. 
Згодна з песняй, якую прапеў Капялюш, падавалася, што так:

\begin{verse}%[\versewidth]
    \itshape
    Я --- капялюш размеркавальны,\\
    Я пачуваюся нармальна,\\
    Увесь год ляжу, як той камень,\\
    Працую ў годзе толькі дзень...
\end{verse}

Калі настала цішыня, Гары сеў на стул, і \emph{акуратна} надзеў на галаву 
800-летні артэфакт забытай старажытнай магіі, думаючы з усёй моцы: 
\emph{Пачакай, пакуль не размяркоўвай мяне! У мяне ёсць да цябе шмат  
пытанняў! На мне калісьці рабілі забывальны заклён?
Калі ты размяркоўваў Цёмнага Лорда, можаш расказаць мне пра яго слабасці? 
Ты ведаеш, чаму я атрымаў чароўную палачку, якая сястра палачкі Сам-Ведаеш-Каго?
Ці ўтрымлівае мой шнар дух Цёмнага Лорда, і таму я часам моцна злуюся?
Гэта самыя важныя пытанні, але, калі ў цябе ёсць час,
можаш расказаць мне нешта пра тое, як пераадкрыць забытую магію, якая цябе стварыла?}

І Размеркавальны Капялюш адказаў \emph{Не. Так. Не. Не. Так і не, наступным разам не задавай падвоеных 
пытанняў. Не.}, і крыкнуў: 

--- РЭЙВЕНКЛО!

\later

\emph{Ох, божачкі вы мае. Ніколі такога не было, і вось...}

--- Што?

\emph{У мяне алергія на твой шампунь...}

І Размеркавальны Капялюш чыхнуў з громкім "А-ЦЬХУ", якое эхам адбілася ад сцен Вялікай
Залы.

--- Ну што, --- сказаў лагодна Дамблдор, --- падобна на тое, што Гары Потэра 
размяркавалі на новы факультэт Ацьху! МакГонагал, прызначаю вас дэканам новага 
факультэта. Лепей вам паспяшыць, каб падрыхтаваць вучэбны план і расклад Ацьху, бо 
заняткі пачынаюцца ўжо заўтра!

--- Але... але... --- пачала заікацца МакГонагал, яе розум быў у стане абсалюстнай
разгубленасці, --- хто будзе дэканам Грыфіндора? --- гэта было ўсё, што яна змагла 
запярэчыць.

Дамблдор пацёр падбародак, выглядаючы задуменным.

--- Снэйп.

Пратэстуючы віск Снэйпа заглушыў нават абурэнне МакГонагал:

--- Тады хто будзе дэканам Слізэрына?

--- Хагрыд.

\later

--- \emph{Пачакай, пакуль не размяркоўвай мяне! У мяне ёсць да цябе шмат  
пытанняў! На мне калісьці рабілі забывальны заклён?
Калі ты размяркоўваў Цёмнага Лорда, можаш расказаць мне пра яго слабасці? 
Ты ведаеш, чаму я атрымаў чароўную палачку, якая сястра палачкі Сам-Ведаеш-Каго?
Ці ўтрымлівае мой шнар дух Цёмнага Лорда, і таму я часам моцна злуюся?
Гэта самыя важныя пытанні, але, калі ў цябе ёсць час,
можаш расказаць мне нешта пра тое, як пераадкрыць забытую магію, якая цябе стварыла?}

Кароткая пауза.

--- Хэй? Мне паўтарыць пытанні?

Размеркавальны Капялюш завішчэў, гэты жудасны выскі гук адбіўся ад сцен Вялікай Залы,
студэнты закрылі вушы рукамі. З воем Капялюш спрыгнуў з галавы Гары Потэра,
і папоўз па падлозе, штурхаючыся сваімі палямі, і амаль дасягнуў галоўнага стала, 
перад тым як выбухнуць.

\later

--- СЛІЗЭРЫН!

Калі Фрэд Уізлі убачыў жахлівы позірк Гары Потэра, ягоныя думкі пабяжалі з хуткасцю, як
ніколі ў жыцці. Адным рухам ён схапіў сваю палачку, прашаптаў  "\emph{Silencio!}",
потым "\emph{Changemyvoiceio!}", і нарэшце "\emph{Ventriliquo!}".

--- Проста пажартаваў, --- сказаў Фрэд замест Капелюша. --- ГРЫФІНДОР!

\later


\emph{Ох, божачкі вы мае. Ніколі такога не было, і вось...}

--- Што?

\emph{Звычайна такія пытанні задаюць дырэктару, які можа потым звярнуцца да мяне.
Але некаторыя з іх не толькі за гранню тваіх магчымасцей, але і за гранню 
магчымасцей Дамблдора.}

--- Як я магу павысіць узровень сваіх магчымасцей?

\emph{Баюся, я не магу адказаць на гэтае пытанне, пакуль ты на сваім бягучым 
узроўні.}


-- Якія опыці мне \emph{даступныя} на маім узроўні?

Пасля гэтага шмат часу не патрабавалася...

--- ROOT!\footnote{{} Гэта камп'ютэрны жарт, не абурайцеся, калі не зразумела.}

\later

\emph{Ох, божачкі вы мае. Ніколі такога не было, і вось...}

--- Што?

\emph{Ведаеш, мне прыходзілася казаць людзям, што яны хутка стануць маці ---
гэта разаб'е твае сэрца, калі ты бы ведаў, што я бачыў у іх розумах, ---
але гэта першы выпадак у гісторыі, калі я павінны сказаць камусьці, што 
ён стане бацькам.}

--- \emph{\shout{Што?}}

\emph{У Драко Малфоя будзе ад цябе дзіця.}

--- \emph{\shout{ШТООООООО?}}

\emph{Паўтараю: У Драко Малфоя будзе ад цябе дзіця.}

--- Але нам толькі адзінаццаць...

\emph{Насамрэч, Драко трынаццаць.}

--- М-м-мужчыны не могуць мець дзяцей!

\emph{Насамрэч ён дзяўчына.}

--- \shout{У нас не было сэкса, ты, ідыёт!}

\emph{\shout{Яна цябе згвалтавала і потым зрабіла на цябе забывальны заклён, прыдурак!}}

Гары Потэр згубіў прытомнасць. Яго цела ўпала на падлогу з глухім бухам.

--- РЭЙВЕНКЛО! --- крыкнуў Капялюш, ляжачы на ягонай галаве. Жарт атрымаўся 
значна трапней, чым ён чакаў.

\later

--- ЭЛЬФ!

Што? Гары ўспомніў, што Драко казаў "а можа адразу ў эльфы?", але што гэта 
значыла дакладна?

Судзячы па сумным позіркам прысутнічаючых, нічога добрага...


% Following two are intranslatable:
% they joke in form "House of XXX", but original 
% "house" translated as "faculty" meaning university division,
% which fails to transmit the humor here.

%\later

%“PANCAKES!”

%\later

%“REPRESENTATIVES!”

\later

\emph{Ох, божачкі вы мае. Ніколі такога не было, і вось...}

--- Што?

\emph{Мне ніколі не даводзілася размяркоўваць чалавека, які адначасова рэінкарнацыя
Гордыка Грыфіндораа, Салазара Слізэрына і Наруто Удзумакі...}

%\later
%“ATREIDES!”

\later

--- А-а-а, зноў падмануў! ХАФЛПАФ! СЛІЗЭРЫН! ХАФЛПАФ!

%\later

%“PICKLED STEWBERRIES!”


% OMAGAD
% is this ref to star trek? 
% some people think it is the only scifi in the world :(
% (the rest think HP is a scifi, which is worse)

%\later

%“KHAAANNNN!”

\later

За галоўным сталом Дамблдор па-ранейшаму добра ўсміхаўся,
ціхія металічныя гукі даносіліся з боку Снэйпа, які задуменна працягваў спрэсоўваць
рэшкі кубка. Мінерва МакГонагал сцісквала кафердру пабялеўшымі пальцамі, ведаючы,
што зараза хаоса перакінулась з Гары Потэра на Размеркавальны Капялюш.

Сцэнарыі, адзін жахлівей аднаго, праігрываліся ў галаве Мінервы. Капялюш скажа,
што Гары надта добра збалансаваны каб быць размеркаваным, і вырашыць, што ён 
будзе належыць усім факультэтам. 
Капялюш скажа, што іншапланецян нельга размяркоўваць наогул.
Капялюш патрабуе, каб Гары Потэра выключылі з Хогвартс. 
Капялюш увойдзе ў кому. Капялюш патрабуе, каб размеркаваць Гары Потэра, яны павінны стварыць
новы факультэт Суднага Дня, і што \emph{Дамблдор прымусіць яе яго ўзначаліць}...

Мінерва ўспомніла, што ён казаў Гары падчас іх жудаснага пахода ў Дыягон-аллею...
пра заблуду планавання, падаецца так... як людзі былі занадта аптымістычныя, нават
калі яны думалі на найгоршыя сцэнары. Інфармацыя такога тыпу паразытавала на тваім
розуме, яна там разводзілася і выклікала начныя кашмары...

Але што было \emph{самым дрэнным}, што магло здарыцца?

Ну... у \emph{найгоршым сцэнары}, Капялюш прызначыць Гары ў цалкам новы факультэт.
Дамблдор будзе настойваць, каб Мінерва выканала гэты загад --- стварыць новы факультэт
дзеля яго аднаго, і ёй трэба будзе цалкам перарабіць расклад заняткаў у першы школьны 
дзень. Потым Дамблдор зробіць яе дэканам новага факультэту, а яе ўлюбённы Грыфіндор 
перадасць... прафесару Бінзу, прывіду-настаўніку гісторыі. І яна будзе марна 
намагацца даваць Потэру загады... здымаць баллы без аніякага рэзультату, пакуль 
яе будуць вінаваціць у адной катастрофе за другой.

Гэта быў найгоршы сцэнар?

Мінерва шчыра не ўяўляла сабе, што можа быць горш за гэта.

І нават, калі будзе найгоршы... няважна, яно ўсё скончыцца ўсяго праз сем гадоў.

Мінерва адчувала, як яе пальцы, схапіўшыя кафедру, разнявольваюцца. 
Гары быў правы, гэта дадавала ўпэўненасці: паглядзець напрамкі ў самую
глыбіню цемры, ведаць, што ты сустрэўся са сваімі страхамі, і гатовы да ўсяго.

Спужаную цішыню залы ўзарвала адно слова:

--- ДЫРЭКТАР! ---  крыкнуў Размеркавальны Капялюш.

За галоўным сталом Дамблдор падняўся. Яго твар быў збянтэжаны.

--- Так? --- сказаў ён капелюшу. --- У чым справа?

--- Ні ў чым, --- сказаў Капялюш. --- Я размеркаваў Гары Потэра на тое месца
ў Хогвартс, якое падыходзіць яму больш за ўсё. На месца Дырэктара.
