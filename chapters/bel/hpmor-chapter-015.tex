\chapter{Добрасумленнасць}

\begin{chapterOpeningQuote}
Думаю, я здолею дзесьці знайсці крыху часу.
\end{chapterOpeningQuote}

\lettrinepara{-F}{\emph{rigideiro!}}

\hplettrineextrapara
Гары ўсадзіў палец у шклянку с вадой, якая стаяла перад ім на стале. Вада павінна была
быць халоднай. Аднак, як яна была звычайнай тэмпературы, так і засталася. Зноў.

Гары пачуваў сябе моцна, моцна падманутым.

У даму Верэсаў было некалькі соцен фэнтэзі-кніг, і Гары прачытаў іх досыць. І 
падавалася, што ў яго ёсць загадкавы цёмны бок. Таму калі шклянка вады адмовілася
супрацоўнічаць некалькі разоў запар, Гары зірнуў па баках, праверыць, ці не назірае 
за ім хтосьці, зрабіў глыбокі ўдых, сканцэнтраваўся, і прымусіў сябе ўззлавацца.
Ён падумаў, як слізэрыны булілі Нэвіла, пра іх вясёлую гульню з выбіваннем толькі 
што падабраных кніг. Успомніў, што Драко Малфой сказаў пра гэтую дзеўчыну Лавгудаў, 
і пра тое, як сапраўды быў уладкаваны Уізенгамот.

І калі гнеў уварваўся ў ягоныя вены, ён дрыжачай ад нянавісці рукой падняў чароўную
палачку, сказаў халодна "\emph{Frigideiro!}", і адбылося абсалютна нічога.

Яго нагла \emph{абшукалі}. Ён жадаў напісать камусьці скаргу, і запатрабаваць
\emph{кампенсацыю} за свой цёмны бок, які відавочна быў \emph{павінен} мець 
неадольныя магічныя сілы, а апынуўся проста \emph{бракованым.}

--- \emph{Frigideiro!} --- пачуўся голас Герміёны з-за суседняй парты. Яе вада 
цалкам стала ільдом, і белыя крысталі пачалі вырастаць на краі шклянкі. 
Герміёна была так засяроджана на сваёй працы, што не заўважала, якімі нядобрымі 
позіркамі на яе глядзелі астатнія вучні, што было а) пагрозліва недарэчна з яе боку, або
б) цудоўна натрэніраванай пастаноўкай, якая дасягала ўзроўню вышэйшага майстэрства.

--- О, вельмі добра, міс Грэнджэр, --- ціўкнуў Філіус Флітвік, прафесар Чаравання і
дэкан факультэта Рэйвенкло, малюхны чалавечак без знешніх прыкмет былога дуэлянта-чэмпіёна.
---~Цудоўна! Надзвычайна!

Гары чакаў у найгоршым выпадку быць другім пасля Герміёны. Ён бы, канешне абраў 
бы варыянт, калі яна павінна была б даганяць яго, але адваротную сітуацыю ён таксама 
мог бы прыняць. Але яшчэ не скончыўся панядзелак, а Гары імкліва набліжаўся да 
канца спіса --- за гэтую пазіцыю 
па-сяброўску спаборнічалі ўсе астатнія магланароджаныя, якрамя Герміёны.
Якая была адна на сваёй вяршыні, без аніякіх супернікаў, небарака.

Прафесар Флітвік стаяў каля адной з дзеўчын, і папраўляў тое, як тая трымала сваю 
палачку.  

Гары з паглядзеў на Герміёну. Ён з цяжкасцю зглынуў. Лагічна, 
што да гэтай ролі яны была вельмі звыклая.

--- Герміёна? --- сказаў ён няўпэўнена. --- Можаш сказаць, што я раблю не так?

Вочы Герміёны ўспыхнулі жудасным святлом ахвотнасці дапамагчы, і недзе ў 
глыбіні Гарынай свядомасці нехта крычаў ад прыніжэння.

Праз пяць хвілін яго вада была ўжо заўважна халадней за паветра, і Герміёна
вербальна пагладзіла яго па галаве, сказала наступным разам вымаўляць больш
акуранта, і сышла, каб дапамагчы камусьці яшчэ.

Прафесар Флітвік прысудзіў ёй бал факультэта за дапамогу Гары.

Гары сціскваў зубы так моцна, што сківіца балела, і гэта таксама не дапамагала 
больш акуратнаму вымаўленню.

\emph{Мне ўсё роўна, што гэта несправядлівае спаборніцтва. Я дакладна ведаю, 
на што буду траціць свае дадатковыя дзве гадзіны. Я буду сядзець у сваім 
куфары, і вучыць, пакуль не даганю Герміёну Грэнджэр.}

\later

--- Трансфігурацыя --- адзін з найбольш складаных і небяспечных відаў магіі,
якія вы будзеце вывучаць у Хогвартс, --- сказала прафесар МакГонагал.
На твары старой строгай вядзьмаркі не было і цені легкадумнасці. --- Любы
парушальнік правілаў маіх заняткаў пакіне гэты кабінет назаўсёды. Я вас папярэдзіла.

Яна кранула палачкай свой пісьмовы стол, які плаўна перафармаваўся ў свінню. Некалькі 
магланароджаных ціха віскнулі. Свіння разявата павярнулася кругом, рохнула, і зноў стала сталом.

Настаўніца трансфігурацыі агледзела вучняў, яе позірк спыніўся на адным з іх.

--- Містэр Потэр, --- сказала прафесар МакГонагал, --- вы атрымалі свае кнігі 
ўжо некалькі дзён таму. Ці пачалі вы чытаць ваш падручнік па трансфігурацыі?

--- Прабачце, прафесар, не, --- адказаў Гары.

--- Вам няма за што пытаць прабачэння, містэр Потэр, калі б перад вамі было такое 
патрабаванне, я бы сказала, --- яна пагрукала касцяшкамі па сваім стале. --- Містэр
Потэр, ці можаце вы выказаць здагадку, гэта быў стол, з якога я трансфігуравала свінню,
або ён спачаку быў свіннёю, і толькі што я на некалькі секунд спыніла трансфігурацыю?
Калі вы прачыталі першую главу, то павінны ведаць адказ.

Бровы Гары крыху нахмурыліся.

--- Думаю, лягчэй было пачачь са свінні, бо калі вы пераўтварыце стол у свінню, 
ён можа не ведаць, як нават хадзіць.

Прафесар МакГонагал пакачала галавой.

--- Вашай віны ў гэтым няма, містэр Потэр, але правільны адказ у тым, што 
з трансфігурацыяй вы  павінны \emph{не выказыаць здагадкі}. У тэстах няправільныя адказы 
будуць атрымліваць значна горшыя адзнакі, чым пытанні, наогул пакінутыя без адказу.
Вы павінны вывучаць тое, чаго не ведаеце, а не гадаць. Калі я задаю пытанне, няважна наколькі 
відавочнае або элементарнае, і ваш адказ "я не ўпэўнены", гэта не будзе палічана за мінус,
а любы, хто засмяецца, атрымае штраф факультэтных балаў. Ці можаце вы сказаць, містэр 
Потэр, чаму гэта правіла існуе?

\emph{Таму, што малейшая памылка трансфігурацыі смяротна пагрозліва.}

--- Не.

--- Гэта правільны адказ. Трансфігурацыя небяспечней за апарыцыю, яку праходзяць 
толькі на шостым курсе. На жаль, каб дасягнуць свайго максімума, 
трансфігурацыю трэба пачынаць вучыць у малым узросце. Таму нашыя заняткі небяспечныя,
і вы павінны баяцца зрабіць памылку, бо ніколі маі студэнты не  
наносілі сабе цяжкія пашкоджаніі, і мой настрой будзе вельмі сапсаваны, 
калі вы будзеце першым класам, які перарве гэты ланцуг.

Некалькі студэнтаў зглынулі.

Прафесар МакГонагал падышла да дошкі. 

--- Ёсць шмат прычын лічыць трансфігурацыю небяспечнай, але адна выбіваецца
над усімі астатнімі, --- яна быццам з паветра дастала кароткае тоўстае пяро, якім 
напісала на дошцы чырвонымі літарамі і падкрэсліла сінім колерам, карыстраючы 
той жа "маркер":

\McGonagallWhiteBoard{ТРАНСФІГУРАЦЫЯ НЕ ПЕРМАНЕНТНА!}

--- Трансфігурацыя не перманентна! --- сказала прафесар МакГонагал. --- Трансфігурацыя не перманентна!
Трансфігурацыя не перманентна! Містэр Потэр, уявіце сабе, што студэнт трансфігураваў 
ваду з кавалка дрэва, і вы яе выпілі. Як думаеце, што здарыцца, калі трансфігурацыя 
расчыніцца? --- была невялікая пауза. --- Прабачце, містэр Потэр, не варта 
было пытацца ў вас. Я забыла, што вы адораны незвычайна песімістычным уяўленнем...

--- Не, усё добра, --- сказаў Гары хрыпла. --- Так, першы адказ, канешне, \emph{"я не ведаю"}, --- 
МакГонагал згодна кіўнула, --- але я магу ўявіць, што малюсенькія кавалачкі дрэва 
раптам з'явяцца ў маім страўніку... і крыві... і калі некая частка гэтай вады 
трапіла ў ткані... скажыце, а дрэва будзе мягкае або цвёрдае, або?.. --- 
магічнага досвіду Гары не хапала, каб зразумець, як, па-першае, малекулы вады будуць 
адпавядаць "малекулам" дрэва, і таму, другое, што адбудзецца з перамешанымі 
праз звычайную канвекцыю малекуламі вады/дрэва, калі магія скончыцца.

Твар МакГонагал быў жорсткім.

--- Як правільна размяркаваў містэр Потэр, вам будзе вельмі дрэнна, і вам будзе 
патрэбна неадкладная дапамога спецыялістаў шпіталя Сант-Манга, калі вы дажывяце 
да яго. Калі ласка, разгарніце падручнікі на старонцы нумар пяць.

Магчная фотаграфія перадавала толькі рух, без гуку, але адразу можна было сказаць,
што жудасна бледная жанчына бесперапынна крычала.

--- Злачынца, які трансфігураваў віно з золата, і даў выпіць гэтай 
жанчыне, "за ўплату доўга", як ён гэта потым назваў, атрымаў дзесяць гадоў Азкабана.
Перагарніце на наступную старонку. Гэта Дэментар. Дэментары --- наглядчыкі Азкабана.
Яны высмактваюць з чалавека магію, жыццё, і любыя шчаслівыя ўспаміны. Злачынца 
праз дзесяць год, у дзень выпуска з турмы, паказан на старонцы нумар сем.
Вы, канешне, зразумелі, што ён мёртвы... я слухаю, містэр Потэр?

--- Прафесар, --- сказаў Гары, --- калі гэта вядзе да такога, ці няма 
нейкага надзейнага спосаба падтрымліваць трансфігурацыю пэўны час?

--- Надзейнага спосаба няма, --- сказала МакГонагал без змены тону. --- Падтрыманне
трансфігурацыі траціць вашую магію, і хуткасць памнажаецца з ростам памераў прадмета.
І вы павінны ўваходзіць з ім у кантакт кожныя некалькі гадзін. Такія катастрофы 
\emph{непапраўныя!}

Прафесар МакГонагал падалася наперад, яе твар быў яшчэ жарстчэй.

--- Вы абсталютна ні пры якіх абставінах не будзеце трансфігураваць 
вадкасць або газ з чаго заўгодна. Ні ваду, ні паветра. Нішто падобнае на ваду або паветра.
Нават калі вадкасці не прызначаны для піцця, яны \emph{выпараюцца}, і іх 
маленькія нябачныя кавалачкі трапляюць у паветра. Вы не будзеце трансфігураваць
нешта, што можна спаліць. Яна дае дым, і нехта гэты дым можа ўдыхнуць.
Вы не будзеце трансфігураваць нешта, што можа любым чынам трапіць унутр іншага чалавека.
Ніякой ежы. Нічога, што нават выглядае, як ежа.  Ніякіх вясёлых пранкаў,
нават калі вы маеце намер папярэдзіць чалавека да таго, як той нешта праглыне.
Вы ніколі такога не зробіце. Кропка. У гэтым кабінеце, або ў Хогвартс, або дзесьці яшчэ.
Ці кожны з вас гэта разумее добра і чотка?

--- Так, --- сказаў Гары, Герміёна, і некалькі яшчэ вучняў. У астатніх, падавалася,
мову заняло.

--- \emph{Ці кожны з вас гэта разумее добра і чотка?}

--- Так, --- кожны сказаў, або прамармытаў, або прашаптаў.

--- Калі вы парушыце гэтыя правілы, вы больш не будзеце вывучаць 
трансфігурацыю ў Хогвартс. Паўтарайце за мной. Я ніколі не буду трансфігураваць
вадкасць або газ.

--- Я ніколі не буду трансфігураваць
вадкасць або газ, --- сказалі студэнты нястройным хорам.

--- Яшчэ раз! Грамчэй! Я ніколі не буду трансфігураваць
вадкасць або газ.

--- Я ніколі не буду трансфігураваць
вадкасць або газ.

--- Я ніколі не буду трансфігураваць
нешта, што выглядае, як ежа, або можа трапіць унутр чалавечага цела.

--- Я ніколі не буду трансфігураваць
нешта, што можа згарэць, бо яно робіць дым.

--- І вы ніколі не будзеце трансфігураваць грошы, уключаючы маглаўскія грошы, --- сказала
прафесар МакГонагал. --- У гоблінаў ёсць спосабы высветліць аўтара падробкі.
Згодна з ратыфікаванымі пагадненнямі, нацыя гоблінаў знаходзіцца 
ў бесперапынным стане вайны з любымі магамі-фальшываманетчыкамі. Яны не будуць 
дасылаць за вамі аўрораў. Яны дашлюць армію.

--- Я ніколі не буду трансфігураваць
нешта, што выглядае, як грошы, --- паўтарылі студэнты.

--- І \emph{самае галоўнае}, --- сказала МакГонагал, --- вы ніколі не будзеце
трансфігураваць жывых істот, \emph{асабіва з саміх сябе!} Гэта скончыцца цяжкай хваробай,
або смерцю, у залежнасці ад спосаба трансфігурацыі і даўжыні яе падтрымання...
Містэр Потэр падняў руку, бо калісьці ён назіраў трансфармацыю анімага --- а
дакладней, пераўтварэнне чалавека ў ката і назад. Адказ на вашае пытанне: 
трансфармацыя анімага --- не \emph{свабодная} трансфігурацыя.

Прафесар МакГонагал дастала невялікі кавалак дрэва з кішэні. Калі яна кранулася 
яго сваёй палачкай, ён стаў крыштальным шарам. Потым яна сказала "\emph{Crystferrium!}",
і шар стаў металічным. Яна кранулася яго яшчэ раз, і ён зноў стаў дрэвам.

--- Заклён "\emph{сrystferrium}" пераўтварае шкляны прадмет у новы прадмет такой жа 
формы, але жалезны. Гэта заклён не можа адрабіць пераўтварэнне. 
Самы агульны варыянт трансфігурацыі --- свабодная трансфігурацыя, якую вы будзеце
вывучаць, --- можа пераўтварыць любы прадмет у любы іншы прадмет, прынамсі 
падобнай формы або аб'ёму. Таму свабодную трансфігурацыю трэба рабіць моўчкі, у адрозненне
ад зачараванняў, дзе для кожнай пары прадметаў будзе патрэбна новая інкантацыя.

МакГонагал пранізліва агледела клас.

--- \emph{Некаторыя} настаўнікі пачынаюць з трансфігурацыённых зачараванняў, 
і толькі потым пераходзяць да свабоднай трансфігурацыі. Так, гэта крыху палягчае 
вывучэнне, спачатку. Але гэта дае пэўны патэрн, які будзе абмяжоўваць вашыя 
здольнасці пазней, і з якога складана выйсці. У мяне вы будзеце вывучаць 
свабодную трансфігурацыю \emph{з самога пачатку}, што патрабуе 
каставання заклёнаў без слоў, цалкам трымаючы зыходную форму, канчаткову форму,
і трансфармацыю ў вашым розуме.


--- І ў адказ на пытанне містэра Потэра, --- працягвала яна пасля невялікай паузы, --- 
як раз свабодную трансфігурацыю ніколі нельга прымяняць на жывыя істоты. 
Існуюць зачараванні і зёлкі, якія бяспечна і адваротна могуць пераўтвараць
жывыя істоты \emph{абмежаванымі} шляхамі. Напрыклад, анімаг без нагі таксама не 
будзе мець нагі і пасля трансфармацыі. Свабодная трансфігурацыя небяспечная.
Ваша цела заўсёды мяняецца, напрыклад удыхае і выдыхае паветра. 
Калі вы, напрыклад, дакранецеся да сябе палачкай, і ўявіце 
сябе з залатымі валасамі, ў выніку яны проста выпадуць.
Калі вы ўявіце сябе з больш чыстай скурай, вас чакае доўгі візіт у Сант-Манга. 
А калі вы паспрабуеце зрабіць з сябе дарослую форму, калі 
трансфігурацыя скончыца, вы памрэце.

Гэта тлумачыла, чаму Гары сустракаў такія выпадкі, як занадта тоўстыя хлопцы, або 
непрыгожыя дзеўчыны. Або старыя людзі, да кучы. Калі б ты мог трансфігураваць сябе кожную раніцу,
такога б не было... Гары падняў руку і паспрабаваў даслаць сігнал прафесару МакГонагал
вачыма.

--- \emph{Што}, містэр Потэр?

--- Ці магчыма трансфігураваць з жывой істоты нежывую, як манета... не, прабачце, прабачце,
--- жалезны шар, напрыклад?

Прафесар МакГонагал пакачала галавой.

--- Містэр Потэр, нават нежывыя прадметы за пэўны час атрымліваюць невялікія змены.
І нават калі пасля вяртання вы не заўважыце розніцы, праз некалькі часоў 
вам стане дрэнна, а праз дзень настане смерць.

--- Гхм, прабачце, прафесар, калі бы я прачытаў першую главу, я бы мог здагадацца,
што стол быў першапачаткова сталом, а не свіннёй, --- сказаў Гары, --- 
але толькі калі я буду мяркаваць, што вы не хаціце забіць свінню, 
што выглядае даволі імаверна, але...

--- Я ўжо прадбачу, што правяраць вашыя тэсты будзе для мяне невычарпальнай крыніцай
задавальнення, містэр Потэр, але калі ў вас ёсць яшчэ пытанні, ці магу я запытаць 
вас прытрымаць іх да канца ўрока?

--- Больш ніякіх пытанняў, прафесар.

--- Праўтарайце за мной, --- сказала прафесар МакГонагал. --- Я ніколі не буду 
прымяняць трансфігурацыю на жывых істотах, асабліва на сябе, калі толькі 
не атрымаю інструкцыі выкарыстаць адмысловае зачараванне або зёлкі.

--- Калі я не цалкам упэўнены, што трансфігурацыя бяспечна, я не буду 
рабіць яе, не парадзіўшыся з прафесарам МакГонагал, або прафесарам Флітвікам, або 
прафесарам Снэйпам, або дырэктарам, якія  адзіныя маюць паўнамоцтвы 
на кантроль трансфігурацыі ў Хогвартс. Запытаць іншага студэнта --  \emph{непрымальна},
нават калі яны вам скажуць, што ўжо задавалі такое ж пытанне настаўнікам.

--- Нават калі прафесар Абароны Хогвартс скажа мне, што трансфігурацыя бяспечная,
і нават калі я ўбачу, як сам прафесар Абароны зробіць яе, і нічога дрэннага не адбудзеца,
я не буду спрабаваць зрабіць гэта сам.

--- У мяне ёсць абсалютнае права адмовіцца рабіць любую трансфігурацыю, якая 
мяне не заўпэўнівае. Улічваючы, што сам дырэктар не можа прымусць мяне зрабіць такое,
я адназначна не буду прымаць падобныя загады ад прафесара Абароны, нават калі той
пагражае зняць сто балаў з факультэта, або адчысліць мяне.

--- Калі я парушу гэтыя правілы, я не больш ніколі не буду вывучаць трансфігурацыю 
ў Хогвартс. 

--- Мы будзем паўтараць гэтыя правілы перад кожным урокам прынамсі месяц, --- 
сказала прафесар МакГонагал. --- А зараз мы пачнем з трансфармацыі зубачыстак ў іголкі... 
убярыце вашыя палачкі, дзякуй, "пачнем" --- мелася на ўвазе пачнем рабіць канспект.

За паўгадзіны да канца ўрока МакГонагал раздала ім зубачысткі.

Праз паўгадзіны ў Герміёны была срэбная на выгляд зубачыстка, астатнія --- 
і магланароджаныя, і магі --- усе засталіся з тым, з чаго пачалі.

Прафесар МакГонагал узнагародзіла Герміёну яшчэ адным балам для Рэйвенкло.

\later

Пасля урока, пакуль Гары збіраў свае рэчы, да яго падышла Герміёна.

--- Ведаеш, --- сказала яна з бязвінным выразам, --- я заработала два бала 
для Рэйвенкло сёння.

--- Я заўважыў, --- сказаў Гары коратка.

--- Гэта, канешне, не так крута, як тваі  \emph{сем} балаў, --- сказала яна. ---
Напэўна, я не такая разумная, як ты.

Гары ўвапхнуў кнігі і світкі ў махляскін, і павярнуўся да Герміёны. Ён ужо 
і забыў пра свае балы.

Герміёна \emph{пахлопала яму вейкамі}.

--- Але ведаеш, урокі ў нас кожны дзень, а хафлпафы ў бядзе... цікава, колькі 
табе патрэбіцца часу, каб знайсці яшчэ? Сёння панядзелак. У цябе часу па чацьвер.

Яны глядзелі адно аднаму ў вочы, не міргаючы.

Гары першы перарваў цішыню.

--- Спадзяюся, ты разумееш, што гэта вайна.

--- Не думала, што ў нас быў мір.

Усі астатнія студэнты глядзелі на іх вельмі вялікімі вачыма. Усі студэнты, і, на жаль,
прафесар МакГонагал.

--- О, містэр Потэр, --- прапела прафесар МакГонагал з іншага канца кабінета, --- у 
мяне ёсць добрыя навіны для вас. Мадам Помфры зацведзіла вашую прапанову на конт
перадухілення праблем са Скрутнасцішнымі бардальёнамі, і мы скончым працу 
да канца наступнага тыдня. Я мяркую, гэта заслугоўвае... скажам, дзесяць балаў для 
Рэйвенкло.

Герміёна шакавана пабляднела ад такой здрады. Гары падумаў, што ягоны ўласны 
твар не надта адрозніваўся.

--- Прафе-есар! --- абурана пачаў Гары.

--- Няма сумневу, што вы заслужылі гэтыя дзесяць балаў, містэр Потэр.
Я не раздаю балы факультэтаў без прычыны. Для вас гэта можа падавацца 
простай справай: вы ўбачылі нешта крохкае, і прапанавалі спосаб абараніць гэта, 
але бардальёны --- рэчы каштоўныя, і дырэктару вельмі не спадабалася, калі 
такі разбіўся апошнім разам, --- прафесар МакГонагал задумалася. --- Ммм... цікава, 
ці быў у нас калісьці студэнт, які здолеў атрымаць сямнаццаць балаў за 
першы школьны дзень?.. Мне трэба будзе паверыць запісы, але падазрую, што 
гэта новы рэкорд. Магчыма, нам варта зрабіць абвестку пра гэта за вячэрай? 

--- \scream{Прафесар!} --- прахрыпеў Гары, --- гэта нашая вайна! Не перашкаджайце!

--- Затое ў вас ёсць часу па чацьвер наступнага тыдня, містэр Потэр. Канешне, калі 
вы не зробіце нейкую дурасць і не \emph{страціце} балы факультэта да таго часу. 
За непаважлівы зварот да прафесара, напрыклад, --- прафесар МакГонагал паклала палец
на шчаку і яшчэ раз задумалася. --- Думаю, вы дойдзеце да негатрыўнага 
ліку да канца пятніцы.

Гары зачыніў рот. Ён даслаў МакГонагал свой лепшы Позірк Смерці, але яе гэта толькі 
забаўляла.

--- Так, дакладна, абвестка за вячэрай, --- працягнула яна задуменна. --- Але не варта абураць 
Слізэрынаў звыш меры, таму прыйдзеца зрабіць яе кароткай. Проста лік балаў, рэкорд...
і калі нехта звернецца да вас па дапамогу з заняткамі і будзе расчараваны пачуць,
што вы не пачыналі чытаць падручнікі, вы заўсёды можаце іх накіраваць да міс Грэнджэр.

--- \emph{Прафесар!} --- сказала Герміёна на актаву вышэй, чым звычайна.

Прафесар МакГонагал праігнаравала яе.

--- Дарэчы, колькі зойме ў міс Грэнджэр здзейсніць нешта, вартае абвесткі за вячэрай?
Але буду чакаць, колькі б тое не заняло.

Гары і Герміёна па ўзаемнай невыказанай згодзе павярнуліся і абурана пакінулі пакой.
Астатнія рэйвенкло паследвалі за імі, як гіпнатызаваныя.

--- Гхм, --- сказаў Гары. --- Усё яшчэ ў сіле пасля вячэры?

--- Атож, --- сказала Герміёна. --- Я не хачу адставаць яд цябе ў вучобе.

--- Ну, дзякуй. І, прабач, але не магу не ўяўляць сабе, пры тваёй цяперашняй 
выбітнасці, на што ты будзеш здольная пасля пачатковай падрыхтоўцы ў рацыянальнасці.

--- Яна і праўда такае цудоўная? Не падобна, каб рацыянальнасць дапамагла табе на 
Чараванні або у Трансфігурацыі.

Была кароткая пауза.

--- Ну... я толькі атрымаў кнігі чатыры дні таму. Вось чаму мне прыйшлося забаратваць гэтыя сямнаццаць 
балаў без маёй палачкі.

--- Чатыры дні? Напэўна, складана прачытаць восем кніг за чатыры дні, але ты мог 
бы прынасмі справіцца з адной! Колькі табе трэба часу з тваёй хуткасцю?
Думаю, ты і сам валодаеш матэматыкай, таму скажы сам, колькі будзе
восем памножыць на чатыры і падзяліць на ноль?

--- Ты не ўлічваеш заняткі, і яшчэ выходныя, таму... бярэм ліміт ад 
восем памножыць на чатыры падзеленае на эпсілон (эспсілон імкнецца да нуля)...
гэта будзе... гэта будзе... 10:47 у нядзелю. 

--- Мне хапіла трох дней, дарэчы.

--- Добра, 14:47 у суботу. Думаю, я здолею дзесьці знайсці крыху часу.

I быў вечар, і была раніца: дзень першы.
