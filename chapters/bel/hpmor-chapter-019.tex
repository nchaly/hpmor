\chapter{Адкладзенае задавальненне}


\lettrine{В}{ыраз} Драко быў суровы, а яго аблямаваная зялёным мантыя выглядала 
прыгажейшай і даражэйшай за абсалютная такія ж самыя мантыі на двух 
хлопцах, якія стаялі па баках.

--- Расказвай, --- сказаў Драко.

--- А ну, кажы!

--- Чуў, шо казаў бос? Гавары!

--- Вы, абодва, наадварод, \emph{забіцеся!}

У вялікай аудыторыі, дзе ўсе чатыры факультэта вывучалі Абарону... э... Баявую 
Магію, хутка быў павінны пачацца іх апошні пятнічны ўрок. 

Апошні пятнічны ўрок. 

Гары спадзяваўся, што ён будзе спакойны. Прынамсі, што цудоўны прафесар Квірэл 
зразумее, што зараз не лепшы час, каб ставіць Гары ў цэнтр увагі. Гары крыху 
адышоў ад ранішняга, але...

...але на ўсялякі выпадак, можна зняць стрэс і зараз.

Гары адкінуўся на сваім стуле, прыняў вельмі ўрачысты выгляд, і сфакусаваў позірк
на Драко і ягоных міньёнах.

--- Хочаш ведаць, чаму? --- абвесціў Гары. --- Я магу адказаць адным словам.
Гэта --- перамога. Перамога за любы кошт. Перамога насуперак страхам. Перамога,
якой бы доўгай і цяжкай ні была б дарога, бо без перамогі няма...

--- \emph{Пра Снэйпа} расказвай! --- прасычэў Драко. --- \emph{Як ты гэта зрабіў?}

Гары скінуў выраз фальшывай урачыстасці, і глянуў проста сур'ёзна.

--- Ты сам бачыў, --- сказаў Гары. --- Усе бачылі. Я проста цокнуў.

--- \emph{Гары! Не дражні мяне!}

О як. Яго павысілі да \emph{Гары}. Цікава. І хаця насамрэч ён быў упэўнены, што 
гэта проста маніпулятыўны ход, Гары адчуваў, што трэба адказаць нейкай узаемнасцю...

Гары дакрануўся пальцам да вуха і выразна паглядеў на міньёнаў.

--- Яны не разбрахаюць, --- сказаў Драко.

--- Драко, ---- сказаў Гары, --- дазволь мне быць на сто адсоткаў шчырым, і 
сказаць, што ўчора я не тое, каб быў моцна ўражаны разумовымі здольнасцямі 
містэра Гойла.

Містэр Гойс скурчыўся.

--- Я таксама, --- сказаў Драко. --- Яму было растлумачана, што ў выніку я застаўся 
вінны табе паслугу. Але ёсць розніца паміж памылкай, і трыманнем сакрэтаў.
У гэтым іх трэніравалі з дзяцінства.

--- Добра, --- Гары сцішыў голас, хаця фонавы шум і так амаль знік, калі падышоў 
Драко. --- Я высветліў адзін з сакрэтаў Северуса, і прымяніў крыху шантажа.

Выраз Драко стаў ячшэ цвярдзей.

--- Добра. А зараз расказвай не тое, што ты пад вялікім сакрэтам ужо расказаў 
ідыётам з Грыфіндора --- што значыць, што гэтую гісторыю ты і 
\emph{планаваў} распаўсюдзіць.

Гары аутаматычна ўсміхнуўся. Драко глядзеў прама ў корань.

--- А што Северус кажа? --- спытаў Гары.

--- Што ён \emph{не ўяўляў}, накольмі чуллівымі могуць быць пачуцці дзяцей, ---
сказаў Гары. --- Нават слізэрынам! Нават \emph{мне!}

--- А ты ўпэўнены, --- сказаў Гары, --- што хочаш ведаць тое, што сам дэкан 
не хоча раскрываць?

--- Так! --- з гатоўнасцю адказаў Драко.

\emph{Цікава.} 

--- Тады ты павінен спачатку адаслаць сваіх міньёнаў, бо мне падаецца, я не магу
давяраць усяму, што ты пра іх думаеш.

Драко кіўнуў. 

--- Акей.

Містэр Крэб і містэр Гойл выглядалі вельмі няшчасна. 

--- Бос... --- сказаў містэр Крэб.

--- Вы не далі містэру Потэру ніякага рэзону вам давяраць, --- сказаў Драко. ---
Ідзіце!

Яны сышлі.

--- У асабістасці, --- сказаў Гары, паніжаючы свой голас яшчэ больш, --- я не
цалкам упэўнены, што яны проста не перададуць мае словы Люцыусу.

--- \emph{Такога} бацька ніколі не зробіць! --- сказаў Драко, сапраўды выглядаючы
абурана. --- Яны \emph{мае} міньёны!

--- Выбачай, Драко, але мне падаецца, што я не магу верыць усяму, што ты ведаеш 
аб сваім бацьке. Уяві, што ў цябе быў бы сакрэт, і калі я бы сказаў, што мой 
бацька так ніколі не зробіць...

--- Напэўна ты правы... --- павольна кіўнуў Драко. --- Гэта ты выбачай мяне, Гары.
Не варта было пытацца пры ніх.

\emph{І як гэта мяне насколькі павысілі? Ці не павінен ён абурацца на мяне зараз?}
Гары падавалася, што ён бачыў у Драко нешта, чым можна было маніпуляваць... калі 
б ягоны розум не быў бы такім стомленым. Інакш ён бы абавязкова 
паспрабаваў прыняць удзел у нейкай надзвычайнай інтрыге.

--- Ладна, --- сказаў Гары. --- Абмен. Я скажу табе тое, што ніхто не ведае, і не 
даведаецца, а ты мне скажаш, што ты і слізэрыны думаюць пра ўсю гэтую сітуацыю.

--- Дагаварыліся!

А цяпер трэба выдаць інфармацыю максімальна размыта і некантрэтна, каб яно не магло 
пашкодзіць, нават, калі і выйдзе вонкі...

--- Я сказаў праўду. Я высветліў адзін з сакрэтаў Северуса, і шантажаваў яго.
Але Северус --- не адзіная асоба, уцягнутая ў справу.

--- \emph{Я так і знаў!} --- сказаў Драко ўсхвалявана.

У Гары з'явілася пачуццё, як у хуткім ліфце. Відавочна, што ён толькі што сказаў 
нешта важнае, і не разумеў, чаму менавіта яно было важна. Гэта быў дрэнны знак.

--- Добра, --- сказаў Драко. Ён усміхаўся вельмі шырока. --- Значыць, на конт рэакцыі 
на факультэце. Спачатку гэтыя ідыёты такія: "Ненавідзім Гары Потэра! 
Давайце начысцім яму бубен!" 

Гары папярхнуўся.

--- Да што ж не так з Размеркавальным Капелюшом? Гэта не Слізэрын, гэта 
Грыфіндор!

--- І не кажы, дурняў і ў нас хапае, --- сказаў Драко, усміхаючыся па-змоўніцку.
--- Тады каму-некаму патрабавалася каля пятнаццаці секунд, каб растлумачыць ім, што 
гэта была б даволі дрэнная паслуга для Снэйпа, таму пакуль ты ў бяспецы. 
Але пасля адразу паднялася наступная хваля ідыятызму: "Ды ён проста выпендрываецца,
чортавы батанік!"

--- І тады?... --- спытаў Гары, усміхаючыся, хаця і не размеў, чаму гэта
было ідыятызмам па думкам Драко.

--- І тады нехта сапраўды разумны ўзяў слова. Відавочна, што Потэр знайшоў спосаб 
крута прыціснуць Снэйпа. І відавочная наступная думка --- што гэта мае 
дачыненне да таго, якім таямнічым чынам Снэйп цісне на Дамблдора. Ці не так?  

--- Без каментароў, --- сказаў Гары. Прынасмі гэтую частку ён зразумеў: факультэт
Слізэрын \emph{сапраўды} цікавіўся, чаму Снэйпа не звальняюць. І яны вывелі, што 
Северус сам павінны шантажаваць Дамблдора. Ці можа гэта быць праўдай? 
Па паводзінам Дамблдора --- не вельмі...

Драко працягваў.

--- І \emph{лагічная} наступная думка, на якую ўказалі разумныя нехта: калі ты 
здолеў націснуць на Спэйна настолькі, што ён пакінуў палову Хогвартс у спакоі,
магчыма, у цябе была і магчымасць пазбавіцца ад яго цалкам, калі б ты таго жадаў.
Ты проста адказаў абразай на абразу, але чамусьці ты вырашыў пакінуць 
нам нашага дэкана...

Усмешка Гары пашырылася.

--- А потым \emph{сапраўды} разумныя людзі, --- выраз Драко стаў вельмі 
сур'ёзным, --- адышлі і абмяркавалі сітуацыю вузкім 
колам, і нехта выказаў думку, што было б вельмі небяспечна пакідаць такога 
ворага недабітым. Калі ты і сапраўды мог зламаць яго схему шантажу Дамблдора,
відавочна, ты бы так і зрабіў. Дамблдор адразу бы выкінуў Снэйпа, або 
забіў бы. Ён быў бы \emph{вельмі} ўдзячны табе за дапамогу, і табе не прыйшлося
бы болей хвалявацца пра Снэйпа, які можа прабрацца ноччу ў тваю спальню
з нейкім цікавым узварам.

Гары трымаў нейтральны выраз. Пра такую магчымасць ён не падумаў, хаця наогул 
было б вельмі, вельмі варта.

--- І з гэтага ты зрабіў выснову?...

--- Снэйп ведае нейкі сакрэт Дамблдора, і \emph{цяпер ты таксама яго ведаеш!}
--- сказаў Драко пераможна. --- Сакрэт не насколькі важны, каб знішчыць 
Дамблдора цалкам, або Снэйп ужо гэта зрабіў бы. Відавочна, што ён карыстае сваю 
ўладу толькі на тое, каб заставацца дэканам Слізэрына, і часам ён не атрымлівае 
ўсяго, чаго жадае, значыць, ягоны ўплыў мае межы. Бацька спрабуе выцягнуць гэты 
сакрэт са Снэйпа \emph{гадамі!}

--- І цяпер, калі Люцыус думае, што я магу мець нейкую інфармацыю. Скажы,
ты ўжо атрымаў саву з дому?

--- Вечерам будзе, --- сказаў Драко і засмяяўся. --- Там будзе напісана, --- яго
голас стаў іншым, болей фармальным, --- \emph{Мой любы сын, я ўжо не раз казаў пра патэнцыял
Гары Потэра. Як ты ўжо, імаверна, зразумеў, яго ўплывовасць праяўляецца ўсё 
мацней. Пры любой магчымасці завесці з ім сяброўства, або прымяніць ціск на яго
ўразлівыя месцы, ты павінны дзейнічаць, і ўсе рэсурсы рода Малфоеў будуць у
тваім распрараджэнні, калі патрэбіцца.}

О, божа.

--- Ого, --- сказаў Гары. --- Адмоўлюся ад каментавання тваёй складанай тэорыі.
Дазволь толькі заўважыць, што мы з табой яшчэ не сябры.

--- Я ведаю, --- сказаў Драко. Яго выраз стаў \emph{вельмі} сур'ёзным, і голас 
стаў яшчэ цішэй, нягледзячы на сцішальны заклён. 

--- Гары, ты не думаў, што калі ты ведаеш нешта, што Дамблдор хацеў бы пакінуць 
у сакрэце, ён можа проста арганізаваць тваю смерць? Дарэчы, тут можна забіць 
двух зайцаў: і пазбавіцца патэнтыйнага суперніка ў барацьбе за лідэрства,
і зрабіць з Хлопца-які-выжыў кашоўнага публічнага пакутніка.

--- Без каментароў, --- сказаў Гары. Пра такое ён таксама не думаў. Гэта 
\emph{не падавалася} ў стылі Дамблдора, але ж...

--- Гары, --- сказаў Драко, --- відавочна, што ў цябе ёсць неверагодны талент, 
але заўважна, што ў цябе не было годнай адукацыі і настаўнікаў, і часам ты робіш 
сапраўды бязглуздыя рэчы, і \emph{табе праўда патрэбны дарадца, інакш гэта 
скончыцца для цябе вельмі дрэнна!}

--- А!.. --- сказаў Гары. --- Дарадца --- кшталту Люцыуса?

--- Кшталту \emph{мяне!} --- сказаў Драко. --- Абяцаю трымаць твае сакрэты ад 
бацькі, ад усіх, я проста буду дапамагаць табе ажыццяўляць твае ідэі!


Ну нічога сабе.

Гары ўбачыў, што зобмі-Квірэл кандыбае праз дзверы.

--- Урок зараз пачнецца, --- сказаў Гары. --- Я падумаю аб тваёй прапанове. 
Ты правы, часам мне вельмі хочацца, каб у мяне быў нейкі запас адукацыі.
Але я не ведаю, ці магу так хутка пачаць давяраць табе...

--- Не можаш, --- сказаў Драко, --- надта рана. Бачыш? Даю добрую параду,
нават калі гэта мне і перашкаджае. Але магчыма нам і сапраўды варта 
не маруздзіць, і хутчэй стаць сябрамі, што сказаш?

--- Я не супраць, --- сказаў Гары, бо гэта і так ужо было часткай яго плана.

--- І яшчэ, --- сказаў Драко паспешна, бо Квірэл амаль дасягнуў свайго стала. 
--- Зараз шмат хто на Слізэрыне думае, \emph{што} ўсё гэта вызначае. Калі 
хочаш Слізэрын у звязнікі, то раю табе падаць нейкі сігнал сяброўства ўсяму 
факультэту. \emph{Хутка}, напрыклад сёння, максімум заўтра.

--- А дазвол Северусу несправядліва даваць балы слізерынам --- недастаткова? --- 
не было ніякай прычыны не прылічыць гэтае дасягненне сабе.

У вычах Драко прабегла разуменне, потым ён сказаў хутка:

--- Гэта не падыходзіць, павер мне, гэта павінна быць нешта простае і зразумелае.
Проста пхні ў сцяну сваю бруднакроўку-суперніцу Грэджэр, або нешта такое ---
усе на Слізэрыне зразумеюць такі сігнал...

--- У Рэйвенкло так не працуе, Драко! Калі ты пхаеш кагосьці, гэта значыць, што 
ў цябе не хапае мазгаўні перамагчы іх звычайнымі спосабамі, і ўсе на Рэйвенкло
зразумеюць, што...

Экран на стале Гары заміргаў і ўключыўся, уздымаючы хвалю настальгіі аб тэлебачанні
і камп'ютэрах.

--- Гхмпф... --- сказаў голас прафесара Квірэла, накіраваны персанальна Гары з 
экрана. --- Калі ласка, займіце свае месцы.


\later

І ўсе вучні селі, і паглядзелі на экраны на сваі месцах, або ўніз на вялікую 
сцэну з белага мармуру, дзе стаяў прафесар Квірэл, нахіліўшыся над сталом, які 
стаяў на невялікай платформе цёмнага мармуру.

--- Сёння, --- сказаў прафесар Квірэл, --- я планаваў навучыць вас вашаму 
першаму абарончаму заклёну, невялікі шчыт, пра-пра-дзядуля сучаснага \emph{Protego.}
Але ў святле сённяшніх падзей у апошнюю хвіліну я вырашыў змяніць тэму сённяшняга
ўрока.

Шукаючы позірк прафесара Квірэла прайшоўся па радах. Гары скурчыўся на сваім месцы
ззаду аудыдыторыі, бо ў яго было прадчуванне, якія менавіта падзеі Квірэл меў на 
ўвазе.

--- Драко з Вялікага і Старажытнейшага Роду Малфоеў, --- сказаў прафесар Квірэл.

Фух...

--- Так, прафесар? --- сказаў Драко. Ягоны голас быў узмоцнены, і таксама ішоў 
з  экрана, адначасова паказваючы аблічча Драко. Потым выява зноў змянілася на прафесара 
Квірэла. 

--- Ці маеце вы жаданне стаць наступным Цёмным Лордам?

--- Гэта дзіўнае пытанне, прафесар, --- сказаў Драко. --- У сэнсе, наколькі 
дурным трэба быць, каб у гэтым прызнацца?

Некалькі студэнтаў засмяяліся, але не ўсе.

--- Сапраўды, --- сказаў прафесар Квірэл. --- І хаця няма сэнсу вас пытаць,
мяне нісколькі не здзівіць, калі ў мяне есць студэнт-два, якія маюць сур'ёзныя амбіцыі 
на гэты конт. Што і казаць, \emph{я сам} планаваў стаць наступным Цёмным 
Лордам, калі я быў яшчэ маладым слізерынам. 

Гэтым разам смех у зале быў мацней.

--- У рэшце рэшт, гэта факультэт амбіцыёзных асоб, --- сказаў прафесар Квірэл, 
усміхаючыся. --- Толькі праз шмат гадоў я зразумеў, што маёй асалодай была 
Баявая Магія, і маёй сапраўднай амбіцыяй было стаць вялікім магам-ваяром,
і аднойчы стаць настаўнікам у Хогвартс. У любым выпадку, калі мне было 
трынаццаць, я перарыў гістарычны аддзел біблятэкі Хогвартс, вывучаючы жыцця і 
лёсы прошлых Цёмных Лордаў, і зрабіў ліст усіх памылак, якія \emph{я} ніколі 
не зрабілю, будзь я Цёмным Лордам... 

Гары хіхікнуў, не стрымаўшыся.

--- Так, містэр Потэр, вельмі забаўляльна. Ці здолееце вы здагадацца, што было 
самым першым пунктам у маім спісе?

\emph{Выдатна...}

--- Эм... ніколі не абірай складаны спосаб перамагчы ворага, калі можаш 
проста скарыстаць Абракадабру?

--- Гэты праклён, містэр Потэр, гучыць як \emph{Avada Kedavra}, --- голас 
прафесара Квірэла стаў чамусьці жарстчэй. --- І не, гэтага не было ў спісе,
які я склаў у трынаццаць гадоў. Паспрабуеце яшчэ раз?

--- Эм... не баўбатаць нікому пра свой геніяльны злобны план?

Прафесар засмяяўся.

--- О, \emph{гэта} было нумарам два. Божа, містэр Потэр, мы з вамі што, 
чыталі адны кнігі?

Больш смеху ў зале, але з нярвовымі ноткамі. Гары моцна сціснуў сківіцы, 
і прамаўчаў. Адмаўляцца не было сэнсу. 

--- Маім \emph{першым} пунктам было "я не буду правакаваць магутных, лютых ворагаў".
Сусветная гісторыя пайшла бы па зусім іншаму шляху, калі б Марнеліт Фэлк'нзбэйн 
або Адольф Гітлер засвоілі бы некалькі простых правілаў. Так, \emph{калі б}, 
містэр Потэр --- проста \emph{ўявіце сабе}, калі нейкім дзіўным чынам у вас
з'явілася б амбіцыя, падобная на тую, што я меў у трынаццаць гадоў, --- нават, 
калі і так, я спадзяюся, вы не жадалі бы стаць \emph{бязглуздым} Цёмным Лордам.

--- Прафесар Квірэл, --- сказаў Гары, скрыгаючы зубамі, --- я \emph{рэйвенкло},
і ў майіх планах няма стаць нікем бязглуздым, кропка. Я ведаю, што тое, што я 
зрабіў сёння --- было тупа. Але яно не было цёмным! І я не быў тым, хто 
пачаў гэтую бойку!

--- Вы, містэр Потэр, ідыёт. Але я таксама быў ідыётам у вашым узросце. Я чакаў,
што ваш адказ будзе такім, і таму адпаведна змяніў план сённяшняга занятка.
Містэр Грэгары Гойл, ці не маглі бы вы выйсці да мяне, калі ласка?

У аудыдыторыі павісла здзіўленая пауза. Гары таксама такога не чакаў. 

Як і містэр Гойл, мяркуючы па яго даволі ўсхваляванаму і няўпэўненаму выгляду,
пакуль ён дабіраўся да сцэны.

Прафесар Квірэл адышоў ад стала. Ён раптам стаў на выгляд сільней, яго 
рукі сціснуліся ў кулакі, і ён плаўным рухам ён стаў у позу, вядомую Гары па 
фільмах пра баявыя майстэрствы. 

Вочы Гары пашырыліся. Ён адразу зразумеў, чаму быў выкліканы містэр Гойл.

--- Большасць магаў, --- сказаў прафесар Квірэл, --- не турбуюцца валоданнем 
таго, што маглы клічуць баявымі майстэрствамі. Ці не мацней палачка за кулак?
Такое стаўленне --- глупства. Кулак і трымае палачку. Калі хочаш быль вялікім
баявым чараўніком, ты \emph{павінны} засвоіць методыку кулачнаго бою, прынамсі
каб уразіць маглаў. Зараз я вам прадэманструю адну жыццёва важную тэхніку,
якую я вывучыў у \emph{додзё}, як маглы клічуць свае школы баявых майстэрстваў, 
і пра якую я раскажу вам пазней. А зараз... --- усё яшчэ ў стойцы, ён зрабіў 
некалькі крокаў наперад. --- Містэр Гойл, я запытаю вас атакаваць мяне.

--- Прафесар Квірэл, --- сказаў містэр Гойл, ягоны голас цяпер таксама ўзмоцнены,
--- магу я даведацца, які ўзровень...

--- Шосты дан. Ні вы, ні я не атрымаем шкоды. Пакажыце ўсё, на што вы здольныя.

Містэр Гойл з палёгкай кіўнуў.

--- Папрашу заўважыць, --- сказаў прафесар Квірэл астатнім, --- што містэр Гойл не хацеў 
нападаць на чалавека, не знаёмага з баявымі майстэрствамі на дастатковым 
узроўні, асцерагаючыся, што камусьці з нас будзе нанесена шкода. Падыход 
містэра Гойла цалкам правільны, і за гэта ён заслужыў тры квірэл-бала. 
А цяпер, нападайце!

Хлопец імгненна кінуўся наперад, махаючы кулакамі, і прафесар блакаваў 
кожны ўдар, адыходзячы назад, потым ён ударыў нагой, Гойл паставіў блок, 
пракруціўся, адначасова прысажываючыся, і зрабіў нагой падсечку, але Квірэл
перапрыгнуў яго нагу, і потым усё стала занадта хутка, каб можна было зразумець,
што адбываецца, і потым Гойл, лежачы на спіне, ударыў абедзвюма нагамі, і 
Квірэл \emph{праляцеў} у паветры, сустрэў падлогу плячом і перакуліўся праз галаву.

--- Стоп! --- пачуўся крык прафесара з падлогі. У ім была крыху чутна паніка.
--- Вы перамаглі.

Містэр Гойл ужо пачаў зноў рушыцца да настаўніка, і пачуўшы, спыніўся так 
хутка, што амаль не упаў ад інэрцыі. На яго твары быў выраз шоку.

Прафесар Квірэл падняў ногі дагары, зрабіў  рывок нагамі і тулавам,
які дазволіў яму падняцца на ногі без рук.

У аудыдыторыі была поўная збятэжанасці цішыня.

--- Містэр Гойл, --- сказаў прафесар Квірэл, --- якую жыццёва важную тэхніку 
я зараз прадэманстраваў?

--- Як правільна падаць, калі вас моцна штурхаюць, --- адказаў містэр Гойл. ---
Гэта адзін з першых...

--- І гэта таксама, --- сказаў прафесар Квірэл.

Павісла пауза.

--- Жыццёва важная тэхніка, якую я прадэмантстраваў, было \emph{паразай}. 
Можаце вярнуцца на месца, містэр Гойл, дзякуй.

Містэр Гойл пакінуў сцэну, выглядаючы даволі збянтэжанымі. Гары адчуваў сябе 
так сама.

Прафесар Квірэл падышоў да стала і зноў над ім нахіліўся.

--- Часам мы забываем базавыя рэчы, бо прайшло так шмат часу з тых пор, як 
мы іх вывучылі. Я зразумеў, што зрабіў тую ж памылку ў сваім плане вашага 
навучання. Нельга вучыцца прыгаць, пакуль не навучыўся падаць. Нельга 
вучыцца драцца, пакуль не зразумееш, як прайграць. 

Твар прафесар Квірэла стаў цвярдзей, і Гары падалося, што па яму прабяжаў выраз
болю і смутку. 

--- Я навучыўся прайграваць ў додзё ў Азіі, дзе, як вядома кожнаму маглу, жывуць
найлепшыя майстры баявых мастацтваў. У гэтым додзё практыкавалі стыль, які меў 
рэпутацыю сярод магаў, бо ён добра спалучаўся з нашымі метадамі магічных дуэляў.
Майстра гэтага додзё --- стары чалавек па маглаўскім меркам --- быў найлепшым 
з жывых настаўнікаў гэтага стылю. Канешне, ён і не падазраваў, што магія існуе.
Я прыйшоў да іх, і быў адным з нешматлікіх, каго яны штогод абіралі на навучанне з 
вялікай колькасці жадаючых. Магчыма, таму, што я крыху паўплываў на іх рашэнне.

Нехта ў зале выдаў знерваваны смешок, і Гары далучыўся. Яму зусім не падабалася,
куды гэта ўсё ідзе.

--- У любым выпадку, падчас адной з маіх першых боек, пасля таго, як мяне 
пабілі даволі зняважлівым чынам, я згубіў кантроль і напаў на майго спарынг-партнёра...


\emph{Ох.}

--- ...нашчасце, толькі з кулакамі, не магічна. Надзіва, Майстра не выгнаў 
мяне ў тую ж секунду. Ён сказаў, што ў маім характары ёсць загана. Ён патлумачыў
мне, і я ведаў, што ён быў правы. І потым ён сказаў, што мне трэба навучыцца 
прайграваць.

Аблічча прафесара было пазбаўлена выразу.

--- Па яго загаду ўсе студэнты ў додзё выстраіліся ў чаргу. Яны падыходзілі 
да мяне па аднаму. Я быў павінен \emph{не абараняцца}. Мне можна было толькі 
прасіць іх аб лістаці. Адзін за адным, яны білі мяне, у галаву, тулава, або 
пхалі мяне на зямлю. Хтосці пляваў на мяне. Яны лаяліся на мяне на сваёй мове.
І кожнаму я павінны быў сказаць: "Я здаюся!", або "Малю цябе спыніцца!", або
"Прызнаю, што ты лепей за мяне".

Гары спрабаваў уявіць сабе гэта, і проста не мог. Такое проста ніяк не магло 
адбыцца з цудоўным прафесарам Квірэлам.

--- Нават ужо тады я быў майстрам Баявой Магіі. Нават без палачкі я мог 
пазабіваць усіх у тым додзё. Але я так не зрабіў. Я навучыўся праігрываць. 
Да гэтых дней я ўспамінаю гэта, як найгоршыя часы ў сваім жыцці. І калі я 
пакінуў тое додзё праз восем месяцаў --- занадта рана, але гэта быў увесь час,
які я мог сабе тады дазволіць, --- Майстра сказаў мне, што спадзяецца, я зразумеў,
чаму быў патрэбны той урок. І я адказаў яму, што гэта быў адзін з  самых каштоўных
урокаў у маім жыцці. Што было --- і застаецца --- праўдай.  

Яго выраз стаў крыху сумны.

--- Вы зараз напэўна думаеце, дзе знаходзіцца гэтае цудоўна додзё, і ці можаце
вы туды паступіць. Дык вось, вы не можаце. Праз некаторы час у тое 
месца ў туманных гарах прыйшоў новы кандыдат. Той-каго-нельга-называць.

Пачуўся гук, як шмат людзей адначасова зрабілі рэзкі ўдых. У Гары скруціла
жывот. Ён уяўляў, што будзе далей.

--- Цёмны Лорд прыйшоў туды не хаваючыся, адкрыта, ззяючыя чырвоныя вочы, і ўсё 
такое. Студэнты спрабавалі забарыкадаваць вароты, але ён проста апарыраваў унутр.
Усім было жахліва, але яны трымалі дысцыпліну. Майстра выйшаў насустрач. І Цёмны Лорд 
запатрабаваў --- не запытаў, а менавіта запатрабаваў, --- каб яго прынялі ў 
навучанне. 

Выраз Квірэла стаў цвядзей. 

--- Магчыма, Майстра чытаў надта кніг, дзе казалася, што сапраўдны майстра 
баявых мастацтваў можа перамагчы нават дземанаў. Ён адмовіў. Цёмны Лорд 
спытаў, чаму ён не можа стаць тут студэнтам. Майстра адказаў, што ў яго  
не хапае цярпення, і тады Цёмны Лорд вырваў яму язык.

Зала калектыўна ахнула.

--- Можаце здагадацца, што адбылося далей. Студэнты спрабавалі драцца з Цёмным 
Лордам, але пападалі там дзе стаялі, здранцавелыя. А потым...

Голас прафесара на секунду запнуўся, і потым працягнуў:

--- Як вы ведаеце, адзін з Недаравальных Праклёнаў --- Круцыатус --- выклікае ў 
ахвяры неверагодны боль. Калі зацягнуць яго на больш чым некалькі хвілін, 
наступае перманентнае вар'яцтва. Аднаго за адным Цёмны Лорд закатаваў усіх 
студэнтаў да вар'яцтва, і потым прыкончыў іх забівальным заклёнам, і Майстра 
быў прымушаны за гэтым назіраць, і памёр ён апошнім, такім жа чынам.
Я дазнаўся пра гэта ад адзінага вучня, якому Цёмны Лорд пакінуў жыццё дзеля таго,
каб той расказаў гэтую гісторыю, ён быў калісьці маім сябрам...

Прафесар Квірэл на некалькі секунда адвярнуўся. Калі ён павярнуўся назад, ён 
зноў выглядаў спакойным і сабраным.

--- У цёмных чараўнікоў ёсць праблемы з самакантролем, --- сказаў ён ціха. --- Гэта 
амаль стоадсоткавая ўцятнасць катэгорыі людзей, маючых здольнасці да бойкі, і 
якія хутка пачынаюць цалкам давярацца гэтай здольнасці. Вы павінны зразумець важную рэч:
Цёмны Лорд \emph{не} перамог у той дзень. Яго мэтай было паступіць у навучанне
баявым майстэрствам, але ён пакінуў тое месца, не атрымаўшы ні воднага ўрока.
І яго жаданне, каб пра гэтую гісторыю даведаліся --- таксама глупства. Яна не 
паказала яго моц, а толькі ўразлівасць, якую можна лёгка скарыстаць. 

Позірк прафесара Квірэла сфакусаваўся на адзінам вучне.

--- Гары Потэр, --- сказаў прафесар Квірэл.

--- Тут, --- сказаў Гары хрыплым голасам.

--- Што \emph{менавіта} вы сёння зрабілі не так?

Гары адчуў млоснасць.

--- Я страціў самакантроль.

--- Гэта не дакладны адказ, --- сказаў прафесар Квірэл. --- Дазвольце мне 
апісаць гэта дасканала. У шмат якіх жывёл у прыродзе існуе рытуал 
сапборніцтва за дамінаванне. Напрыклад, яны разбягаюцца, і б'юцца ілбамі, замест 
таго, калі калоць адзін адно рагамі. Кошкі б'юцца лапамі, схаваўшы кіпцюры. 
Але чаму? Кіпцюры дадаюць табе шанцы на перамогу, ці не так? Бо іх вораг таксама
можа прымяніць кіпцюры, і замест таго, каб высветліць іерархію дамінавання, абодва 
могуць у выніку быць сур'ёзна параненымі. 

Позірк прафесара Квірэла з экрана свідраваў Гары наскрозь.

--- Вы прадэманстравалі сёння, містэр Потэр, што --- у адрозненне ад 
дзікіх жывёл --- вы не ведаеце, як прайграць спаборніцтва дамінавання. Калі 
\emph{прафесар Хогвартс} кінуў вам выклік, вы не адвярнуліся. Калі вы зразумелі,
што можаце прайграць, вы паказалі свае кіпцюры, негледзячы на ўласную небяспеку.
Вы \emph{эскаліравалі} сітуацыю, а потым вы эскаліравалі яе \emph{яшчэ раз.}
Усё пачалося з аплявухі прафесара Снэйпа, які, відавочна дэманстраваў вам 
сваю дамінантнасць. Замест таго, каб прайграць, вы пачалі адказваць, страцілі
дзесяць балаў Рэйвенкло, страцілі яшчэ, і ўжо праз хвіліну вы пачалі гаварыць 
аб тым, каб пакінуць Хогвартс. Той факт, што і пасля гэтага вы эскаліравалі 
праблему, толькі пацвярджае, што вы ідыёт, нягледзчы нават на тое, што ў выніку
вы нейкім чынам перамаглі.

--- Я разумею, --- сказаў Гары. У яго горле перасохла. \emph{Гэта} было дасканалае 
апісанне. \emph{Пужальна} дасканалае. Зараз, калі прафесар Квірэл сказаў гэта ў 
голас, Гары з жахам зразумеў, наколькі ён дасканала патлумачыў падзеі раніцы.
Калі нехта разумее цябе гэтак дакладна, ты павінен задацца пытаннем, а што калі 
і астатнія яго словы --- праўда, напрыклад пра твой смяротны намер.

--- \emph{Наступным разам,} містэр Потэр, калі вы абярэце шлях эскалацыі канфлікту
замест таго, каб прайграць, вы можаце згубіць усе свае стаўкі. Я не ведаю, якімі 
яны былі сёння. Але я магу здагадацца, што яны былі значна, значна вышэй, чым 
страта дзесяці балаў для Рэйвенкло.

Кшталту лёсу магічнай Брытаніі. Вось да чаго ён сёння дайшоў.

--- Вы можаце запярэчыць, што вы намагаліся выратаваць увесь Хогвартс, а такая
важная мэта вартая вялікіх рызыкаў. Гэта --- \emph{хлусня.} Калі б вы...

--- Калі б я проста прыняў аплявуху, перачакаў, а потым абраў бы найлепшы момант для 
помсты... --- сказаў Гары хрыпла. --- Зразумела. Але гэта значыла прайграць. 
Дазволіць яму дамінаваць. Гэта тое самае, чаго не змог зрабіць Цёмны Лорд з 
Майстрам, у якога ён жадаў вучыцца.

Прафесар Квірэл кіўнуў.

--- Я бачу, што вы добра зразумелі. І таму, містэр Потэр, сёння вы вывучыце, як 
прайграваць.

--- Я...

--- Я не жадаю чуць ніякіх супярэчанняў, містэр Потэр. Відавочна, што вам 
жыццёва патрэбны гэты урок, і ў вас дастаткова сілы, каб яго засвоіць. 
Абяцаю, што ваш досвід не будзе такім жа жорсткім, як быў мой, хаця вы і 
можаце запомніць гэта, як найгоршыя пятнаццаць хвілін у вашым маладым жыцці. 

Гары зглынуў.

--- Прафесар, --- сказаў ён тонкім голасам, --- а мы можам зрабіць гэта калісьці
пазжэй?

--- Не, --- сказаў прафесар Квірэл. --- Прайшло толькі пяць дзён вашага першага 
курса ў Хогвартс, і ўжо столькі здарылася. Сёння пятніца. Наш наступны занятак па 
Баявой Магіі ў сераду. Субота, нядзеля, панядзелак, ауторак, серада... не, мы не 
можам дазволіць сабе чакаць так доўга.

Нехта засмяяўся, але ніхто не падхапіў.

--- Калі ласка, лічыце гэта загадам вашага настаўніка, містэр Потэр. Я маю на 
ўвазе, што калі вы адмовіцеся, я, у сваю чаргу, адмоўлюся вучыць вас баявым
заклёнам, бо ведаю, што хутка да мяне дойдуць звесткі пра тое, што вы 
моцна пашкодзілі кагосьці або нават забілі. Нажаль, я чуў чуткі, што 
вашыя пальцы --- ужо грозная зброя. Прашу вас не карыстаць іх падчас гэтага 
ўрока.

Нярвовыя напруджаныя смяшкі прабеглі па аудыторыі.

Гары адчуваў, што зараз заплача.

--- Прафесар, калі вы не спыніцеся, гэта мяне вельмі 
ўззлуе, а я праўда-праўда не хацеў бы больш сёння злавацца...

--- Сэнс не ў тым, каб \emph{пазбягаць} гневу, --- адказаў прафесар Квірэл, 
выглядаючы вельмі змрочна. --- Гнеў --- натуральнае пачуццё. Вы павінны навучыцца 
паразе нават, калі адчуваеце гнеў. Або прынасмі \emph{ўдаваць,} што 
ваш супернік перамог, і пачаць планаваць сваю помсту. Як я зрабіў раней з 
містэрам Гойлам, калі, канешне, вы толькі не думаеце, што ён і сапраўды 
апынуўся лепей за мяне...

--- Я не лепей! --- крыху гістэрычна крыкнуў містэр Гойл са свайго месца. --- 
Я ведаю, што вы толькі ўдавалі! Калі ласка, не пачынайце планаваць помсту!

Гары зноў адчуваўся млосна. Прафесару Квірэлу не было вядома пра Гарын цёмны бок.

--- Прафесар, нам праўда варта абмяркаваць гэта пасля заняткаў...

--- І мы абмяркуем, --- паабяцаў прафесар Квірэл. --- Пасля таго, як вы вывучыцеся
прайграваць. Паверце мне на слова, я гарантую адсутнасць усяго, што можна вас параніць,
або нават вызваць нейкі значны боль. Ваш сапраўдны боль узнікне ад таго,
наколькі цяжка вам будзе прыняць паразу, і прымусіць сябе прайграць, замест таго, каб 
біцца ў адказ і эскаліраваць сітуацыю да вашай перамогі.

Подых Гары стаў частым, панічным. Ён быў спужаны больш, чым калі пакінуў 
клас зеллеварэння. 

--- Прафесар Квірэл, --- здолеў ён сказаць, --- я не хачу, каб вас зволь...

--- Мяне не звольняць, --- сказаў прафесар Квірэл, --- калі \emph{вы} скажаце 
ім потым, што гэты ўрок быў вам патрэбны. А я думаю, што вы так і скажаце, --- 
на пэўны момант голас прафесара Квірэла стаў вельмі сухі. --- Паверце мне, 
яны заплюшчвалі вочы на значна горшыя рэчы, якія адбываліся ў школьных 
калідорах. Гэты занятак адрозніваецца толькі тым, што падзеі адбываюцца ў 
класным пакоі.

--- Прафесар Квірэл, --- прашаптаў Гары, ведачы, што яго голас далятае да ўсіх 
праз экраны, --- вы шчыра думаеце, што калі я не згаджуся, то ў будучыні 
змагу нанесці камусьці шкоду?

--- Так.

--- Тады, --- сказаў Гары, і галава ў яго закруцілася, --- я згодны.

Прафесар Квірэл павярнуўся ў бок Слізэрына. 

--- Тады... з поўнай згоды свайго настаўніка, і ва ўмовах, калі Снэйп не 
можа кантраляваць вашыя дзеянні... ці жадае нехта з вас паказаць нам сваю
перавагу над Хлопцам-які-выжыў? Пхнуць яго, паваліць на зямлю, пачуць, яе 
ён будзе маліць літасці?

Пяць рук узняліся над сталамі.

--- Усе, хто падняў рукі --- вы поўныя ідыёты. Якую частку з выразу 
\emph{"удаваць, што ваш супернік перамог"} вы не ўцямілі? Калі Гары Потэр 
сапраўды стане наступным Цёмным Лордам пасля заканчэння Хогвартс, 
ён лёгка вас зловіць і заб'е паасобку. 

Пяць рук хутка зніклі.

--- Я не буду Цёмным Лордам, --- сказаў Гары слабым голасам. --- Клянуся не 
чыніць помсты тым, хто дапаможа мне навучыцца прайграваць. Прафесар Квірэл,
\emph{калі ласка}... вы можаце спыніць гэта?

Прафесар Квірэл уздыхнуў.

--- Мне вельмі шкада, містэр Потэр. Я разумею, што гэта можа вас раздражняць
незалежна ад таго, ці маеце вы намер станавіцца наступным Цёмным Лордам.
Але \emph{тым} вучням таксама патрабуецца важны жыццёвы ўрок. Ці прымеце вы 
квірэл-бал у якасці прабачэння? 

--- Дайце два, --- сказаў Гары.

Нечаканы смех крыху расчыніў напруджанне ў зале. 

--- Добра.

--- І пасля заканчэння Хогвартс я злаўлю вас і \emph{заказытаю}. 

Яшчэ смех у зале, аднак прафесар Квірэл не ўсміхнуўся.

Гары адчуваў, быццам б'ецца з анакондай, намагаючыся накіраваць размову 
па вузкаму шляху, які зможа пераканаць людзей, што ён не Цёмны Лорд... 
\emph{і чаму} прафесар Квірэл быў у гэтым так упэўнены?

--- Прафесар, --- сказаў Драко звычайным неўзмоцненым голасам. --- У мяне таксама 
няма амбіцый стаць бязглуздым Цёмным Лордам. 

У зале павісла шакаваная цішыня. 

\emph{Ты не павінен гэта рабіць!} --- амаль не выкрыкнуў Гары, але стрымаўся. 
Паміж іншым, Драко мог хацець, каб ніхто не ведаў, што ён дапамагаў Гары 
праз іх сяброўства... або жаданне ўдаваць іх сяброўства...

Назваўшы гэта "жаданнем удаваць сяброўства", Гары адчуў сорам. Калі Драко 
хацеў яго ўразіць учынкам, у любым выпадку ў яго атрымалася выдатна.  

Прафесар Квірэл паглядзеў на Драко змрочна. 

--- Вас хвалюе, чаму гэта вы не можаце ўдаваць, што прайгралі, містэр 
Малфой? Што ўразлівасць містэра Потэра, якую я апісаў, таксама ўласціва і вам?
Усё ж, думаю, што \emph{вашае} выхаванне з гэтым павінна было справіцца. 

--- Ва ўсім, што тычыцца размоваў --- так, --- сказаў Драко, ужо з экранаў.
--- Не тады, калі справа даходзіць да пханняў і валянняў па зямлі. Я хачу 
стаць такім жа моцным, як і вы, прафесар Квірэл.

Бровы прафесара Квірэла папаўзлі ўверх, і там і засталіся.

--- Баюся, містэр Малфой, --- сказаў ён праз некалькі секунд, --- што гэты ўрок 
быў падрыхтаваны адмыслова для містэр Потэра, ён ўключае ўдзел некаторых старэйшых 
слізэрынаў, і не спрацуе ў вашым выпадку. Калі хочаце ведаць маё прафесійнае 
меркаванне, то я думаю, што ў вас ужо ёсць моц. Калі б я дазнаўся, што вы 
ўчынілі нешта падобнае на тое, што ўчыніў сёння містэр Потэр, вам бы дастаўся 
ад мяне персанальны ўрок. Але я цалкам упэўнены, што да мяне ніколі не 
дайдуць чуткі такога роду пра вас.

--- Я разумею, прафесар, --- сказаў Драко.

Прафесар Квірэл агледзеў залу. 

--- Хто яшчэ жалае стаць мацней?

Некаторыя вучні знервавана глядзелі па баках. Некаторыя выглядалі, быццам 
гатовыя нешта сказаць, але так і не здолелі. У выніку, ніхто нічога не адказаў.

--- Дарко Малфой стане генералам адной з вашых армій, --- сказаў прафесар Квірэл,
--- калі ён згодны ўдзельнічаць у маіх пазакласных занятках. А цяпер, калі ласка,
містэр Потэр, падыдзіце да мяне.


\later

--- \emph{Так}, --- сказаў яму прафесар Квірэл, --- \emph{гэта павінна адбыцца 
перад вашымі сябрамі, бо перад імі вы атакавалі Снэйпа, і перад імі 
вы павінны прайграць.}

Таму першакурснікі глядзелі. У магічна ўсталяванай цішыні, і з загадам ад 
абодвух Гары і настаўніка не ўлазіць. Герміёна сядзела, адвярнуўшы твар, але
яна нічога не спрабавала сказаць, верагодна таму, што таксама была тады на 
ўроку зеллеварэння.

Гары стаяў на сінім гімнастычным маце, які можна знайсці ў любой 
маглаўскай спартыўнай зале. Прафесар Квірэл падклаў яго для таго, каб 
Гары не пашкодзіўся, калі яго будуць валяць па зямлі.  

Гары баяўся таго, што ён можа зрабіць. Калі прафесар Квірэл быў правы на конт 
ягонага смяротнага намеру...

Палачка Гары ляжала на настаўніцкам стале, не таму, што Гары не ведаў ніякіх 
абарочных заклёнаў, а каб (як ён падазраваў) Гары не мог уваткнуць яе камусьці 
ў вока. Таксам там ляжаў ягоны махляскін, які ўтрымліваў часаварот --- у абарончым 
кантэйнеры, але ўсё роўна патэнцыйна крохкі.

Гары спытаў прафесара Квірэла трансфігураваць яму баксёрскія пальчаткі, і магічна 
зашпіліць іх на яго руках. Той паглядзеў на Гары позіркам, поўным ціхага разумення, і 
адмовіўся.

\emph{Толькі не пашкодзь іх вочы, толькі не ў вочы, бо гэта будзе канец твайго 
жыцця ў Хогвартс, арышт, турма, і ганебная смерць...} --- такія думкі спрабаваў убіць 
у сваю свядомасць Гары, спадзяваючыся, што яна застанецца там, на выпадак,
калі яго смяротны намер возьме верх.

Прафесар Квірэл вярнуўся ў залу. За ім зайшлі трынаццаць старэйшых слізерынаў
з розных курсаў. У адным з іх Гары пазнуў студэнта, у ягога ён кінуў пірагамі.
Таксама прысутнічалі два іншых удзельніка той канфрантацыі. Студэнт, які тады спрабаваў 
спыніць іх, не прыйшоў.

--- Паўтараю, --- сказаў строга прафесар Квірэл, --- Потэру фізічна не шкодзіць.
Любыя \emph{няшчасныя} выпадкі будуць меркавацца як наўмысныя. Зразумела?

Старэйшыя слізэрыны заківалі, усміхаючыся.

--- Тады давайце абламаем крыху рагоў Хлопцу-які-выжыў, --- сказаў прафесар 
Квірэл з крывой усмешкай, якую зразумелі толькі першакурснікі.

Па нейкай невыказанай узаемнай згодзе першым у чарзе стаяў "хлопец з пірагамі".

--- Потэр, --- сказаў прафесар Квірэл, --- пазнаёмцеся з містэрам Перэгрынам Дэрыкам.
Ён лепей за вас, і ён збіраецца вам гэта даказаць.

Дэрык зрабіў крок наперад, другі...  Гарын мозг бязладна крычаў, што ён павінны не бяжаць,
павінны не драцца ў адказ...

Дэрык спыніўся на адлегласці выцягнутай рукі ад Гары.

Гары не быў пакуль што ўззлаваным, толькі спужаным. Бо перад ім стаяў амаль дарослы
мужык на паўметра вышэй за Гары, з чотка аформленай мускулатурай, шаціннем на 
шчаках, і ўсмешкай жудаснага чакання. 

--- Маліце яго не шкодзіць вам, --- сказаў прафесар Квірэл. --- Магчыма, убачыўшы,
наколькі вы нікчэмны, яму стане нудна і ён сыйдзе.

Астатнія слізэрыны агідна засмяяліся.

--- Калі ласка, --- сказаў Гары дрыжачым голасам, --- не бейце мяне...

--- Пагучала не вельмі шчыра, --- сказаў прафесар Квірэл.

Усмешка Дэрыка пашырылася. Няўдалы імбецыл адчуваў сябе вельмі крутым, і...

...тэмпература крыві Гары імкліва падала...

--- Калі ласка, не бейце, --- паспрабаваў ён яшчэ раз.

Прафесар Квірэл пачакаў галавой. 

--- Якім чынам, дзеля Мерліна, скажыце, вы здолелі сказаць гэта так, што яно
прагучала, як абраза, Потэр? Вы можаце чакаць толькі аднаго адказу ад містэра 
Дэрыка.

Дэрык з гатоўнасць ступіў наперад, пхнуўшы яго грузямі.

Гары адкінула на некалькі футаў, потым ён здолеў спыніцца, стаў прама і 
паглядзеў на Дэрыка ледзяным позіркам.

--- Дрэнна, --- сказаў прафесар Квірэл, --- вельмі дрэнна.

--- Ты ў мяне ўрэзаўся, Потэр, --- сказаў Дэрык. --- Прасі прабачэння.

--- Прабач! 

--- Гэта не гучыць выбачальна, --- сказаў Дэрык.

Гарыны вочы выбліснулі абурэннем --- ён \emph{ужо} пастараўся, каб гучала ўмольна...

Дэрык пхнуў Гары, і ён упаў на мат, прызямліўшыся на рукі і калені.

Сіняя тканіна мата лунала ў палі зроку Гары, дзесьці блізка і далёка адначасова.

Ён пачынаў сумнявацца ў рэальных матывах прафесара Квірэла.

Нага апусцілася на Гарын азадак, і праз імгненне яго пхнулі ўбок, так што ён
расцягнуўся на спіне.

Дэрык засмяяўся.

--- Гэта \emph{весела,} --- сказаў ён.

Усё, што трэба было сказаць Гары --- што ўсё скончана. Наведаць кабінет Майстра.
Назіраць, як пакідае Хогвартс гэты так званы \emph{прафесар Абароны}, а з ім 
і ўсе яго бязглуздыя прыдумкі... прафесар МакГонагал, канешне, уззлуецца, але... 

(Аблічча МакГонагал з'явілася ва ўяўленні Гары, аднак яно было не злое, а 
проста сумнае...)

--- Скажыце, што ён лепей за вас, Потэр, --- пачуўся голас прафесара Квірэла.

--- Ты. Лепей. За. Мяне.

Гары пачаў падымацца, але Дэрык паклаў нагу яму на грудзі, і пхнуў назад на мат.

Наваколле станавілася празрыстым, як крышталь. Схема падзей і іх наступстваў 
ляжала перад ім у неймавернай яснасці. Гэты дурань не чакае, што Гары можа 
адказаць, і хуткі ўдар нагамі у пах вырубіць яго на некалькі хвілін...

--- Паспрабуйце яшчэ раз, --- сказаў прафесар Квірэл, але Гары ўжо з раптоўным 
рухам перакуліўся, падняўся на ногі, і накіраваўся ў бок свайго сапраўднага 
ворага --- настаўніка...

--- У вас не хапае цярпення, --- сказаў Прафесар Квірэл.

Гары спатыкнуўся. Ягоны розум намаляваў выяву сівога старога, у яго з рота 
ішла кроў, бо Гары вырваў яму горла...

...Дэрык зноў паваліў Гары на мат і з размаху сеў на яго, выгнаўшы ўсё паветра 
з яго лёгкіх.

--- Спыніся! ---- закрычаў Гары. --- Калі ласка, спыніся!

--- Лепей, --- сказаў прафесар Квірэл. --- Гэта нават прагучала шчыра.

Гэта і \emph{было} шчыра. Самае жудаснае было ў тым, што гэта было шчыра. Гары 
цяжка і часта дыхаў, страх і гнеў бароліўся ў ім...

--- Здавайся, --- сказаў прафесар Квірэл.

--- Я... здаюся... --- выціснуў з сябе Гары.

--- Мне падабаецца, --- сказаў Дэрык дзесьці з вышыні. --- Здавайся яшчэ.


\later

Мноства рук штурхалі Гары з аднаго на другі бок кола старэйшых слізерынаў. Гары
ўжо не стрымліваў свой плач, і проста намагаўся трымацца на нагах.

--- Хто ты такі, Потэр? --- сказаў Дэрык.

--- ...лузер, я лузер, я здаюся, ты перамог, ты лепей, спыніся, калі ласка...

Хтосьці падставіў нагу, і Гары, зачапіўшыся, зваліўся на падлогу. Праз секунду 
ён стаў падымацца.

--- \emph{Дастаткова!} --- пачуўся голас прафесара Квірэла, настолькі рэзкі, што 
мог, напэўна, рэзаць сталь. --- Тры крокі назад ад містэра Потэра!

Гары бачыў здзіўленыя выразы на іх тварах. Мароз у ягонай крыві, які так доўга 
стрымліваўся, усміхнуўся з халодным задавальненнем.

Потым Гары ўпаў на мат.

Прафесар Квірэл звярнуўся да старэйшых слізерынаў. Напрыканцы яго прамовы выглядалі
 яны не вельмі.

--- ...і я думаў, што нашчадак рода Малфоеў таксама хоча вам нешта патлумачыць.

Пачуўся голас Драко. Ён гучаў амаль так сама рэзка, як і голас настаўніка, 
у ім сталі чутны ноткі Люцыуса, і ён казаў нешта пра 
\emph{...кінуць цень на факультэт Слізэрын...} і 
\emph{...хто ведае, колькі людзей яго падтрымліваюць толькі ў гэтай школе...} і яшчэ
\emph{...поўная адчутнасць чуцця, не кажучы аб розуме...} і
\emph{...тупыя грамілы, годныя толькі ў лёкаі...} і нешта ў свядомасці Гары, 
нягледзячы на ўсё, што ён ведаў, прылічала Драко да сваіх звязнікаў.

У Гары балела ўсё цела, і магчыма, ён увесь быў у сіняках, яму было холадна,
ягоны розум цалкам знясілены. Ён падумаў пра спеў Фоукса, але не мог успомніць
мелодыю, і яму да галавы толькі прыйшло цвырканне вераб'ёў.

Потым Драко скончыў, і прафесар адпусціў слізерынаў, і тады Гары расплюшчыў вочы,
і паспрабаваў сесьці. 

--- Па... пачакайце, --- сказаў ён, з цяжкасцю прамаўляючы словы, --- я хачу ім 
нешта...

--- Стаяць, --- сказаў холадна прафесар Квірэл.

Гары, хіляючыся, устаў на ногі. Ён стараўся не глядзець у бок аднакурснікаў.
Ён не хацеў бачыць іх позіркі. Не хацеў бачыць іх жаль.

Таму ён глядзеў толькі на старэйшых слізерынаў, якія ўсе яшчэ былі ў стане шоку.

Яго цёмны бок, чакаў гэтага моманту, увесь гэты час удаваючы, што прайгравае.

--- Ніхто не будзе...

--- Стойце, --- сказаў прафесар Квірэл. --- Калі гэта тое, што я думаю, ім лепей 
гэтага не чуць. Усім нам патрэбны асабісты ўрок, містэр Потэр. 

--- Д.. добра...

--- Можаце ісці.

Старэйшыя слізерыны амаль выбеглі з залы. Дзверы за імі зачыніліся самі.

--- Ніхто не будзе чыніць ім помсты, --- сказаў хрыпла Гары. --- Я прашу аб 
гэтым усіх, хто ліча мяне сваім сябрам. Я атрымаў свой урок, яны дапамаглі мне 
яго вывучыць, і яны ў свой час атрымаюць свой урок. Калі будзеце камусьці расказваць,
не забудзьцеся і пра гэтую маю просьбу.

Гары паглядзеў на прафесара Квірэла.

--- Вы прайгралі, --- сказаў прафесар, ягоны голас першы раз за урок быў мягчэй.
Было дзіўна, бо падавалася, што гэты голас фізічна не можа так гучаць.

Гары сапраўды прайграў. Былі моманты, калі ягоны гнеў цалкам знікаў, выціснуты 
страхам, і ў тыя моманты ён маліў шчыра...

--- І вы ўсё яшчэ жывы? --- спытаў прафесар Квірэл з той жа дзіўнай мягкасцю.

Гары здолеў кіўнуць.

--- Не кожная параза так выглядае. Існуюць кампрамісы і перамовы. Існуюць іншыя 
спосабы задобрыць булінг. Гэта асобнае мастацтва --- маніпуляваць людзьмі, дазваляючы 
ім дамінаваць над сабой. Але па-першае, параза павінна быць \emph{магчымасцю}.
Ці запомніце вы, як прайгралі?

--- Так.

--- І здолееце прайграць у будучыні?

--- Я... думаю, да...

--- І я таксама так думаю, --- прафесар Квірэл пакланіўся так нізка, што ягоныя 
рэдкія валасы амаль крануліся падлогі. 

--- Віншую вас, Гары Потэры, вы перамаглі.

І без анякага сігналу, агульная хваля апладысментаў наляцела на яго як 
цунамі.

Гары быў у шоку. Ён рызыкнуў зірнуць у бок аднагрупнікаў, і ўбачыў на іх 
тварах не жаль, а ўсхваляванне. У далоні пляскалі Рэйвенкло, Грыфіндор, Хафлпаф,
і нават некаторыя са Слізэрына, магчыма таму, што Драко таксама пляскаў.
Некаторыя ўставалі са сваіх стулаў, а палова Грыфіндора ўжо стаяла на сталах.

І Гары стаяў у цэнтры гэтага патока павагі, адчуваючы сябе мацней, і нават 
крыху вылечаным.

Прафесар Квірэл пачакаў, пакуль апладысменты не схціхлі. На гэта спатрэбіўся 
пэўны час.

--- Здзіўлены, містэр Потэр? --- сказаў прафесар Квірэл. --- Вы толькі што
высветлілі, што сапраўдны сусвет не \emph{заўсёды} адпавядае вашым самым 
страшным кашмарам. Так, калі вы былі б нейкай ананімнай ахвярай булінга, 
вы бы згубілі іх павагу, хаця яны бы вас і пажалелі з добрых пачуццяў. Баюся,
гэта ў чалавечай прыродзе. Але \emph{вас} яны ўжо ведалі, як моцную фігуру, 
і сталі сведкамі таго, як вы сустрэлі свае страхі, і працягвалі з імі 
біцца, хаця і маглі проста сысці ў любы момант. Ці страціў я вашую павагу,
калі расказаў пра сваю ўласную ганьбу?

Гары адчуў, як пячэ ў горле, і адчайна сціснуў яго. Ён не насколькі верыў 
у гэтую містычную павагу, каб не пачаць плакаць зноў перад усім курсам.

--- Ваша \emph{экстраардынарнае} дасягенне заслугоўвае экстраардынарнай
узнагароды, містэр Потэр. Калі ласка, прыміце яе ад імя майго факультэту, і 
помніце, што не ўсе слізерыны аднолькавыя. Ёсць слізерыны, а есць \emph{слізэрыны}.


Прафесар Квірэл усміхаўся вельмі шырока, калі ён дадаў:

--- Пяцьдзесят адзін бал да Рейвенкло.

На долю секунды павісла шакаваная пауза, а потым сектар Рэйвенкло выбухнуў 
радасным смехам, свістам і ўлюлюканнем.

(У гэты момант Гары адчуў пэўны ўкол, бо прафесар МакГонагал была права, 
наступствы павінны наступаць, за дрэнныя ўчынкі патрабуецца расплата, нельга
проста ўзяць і вярнуць усё, як было...)

Але ён бачыў шчаслівыя твары сваіх аднагрупнікаў, і ведаў, што ніколі не зможа 
адмовіць.

Ягоны розум зрабіў прапанову. Прапанова была добрая. Гары нават не мог паверыць,
што яго розум наогул можа трымаць яго на нагах, не кажучы пра добрыя прапановы.

--- Прафесар Квірэл, --- сказаў Гары, настолькі роўна, наколькі мог з палаючым горлам,
--- вы --- цалкамі такі, якім і павінен быць прадстаўнік вашага факультэту. Думаю,
менавіта так сабе і ўяўляў свой факультэт Салазар Слізэрын, калі яны заснавалі 
Хогвартс. Я ўдзячны вам і вашаму факультэту, --- Драко амаль незаўважна ківаў, 
і рабіў пальцам кругавыя рухі, \emph{працягвай у тым жа родзе}, --- і я думаю,
што гэта заслугоўвае траякратнага "ура" у гонар Слізэрына. Усе разам!, --- 
ён зрабіў паузу. 

--- \emph{Ура!} --- толькі некалькі людзей далучыліся да яго.

--- \emph{Ур-ра!} --- гэтым разам большасць рэйвенкло яго падтрымала. 

--- \emph{Ур-р-ра!} --- тут былі амаль усе рэйвенкло, некаторыя хафлпафы, і 
амаль чвэрць грыфіндораў.

Далонь Драко хутка склалася ў кулак з паднятым адзінцом і адразу распаволілася.

Большасць слізерынаў сядзела з выразам недаверу і шоку. Некаторыя глядзелі 
пытальна на настаўніка. Блэйз Забіні разглядваў Гары разважлівым, быццам
нешта падлічваючым позіркам.

Прафесар Квірэл пакланіўся.

--- Гэта я ўдзячны \emph{вам}, Гары Потэр, --- сказаў ён, усё яшчэ з шырокай 
усмешкай. Ён павярнуўся да аўдыторыі. --- А зараз, паверыце вы, ці не --- у нас 
засталося яшчэ паўгадзіны занятка, і гэтага будзе як раз дастаткова,
каб прадставіць вам заклён Простага Шчыта. Містэр Потэр, канеше, пакіне нас дзеля 
заслужанага адпачынку. 

--- Я магу...

--- Ну што за ідыёт, --- сказаў весела прафесар Квірэл. Усе адразу засмяяліся. --- 
Вашыя аднакурснікі навучаць вас потым, або я патрачу асабісты час, калі спратрэбіцца.
Але \emph{зараз} вы знойдзеце трэція дзверы налева за сцэнай, за імі будзе
ложак, крыху выбітна смачных закусак, і крыху лёгкай чытаніны са школьнай бібліятэкі.
Вашыя рэчы, у прыватнасці вашыя падручнікі, застануцца тут. Ідзіце.

Гары паслухаўся.
