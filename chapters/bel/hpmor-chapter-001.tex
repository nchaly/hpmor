\chapter{Дзень вельмі малой верагоднасці}


\begin{chapterOpeningQuote}
    \noindent
    У месяцовым святле срэбны бляск адабіваецца на імгненне...
    
    \vspace*{2ex}
    (чорная мантыя, падзенне)
    
    \vspace*{2ex}
    ...кроў л'ецца літрамі, і жудасны крык.
\end{chapterOpeningQuote}


\lettrine{К}{ожны} сантыметр сцены быў заняты кніжнымі шафамі.
У кожнай шафе было па шэсць паліц, якія амаль дасягалі столі.
Большаць паліц былі застаўлены кнігамі ў цвердых сур'ёзных пераплётах: фізіка, матэматыка, гісторыя,
і гэтак далей. Некаторыя паліцы ў два рады забітыя навуковай фантастыкай у мяккіх пераплётах, прычым 
заднія рады абапіраліся на старыя каробкі або дошкі, каб пярэднія рады іх не загароджвалі.
Але паліц яўна было мала. Кнігі ляжалі паўсюль: на стале, канапе, і невялікія іх кіпы нібыта вырасталі 
пад вокнамі. 

Гэта была гасцёўня ў доме выдатнага вучонана~--- прафесара Майкла Верэс-Эванса, яго жонкі, місіс
Петуніі Эванс-Верэс, і іх прыёмнага сына, Гары Джэймса Потэра-Эванса-Верэса.

Пасярэдзіне пакоя на століке ляжаў ліст у пергаментным канверце без маркі, на якім смагардавым
атрамантам было выведзена \emph{Містэру Г.~Потэру}.

Прафесар і яго жонка вялі размову даволі рэзка, але да крыку справа не дайшла. Прафесар
лічыў крык дурным тонам.

--- Гэта такі жарт?~--- спытаў Майкл. Па яго тону было зразумела, што яго вельмі палохала 
магчымасць адсутнасці жарта.

--- Мая сястра была ведзьмай,~--- паўтарыла Петунія. Яна выглядала спалоханай, але здавацца не
збіралася.~---~Яе муж быў чараўніком.

--- Гэта нонсэнс!~--- рэзка сказаў Майкл.~---Яны былі на нашам вяселлі... яны завіталі на Раство...

--- Я іх папярэдзіла... ты павінен быў нічога не ведаць, --- прашаптала Петунія. --- Але гэта праўда.
Я сама бачыла іх здольнасці...

Прафесар закаціў вочы.

--- Дарагая, я разумею, што можаш быць не знаёмая з крытычнай літаратурай. Але ты нават не 
ўяўляеш сабе, наколькі лёгка для трэніраванага "мага" падрабіць нешта нібыта цудоўнае.
Памятаеш, як я навучыў Гары гнуць лыжкі? Напрыклад, калі яны ўдаюць, што ведаюць твае думкі,
то гэта адмысловая тэхніка, яна называецца "халоднае чытанне"...

--- То было далёка ад згібання лыжак...

--- Што то было, напрыклад?

Петунія пакусала сябе за губу.

--- Мне цяжка гэта сказаць. Ты падумаеш, што я... --- яна зглынула. --- Слухай, Майкл. Я не 
заўсёды была... такой, --- яна паказала на сябе, маючы на ўвазе сваю ладную постаць. --- Лілі
зрабіла мяне такой. Таму што я спытала. Я маліла яе, гадамі. Лілі была прыгажуня, а я...
я ненавідзела яе за гэта, а ў яе да тагож была і магія, можаш уявіць сабе мае становішча?
І я \emph{маліла} яе скарыстаць крыху яе магіі на мне... нават калі ў мяне няма магічных
здольнасцяў, так прынамсі я была бы прыгожай.

У яе вачах блішчэлі слёзы.

--- І Лілі заўсёды мне адмаўляла, і яна выдумляла нейкія недарэчныя апраўданні, кшталту, што 
канец света настане, калі яна зробіць добрую справу сваёй сястры, або што цэнтаўр ёй забараніў ---
самае бязглузды варыянт, --- і я ненавідзела яе за гэта. Скончыўшы каледж, я нейкі час сустракалася
з хлопцам, Вернанам Дурслі, ён быў тлусты, і толькі ён размаўляў са мной у каледжы. І ён 
аднойчы раскаўаў мне, што свайго першага сына ён назаве Дадлі. І тады я падумала: \emph{што за
бацька даць свайму дзіця імя Дадлі Дурслі?} Быццам усё маё будучае жыццё развярнулася перад
маімі вачыма ў той момант, і гэта было жудасна. Я напісала сястры, і сказала, што калі яна мне 
не дапаможа, то мне... лепей...

Петунія ўсхліпнула.

--- Урэшце, --- працягвала яна ціха, --- яна здалася. Яна сказала, што гэта небяспечна, і я
адказала, што мне ўжо ўсё роўна, і я выпіла тыя зёлкі, і некалькі тыдняў хварэла, а потым мая
скура ачысцілася, і я набрала вагу, і... я стала прыгожай, і людзі сталі \emph{добрыя} да мяне, ---
яе голас перарваўся, --- і я перастала ненавідзець сваю сястру, асабліва, калі ўзнала, да
чаго нарэшце магія яе давяла...

--- Дарагая, --- сказаў мягка Майкл, --- ты проста захварэла, і пасля добрага пасцельнага 
рэжыму твая вага і скура сталі лепей у працэсе выздараўлення, самі па сабе. Або пасля хваробы
твая дыета змянілася, і таму...

--- Яна была ведзьма, --- паўтарыла Петунія. --- Я бачыла сваімі вачамі.

--- Петунія, --- сказаў Майкл. У яго голасе пачынала чуцца раздражненне. --- Ты \emph{ведаеш}, што
так не бывае. Ці сапраўды мне трэба тлумачыць табе?

Яна заламвала рукі, і падавалася, што зараз яна заплача.

--- Любоў мая, я ведаю, што не магу цябе пераканаць лагічна, але, калі ласка, ты павінен мне 
проста паверыць...

--- \emph{Тата! Мама!}

Яны абодва замаўчалі, і здзіўлена паглядзелі на Гары, быццам забыўшыся на тое, што ў пакоі 
яшчэ нехта быў.

Гары зрабіў вялікі ўдых.

--- Мама, у тваіх бацькоў не было магіі, так?

--- Так, --- сказала Петунія збянтэжана.

--- Такім чынам, калі Лілі атрымала ліст, ніхто з тваёй сям'і не ведаў пра магію. Як здарылася,
што яны паверылі?

--- А-а... --- сказала Петунія. --- Лілі не проста атрымала ліст. Яго прывёз настаўнік з Хогвартса.
Ён... --- яна хутка зірнула на Майкла, --- ён паказаў нам нешта магічнае.

--- І таму ім не трэба было спрачацца на гэты конт, --- сказаў Гары цвёрда. Ён адчайна спадзяваўся,
што ў гэты раз, хаця бы раз у жыцці, бацькі яго паслухаюць. --- Калі мы зможам заклікаць да нас
настаўніка з Хогвартса, і сваімі вачыма ўбачыць магію, тата згадіцца, што яна існуе. І калі не ---
то мама згадзіцца, што магія не існуе. У гэтым --- сутнасць навуковага метада, які і прыдумалі, каб
нам не прыходзілася бясконца спрачацца.

Прафесар павярнуўся да Гары, і праглядзеў на яго зверху ўніз, і як заўсёды ў такіх сітуацыях,
крыху пагардліва.

--- Да ладна, Гары! Сур'ёзна, \emph{магія?} Я быў упэўнены, што ты не мог успрыняць гэта сур'ёзна,
нават калі табе толькі дзесяць. Магія --- самая антынавуковая рэч на свеце!

У Гары ад крыўды аж скрывіўся рот. Усё жыццё з ім абыходзіліся добра, магчыма значна лепей, чым
большасць генетычных бацькоў ставяцца да сваіх уласных дзяцей. Гары вучыўся ў лепшых пачатковых
школах, і калі справы з імі не склаліся, яму забяспечылі асабістых рэпетытараў з невычарпальных
працоўных рэзерваў галодных студэнтаў. Яго заўседы заахвочвалі вывучаць усё, што прыцягвала яго
ўвагу, набывалі кнігі, якімі ён цікавіўся, спансіравалі ўдзел ва ўсіх алімпіядах, дзе ён хацеў
удзельнічаць. Ён атрымліваў любыя рэзонныя рэчы, якія жадаў, акрамя, магчыма, малейшай кроплі 
павагі. Не варта чакаць ад Оксфардскага прафесара біяхіміі, што ён будзе слухаць
параду дзесяцігадовага хлопчыка. Вядома, ён выслухае, каб \emph{прадэманстраваць інтарэс},
бо так робяць усе Добрыя Бацькі. Але прыймаць дзіця \emph{сур'ёзна?} Наўрад ці. 

Часам Гары хацелася наараць на свайго бацьку.

--- Мам, --- сказаў Гары. --- Калі ты хочаш перамагчы ў спрэчцы з татам, паглядзі ў другім
раздзеле першага тома "фейнманаўскіх лекцый па фізіцы". Там ёсць абзац пра тое, як філосафы
бясконца кажуць аб тым, што і як трэба рабіць у навуцы, і гэта ўсё хлусня, бо адзінае правіла
навуцы ў тым, што даказаць нешта можна толькі назіраннем, калі ты глядзіш, як нешта ў свеце
адбываецца, і рапартуеш, што бачыў. Ммм... І я не магу з ходу прыпомніць, у якой кнізе
прачытаць пра тое, что любая праблема ў навуцы вырашаецца праз эксперымент, а не праз
спрэчкі...  

Маці праглядзела на Гары і ўсміхнулася.

--- Дзякуй, Гары. Але... --- яна пазняла позірк на свайго мужа. --- Я не хачу перамагаць у гэтай
спрэчцы з тваім бацькам. Я хачу, каб мой муж паслухаў сваю жонку, і проста ёй паверыў...

Гары на імгненне прыкрыў вочы. \emph{Безнадзейна.} Абодва яго бацькі былі проста безнадезныя.

І спрэчка зараз перайшла ў \emph{той самы} рэжым, калі маці спрабавала прымусіць бацьку адчуваць
сябе вінаватым, а бацька справаў вызваць у маці пачуццё недарэчнасці. 

--- Я іду ў свой пакой, --- абвесціў Гары. Яго голас крыху дрыжэў. --- Паспрабуйце не надта сварыцца
на гэты конт. Мы хутка высветлім праўду, так?

--- Канешне, Гары, --- сказаў бацька, а маці падбадзерыла яго пацалункам, і яны 
працягнулі спрачацца, як толькі Гары падняўся на другі паверх. 

Ён зачыніў за сабой дзверы, і праспрабаваў абдумаць сітуацыю.

Самае смешнае, што ён быў \emph{павінны} згадзіцца з бацькам. Ён ні разу не бачыў доказаў існавання
магіі, але маці калаза, што існуе цэлы магічны свет. Як мыгчыма трымаць у сакрэце такое? З дапамогай
яшчэ большай магіі? Гэта выглядала даволі падазроным агрументам.

Па ўсяму выходзіла, што маці або жартавала, або хлусіла, або звар'яцела (у парадку ўзрастання
жахлівасці). То, што ліст з'явіўся ў паштовай скрыні без маркі, можна было патлумачыць тым, што
яна сама яго туда паклала. Лёгкае вар'яцтва было значна, значна менш неверагодным, чым тое, што
ў свеце сапраўды магло адбывацца такое.

Але чамусці Гары быў цалкам упэўнены, што магія была рэальнасцю, з таго самога моманту, як ён
убачыў меркаваны ліст са Школы Вядзьмарства і Чарадзейства пад назвай Хогвартс.

Гары паморшчыўся і пацёр лоб. \emph{Не вер усяму, што думаеш,} як казала адна з яго кніг.

Але гэтая дзіўная ўпэўненасць... Гары адчуваў, што ўжо проста чакае, што так, прыедзе настаўнік з 
Хогвартса, махне чароўнай палачкай, і атрымаецца магія. Яшчэ было дзіўна, што гэтая ўпэўненасць
ніяк не спрабавала абараніцца ад фальсіфікацый --- не прыдумляла загадзя прычыны, напрыклад,
чаму настаўнік не з'явіцца, або чаму ён толькі будзе здольны гнуць лыжкі.

\emph{Скуль ты ўзялося, дзіўнае маленькае прадказанне?} --- Гары накіраваў думку ва ўласны мозг. ---
\emph{Чаму я веру ў тое, у што я веру?}

Звычайна ў Гары добра атрымлівалася адказваць на гэтае пытанне, але ў гэтым канкрэтным выпадку ў 
яго не было ні малейшага паняцця, аб чым думае яго мозг.

Гары мысленна паціснуў плячыма. Калі бачыш на дзярах плоскую пласціну --- таўкай, калі бачыш ручку
--- цягні, а калі сустракаеш цікавую гіпотэзу, то проста ідзі і правярай яе.

Ён узяў ліст паперы з шуфлядкі, і пачаў пісаць.

\medskip

{\LARGE \hpFontFasthand{}Паважанай намесніцы дырэктара}

\medskip


Гары задумаўся, потым змяў ліст, кінуў у карзіну, узяў новы ліст, і, пстрыкнуўшы сваім механічным
алоўкам, высунуў стрыжаць на чарговы міліметр. Абставіны патрабавалі сапраўднай каліграфіі.

\begin{writtenNote}
\letterAddress{Паважанай намесніцы дырэктара\\Мінерве МакГонагал,\\або каму гэта можа тычыцца:}

Я толькі што атрымаў ваш ліст аб прыёме ў Хогвартс, адрэсаваны містэру Г. Потэру. Вы можаце
гэтага не ведаць, але мае генетычныя бацькі, Джэймс і Лілі Потэр (у дзявоцстве --- Лілі Эванс)
памерлі. Я быў усыноўлены сястрой Лілі, Петуніяй Эванс-Верэс, і яе мужам, Майклам Верэс-Эванс.

Я вельмі зацікаўлены ідэяй вучыцца ў Хогвартсе, пры умовах, што такое месца сапраўды існуе.
Мая маці Петунія кажа, што ведае пра магію, але сама карыстаць яе не можа. Мой бацька настроены
вельмі скептычна. Сам я неўпэўнены. Таксама, я не ведаю, як здабыць кнігі і абсталяванне,
пералічаныя ў вашам лісце.

Маці згадала, што вы дасылалі прадстаўніка Хогвартс да бацькоў Лілі Потэр (тады --- Лілі Эванс),
каб прадэманстраваць, што магія рэальна, і, я мяркую, дапамагчы Лілі набыць матэрыялы для школы.
Калі вы можаце зрабіць тое ж самае для маёй сям'і, я быў бы вам вельмі ўдзячны. 


\letterClosing[Шчыра ваш,]{Гары Джэймс Потэр-Эванс-Верэс.}

\end{writtenNote}


Гары дадаў свой адрас, потым паклаў ліст у канверт, і адрэсаваў яго ў Хогвартс. Падумаўшы
яшчэ крыху, ён знайшоў свечку, падпаліў яе, і накапаў воску на клапан. Пасля канцом
перачыннага ляза ён начарціў на пячатке свае ініцыялы "Г.Дж.П.Э.В." Калі ён і збіраўся 
занурацца ў вар'яцтва, то ён будзе рабіць гэта стыльна.

Потым ён выйшаў са свайго пакоя і спусціўся на ніз. Бацька ўсё яшчэ сядзеў у гасцёўні, і чытаў
кнігу па вышэйшай матэматыцы, каб паказаць, які ён разумны. Маці была на кухні і гатавала
адну з улюбёных бацькіных страваў, каб паказаць, як яна яго кахае. Выглядала, што яны наогул
перасталі размаўяць адно з адным. Якімі бы страшнымі не былі спрэчкі, \emph{такая цішыня} была
чамусьці значна горш.

--- Мам, --- сказаў Гары ў палохаючую цішыню, --- я збіраюся праверыць адну гіпотэзу. Па тваёй
тэорыі, як я магу даслаць саву ў Хогвартс?

Маці адвярнулася ад пліты, каб паглядзець на яго, выгляд у яе быў шакаваны. 

--- Я не... я не ведаю. Думаю, проста трэба мець сваю магічную саву...

Гэта было павінна гучаць падазрона, кшталту \emph{выбачай, але ты не можаш праверыць сваю гіпотэзу}, 
але дзіўная ўпэўненасць Гары прымушала яго ісці да канца.

--- Ну, ліст неяк трапіў да нас, --- сказаў Гары, --- таму я збіраюся проста выйсці вонкі, 
памахаць ім у паветры, і крыкнуць "Ліст у Хогвартс!", і пагляжу, ці з'явіцца сава, каб забраць
канверт. Тата, хочаш паглядзець?

Бацька амаль незаўважна пакачаў галавой, і працягваў чытаць. \emph{Ну канешне,} падумаў Гары.
Магія была ганебнай рэччу, у якую верылі толькі дурні. Калі яго бацька дойдзе да праверкі
гіпотэзы... не, нават да таго, каб проста паглядзець на праверку, гэта будзе выглядаць, 
быццам ён прыймае ўдзел і нават сам пачаў верыць...

Толькі калі Гары выйшаў на задні двор, яго прасякла думка, што калі сава сапраўды \emph{прыляціць} і
схопіць ліст з яго рукі, як ён будзе расказваць гэта бацьку.

\emph{Ну што ж, гэта не можа адбыцца ў рэальнасці, так?.. У незалежнасці ад таго, наколькі
мой мозг хоча ў гэта верыць. Калі сава схопіць канверт, у мене будзе хвалявацца аб чысмці 
значна большым, чым меркаванне таты.}

Гары моцна ўздыхнуў і рашуча падняў канверт у паветра.

Ён зглынуў.

Ідэя выкрыкнуць \emph{Ліст у Хогвартс!}, стоячы на заднім двары і трымаючы канверт высока ў
паветры... шчыра кажучы, зараз гэтая ідэя выглядала даволі няёмка.

\emph{Не. Я лепей за тату. Я буду прымяняць навуковы метад, нават калі мне ад гэтага няёмка.}

--- Ліст... --- паспрабаваў сказаць Гары, але атрымаўся нейкі прыдушаны хрып. 

Гары собраў волю ў кулас, і закрычаў у чыстае неба:

--- \emph{Ліст у Хогвартс! Можна мне саву, калі ласка?}

--- Гары? --- пачуўся здзіўлены голас з боку суседзяў.

Гары ірвануў руку ўніз, быццам яна загарэлася, і схаваў канверт за спінаю, быццам то былі
скрадзеныя грошы. Яго твар палаў ад сораму. 

Галава пажылой жанчыны з'явілася над адным з суседскіх плотаў, касматыя сівыя валасы выбіваліся
з-пад сетачкі. Місіс Фігг, якая час ад часу працавала нянькай. Гары.

--- Што ты робіш, Гары?

--- Нічога, --- сціснутам голасам адказаў ён. --- Проста правяраю адну вельмі недарэчную тэорыю...

--- Няўжо ты ўрэшце атрымаў ліст з Хогвартса?

Гары замёр.

--- Так, --- механічна сказаў ён праз некалькі секунд. --- Атрымаў... Яны сказалі, што
чакаюць ад мяне саву да трыццаць першага ліпеня, але...

--- Але ў цябе няма савы. Ох, бяда. І аб чым толькі яны там думаюць, дасылаючы такім, як ты, проста
стандартную форму.

Маршчыністая рука выцягнулася цераз плот, далонь расчынілася ў чаканні. Галава ў Гары
здалася і наогул перастала нешта думаць. Ён аўтаматычна працягнуў ён канверт.

--- Не хвалюйся, даражэнькі, --- сказал місіс Фігг, --- я адашлю кагосьці, вокам міргнуць не
паспееш.

І яе галава знікла за плотам.

На заднім двары доўга панавала цішыня.

Потым пачуўся безвыразны голас хлапчука:

--- Што за?...
